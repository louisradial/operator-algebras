% vim: spl=en_us
\section{Bounded operators in Hilbert spaces}
We recall \crefname{thm:orthogonal_decomposition}, that for every vector \(v \in \hilbert\), there exist unique \(v_\parallel \in V\) and \(v_\perp \in V^\perp\) such that \(v = v_\parallel + v_\perp\), and \(v_\parallel\) is the unique best approximation of \(v\) on \(V\). This motivates the definition of a map that realizes the best approximation.
\begin{definition}{Orthogonal projector}{orthogonal_projector}
    Let \(\hilbert\) be a Hilbert space. If \(V \subset \hilbert\) is a closed linear subspace of \(\hilbert\), the \emph{orthogonal projector onto \(V\)} is the map
    \begin{align*}
        P_V : \hilbert &\to \hilbert\\
                     v &\mapsto v_\parallel
    \end{align*}
    where \(v_\parallel \in V\) is the unique best approximation of \(v\) on \(V\). If \(x \in \hilbert\), we may write \(P_x\) as shorthand for the orthogonal projector \(P_{\lspan{\set{x}}} : \hilbert \to \hilbert\).
\end{definition}

\begin{proposition}{Orthogonal projector is a bounded operator}{orthogonal_projector_bounded}
    Let \(\hilbert\) be a Hilbert space. If \(V\) is a closed linear subspace of \(\hilbert\), then \(P_V \in \bounded(\hilbert)\) with \(\norm{P_V} \leq 1\).
\end{proposition}
\begin{proof}
    Let \(x, y \in \hilbert\) and \(\alpha \in \mathbb{C}\), then \cref{thm:orthogonal_decomposition} guarantees the existence and uniqueness of \(x_\parallel, y_\parallel \in V\) and \(x_\perp, y_\perp \in V^\perp\) such that \(x = x_\parallel + x_\perp\) and \(y = y_\parallel + y_\perp\), then
    \begin{equation*}
        x + \alpha y = (x_\parallel + \alpha y_\parallel) + (x_\perp + \alpha y_\perp).
    \end{equation*}
    That is, \(x_\parallel + \alpha y_\parallel\) is the best approximation of \(x + \alpha y\) in \(V\), hence \(P_V(x + \alpha y) = P_V(x) + \alpha P_V(y)\). It is clear \(P_V\) is bounded since
    \begin{equation*}
        \sup_{v \in \hilbert} \frac{\norm{P_Vv}}{\norm{v}} = \sup_{v \in \hilbert} \frac{\norm{v_\parallel}}{\sqrt{\norm{v_\parallel}^2 + \norm{v_\perp}^2}} \leq 1,
    \end{equation*}
    that is, \(\norm{P_V} \leq 1\).
\end{proof}

An important consequence of the orthogonal decomposition theorem is that every bounded linear operator defined on a subset of a Hilbert space can be extended to the entire linear space.
\begin{theorem}{Isometric extension of a bounded operator defined on a Hilbert space}{isometric_extension_bounded_hilbert}
    Let \(T : \domain{T} \subset \hilbert \to X\) be a bounded linear operator, where \(\hilbert\) is a Hilbert space and \(X\) is a Banach space. Then \(T\) can be uniquely extended to \(\hat{T} : \hilbert \to X\), where the extension satisfies \(\hat{T}((\cl_\hilbert\domain{T})^\perp) = \set{0}\) and \(\norm{\hat{T}} = \norm{T}\).
\end{theorem}
\begin{proof}
    Notice \(\domain{T}\) is dense in the Banach space \(V = \cl_\hilbert{\domain{T}}.\) Using \cref{thm:blt} if necessary, there exists a unique bounded linear extension \(\tilde{T} : V \to X\) to \(T\) such that \(\norm{\tilde{T}} = \norm{T}\).

    We define the map
    \begin{align*}
        \hat{T} : \hilbert &\to X\\
                         v &\mapsto \tilde{T} \circ P_V (v),
    \end{align*}
    which is manifestly a bounded linear operator, as a composition of bounded linear operators. Notice \(P_Vu = u\) for all \(u \in \domain{T}\subset V\), since \(u\) is its best approximation in \(V\). Then, since \(\tilde{T}\) extends \(T\), \(\hat{T}\) extends \(T\). As \(\ker{P_V} = V^\perp\), it follows that \(\hat{T}(V^\perp) = \set{0}\). Moreover, we have
    \begin{equation*}
        \norm{\hat{T}} = \sup_{v \in \hilbert}{\frac{\norm{\hat{T}v}}{\norm{v}}}= \sup_{v \in \hilbert}{\frac{\norm{\tilde{T}v_\parallel}}{\sqrt{\norm{v_\parallel}^2 + \norm{v_\perp}^2}}} \leq \sup_{v\in V}{\frac{\norm{\tilde{T}v}}{\norm{v}}} = \norm{\tilde{T}} = \norm{T}
    \end{equation*}
    and
    \begin{equation*}
        \norm{\hat{T}} = \sup_{v \in \hilbert}{\frac{\norm{\hat{T}v}}{\norm{v}}} \geq \sup_{v\in\domain{T}}{\frac{\norm{Tv}}{\norm{v}}} = \norm{T},
    \end{equation*}
    and we conclude \(\norm{\hat{T}} = \norm{T}\).

    Suppose there exists \(S \in \bounded(\hilbert, X)\) that extends \(T\) such that \(\norm{S} = \norm{T}\) and \(S(V^\perp) = \set{0}\). Then, \(\restrict{S}{V}\) is a bounded extension to \(T\) with \(\norm{\restrict{S}{V}} = \norm{T}\), hence the uniqueness of \(\tilde{T}\) yields \(\restrict{S}{V} = \tilde{T}\). For every \(v \in V\) we have
    \begin{equation*}
        S(v) = S\circ P_V(v) + S(v - P_V(v)) = \restrict{S}{V} \circ P_V(v) = \tilde{T}\circ P_V (v) = \hat{T}(v),
    \end{equation*}
    then \(S = \hat{T}\), showing uniqueness.
\end{proof}
Due to \cref{thm:isometric_extension_bounded_hilbert}, we can always assume a \emph{bounded} linear operator is defined on an entire Hilbert space, rather than just a linear subspace.

\subsection{Adjoint of an operator}
We consider densely defined linear operators, not necessarily bounded.
\begin{proposition}{Domain of the adjoint map is a linear subspace}{domain_adjoint}
    Let \(\hilbert_1, \hilbert_2\) be Hilbert spaces. If \(T : \domain{T} \subset \hilbert_1 \to \hilbert_2\) is a densely defined linear operator, then
    \begin{equation*}
        D = \setc{u \in \hilbert_2}{\exists \eta \in \hilbert_1 : \inner{u}{Tv}_{\hilbert_2} = \inner{\eta}{v}_{\hilbert_1}, \forall v \in \domain{T}}
    \end{equation*}
    is a linear subspace. Moreover, if \(u \in D\), there exists a unique \(\eta \in \hilbert_1\) such that
    \begin{equation*}
        \inner{u}{Tv}_{\hilbert_2} = \inner{\eta}{v}_{\hilbert_1}
    \end{equation*}
    for all \(v \in \domain{T}\).
\end{proposition}
\begin{proof}
    Notice \(0 \in D\), since for all \(v \in \domain{T}\) we may choose \(\eta = 0\) such that \(\inner{0}{Tv}_{\hilbert_2} = \inner{0}{v}_{\hilbert_1}\). Let \(x, y \in D\), then there exist \(\eta_x, \eta_y \in \hilbert_1\) such that \(\inner{x}{Tv}_{\hilbert_2} = \inner{\eta_x}{v}_{\hilbert_1}\) for all \(v \in \domain{T}\) and analogously for \(y\). Let \(\alpha \in \mathbb{C}\), then
    \begin{equation*}
        \inner{\eta_x + \alpha \eta_y}{v}_{\hilbert_1} = \inner{\eta_x}{v}_{\hilbert_1} + \conj{\alpha} \inner{\eta_y}{v}_{\hilbert_2} = \inner{x}{Tv}_{\hilbert_2} + \conj{\alpha}\inner{y}{Tv}_{\hilbert_2} = \inner{x + \alpha y}{Tv}_{\hilbert_2}
    \end{equation*}
    for all \(v \in \domain{T}\). That is, \(x + \alpha y \in D\) and we conclude \(D\) is a linear subspace.

    Let \(u \in D\), then there exists \(\eta \in \hilbert_1\) such that \(\inner{u}{Tv}_{\hilbert_2} = \inner{\eta}{v}_{\hilbert_1}\) for all \(v \in \domain{T}\). Suppose there exists \(\tilde{\eta} \in \hilbert_1\) satisfying the same property, then
    \begin{equation*}
        \inner{\tilde\eta}{v}_{\hilbert_1} = \inner{u}{Tv}_{\hilbert_2} = \inner{\eta}{v}_{\hilbert_1} \implies \inner{\tilde{\eta} - \eta}{v}_{\hilbert_1} = 0
    \end{equation*}
    for all \(v \in \domain{T}\). Hence, \(\tilde{\eta} - \eta \in \domain{T}^\perp\). Taking the orthogonal complement yields \(\cl_\hilbert\domain{T} \subset \set{\tilde{\eta} - \eta}^\perp\), and we conclude \(\eta = \tilde{\eta}\) since \(T\) is densely defined.
\end{proof}
This result defines the \emph{adjoint} map of a densely defined linear operator.
\begin{definition}{Adjoint of an operator}{adjoint_densely_defined}
    Let \(\hilbert_1, \hilbert_2\) be Hilbert spaces. The \emph{adjoint map of a densely defined operator \(T : \domain{T} \subset \hilbert_1 \to \hilbert_2\)} is the map
    \begin{align*}
        T^* : \domain{T^*} \subset \hilbert_2 &\to \hilbert_1\\
                                            u &\to \eta_u,
    \end{align*}
    where \(\domain{T^*}\) is the linear subspace of \(\hilbert_2\) given by
    \begin{equation*}
        \domain{T^*} = \setc{u \in \hilbert_2}{\exists \eta_u \in \hilbert_1 : \inner{u}{Tv}_{\hilbert_2} = \inner{\eta_u}{v}_{\hilbert_1} \forall v \in \domain{T}}
    \end{equation*}
    and \(\eta_u\) is the unique element in \(\hilbert_1\) such that \(\inner{u}{Tv}_{\hilbert_2} = \inner{\eta_u}{v}_{\hilbert_1}\) for all \(v \in \domain{T}\).
\end{definition}
\begin{remark}
    In \cref{prop:domain_adjoint} while showing the domain of the adjoint map \(T^*\) is a linear subspace of \(\hilbert_2\), we have shown that \(T^*\) is a linear map. Henceforth we will refer to the adjoint map as the adjoint operator.
\end{remark}
\begin{remark}
    From the previous definition we see that
    \begin{equation*}
        \inner{T^*u}{v}_{\hilbert_1} = \inner{u}{Tv}_{\hilbert_2}
    \end{equation*}
    for all \(v \in \domain{T}\) and all \(u \in \domain{T^*}\).
\end{remark}

In Quantum Mechanics, the concept of the adjoint operator is of the utmost importance as the \emph{self-adjoint} operators represent observables. As we will later see, the Spectral Theorem holds for self-adjoint operators, which is intimately related to the probabilistic interpretation of Quantum Mechanics.
\begin{definition}{Self-adjoint operator}{self_adjoint}
    Let \(\hilbert\) be a Hilbert space. A \emph{self-adjoint operator \(T : \domain{T} \subset \hilbert \to \hilbert\)} is a map such that \(T = T^*\).
\end{definition}
\begin{remark}
    It is important to stress that \(T = T^*\) means \(\domain{T} = \domain{T^*}\) and \(Tu = Tu\) for all \(u \in \domain{T}\).
\end{remark}
\begin{remark}
    For self-adjoint operators we have
    \begin{equation*}
        \inner{Tu}{v} = \inner{u}{Tv}
    \end{equation*}
    for all \(u, v\in \domain{T}\).
\end{remark}

We now give a property of the operator norm of a self-adjoint bounded operator.
\begin{theorem}{Norm of a self-adjoint bounded operator}{norm_self_adjoint}
    Let \(T \in \bounded(\hilbert)\) be a self-adjoint bounded operator defined on a Hilbert space \(\hilbert\). Then its norm satisfies
    \begin{equation*}
        \norm{T} = \sup_{x \in \hilbert}\frac{\abs*{\inner{x}{Tx}}}{\norm{x}^2}.
    \end{equation*}
\end{theorem}
\begin{proof}
    Let
    \begin{equation*}
        M = \sup_{x \in \hilbert}{\frac{\abs*{\inner{x}{Tx}}}{\norm{x}^2}},
    \end{equation*}
    which is finite by the \nameref{thm:cauchy_schwarz}, which yields \(M \leq \norm{T}\). For all \(u, v \in \hilbert\), we have by symmetry that
    \begin{equation*}
        \inner{u}{Tv} = \inner{Tu}{v} = \conj{\inner{v}{Tu}}.
    \end{equation*}
    As a result, for all \(x, y \in \hilbert\) and \(\lambda \in \mathbb{C}\),
    \begin{equation*}
        \inner{x + \lambda y}{T(x + \lambda y)} = \inner{x}{Tx} + 2\Re{\left(\inner{x}{T(\lambda y)}\right)}+ \abs{\lambda}^2 \inner{y}{Ty}
    \end{equation*}
    and
    \begin{equation*}
        \inner{x - \lambda y}{T(x - \lambda y)} = \inner{x}{Tx} - 2\Re{\left(\inner{x}{T(\lambda y)}\right)} + \abs{\lambda}^2 \inner{y}{Ty},
    \end{equation*}
    which yields
    \begin{equation*}
        4 \Re{\left(\inner{x}{T(\lambda y)}\right)} = \inner{x + \lambda y}{T(x + \lambda y)} - \inner{x - \lambda y}{T(x - \lambda y)}.
    \end{equation*}
    Using the estimation provided by \(M\), we have
    \begin{align*}
        4\abs*{\Re{\left(\lambda \inner{x}{Ty}\right)}} &= \abs*{\inner{x + \lambda y}{T(x + \lambda y)} - \inner{x - \lambda y}{T(x - \lambda y)}}\\
                                                        &\leq \abs*{\inner{x + \lambda y}{T(x + \lambda y)}} + \abs*{\inner{x - \lambda y}{T(x - \lambda y)}}\\
                                                        &\leq M\left(\norm{x + \lambda y}^2 + \norm{x - \lambda y}^2\right)\\
                                                        &\leq 2M\left(\norm{x}^2 + \abs{\lambda}^2 \norm{y}^2\right),
    \end{align*}
    where we have used the parallelogram identity. Set \(\lambda\) such that \(\lambda \inner{x}{Ty} = \abs*{\inner{x}{Ty}}\) and \(\abs*{\lambda} = 1\), then
    \begin{equation*}
        \abs*{\inner{x}{Ty}} \leq \frac12 M \left(\norm{x}^2 + \norm{y}^2\right),
    \end{equation*}
    for all \(x, y \in \hilbert\). If \(Tx = 0\), it is clear that \(\norm{Tx} \leq M \norm{x}\) holds trivially, so we may suppose \(Tx \neq 0\). Then, we set \(y = \frac{\norm{x}}{\norm{Tx}}Tx\), which yields
    \begin{equation*}
        M \norm{x}^2 \geq \abs*{\inner*{x}{\frac{\norm{x}}{\norm{Tx}}T(Tx)}} = \frac{\norm{x}}{\norm{Tx}} \abs*{\inner{Tx}{Tx}} = \norm{x}\cdot \norm{Tx}.
    \end{equation*}
    We have thus shown
    \begin{equation*}
        M \geq \frac{\norm{Tx}}{\norm{x}}
    \end{equation*}
    for all \(x \in \hilbert\), hence, \(M \geq \norm{T}\).
\end{proof}

\subsection{Adjoint of a bounded operator}
We consider bounded operators \(A : \hilbert_1 \to \hilbert_2\), where \(\hilbert_1\) and \(\hilbert_2\) are Hilbert spaces. As a consequence of \nameref{thm:riesz_representation}, we'll show any bounded operator \(A \in \bounded(\hilbert_1, \hilbert_2)\) defines a unique operator \(B \in \bounded(\hilbert_2, \hilbert_1)\) such that \(\inner{y}{Ax}_{\hilbert_2} = \inner{By}{x}_{\hilbert_1}\), for all \(x \in \hilbert_1\) and \(y \in \hilbert_2\). We begin proving a lemma showing a bicontinuous sesquilinear form defines a bounded operator.
\begin{lemma}{Bicontinuous sesquilinear form defines a unique operator}{sesquilinear_adjoint}
    Let \(\mathscr{A} : \hilbert_2 \times \hilbert_1 \to \mathbb{C}\) be a sesquilinear form defined on Hilbert spaces \(\hilbert_1\) and \(\hilbert_2\). Suppose \(\mathscr{A}\) is \emph{bicontinuous}, that is, there exists \(M > 0\) such that \(\abs{\mathscr{A}(u,v)} \leq M\norm{u}_{\hilbert_2} \norm{v}_{\hilbert_1}\) for all \(u \in \hilbert_2\) and all \(v \in \hilbert_1\). Then, there exists a unique linear operator \(B \in \bounded(\hilbert_2, \hilbert_1)\) such that
    \begin{equation*}
        \mathscr{A}(u,v) = \inner{Bu}{v}_{\hilbert_1},
    \end{equation*}
    for all \(u \in \hilbert_2\) and all \(v \in \hilbert_1\).
\end{lemma}
\begin{proof}
    As \(\mathscr{A}\) is bicontinuous, it follows that for each \(u \in \hilbert_2\), the map
    \begin{align*}
        \ell_u : \hilbert_1 &\to \mathbb{C}\\
                          v &\mapsto \mathscr{A}(u,v)
    \end{align*}
    is a bounded linear functional on \(\hilbert_1\). That is, there exists a map \(\ell : \hilbert_2 \to \hilbert_1^\dag\) defined by \(\ell(u) = \ell_u\). Let \(u_1, u_2 \in \hilbert_2\) and \(\alpha \in \mathbb{C}\), then
    \begin{align*}
        \ell(u_1 + \alpha u_2)(v) = \ell_{u_1 + \alpha u_2}(v) &= \mathscr{A}(u_1 + \alpha u_2, v)\\
                                                               &= \mathscr{A}(u_1, v) + \conj{\alpha} \mathscr{A}(u_2, v)\\
                                                               &= \ell_{u_1}(v) + \conj{\alpha} \ell_{u_2}(v)\\
                                                               &= \left[\ell(u_1) + \conj{\alpha} \ell(u_2)\right](v)
    \end{align*}
    for all \(v \in \hilbert_1\). We infer \(\ell(u_1 + \alpha u_2) = \ell(u_1) + \conj{\alpha}\ell(u_2)\), thus showing \(\ell\) is an antilinear map.

    We claim the map \(B = \riesz \circ \ell\) is a bounded linear operator such that \(\mathscr{A}(u,v) = \inner{B(u)}{v}_{\hilbert_1}\), where \(\riesz : \hilbert_1^\dag \to \hilbert_1\) is the Riesz representation map. \nameref{thm:riesz_representation} ensures that \(B(u)\) is the unique vector in \(\hilbert_1\) such that \(\ell_u(v) = \inner{B(u)}{v}_{\hilbert_1}\) for all \(v \in \hilbert_1\). As a result, for all \(u \in \hilbert_2\) and \(v \in \hilbert_1\), we have \(\inner{B(u)}{v}_{\hilbert_1} = \mathscr{A}(u,v)\). It is clear \(B\) is a linear map since it is the composition of two antilinear maps. Moreover, since \(\mathscr{A}\) is bicontinuous, we have
    \begin{equation*}
        \norm{Bu}_{\hilbert_1}^2 = \abs*{\inner{Bu}{Bu}_{\hilbert_1}} = \abs*{\mathscr{A}(u, Bu)} \leq M \norm{u}_{\hilbert_2} \norm{Bu}_{\hilbert_1},
    \end{equation*}
    that is, \(M\) is an upper bound for the set \(\setc*{\frac{\norm{Bu}_{\hilbert_1}}{\norm{u}_{\hilbert_2}}}{u \in \hilbert_2 \setminus \set{0}}\), hence \(B \in \bounded(\hilbert_2, \hilbert_1)\), proving our claim.

    Let \(\tilde{B} \in \bounded(\hilbert_2, \hilbert_1)\) such that \(\mathscr{A}(u,v) = \inner{\tilde{B}u}{v}_{\hilbert_1}\) for all \(u \in \hilbert_2\) and \(v \in \hilbert_1\). Then, \(\inner{Bu}{v}_{\hilbert_1} = \inner{\tilde{B}u}{v}_{\hilbert_1}\), which implies \(\inner{(B - \tilde{B})u}{v}_{\hilbert_1} = 0\) for all \(u \in \hilbert_2\) and all \(v \in \hilbert_1\). Since the inner product is non-degenerate this yields \((B - \tilde{B})u = 0\) for all \(u \in \hilbert_2\), hence \(B = \tilde{B}\), which shows uniqueness.
\end{proof}

We now use a bounded operator to define a bicontinuous sesquilinear form, hence uniquely defining another bounded operator. Then, we show this operator is precisely the adjoint operator.
\begin{theorem}{Existence of the adjoint of a bounded operator}{adjoint_bounded_hilbert}
    Let \(\hilbert_1, \hilbert_2\) be Hilbert spaces. If \(A \in \bounded(\hilbert_1, \hilbert_2)\) is a bounded linear operator, then there exists a unique bounded linear operator \(B \in \bounded(\hilbert_2, \bounded_1)\) such that
    \begin{equation*}
        \inner{y}{Ax}_{\hilbert_2} = \inner{By}{x}_{\hilbert_1}
\end{equation*}
    for all \(x \in \hilbert_1\) and \(y \in \hilbert_2\). Furthermore, \(B = A^*\).
\end{theorem}
\begin{proof}
    We claim the map
    \begin{align*}
        \mathscr{A} : \hilbert_2 \times \hilbert_1 &\to \mathbb{C}\\
                                             (y,x) &\mapsto \inner{y}{Ax}_{\hilbert_2}
    \end{align*}
    is a bicontinuous sesquilinear form. Sesquilinearity of \(\mathscr{A}\) follows from linearity of the operator \(A\) and from sesquilinearity of the inner product defined on \(\hilbert_2\). \nameref{thm:cauchy_schwarz} yields
    \begin{equation*}
        \abs{\mathscr{A}(y,x)} = \abs*{\inner{y}{Ax}_{\hilbert_2}} \leq \norm{y}_{\hilbert_2} \norm{Ax}_{\hilbert_2} \leq \norm{A}_{\bounded(\hilbert_1, \hilbert_2)} \norm{x}_{\hilbert_1} \norm{y}_{\hilbert_2},
    \end{equation*}
    for all \(x \in \hilbert_1\) and \(y \in \hilbert_2\), hence \(\mathscr{A}\) is bicontinuous. By \cref{lem:sesquilinear_adjoint}, there exists a unique bounded linear operator \(B \in \bounded(\hilbert_2, \hilbert_1)\) such that \(\mathscr{A}(y,x) = \inner{By}{x}_{\hilbert_1}\), that is,
    \begin{equation*}
        \inner{y}{Ax}_{\hilbert_2} = \inner{By}{x}_{\hilbert_1},
    \end{equation*}
    for all \(x \in \hilbert_1\) and \(y \in \hilbert_2\).

    The existence of \(B\) implies that \(\domain{T^*} = \hilbert_2\). Indeed, by definition we have
    \begin{equation*}
        \domain{A^*} = \setc{y \in \hilbert_2}{\exists \eta_y \in \hilbert_1 : \inner{y}{Ax}_{\hilbert_2} = \inner{\eta}{x}_{\hilbert_1} \forall x \in \hilbert_1},
    \end{equation*}
    but for all \(y \in \hilbert_2\), we have \(By \in \hilbert_1\) such that \(\inner{y}{Ax}_{\hilbert_2} = \inner{By}{x}_{\hilbert_1}\) for all \(x \in \hilbert_1\), hence \(y \in \domain{T^*}\). Moreover, the uniqueness of each \(\eta_y\) yields \(\eta_y = A^*y = By\) for all \(y \in \hilbert_2\), hence \(B = A^*\).
\end{proof}

Notice \cref{thm:adjoint_bounded_hilbert} establishes a map from \(\bounded(\hilbert_1, \hilbert_2)\) to \(\bounded(\hilbert_2, \hilbert_1)\).
\begin{definition}{Adjoint operation on bounded operators}{adjoint_bounded_hilbert}
    Let \(\hilbert_1, \hilbert_2\) be Hilbert spaces. The map
    \begin{align*}
        ^* : \bounded(\hilbert_1, \hilbert_2) &\to \bounded(\hilbert_2, \hilbert_1)\\
                                           A &\mapsto A^*,
    \end{align*}
    where \(A^*\) is the adjoint operator of \(A\), is called an \emph{adjoint operation}, later \emph{involution}, on \(\bounded(\hilbert_1, \hilbert_2)\).
\end{definition}
The Banach space adjoint and the Hilbert space adjoint unfortunately share the same name, despite not being the same operation. They are, however, related by the Riesz representation map.

\begin{proposition}{Banach space adjoint and Hilbert space adjoint}{adjoint_relation}
    Let us denote the Banach space adjoint operation by \(' : \bounded(\hilbert_1, \hilbert_2) \to \bounded(\hilbert_2^\dag, \hilbert_1^\dag)\) and the Hilbert space adjoint operation by \(^* : \bounded(\hilbert_1, \hilbert_2) \to \bounded(\hilbert_2, \hilbert_1)\), where \(\hilbert_1, \hilbert_2\) are Hilbert spaces. If \(T \in \bounded(\hilbert_1, \hilbert_2)\) is a bounded operator, then \(T^* = \riesz_{\hilbert_1} \circ T' \circ \riesz^{-1}_{\hilbert_2}\).
\end{proposition}
Let \(T \in \bounded(\hilbert_1, \hilbert_2)\), with \(T' \in \bounded(\hilbert_2^\dag, \hilbert_1^\dag)\) being the Banach space adjoint and \(T^* \in \bounded(\hilbert_2, \hilbert_1)\) the Hilbert space adjoint, then for all \(x \in \hilbert_1\) and \(y \in \hilbert_2\) we have
    \begin{align*}
        \inner{\riesz_{\hilbert_1} \circ T' \circ \riesz_{\hilbert_2}^{-1}(y)}{x}_{\hilbert_1}
        &= \inner{\riesz_{\hilbert_1}(\riesz_{\hilbert_2}^{-1}(y) \circ T)}{x}_{\hilbert_1}\\
        &= \riesz_{\hilbert_2}^{-1}(y) \circ T(x)\\
        &= \inner{y}{Tx}_{\hilbert_2},
    \end{align*}
    hence \(T^* = \riesz_{\hilbert_1} \circ T' \circ \riesz^{-1}_{\hilbert_2}\) by uniqueness of the Hilbert space adjoint.

\begin{theorem}{Adjoint operation properties}{involution_properties}
    We'll denote Hilbert spaces by \(\hilbert_n\) and the adjoint operations on \(\bounded(\hilbert_n, \hilbert_m)\) simply as \(^*\). Then, the adjoint operation
    \begin{enumerate}[label=(\alph*)]
        \item is an involution: \((A^*)^* = A\), for all \(A \in \bounded(\hilbert_1, \hilbert_2)\);
        \item is an isometry: \(\norm{A^*} = \norm{A},\) for all \(A \in \bounded(\hilbert_1, \hilbert_2)\);
        \item satisfies the \(C^*\) property: \(\norm{A^* \circ A} = \norm{A}^2\), for all \(A \in \bounded(\hilbert_1, \hilbert_2)\);
        \item is antilinear: \((\alpha A + \beta B)^* = \conj{\alpha} A^* + \conj{\beta} B^*\) for all \(\alpha, \beta \in \mathbb{C}\) and \(A, B \in \bounded(\hilbert_1, \hilbert_2)\);
        \item The adjoint operation is antidistributive: \((B \circ A)^* = A^*\circ B^*\) for all \(A \in \bounded(\hilbert_1, \hilbert_2)\) and \(B \in \bounded(\hilbert_2, \hilbert_3)\);
        \item preserves the identity: if \(\unity \in \bounded(\hilbert_1)\) is the identity operator, then \(\unity^* = \unity\);
        \item maps the inverse to the inverse of the adjoint: if \(A \in \bounded(\hilbert_1, \hilbert_2)\) admits an inverse map \(A^{-1} \in \bounded(\hilbert_2, \hilbert_1)\), then \((A^{-1})^* = (A^*)^{-1}\).
    \end{enumerate}
\end{theorem}
\begin{proof}[Proof of (a)]
    Let \(A \in \bounded(\hilbert_1, \hilbert_2)\), then
    \begin{equation*}
        \inner{(A^*)^*x}{y}_{\hilbert_2} = \inner{x}{A^*y}_{\hilbert_1} = \conj{\inner{A^*y}{x}_{\hilbert_1}} = \conj{\inner{y}{Ax}_{\hilbert_2}} = \inner{Ax}{y}_{\hilbert_2}
    \end{equation*}
    for all \(x \in \hilbert_1\) and \(y \in \hilbert_2.\) By non-degeneracy of the inner product, we have \(A = (A^*)^*\), showing involutivity.
\end{proof}
\begin{proof}[Proof of (b)]
    We consider the Banach space adjoint operation
    \begin{align*}
        ' : \bounded(\hilbert_1, \hilbert_2) &\to \bounded(\hilbert_2^\dag, \hilbert_1^\dag)\\
                                           A &\mapsto A'
    \end{align*}
    which we have shown in \cref{prop:adjoint_Banach} to be an isometry. Since the maps \(\riesz_{\hilbert_1}\) and \(\riesz_{\hilbert_2}^{-1}\) are bijective isometries, it follows that
    \begin{align*}
        \norm{A^*}_{\bounded(\hilbert_2, \hilbert_1)} = \sup_{x \in \hilbert_2}{\frac{\norm{A^*x}_{\hilbert_1}}{\norm{x}_{\hilbert_2}}}
        &= \sup_{x \in \hilbert_2}{\frac{\norm{\riesz_{\hilbert_1}\circ A' \circ \riesz^{-1}_{\hilbert_2}(x)}_{\hilbert_1}}{\norm{x}_{\hilbert_2}}}\\
        &= \sup_{x \in \hilbert_2}{\frac{\norm{A' \circ \riesz^{-1}_{\hilbert_2}(x)}_{\hilbert_1^\dag}}{\norm{\riesz_{\hilbert_2}^{-1}(x)}_{\hilbert_2^\dag}}}\\
        &= \sup_{\ell \in \hilbert_2^\dag}{\frac{\norm{A'(\ell)}_{\hilbert_1^\dag}}{\norm{\ell}_{\hilbert_2^\dag}}}\\
        &= \norm{A'}_{\bounded(\hilbert_2^\dag, \hilbert_1^\dag)} = \norm{A}_{\bounded(\hilbert_1,\hilbert_2)}
    \end{align*}
    for all \(A \in \bounded(\hilbert_1, \hilbert_2)\).
\end{proof}
\begin{proof}[Proof of (c)]
    Let \(A \in \bounded(\hilbert_1, \hilbert_2)\), then \(A^* \circ A \in \bounded(\hilbert_1)\) is a bounded operator as a composition of continuous maps. It should be clear that \(\norm{A^*\circ A}_{\bounded(\hilbert_1)} \leq \norm{A}_{\bounded(\hilbert_1, \hilbert_2)}^2\). Indeed,
    \begin{equation*}
        \norm{A^* \circ A}_{\bounded(\hilbert_1)} = \opnorm{A^*\circ A}{x}{\hilbert_1}{\hilbert_1} \leq \norm{A^*}_{\bounded(\hilbert_2, \hilbert_1)}\opnorm{A}{x}{\hilbert_1}{\hilbert_2} = \norm{A}_{\bounded(\hilbert_1, \hilbert_2)}^2.
    \end{equation*}
    For all \(x \in \hilbert_1\), \(\inner{A^*\circ Ax}{x}_{\hilbert_1}\) is a non-negative real number since
    \begin{equation*}
        \inner{A^*\circ Ax}{x}_{\hilbert_1} = \inner{Ax}{Ax}_{\hilbert_2} = \norm{Ax}_{\hilbert_2}^2 \geq 0.
    \end{equation*}
    The \nameref{thm:cauchy_schwarz} yields
    \begin{equation*}
        \norm{Ax}^2_{\hilbert_2} = \inner{A^*\circ A x}{x}_{\hilbert_1} \leq \norm{A^* \circ Ax}_{\hilbert_1} \norm{x}_{\hilbert_1} \leq \norm{A^* \circ A}_{\bounded(\hilbert_1)} \norm{x}_{\hilbert_1}^2
    \end{equation*}
    for all \(x \in \hilbert_1\). That is, \(\norm{A^* \circ A}_{\bounded(\hilbert_1)} \geq \norm{A}_{\bounded(\hilbert_1, \hilbert_2)}^2\).
\end{proof}
\begin{proof}[Proof of (d)]
    Let \(\alpha, \beta \in \mathbb{C}\) and \(A, B \in \bounded(\hilbert_1, \hilbert_2)\), then for all \(x \in \hilbert_1\) and \(y \in \hilbert_2\) we have
    \begin{align*}
        \inner{(\alpha A + \beta B)^*y}{x}_{\hilbert_1} &= \inner{y}{(\alpha A + \beta B)x}_{\hilbert_2}\\
                                                        &= \alpha\inner{y}{Ax}_{\hilbert_2} + \beta\inner{y}{Bx}_{\hilbert_2}\\
                                                        &= \alpha \inner{A^*y}{x}_{\hilbert_1} + \beta \inner{B^*y}{x}_{\hilbert_1}\\
                                                        &= \inner{\conj{\alpha}A^*y + \conj{\beta}B^*y}{x}_{\hilbert_1}\\
                                                        &= \inner{(\conj{\alpha}A^* + \conj{\beta}B^*)y}{x}_{\hilbert_1},
    \end{align*}
    which yields
    \begin{equation*}
        \inner*{\left[\left(\alpha A + \beta B\right)^* - \left(\conj{\alpha}A^* + \conj{\beta}B^*\right)\right]y}{x}_{\hilbert_1} = 0.
    \end{equation*}
    As the inner product is non-degenerate, we have \(\left[\left(\alpha A + \beta B\right)^* - \left(\conj{\alpha}A^* + \conj{\beta}B^*\right)\right]y = 0\) for all \(y \in \hilbert_2\), hence \(\left(\alpha A + \beta B\right)^* = \conj{\alpha}A^* + \conj{\beta}B^*\).
\end{proof}
\begin{proof}[Proof of (e)]
    Let \(A \in \bounded(\hilbert_1, \hilbert_2)\) and let \(B \in \bounded(\hilbert_2, \hilbert_3)\), then for all \(x \in \hilbert_1\) and \(y \in \hilbert_3\), we have
    \begin{align*}
        \inner{(B\circ A)^*y}{x}_{\hilbert_1} &= \inner{y}{B \circ Ax}_{\hilbert_3}\\
                                              &= \inner{B^*y}{Ax}_{\hilbert_2}\\
                                              &= \inner{A^* \circ B^*y}{x}_{\hilbert_1}.
    \end{align*}
    The non-degeneracy of the inner product yields \((B \circ A)^* = A^* \circ B^*\).
\end{proof}
\begin{proof}[Proof of (f)]
    Let \(\unity \in \bounded(\hilbert_1)\) be the identity map, then for all \(x, y \in \hilbert_1\),
    \begin{equation*}
        \inner{\unity^*y}{x}_{\hilbert_1} = \inner{y}{\unity x}_{\hilbert_1} = \inner{y}{x},
    \end{equation*}
    hence \(\unity^*y = y\), thus showing \(\unity^* = \unity\).
\end{proof}
\begin{proof}[Proof of (g)]
    Let \(A \in \bounded(\hilbert_1, \hilbert_2)\) be a bijective bounded operator and \(A^{-1} \in \bounded(\hilbert_2, \hilbert_1)\) its inverse map. By (e) and (f), we have
    \begin{equation*}
        A^* \circ (A^{-1})^* = (A^{-1} \circ A)^* = \unity_{\hilbert_1}^* = \unity_{\hilbert_1}
    \end{equation*}
    and
    \begin{equation*}
        (A^{-1})^* \circ A^* = (A \circ A^{-1})^* = \unity_{\hilbert_2}^* = \unity_{\hilbert_2}
    \end{equation*}
    and we conclude \((A^{-1})^* = (A^*)^{-1}\).
\end{proof}

Due to the definition of the adjoint operator \(A^*\) to a bounded operator \(A\), the kernel of \(A\) is equal to the orthogonal complement of the image of \(A^*\).
\begin{proposition}{Kernel and range of the adjoint map}{kernel_range_adjoint}
    Let \(\hilbert\) be a Hilbert space. If \(A \in \bounded(\hilbert)\) is a bounded operator, then \(\ker A = \range{A^{\ast}}^\perp\).
\end{proposition}
\begin{proof}
    Let \(x \in \ker A\). For all \(y \in \hilbert\) we have
    \begin{equation*}
        \inner{A ^{\ast}y}{x} = \inner{y}{Ax} = 0,
    \end{equation*}
    hence \(x \in \range{A ^{\ast}}^\perp\). That is, \(\ker A \subset \range{A ^{\ast}}^\perp\).

    Let \(x \in \range{A ^{\ast}}^\perp\). Then, for all \(y \in \hilbert\), we have
    \begin{equation*}
        \inner{y}{Ax} = \inner{A ^{\ast}y}{x} = 0,
    \end{equation*}
    hence \(Ax = 0\) by non-degeneracy of the inner product. Thus showing that \(\range{A ^{\ast}}^\perp \subset \ker A\).
\end{proof}

\subsection{Symmetric operators}
Closely related to self-adjointness of a bounded operator is the notion of a symmetric operator.
\begin{definition}{Symmetric operator}{symmetric_operator}
    Let \(\hilbert\) be a Hilbert space. A densely defined operator \(T : \domain{T} \subset \hilbert \to \hilbert\) is \emph{symmetric} if for all \(x, y \in \domain{T}\) it satisfies \(\inner{x}{Ay} = \inner{Ax}{y}\).
\end{definition}
It is clear that if a densely defined operator \(T : \domain{T} \subset \hilbert \to \hilbert\) is self-adjoint, then it is a symmetric operator. The following theorem shows a globally defined symmetric operator is bounded and self-adjoint.
\begin{theorem}{Hellinger-Toeplitz theorem}{Hellinger_Toeplitz}
    Let \(\hilbert\) be a Hilbert space. Suppose the operator \(T : \domain{T} \subset \hilbert \to \hilbert\) is globally defined, that is, \(\domain{T} = \hilbert\). If \(T\) is symmetric, then it is bounded and self-adjoint.
\end{theorem}
\begin{proof}
    We consider a sequence \(\family{v_n}{n \in \mathbb{N}} \subset \graph{T} \subset \hilbert \oplus \hilbert\) that converges to \(\tilde{v} = (\tilde{x}, \tilde{y}) \in \hilbert\oplus\hilbert\), that is, there exist sequences \(\family{x_n}{n \in \mathbb{N}} \subset \hilbert\) and \(\family{Tx_n}{n \in \mathbb{N}} \subset \hilbert\) such that \(x_n \to \tilde{x}\) and \(Tx_n \to \tilde{y}\). Let \(z \in \hilbert\), then
    \begin{equation*}
        \inner{z}{y} = \inner*{z}{\lim_{n \to \infty}{Tx_n}} = \lim_{n\to \infty}{\inner{z}{Tx_n}}
    \end{equation*}
    by the continuity of the inner product and the convergence of the sequence. Symmetry yields
    \begin{align*}
        \inner{z}{y} &= \lim_{n\to\infty}{\inner{Tz}{x_n}} \\&= \inner*{Tz}{\lim_{n\to\infty}{x_n}} \\&= \inner{Tz}{\tilde{x}}\\
                     &= \inner{z}{T\tilde{x}}
    \end{align*}
    for all \(z \in \hilbert\). Hence \(y = T\tilde{x}\) follows from the non-degeneracy of the inner product, and we conclude \(\tilde{y} \in \range{T}\). We have thus shown \(v_n \to \tilde{v} \in \graph{T}\), that is, \(T\) is bounded by \cref{thm:closed_graph}.

    Since \(T\) is bounded, there exists a unique bounded operator \(T^* \in \bounded(\hilbert)\) such that
    \begin{equation*}
        \inner{x}{Ty} = \inner{T^*x}{y}
    \end{equation*}
    for all \(x, y \in \hilbert\). Symmetry then yields
    \begin{equation*}
        \inner{(T - T^*)x}{y} = 0
    \end{equation*}
    for all \(x, y \in \hilbert.\) By non-degeneracy, \((T - T^*)x = 0\) for all \(x \in \hilbert\), hence \(T = T^*\), that is, \(T\) is self-adjoint.
\end{proof}
An important insight from this result is that if an operator is unbounded, then it cannot be both globally defined and symmetric.

\subsection{Normal operators}
It is clear a bounded operator \(A \in \bounded(\hilbert)\) can be written as a sum of self-adjoint bounded operators by defining the real and imaginary parts of \(A\),
\begin{equation*}
    \Re{(A)} = \frac{1}{2}\left(A + A^*\right)
    \quad\text{and}\quad
    \Im{(A)} = \frac{1}{2i}\left(A - A^*\right),
\end{equation*}
then \(A = \Re{(A)} + i\Im{(A)}\).
\begin{proposition}{Necessary and sufficient condition for a normal operator}{normal_real_imaginary}
    Let \(\hilbert\) be a Hilbert space. A bounded operator commutes with its adjoint if and only if its real part commutes with its imaginary part.
\end{proposition}
\begin{proof}
    Let \(A \in \bounded(\hilbert)\), then
    \begin{equation*}
        4i\Re(A)\circ \Im(A) = (A + A^*) \circ (A - A^*) = A\circ A - A\circ A^* + A^* \circ A - A^* \circ A^*
    \end{equation*}
    and
    \begin{equation*}
        4i\Im(A)\circ \Re(A) = (A - A^*) \circ (A + A^*) = A\circ A + A\circ A^* - A^* \circ A + A^* \circ A^*.
    \end{equation*}
    The commutator of the real and imaginary parts of \(A\) is
    \begin{align*}
        \left[\Re(A), \Im(A)\right] &= \Re(A)\circ \Im(A) - \Im(A)\circ \Re(A)\\
                                    &= - \frac1{2i} A\circ A^* + \frac{1}{2i} A^* \circ A\\
                                    &= \frac{i}{2} [A, A^*],
    \end{align*}
    which proves our claim.
\end{proof}
A bounded operator that commutes with its adjoint is called a normal bounded operator.
\begin{definition}{Normal bounded operator}{normal}
    Let \(\hilbert\) be a Hilbert space. A \todo[\emph{normal bounded operator}] \(T : \hilbert \to \hilbert\) is an operator satisfying \(T \circ T^* = T^* \circ T\).
\end{definition}
\begin{lemma}{Application of the polarization identity}{normal}
    Let \(\hilbert\) be a Hilbert space. The bounded operators \(A, B \in \bounded(\hilbert)\) satisfy \(A^* \circ A = B^*\circ B\) if and only if \(\norm{Ax} = \norm{Bx}\) for all \(x \in \hilbert\).
\end{lemma}
\begin{proof}
    Suppose \(A^* \circ A = B^* \circ B\), then for all \(x \in \hilbert\), we have
    \begin{equation*}
        \norm{Ax}^2 = \inner{Ax}{Ax} = \inner{A^*\circ Ax}{x} = \inner{B^*\circ Bx}{x} = \inner{Bx}{Bx} = \norm{Bx}^2,
    \end{equation*}
    hence \(\norm{Ax} = \norm{Bx}\).

    Suppose that for all \(x \in \hilbert\) we have \(\norm{Ax} = \norm{Bx}\). Then,
    \begin{equation*}
        \inner{x}{A^* \circ Ax} = \inner{(A^*)^*x}{Ax} = \inner{Ax}{Ax} = \inner{Bx}{Bx} = \inner{(B^*)^*x}{Bx} = \inner{x}{B^* \circ Bx}
    \end{equation*}
    for all \(x \in \hilbert\). Let \(u, v \in \hilbert\), then
    \begin{align*}
        \inner{u}{A^* \circ A v} &= \frac14\sum_{n=0}^3 i^n\inner{u + i^{-n}v}{A^*\circ A(u + i^{-n}v)}\\
                                 &= \frac14 \sum_{n=0}^3 i^n\inner{u + i^{-n}v}{B^* \circ B(u + i^{-n}v)}\\
                                 &= \inner{u}{B^* \circ B v}
    \end{align*}
    follows from \cref{prop:polarization_identity}. That is, for all \(u, v \in \hilbert\), \(\inner{u}{(A^*\circ A - B^*\circ B)v} = 0\), hence the non-degeneracy of the inner product yields \(A^*\circ A = B^*\circ B\).
\end{proof}

\begin{proposition}{Necessary and sufficient condition for a normal operator}{normal}
    A bounded operator \(A \in \bounded(\hilbert)\) in a Hilbert space \(\hilbert\) is normal if and only if \(\norm{Ax} = \norm{A^* x}\) for all \(x \in \hilbert\).
\end{proposition}
\begin{proof}
    Setting \(B = A^*\) in \cref{lem:normal} yields
    \begin{align*}
        \forall x \in \hilbert, \norm{Ax} = \norm{A^*x} &\iff A^*\circ A = (A^*)^*\circ A^*\\
                                                        &\iff A^* \circ A = A \circ A^*\\
                                                        &\iff A\text{ is normal,}
    \end{align*}
    as desired.
\end{proof}

\subsection{Unitary operators}
Unitary operators are the structure preserving map of Hilbert spaces. In order to study them, we first consider maps that preserve the inner product.
\begin{theorem}{Inner product preserving map equivalent to linear isometry}{inner_product_linear_isometry}
    Let \(\hilbert_1, \hilbert_2\) be Hilbert spaces. The map \(U : \hilbert_1 \to \hilbert_2\) preserves the inner product, that is,
    \begin{equation*}
        \inner{Ux}{Uy}_{\hilbert_2} = \inner{x}{y}_{\hilbert_1}
    \end{equation*}
    for all \(x, y \in \hilbert_1\) if and only if \(U\) is a linear isometry.
\end{theorem}
\begin{proof}
    Suppose \(U\) preserves the inner product. Let \(x, y \in \hilbert\) and \(\lambda \in \mathbb{C}\). It is clear \(U\) is an isometry since
    \begin{equation*}
        \norm{U(x)}^2 = \inner{U(x)}{U(x)} = \inner{x}{x} = \norm{x}^2.
    \end{equation*}
    The map is also homogeneous since
    \begin{align*}
        \norm{\lambda U(x) - U(\lambda x)}^2
        &= \inner{\lambda U(x) - U(\lambda x)}{\lambda U(x) - U(\lambda x)}\\
        &= \norm{\lambda U(x)}^2 + \norm{U(\lambda x)}^2 - 2\Re\left(\inner{\lambda U(x)}{U(\lambda x)}\right)\\
        &= \abs{\lambda}^2\norm{U(x)}^2 + \norm{\lambda x}^2 - 2\Re\left(\conj{\lambda} \inner{U(x)}{U(\lambda x)}\right)\\
        &= 2\abs{\lambda}^2 \norm{x}^2 - 2\Re\left(\lambda \inner{\lambda x}{x}\right)\\
        &= 0.
    \end{align*}
    Notice
    \begin{align*}
        \norm{U(x) + U(y)}^2 &= \norm{U(x)}^2 + \norm{U(y)}^2 + 2\Re(\inner{U(x)}{U(y)})\\
                             &= \norm{x}^2 + \norm{y}^2 + 2\Re(\inner{x}{y})\\
                             &= \norm{x + y}^2,
    \end{align*}
    then
    \begin{align*}
        \norm{U(x) + U(y) - U(x + y)}^2
        &= \norm{U(x) + U(y)}^2 + \norm{U(x+y)}^2 - 2\Re\left(\inner{U(x) + U(y)}{U(x + y)}\right)\\
        &=2\norm{x+y}^2 - 2\Re(\inner{x}{x+y} + \inner{y}{x+y})\\
        &= 0,
    \end{align*}
    hence \(U\) is linear.

    Suppose \(U\) is a linear isometry. Let \(x, y \in \hilbert_1\), then \cref{prop:polarization_identity} yields
    \begin{align*}
        \inner{Ux}{Uy} &= \frac14 \sum_{n=0}^3 i^{n} \norm{Ux + i^{-n}Uy}^2\\
                                    &= \frac14 \sum_{n=0}^3 i^{n} \norm{U(x + i^{-n}y)}\\
                                    &= \frac14 \sum_{n=0}^3 i^{n} \norm{x + i^{-n}y}\\
                                    &= \inner{x}{y},
    \end{align*}
    hence \(U\) preserves the inner product.
\end{proof}
We have shown in \cref{prop:isometry_bounded} that linear isometries are bounded with unitary operator norm, hence this result shows inner product preserving maps are bounded linear operators. In particular, these maps admit adjoint operators.
\begin{lemma}{Adjoint of an inner product preserving map}{adjoint_inner_product_preserving}
    Let \(\hilbert_1, \hilbert_2\) be Hilbert spaces. If the map \(U : \hilbert_1 \to \hilbert_2\) preserves the inner product, then \(U^* \circ U = \unity_{\hilbert_1}\).
\end{lemma}
\begin{proof}
    Since \(U\) is bounded, we have
    \begin{equation*}
        \inner{x}{y} = \inner{U x}{U y} = \inner{x}{U^* \circ Uy} \implies \inner{x}{(\unity_{\hilbert_1} - U^* \circ U)y} = 0
    \end{equation*}
    for all \(x, y \in \hilbert_1\), hence \(U^* \circ U = \unity_{\hilbert_1}\) follows from the non-degeneracy of the inner product.
\end{proof}

\cref{prop:isometry_Banach} guarantees us that the image of a linear isometry is a Banach space. Hence, the image of an inner product preserving map is a Hilbert space.
\begin{definition}{Unitary operator and isomorphic Hilbert spaces}{unitary_operator}
    Let \(\hilbert_1, \hilbert_2\) be Hilbert spaces. A \emph{unitary operator} is a surjective map \(U : \hilbert_1 \to \hilbert_2\) that preserves the inner product. If there exists such a map, we say \(\hilbert_1\) and \(\hilbert_2\) are \emph{isomorphic}.
\end{definition}

Every choice of complete orthonormal basis on a separable Hilbert space defines a unitary operator onto the space of square-summable sequences \(\ell_2\).
\begin{theorem}{Separable Hilbert spaces isomorphic to \(\ell_2\)}{separable_Hilbert_spaces_l2}
    Let \(\hilbert\) be a separable Hilbert space, then it is isomorphic to \(\ell_2\).
\end{theorem}
\begin{proof}
    Since \(\hilbert\) is a separable Hilbert space, there exists a countable complete orthonormal basis \(\family{e_n}{n\in \mathbb{N}}\subset \hilbert\). By \cref{thm:basis_separable,prop:convergent_series}, for every \(x \in \hilbert\), the sequence \(n \mapsto \inner{e_n}{x}\) is square-summable. That is, the complete orthonormal basis defines a map
    \begin{align*}
        j : \hilbert &\to \ell_2\\
                   x &\mapsto \family{\inner{e_n}{x}}{n \in \mathbb{N}}.
    \end{align*}
    Let \(x, y \in \hilbert\), then
    \begin{align*}
        \inner{x}{y} &= \inner*{\lim_{n \to \infty}\sum_{k = 1}^n \inner{e_n}{x}e_n}{y}\\
                     &= \lim_{n \to \infty} \sum_{k=1}^n \conj{\inner{e_n}{x}} \inner{e_n}{y}\\
                     &= \inner{j(x)}{j(y)},
    \end{align*}
    that is, \(j\) preserves the inner product.

    Let \(\Lambda = \family{\lambda_n}{n \in \mathbb{N}} \in \ell_2\) be a square-summable sequence, then the series \(\family{\sum_{k = 1}^n \lambda_k e_k}{n \in \mathbb{N}} \subset \hilbert\) converges to some \(\tilde{x} \in \hilbert\), by \cref{prop:convergent_series}. Notice \(\inner{e_n}{\tilde{x}} = \lambda_n\) follows from the uniqueness of the best approximation, hence we conclude \(j(\tilde{x}) = \Lambda\). That is, \(j\) is a unitary operator.
\end{proof}
% It is immediate that the adjoint of a unitary operator is unitary.
% \begin{proposition}{Adjoint of a unitary operator}{adjoint_unitary}
%     Let \(\hilbert_1, \hilbert_2\) be Hilbert spaces. If \(U : \hilbert_1 \to \hilbert_2\) is a unitary operator, then \(U^{-1} = U^*\) and \(U^*\) is a unitary operator.
% \end{proposition}
% \begin{proof}
%     Since \(U\) is a surjective linear isometry, it is bijective, hence \(U^* = U^{-1}\) by \cref{prop:adjoint_inner_product_preserving}.  unitary. With
% \end{proof}

We now give necessary and sufficient conditions for a unitary operator.
\begin{theorem}{Necessary and sufficient conditions for a unitary operator}{unitary_iff}
    Let \(U : \hilbert_1 \to \hilbert_2\) be a map, where \(\hilbert_1, \hilbert_2\) are Hilbert spaces. The following statements are equivalent:
    \begin{enumerate}[label=(\alph*)]
        \item \(U\) is a unitary operator;
        \item \(U \in \bounded(\hilbert_1, \hilbert_2)\) and \(U^* = U^{-1}\);
        \item \(U\) and \(U^*\) are linear isometries;
        \item \(U\) is a surjective linear isometry.
    \end{enumerate}
\end{theorem}
\begin{proof}
    Suppose \(U\) is unitary. Then \cref{thm:inner_product_linear_isometry} ensures it is bijective and bounded. \cref{lem:adjoint_inner_product_preserving} yields
    \begin{equation*}
        U^* \circ U \circ U^{-1} = \unity_{\hilbert_1} \circ U^{-1} \implies U^* = U^{-1},
    \end{equation*}
    thus showing (a)\(\implies\)(b).

    Suppose \(U\) is bounded and \(U^* = U^{-1}\). For all \(u, v \in \hilbert_1\) and all \(x, y \in \hilbert_2\), we have
    \begin{equation*}
        \inner{Uu}{Uv} = \inner{u}{U^*\circ Uv} = \inner{u}{v}
        \quad\text{and}\quad
        \inner{U^*x}{U^*y} = \inner{x}{U\circ U^*y} = \inner{x}{y}
    \end{equation*}
    then \(U\) and \(U^*\) are linear isometries. We conclude (b)\(\implies\)(c).

    Suppose \(U\) and \(U^*\) are linear isometries. By \cref{thm:inner_product_linear_isometry}, \(U\) and \(U^*\) are inner product preserving maps. Then, for all \(v, w \in \hilbert_1\) and all \(x, y \in \hilbert_2\), we have
    \begin{equation*}
        \inner{v}{w} = \inner{Uv}{Uw} = \inner{U^*\circ Uv}{w} = 0
        \quad\text{and}\quad
        \inner{x}{y} = \inner{U^*x}{U^*y} = \inner{U \circ U^*x}{y} = 0,
    \end{equation*}
    hence \(U^*\circ U = \unity_{\hilbert_1}\) and \(U\circ U^* = \unity_{\hilbert_2}\) follows from the non-degeneracy of the inner product. We have thus shown \(U\) is bijective, that is, (c)\(\implies\)(d).

    Suppose \(U\) is a surjective linear isometry. By \cref{thm:inner_product_linear_isometry}, it follows that \(U\) preserves the inner product. Then, (d)\(\implies\)(a).
\end{proof}
\begin{corollary}
    Let \(\hilbert_1, \hilbert_2\) be Hilbert spaces. If \(U : \hilbert_1 \to \hilbert_2\) is unitary if and only if \(U^* : \hilbert_2 \to \hilbert_1\) is unitary.
\end{corollary}
\begin{proof}
    This should be clear from the equivalence of (a) and (c). We have
    \begin{align*}
        U \text{ is unitary} &\iff U\text{ and }U^*\text{ are linear isometries} \\
                             &\iff U^*\text{ and }U\text{ are linear isometries}\\
                             &\iff U^*\text{ is unitary},
    \end{align*}
    as claimed.
\end{proof}
\begin{corollary}
    Let \(\hilbert\) be a Hilbert space. If \(U : \hilbert \to \hilbert\) is unitary, then \(U\) is normal.
\end{corollary}
\begin{proof}
    Since \(U \circ U^* = U^* \circ U = \unity\), it follows that \(U\) commutes with \(U^*\).
\end{proof}


% vim: spl=en_us
\section{Bounded operators in Hilbert spaces}
We recall \crefname{thm:orthogonal_decomposition}, that for every vector \(v \in \hilbert\), there exist unique \(v_\parallel \in V\) and \(v_\perp \in V^\perp\) such that \(v = v_\parallel + v_\perp\), and \(v_\parallel\) is the unique best approximation of \(v\) on \(V\). This motivates the definition of a map that realizes the best approximation.
\begin{definition}{Orthogonal projector}{orthogonal_projector}
    Let \(\hilbert\) be a Hilbert space. If \(V \subset \hilbert\) is a closed linear subspace of \(\hilbert\), the \emph{orthogonal projector onto \(V\)} is the map
    \begin{align*}
        P_V : \hilbert &\to \hilbert\\
                     v &\mapsto v_\parallel
    \end{align*}
    where \(v_\parallel \in V\) is the unique best approximation of \(v\) on \(V\). If \(x \in \hilbert\), we may write \(P_x\) as shorthand for the orthogonal projector \(P_{\lspan{\set{x}}} : \hilbert \to \hilbert\).
\end{definition}

\begin{proposition}{Orthogonal projector is a bounded operator}{orthogonal_projector_bounded}
    Let \(\hilbert\) be a Hilbert space. If \(V\) is a closed linear subspace of \(\hilbert\), then \(P_V \in \bounded(\hilbert)\) with \(\norm{P_V} \leq 1\).
\end{proposition}
\begin{proof}
    Let \(x, y \in \hilbert\) and \(\alpha \in \mathbb{C}\), then \cref{thm:orthogonal_decomposition} guarantees the existence and uniqueness of \(x_\parallel, y_\parallel \in V\) and \(x_\perp, y_\perp \in V^\perp\) such that \(x = x_\parallel + x_\perp\) and \(y = y_\parallel + y_\perp\), then
    \begin{equation*}
        x + \alpha y = (x_\parallel + \alpha y_\parallel) + (x_\perp + \alpha y_\perp).
    \end{equation*}
    That is, \(x_\parallel + \alpha y_\parallel\) is the best approximation of \(x + \alpha y\) in \(V\), hence \(P_V(x + \alpha y) = P_V(x) + \alpha P_V(y)\). It is clear \(P_V\) is bounded since
    \begin{equation*}
        \sup_{v \in \hilbert} \frac{\norm{P_Vv}}{\norm{v}} = \sup_{v \in \hilbert} \frac{\norm{v_\parallel}}{\sqrt{\norm{v_\parallel}^2 + \norm{v_\perp}^2}} \leq 1,
    \end{equation*}
    that is, \(\norm{P_V} \leq 1\).
\end{proof}

An important consequence of the orthogonal decomposition theorem is that every bounded linear operator defined on a subset of a Hilbert space can be extended to the entire linear space.
\begin{theorem}{Isometric extension of a bounded operator defined on a Hilbert space}{isometric_extension_bounded_hilbert}
    Let \(T : \domain{T} \subset \hilbert \to X\) be a bounded linear operator, where \(\hilbert\) is a Hilbert space and \(X\) is a Banach space. Then \(T\) can be uniquely extended to \(\hat{T} : \hilbert \to X\), where the extension satisfies \(\hat{T}((\cl_\hilbert\domain{T})^\perp) = \set{0}\) and \(\norm{\hat{T}} = \norm{T}\).
\end{theorem}
\begin{proof}
    Notice \(\domain{T}\) is dense in the Banach space \(V = \cl_\hilbert{\domain{T}}.\) Using \cref{thm:blt} if necessary, there exists a unique bounded linear extension \(\tilde{T} : V \to X\) to \(T\) such that \(\norm{\tilde{T}} = \norm{T}\).

    We define the map
    \begin{align*}
        \hat{T} : \hilbert &\to X\\
                         v &\mapsto \tilde{T} \circ P_V (v),
    \end{align*}
    which is manifestly a bounded linear operator, as a composition of bounded linear operators. Notice \(P_Vu = u\) for all \(u \in \domain{T}\subset V\), since \(u\) is its best approximation in \(V\). Then, since \(\tilde{T}\) extends \(T\), \(\hat{T}\) extends \(T\). As \(\ker{P_V} = V^\perp\), it follows that \(\hat{T}(V^\perp) = \set{0}\). Moreover, we have
    \begin{equation*}
        \norm{\hat{T}} = \sup_{v \in \hilbert}{\frac{\norm{\hat{T}v}}{\norm{v}}}= \sup_{v \in \hilbert}{\frac{\norm{\tilde{T}v_\parallel}}{\sqrt{\norm{v_\parallel}^2 + \norm{v_\perp}^2}}} \leq \sup_{v\in V}{\frac{\norm{\tilde{T}v}}{\norm{v}}} = \norm{\tilde{T}} = \norm{T}
    \end{equation*}
    and
    \begin{equation*}
        \norm{\hat{T}} = \sup_{v \in \hilbert}{\frac{\norm{\hat{T}v}}{\norm{v}}} \geq \sup_{v\in\domain{T}}{\frac{\norm{Tv}}{\norm{v}}} = \norm{T},
    \end{equation*}
    and we conclude \(\norm{\hat{T}} = \norm{T}\).

    Suppose there exists \(S \in \bounded(\hilbert, X)\) that extends \(T\) such that \(\norm{S} = \norm{T}\) and \(S(V^\perp) = \set{0}\). Then, \(\restrict{S}{V}\) is a bounded extension to \(T\) with \(\norm{\restrict{S}{V}} = \norm{T}\), hence the uniqueness of \(\tilde{T}\) yields \(\restrict{S}{V} = \tilde{T}\). For every \(v \in V\) we have
    \begin{equation*}
        S(v) = S\circ P_V(v) + S(v - P_V(v)) = \restrict{S}{V} \circ P_V(v) = \tilde{T}\circ P_V (v) = \hat{T}(v),
    \end{equation*}
    then \(S = \hat{T}\), showing uniqueness.
\end{proof}

\begin{proposition}{Isometric extension of a bounded operator defined on a Hilbert space}{extension_bounded_hilbert}
    Let \(T : \domain{T} \subset \hilbert \to X\) be a bounded linear operator, where \(\hilbert\) is a Hilbert space and \(X\) is a Banach space. Then there exists an extension \(\hat{T} : \hilbert \to X\) to \(T\) defined by \(\hat{T}(v) = \lim_{n \to \infty} T(\tilde{v}_n)\) where \(\family{\tilde{v}_n}{n \in \mathbb{N}} \subset \domain{T}\) is a sequence that converges to \(v_\parallel = P_{\cl_{\hilbert}{\domain{T}}}v\). Moreover, \(\hat{T}\) is bounded and \(\norm{\hat{T}} = \norm{T}\).
\end{proposition}
\begin{proof}
    If \(\domain{T}\) is dense in \(\hilbert,\) we may use \cref{thm:blt} and end up with the above described map, which is the unique bounded extension and satisfies \(\norm{\hat{T}} = \norm{T}\).

    Suppose \(\domain{T}\) is not dense in \(\hilbert\). Then \(\family{T\tilde{v}_n}{n \in \mathbb{N}} \subset X\) is a Cauchy sequence in the Banach space \(X\) since \(T\) is uniformly continuous. By completeness, \(\family{T\tilde{v}_n}{n \in \mathbb{N}}\) is convergent. Let \(\family{\tilde{w}_n} \subset \domain{T}\) be another sequence that converges to \(v_\parallel\), then by the same argument \(\family{T\tilde{w}_n}{n \in \mathbb{N}}\) is convergent. Both sequences converge to the same value in \(X\), that is, \(\hat{T}\) is well-defined. Indeed, since \(T\) is bounded, we have
    \begin{align*}
        \norm{T\tilde{w}_n - T\tilde{v}_n} &\leq \norm{T}\cdot \norm{\tilde{w}_n - \tilde{v}_n}\\
                                           &\leq \norm{T} \cdot \left(\norm{\tilde{w}_n - v_\parallel} + \norm{v_\parallel - \tilde{v}_n}\right),
    \end{align*}
    hence we may take \(n\) arbitrarily large as to make the right hand side arbitrarily small, and we conclude \(T\tilde{w}_n = T\tilde{v}_n = \hat{T}(v)\).

    Since the orthogonal projector and \(T\) are linear, \(\hat{T}\) must be linear. Indeed, let \(x, y \in \hilbert\) and \(\alpha \in \mathbb{C}\), then there exist sequences \(\family{\tilde{x}}{n \in \mathbb{N}} \subset \domain{T}\) and \(\family{\tilde{y}}{n\in \mathbb{N}} \subset \domain{T}\) such that \(\tilde{x}_n \to x_\parallel\), \(\tilde{y}_n \to y_\parallel\), and \(\tilde{x}_n + \alpha \tilde{y}_n \to x_\parallel + \alpha y_\parallel\). Linearity and continuity yield
    \begin{equation*}
        \hat{T}(x + \alpha y) = \lim_{n \to \infty}{T(\tilde{x}_n + \alpha \tilde{y}_n)}= \lim_{n \to \infty}{T(\tilde{x}_n)}+ \alpha \lim_{n\to\infty}{T(\tilde{y}_n)} = \hat{T}(x) + \alpha \hat{T}(y),
    \end{equation*}
    that is, \(\hat{T}\) is a linear operator. Moreover, it is clear that \(\hat{T}\) extends \(T\), since \(\domain{T} \subset \cl_\hilbert{\domain{T}}\), then for every \(u \in \domain{T}\), the best approximation of \(u\) in \(\cl_\hilbert{\domain{T}}\) is \(u\), hence we may consider the constant sequence that converges to \(u\) to evaluate \(\hat{T}(u) = T(u)\).
\end{proof}
Due to \cref{thm:isometric_extension_bounded_hilbert}, we can always assume a \emph{bounded} linear operator is defined on an entire Hilbert space, rather than just a linear subspace.

\subsection{Adjoint of a bounded operator}
We consider bounded operators \(A : \hilbert_1 \to \hilbert_2\), where \(\hilbert_1\) and \(\hilbert_2\) are Hilbert spaces. As a consequence of \nameref{thm:riesz_representation}, we'll show any bounded operator \(A \in \bounded(\hilbert_1, \hilbert_2)\) defines a unique operator \(B \in \bounded(\hilbert_2, \hilbert_1)\) such that \(\inner{y}{Ax}_{\hilbert_2} = \inner{By}{x}_{\hilbert_1}\), for all \(x \in \hilbert_1\) and \(y \in \hilbert_2\).
\begin{lemma}{Bicontinuous sesquilinear form defines a unique operator}{sesquilinear_adjoint}
    Let \(\mathscr{A} : \hilbert_2 \times \hilbert_1 \to \mathbb{C}\) be a sesquilinear form defined on Hilbert spaces \(\hilbert_1\) and \(\hilbert_2\). Suppose \(\mathscr{A}\) is \emph{bicontinuous}, that is, there exists \(M > 0\) such that \(\abs{\mathscr{A}(u,v)} \leq M\norm{u}_{\hilbert_2} \norm{v}_{\hilbert_1}\) for all \(u \in \hilbert_2\) and all \(v \in \hilbert_1\). Then, there exists a unique linear operator \(B \in \bounded(\hilbert_2, \hilbert_1)\) such that
    \begin{equation*}
        \mathscr{A}(u,v) = \inner{Bu}{v}_{\hilbert_1},
    \end{equation*}
    for all \(u \in \hilbert_2\) and all \(v \in \hilbert_1\).
\end{lemma}
\begin{proof}
    As \(\mathscr{A}\) is bicontinuous, it follows that for each \(u \in \hilbert_2\), the map
    \begin{align*}
        \ell_u : \hilbert_1 &\to \mathbb{C}\\
                          v &\mapsto \mathscr{A}(u,v)
    \end{align*}
    is a bounded linear functional on \(\hilbert_1\). That is, there exists a map \(\ell : \hilbert_2 \to \hilbert_1^\dag\) defined by \(\ell(u) = \ell_u\). Let \(u_1, u_2 \in \hilbert_2\) and \(\alpha \in \mathbb{C}\), then
    \begin{align*}
        \ell(u_1 + \alpha u_2)(v) = \ell_{u_1 + \alpha u_2}(v) &= \mathscr{A}(u_1 + \alpha u_2, v)\\
                                                               &= \mathscr{A}(u_1, v) + \conj{\alpha} \mathscr{A}(u_2, v)\\
                                                               &= \ell_{u_1}(v) + \conj{\alpha} \ell_{u_2}(v)\\
                                                               &= \left[\ell(u_1) + \conj{\alpha} \ell(u_2)\right](v)
    \end{align*}
    for all \(v \in \hilbert_1\). We infer \(\ell(u_1 + \alpha u_2) = \ell(u_1) + \conj{\alpha}\ell(u_2)\), thus showing \(\ell\) is an antilinear map.

    We claim the map \(B = \riesz \circ \ell\) is a bounded linear operator such that \(\mathscr{A}(u,v) = \inner{B(u)}{v}_{\hilbert_1}\), where \(\riesz : \hilbert_1^\dag \to \hilbert_1\) is the Riesz representation map. \nameref{thm:riesz_representation} ensures that \(B(u)\) is the unique vector in \(\hilbert_1\) such that \(\ell_u(v) = \inner{B(u)}{v}_{\hilbert_1}\) for all \(v \in \hilbert_1\). As a result, for all \(u \in \hilbert_2\) and \(v \in \hilbert_1\), we have \(\inner{B(u)}{v}_{\hilbert_1} = \mathscr{A}(u,v)\). It is clear \(B\) is a linear map since it is the composition of two antilinear maps. Moreover, since \(\mathscr{A}\) is bicontinuous, we have
    \begin{equation*}
        \norm{Bu}_{\hilbert_1}^2 = \abs*{\inner{Bu}{Bu}_{\hilbert_1}} = \abs*{\mathscr{A}(u, Bu)} \leq M \norm{u}_{\hilbert_2} \norm{Bu}_{\hilbert_1},
    \end{equation*}
    that is, \(M\) is an upper bound for the set \(\setc*{\frac{\norm{Bu}_{\hilbert_1}}{\norm{u}_{\hilbert_2}}}{u \in \hilbert_2 \setminus \set{0}}\), hence \(B \in \bounded(\hilbert_2, \hilbert_1)\), proving our claim.

    Let \(\tilde{B} \in \bounded(\hilbert_2, \hilbert_1)\) such that \(\mathscr{A}(u,v) = \inner{\tilde{B}u}{v}_{\hilbert_1}\) for all \(u \in \hilbert_2\) and \(v \in \hilbert_1\). Then, \(\inner{Bu}{v}_{\hilbert_1} = \inner{\tilde{B}u}{v}_{\hilbert_1}\), which implies \(\inner{(B - \tilde{B})u}{v}_{\hilbert_1} = 0\) for all \(u \in \hilbert_2\) and all \(v \in \hilbert_1\). Since the inner product is non-degenerate this yields \((B - \tilde{B})u = 0\) for all \(u \in \hilbert_2\), hence \(B = \tilde{B}\), which shows uniqueness.
\end{proof}

\begin{theorem}{Existence of the adjoint of a bounded operator}{adjoint_bounded_hilbert}
    Let \(\hilbert_1, \hilbert_2\) be Hilbert spaces. If \(A \in \bounded(\hilbert_1, \hilbert_2)\) is a bounded linear operator, then there exists a unique bounded linear operator \(B \in \bounded(\hilbert_2, \bounded_1)\) such that
    \begin{equation*}
        \inner{y}{Ax}_{\hilbert_2} = \inner{By}{x}_{\hilbert_1}
    \end{equation*}
    for all \(x \in \hilbert_1\) and \(y \in \hilbert_2\).
\end{theorem}
\begin{proof}
    We claim the map
    \begin{align*}
        \mathscr{A} : \hilbert_2 \times \hilbert_1 &\to \mathbb{C}\\
                                             (y,x) &\mapsto \inner{y}{Ax}_{\hilbert_2}
    \end{align*}
    is a bicontinuous sesquilinear form. Sesquilinearity of \(\mathscr{A}\) follows from linearity of the operator \(A\) and from sesquilinearity of the inner product defined on \(\hilbert_2\). \nameref{thm:cauchy_schwarz} yields
    \begin{equation*}
        \abs{\mathscr{A}(y,x)} = \abs*{\inner{y}{Ax}_{\hilbert_2}} \leq \norm{y}_{\hilbert_2} \norm{Ax}_{\hilbert_2} \leq \norm{A}_{\bounded(\hilbert_1, \hilbert_2)} \norm{x}_{\hilbert_1} \norm{y}_{\hilbert_2},
    \end{equation*}
    for all \(x \in \hilbert_1\) and \(y \in \hilbert_2\), hence \(\mathscr{A}\) is bicontinuous. By \cref{lem:sesquilinear_adjoint}, there exists a unique bounded linear operator \(B \in \bounded(\hilbert_2, \hilbert_1)\) such that \(\mathscr{A}(y,x) = \inner{By}{x}_{\hilbert_1}\), that is,
    \begin{equation*}
        \inner{y}{Ax}_{\hilbert_2} = \inner{By}{x}_{\hilbert_1},
    \end{equation*}
    for all \(x \in \hilbert_1\) and \(y \in \hilbert_2\).
\end{proof}

Notice \cref{thm:adjoint_bounded_hilbert} establishes a map from \(\bounded(\hilbert_1, \hilbert_2)\) to \(\bounded(\hilbert_2, \hilbert_1)\).
\begin{definition}{Adjoint of a bounded operator}{adjoint_bounded_hilbert}
    Let \(\hilbert_1, \hilbert_2\) be Hilbert spaces. The map
    \begin{align*}
        ^* : \bounded(\hilbert_1, \hilbert_2) &\to \bounded(\hilbert_2, \hilbert_1)\\
                                           A &\mapsto A^*,
    \end{align*}
    where \(A^*\) is the unique bounded operator such that
    \begin{equation*}
        \inner{y}{Ax}_{\hilbert_2} = \inner{A^*y}{x}_{\hilbert_1}
    \end{equation*}
    for all \(x \in \hilbert_1\) and \(y \in \hilbert_2\), is called an \emph{adjoint operation}, later \emph{involution}, on \(\bounded(\hilbert_1, \hilbert_2)\). The operator \(A^*\) is called the \emph{adjoint of \(A\)}.
\end{definition}
\begin{remark}
    The Banach space adjoint and the Hilbert space adjoint are unfortunately not the same operation, however the Riesz representation let's us relate them. Let \(T \in \bounded(\hilbert_1, \hilbert_2)\), with \(T' \in \bounded(\hilbert_2^\dag, \hilbert_1^\dag)\) being the Banach space adjoint and \(T^* \in \bounded(\hilbert_2, \hilbert_1)\) the Hilbert space adjoint, then for all \(x \in \hilbert_1\) and \(y \in \hilbert_2\) we have
    \begin{align*}
        \inner{\riesz_{\hilbert_1} \circ T' \circ \riesz_{\hilbert_2}^{-1}(y)}{x}_{\hilbert_1}
        &= \inner{\riesz_{\hilbert_1}(\riesz_{\hilbert_2}^{-1}(y) \circ T)}{x}_{\hilbert_1}\\
        &= \riesz_{\hilbert_2}^{-1}(y) \circ T(x)\\
        &= \inner{y}{Tx}_{\hilbert_2},
    \end{align*}
    hence \(T^* = \riesz_{\hilbert_1} \circ T' \circ \riesz^{-1}_{\hilbert_2}\) by uniqueness of the Hilbert space adjoint.
\end{remark}

\begin{theorem}{Adjoint operation properties}{involution_properties}
    We'll denote Hilbert spaces by \(\hilbert_n\) and the adjoint operations on \(\bounded(\hilbert_n, \hilbert_m)\) simply as \(^*\). Then,
    \begin{enumerate}[label=(\alph*)]
        \item The adjoint operation is an involution: \((A^*)^* = A\), for all \(A \in \bounded(\hilbert_1, \hilbert_2)\);
        \item The adjoint operation is an isometry: \(\norm{A^*} = \norm{A},\) for all \(A \in \bounded(\hilbert_1, \hilbert_2)\);
        \item The adjoint operation satisfies the \(C^*\) property: \(\norm{A^* \circ A} = \norm{A}^2\), for all \(A \in \bounded(\hilbert_1, \hilbert_2)\);
        \item The adjoint operation is antilinear: \((\alpha A + \beta B)^* = \conj{A} A^* + \conj{\beta} B^*\) for all \(\alpha, \beta \in \mathbb{C}\) and \(A, B \in \bounded(\hilbert_1, \hilbert_2)\);
        \item The adjoint operation is antidistributive: \((B \circ A)^* = A^*\circ B^*\) for all \(A \in \bounded(\hilbert_1, \hilbert_2)\) and \(B \in \bounded(\hilbert_2, \hilbert_3)\);
        \item The adjoint operation preserves the identity: if \(\unity \in \bounded(\hilbert_1)\) is the identity operator, then \(\unity^* = \unity\);
        \item The adjoint operation maps the inverse to the inverse of the adjoint: if \(A \in \bounded(\hilbert_1, \hilbert_2)\) admits an inverse map \(A^{-1} \in \bounded(\hilbert_2, \hilbert_1)\), then \((A^{-1})^* = (A^*)^{-1}\).
    \end{enumerate}
\end{theorem}
\begin{proof}[Proof of (a)]
    Let \(A \in \bounded(\hilbert_1, \hilbert_2)\), then
    \begin{equation*}
        \inner{(A^*)^*x}{y}_{\hilbert_2} = \inner{x}{A^*y}_{\hilbert_1} = \conj{\inner{A^*y}{x}_{\hilbert_1}} = \conj{\inner{y}{Ax}_{\hilbert_2}} = \inner{Ax}{y}_{\hilbert_2}
    \end{equation*}
    for all \(x \in \hilbert_1\) and \(y \in \hilbert_2.\) By non-degeneracy of the inner product, we have \(A = (A^*)^*\), showing involutivity.
\end{proof}
\begin{proof}[Proof of (b)]
    We consider the Banach space adjoint operation
    \begin{align*}
        ' : \bounded(\hilbert_1, \hilbert_2) &\to \bounded(\hilbert_2^\dag, \hilbert_1^\dag)\\
                                           A &\mapsto A'
    \end{align*}
    which we have shown in \cref{prop:adjoint_Banach} to be an isometry. Since the maps \(\riesz_{\hilbert_1}\) and \(\riesz_{\hilbert_2}^{-1}\) are bijective isometries, it follows that
    \begin{align*}
        \norm{A^*}_{\bounded(\hilbert_2, \hilbert_1)} = \sup_{x \in \hilbert_2}{\frac{\norm{A^*x}_{\hilbert_1}}{\norm{x}_{\hilbert_2}}}
        &= \sup_{x \in \hilbert_2}{\frac{\norm{\riesz_{\hilbert_1}\circ A' \circ \riesz^{-1}_{\hilbert_2}(x)}_{\hilbert_1}}{\norm{x}_{\hilbert_2}}}\\
        &= \sup_{x \in \hilbert_2}{\frac{\norm{A' \circ \riesz^{-1}_{\hilbert_2}(x)}_{\hilbert_1^\dag}}{\norm{\riesz_{\hilbert_2}^{-1}(x)}_{\hilbert_2^\dag}}}\\
        &= \sup_{\ell \in \hilbert_2^\dag}{\frac{\norm{A'(\ell)}_{\hilbert_1^\dag}}{\norm{\ell}_{\hilbert_2^\dag}}}\\
        &= \norm{A'}_{\bounded(\hilbert_2^\dag, \hilbert_1^\dag)} = \norm{A}_{\bounded(\hilbert_1,\hilbert_2)}
    \end{align*}
    for all \(A \in \bounded(\hilbert_1, \hilbert_2)\).
\end{proof}

% vim: spl=en_us
\section{States}
As usual we denote the topological dual of a C*-algebra \(\algebra{A}\) by \(\algebra{A}^\dag\), which is a Banach space with respect to the operator norm, and contained in the algebraic dual \(\algebra{A}'\), which consists of all linear functionals \(\omega : \algebra{A} \to \mathbb{C}\).
\begin{definition}{Positive linear functional and states}{state}
    Let \(\algebra{A}\) be a C*-algebra. The linear functional \(\omega \in \algebra{A}'\) is \emph{positive} if \(\omega(a^*a) \geq 0\) for all \(a \in \algebra{A}\). A \emph{state} is a positive linear functional \(\omega\) with \(\norm{\omega} = 1\).
\end{definition}

A positive linear functional defines a positive sesquilinear form on \(\algebra{A}\), then, in this sense, this linear functional satisfies the Cauchy-Schwarz inequality.
\begin{proposition}{Properties of a positive sesquilinear form}{sesquilinear_positive_form}
    Let \(\inner{\noarg}{\noarg} : V \to \mathbb{C}\) be a sesquilinear form on a complex linear space \(V\), that is, it is anti-linear in the first argument, linear in the second argument. If it is positive, that is, it satisfies \(\inner{v}{v} \geq 0\) for all \(v \in V\), then
    \begin{enumerate}[label=(\alph*)]
        \item this sesquilinear form is Hermitian, \(\inner{u}{v} = \conj{\inner{v}{u}}\); and
        \item the Cauchy-Schwarz inequality holds, \(\abs{\inner{u}{v}}^2 \leq \inner{u}{u} \inner{v}{v}\),
    \end{enumerate}
    for all \(u,v \in V.\)
\end{proposition}
\begin{proof}
    As the sesquilinear form is positive, we have
    \begin{equation*}
        \inner{u+\lambda v}{u + \lambda v} = \inner{u}{u} + \lambda \inner{u}{v} + \conj{\lambda} \inner{v}{u} + \abs{\lambda}^2\inner{v}{v} \geq 0,
    \end{equation*}
    for all \(u,v \in V\) and \(\lambda \in \mathbb{C}\). The imaginary part of the left-hand side must be zero, hence
    \begin{equation*}
        0 = \Im\left(\lambda \inner{u}{v} + \conj{\lambda} \inner{v}{u}\right) = \Re(\lambda) \Im\left(\inner{u}{v} + \inner{v}{u}\right) + \Im(\lambda)\Re\left(\inner{u}{v} - \inner{v}{u}\right),
    \end{equation*}
    and by setting \(\lambda = 1\) and \(\lambda = i\) we conclude
    \begin{equation*}
        \Im\left(\inner{u}{v}\right) = - \Im\left(\inner{v}{u}\right)
        \quad\text\quad
        \Re\left(\inner{u}{v}\right) = \Re\left(\inner{v}{u}\right),
    \end{equation*}
    that is, \(\inner{u}{v} = \conj{\inner{v}{u}}\). As a consequence we have
    \begin{align*}
        0 \leq \inner{u+\lambda v}{u + \lambda v} &= \inner{u}{u} + 2 \Re\left(\lambda \inner{u}{v}\right) + \abs{\lambda}^2\inner{v}{v}\\
                                                  &= \inner{u}{u} + 2 \Re(\lambda) \Re\left(\inner{u}{v}\right) - 2 \Im(\lambda) \Im\left(\inner{u}{v}\right) + \abs{\lambda}^2 \inner{v}{v}.
    \end{align*}
    If \(\inner{v}{v} = 0\), then the inequality can only hold for all \(\lambda\in \mathbb{C}\) if \(\inner{u}{v} = 0\), case for which the Cauchy-Schwarz inequality holds trivially, so we must assume \(\inner{v}{v} \neq 0\) without loss of generality. Notice
    \begin{equation*}
        \inner{v}{v}\left[\Re(\lambda) + \frac{\Re(\inner{u}{v})}{\inner{v}{v}}\right]^2 = \inner{v}{v} \Re(\lambda)^2 + \frac{\Re(\inner{u}{v})^2}{\inner{v}{v}} + 2\Re(\lambda) \Re(\inner{u}{v}
    \end{equation*}
    and
    \begin{equation*}
        \inner{v}{v}\left[\Im(\lambda) - \frac{\Im(\inner{u}{v})}{\inner{v}{v}}\right]^2 = \inner{v}{v} \Im(\lambda)^2 + \frac{\Im(\inner{u}{v})^2}{\inner{v}{v}} - 2\Im(\lambda) \Im(\inner{u}{v},
    \end{equation*}
    then
    \begin{equation*}
        \inner{u+\lambda v}{u + \lambda v} = \inner{u}{u} + \inner{v}{v}\left[\left(\Re(\lambda) + \frac{\Re(\inner{u}{v})}{\inner{v}{v}}\right)^2 + \left(\Im(\lambda) - \frac{\Im(\inner{u}{v})}{\inner{v}{v}}\right)^2\right] - \frac{\abs{\inner{u}{v}}^2}{\inner{v}{v}},
    \end{equation*}
    and thus positivity yields
    \begin{equation*}
        \inner{u}{u} \geq \frac{\abs{\inner{u}{v}}^2}{\inner{v}{v}}
    \end{equation*}
    and the Cauchy-Schwarz inequality follows.
\end{proof}
\begin{corollary}
    Let \(\algebra{A}\) be a C*-algebra and let \(\omega \in \algebra{A}'\) be a positive linear functional. Then
    \begin{enumerate}[label=(\alph*)]
        \item \(\omega(a^*b) = \conj{\omega(b^*a)}\), and
        \item \(\abs{\omega(a^*b)}^2 \leq \omega(a^*a) \omega(b^*b)\),
    \end{enumerate}
    for all \(a, b \in \algebra{A}\).
\end{corollary}
\begin{proof}
    The map
    \begin{align*}
        \inner{\noarg}{\noarg} : \algebra{A} \times \algebra{A} &\to \mathbb{C}\\
                                                          (a,b) &\mapsto \omega(a^*b),
    \end{align*}
    is a positive sesquilinear form. Indeed, let \(a, b,c,d \in \algebra{A}\) and \(\lambda,\mu \in \mathbb{C}\), then
    \begin{align*}
        \inner{a + \lambda b}{c + \mu d} &= \omega\left(a^*c + \mu a^*d + \conj{\lambda} b^* c + \conj{\lambda}\mu b^*d\right)\\
                                         &= \omega(a^*c) + \mu \omega(a^*d) + \conj{\lambda} \omega(b^*c) + \conj{\lambda}\mu \omega(b^*d)\\
                                         &= \inner{a}{c} + \mu \inner{a}{d} + \conj{\lambda} \inner{b}{c} + \conj{\lambda}\mu \inner{b}{d},
    \end{align*}
    hence it is a sesquilinear form and it is positive since \(\inner{a}{a} = \omega(a^*a) \geq 0\) by hypothesis. The corollary then follows from the previous proposition.
\end{proof}

\begin{proposition}{Positive linear functionals are continuous}{state_continuous}
    Let \(\omega : \algebra{A} \to \mathbb{C}\) be a linear functional over a C*-algebra \(\algebra{A}\). The following statements are equivalent:
    \begin{enumerate}[label=(\alph*)]
        \item \(\omega\) is positive;
        \item \(\omega\) is continuous and \(\norm{\omega} = \lim_{\lambda} \omega(e_{\lambda})\) for some approximate identity \(\family{e_{\lambda}}{\lambda \in \Lambda} \subset \algebra{A}_+\).
    \end{enumerate}
\end{proposition}
\begin{proof}
\end{proof}

% vim: spl=en_us
\chapter{C*-algebras}
Bounded operators on a Hilbert space give rise to an abstraction that proves fruitful. Previously we have shown that bounded operators on a Hilbert space can be made into a unital C*-algebra with the operations of map composition and adjoint operation.
\begin{definition}{Algebra}{algebra}
    An \emph{algebra} \(\algebra{A}\) is a linear space \(\algebra{A}\) over \(\mathbb{C}\) where there is a product \(\cdot : \algebra{A} \times \algebra{A} \to \algebra{A}\) satisfying
\begin{enumerate}[label=(\alph*)]
    \item Distributivity with respect to the vector addition:
        \begin{equation*}
            a\cdot(b + c) = a\cdot b + a\cdot c
            \quad\text{and}\quad
            (b + c) \cdot a = b \cdot a + c \cdot a
        \end{equation*}
        for all \(a, b, c \in \algebra{A}\); and
    \item Compatibility with scalar multiplication:
        \begin{equation*}
            \alpha(a \cdot b) = (\alpha a)\cdot b = a \cdot (\alpha b),
        \end{equation*}
        for all \(a, b \in \algebra{A}\) and \(\alpha \in \mathbb{C}\).
\end{enumerate}
If the product is associative, that is,
\begin{equation*}
    a \cdot (b \cdot c) = (a \cdot b) \cdot c
\end{equation*}
for all \(a, b, c \in \algebra{A}\), we say \(\algebra{A}\) is an associative algebra and we denote the product simply by juxtaposition with no ambiguity.

If the product is commutative, that is,
\begin{equation*}
    a \cdot b = b \cdot a
\end{equation*}
for all \(a, b \in \algebra{A}\), we say \(\algebra{A}\) is an abelian algebra.

If there exists \(\unity \in \algebra{A}\) satisfying
\begin{equation*}
    a \cdot \unity = \unity \cdot a = a
\end{equation*}
for all \(a \in \algebra{A}\), then we say \(\algebra{A}\) is a unital algebra and call \(\unity\) an identity element.
\end{definition}

It is easy to see a unital algebra has a unique identity element.
\begin{proposition}{Unital algebra has a unique identity element}{unital_algebra_unique_identity}
    Let \(\algebra{A}\) be a unital algebra and let \(\unity \in \algebra{A}\) be an identity element. Then, \(\unity\) is the only identity element of \(\algebra{A}\).
\end{proposition}
\begin{proof}
    Let \(e \in \algebra{A}\) be an identity element of \(\algebra{A}\). Since \(\unity \in \algebra{A}\), we have \(e \cdot \unity = \unity \cdot e = \unity\). Since \(\unity\) is an identity element, we also have \(e \cdot \unity = \unity \cdot e = e\), hence \(e = \unity\).
\end{proof}

We'll henceforth consider associative algebras. For any \(n \in \mathbb{N}\), we denote \(a^n = a^{n-1}a = aa^{n-1}\), with \(a^1 = a\), for all vectors \(a\) in the algebra.
\begin{proposition}{Identities for the difference of powers}{difference_powers}
    Let \(\algebra{A}\) be an associative algebra. Then
    \begin{equation*}
        x^{n+1} - y^{n+1} = \frac12 \left[(x^n + y^n)(x - y) + (x^n - y^n)(x+y)\right]
    \end{equation*}
    and
    \begin{equation*}
        x^{n+1} - y^{n+1} = \frac12(x^n + y^n)(x-y) + \sum_{k=1}^{n-1}\frac{1}{2^{k+1}}(x^{n-k}+y^{n-k})(x-y)(x+y)^k + \frac1{2^n}(x - y)(x+y)^n
    \end{equation*}
    hold for all \(x,y \in \algebra{A}\) and all \(n \in \mathbb{N}\).
\end{proposition}
\begin{proof}
    Let \(u,v \in \algebra{A}\), then
    \begin{align*}
        \frac12\left[(u^n + v^n)(u - v) + (u^n - v^n)(u+v)\right] &= \frac12 \left[u^{n+1} - u^n v + v^n u - v^{n+1} + u^{n+1} + u^nv - v^nu -v^{n+1}\right]\\&= u^{n+1} - v^{n+1}
    \end{align*}
    for all \(n \in \mathbb{N}\).

    For ease of notation, let \(S_n(x,y)\) denote the sum in the second identity for \(x,y \in \algebra{A}\) and \(n \in \mathbb{N}\).
    Let \(M \subset \mathbb{N}\) be the set of natural numbers for which the second identity holds for all \(x,y \in \algebra{A}\). For \(n = 1\), \(S_1(x,y)\) yields the zero vector, then
    \begin{equation*}
        \frac12 (x + y)(x - y) + \frac12(x-y)(x+y) = x^2 - y^2,
    \end{equation*}
    that is, \(1 \in M\). In particular, \(M\) is non-empty. Let \(m \in M\), then from the first identity we have
    \begin{equation*}
        x^{m+2} - y^{m+2} = \frac{1}{2}\left[(x^{m+1} + y^{m+1})(x - y) + (x^{m+1} - y^{m+1})(x+y)\right]
    \end{equation*}
    for all \(x,y \in \algebra{A}\), hence
    \begin{equation*}
        \begin{split}
            x^{m+2} - y^{m+2} &= \frac12 (x^{m+1} + y^{m+1})(x - y) + \\
                              &+\frac12\left[\frac12(x^{m} + y^{m})(x-y) + S_{m}(x,y) + \frac1{2^{m}}(x - y)(x+y)^{m} \right] (x+y)\\
                              % &= \frac12 (x^m + y^m)(x - y) + \frac{1}{2^m}(x-y)(x+y)^m+\\
                              % &+ \left[\frac14 (x^{m-1} + y^{m-1})(x-y) + \frac12 S_{m-1}(x,y)\right](x+y)\\
                                &=\frac12 (x^{m+1} + y^{m+1})(x - y) + \frac{1}{2^{m+1}}(x-y)(x+y)^{m+1}+\\
                              &+\frac14 (x^{m} + y^{m})(x-y)(x+k) +  \sum_{k=1}^{m-1}\frac{1}{2^{k+2}}(x^{m-k} + y^{m-k})(x-y)(x+y)^{k+1}\\
                              &=\frac12 (x^{m+1} + y^{m+1})(x - y) + \frac{1}{2^{m+1}}(x-y)(x+y)^{m+1}+\\
                              &+\frac14 (x^{m} + y^{m})(x-y)(x+k) +  \sum_{k=2}^{m}\frac{1}{2^{k+1}}(x^{m+1-k} + y^{m+1-k})(x-y)(x+y)^{k}\\
                              &=\frac12 (x^{m+1} + y^{m+1})(x - y) + S_{m+1}(x,y) + \frac{1}{2^{m+1}}(x-y)(x+y)^{m+1},
        \end{split}
    \end{equation*}
    that is, \(m+1 \in M\). By the principle of finite induction, we have shown that \(M = \mathbb{N}\).
\end{proof}

If the underlying linear space of an algebra is equipped with a norm, we may study the algebraic structure with respect to the \emph{uniform topology}, the topology induced by the norm.
\begin{definition}{Normed algebra}{normed_algebra}
    A \emph{normed algebra} \(\algebra{A}\) is an associative algebra equipped with a norm \(\norm{\noarg} : \algebra{A} \to \mathbb{R}\) satisfying
    \begin{enumerate}[label=(\alph*)]
        \item \(\norm{ab} \leq \norm{a}\cdot\norm{b}\) for all \(a,b \in \algebra{A}\); and
        \item if \(\algebra{A}\) is unital, \(\norm{\unity} = 1\).
    \end{enumerate}
\end{definition}

In an algebra \(\algebra{A}\), if we fix \(v \in \algebra{A}\) we'll denote the restriction the of the product to either \(\set{v} \times \algebra{A}\) or \(\algebra{A} \times \set{v}\) by \(L_v\) and \(R_v\), respectively. That is, the maps
\begin{align*}
    L_v : \algebra{A} &\to \algebra{A} &
    R_v : \algebra{A} &\to \algebra{A}\\
    u &\mapsto vu&
    u &\mapsto uv,
\end{align*}
represent the left and right product by \(v\), which are clearly linear from the compatibility of the product with scalar multiplication and the distributivity over addition. The requirement of a submultiplicative norm makes the product continuous with respect to the uniform topology.
\begin{proposition}{Product is continuous with respect to the uniform topology}{product_uniform}
    Let \(\algebra{A}\) be a normed algebra. Then, for all \(v \in \algebra{A}\) the maps \(L_v\) and \(R_v\) are continuous.
\end{proposition}
\begin{proof}
    Let \(\family{w_n}{n \in \mathbb{N}} \subset \algebra{A}\) be a sequence that converges against \(\tilde{w} \in \algebra{A}\). Then, for all \(n \in \mathbb{N}\),
    \begin{equation*}
        \norm{L_v\tilde{w} - L_vw_n} = \norm{L_v(\tilde{w} - w_n)} \leq \norm{v} \norm{\tilde{w} - w_n},
    \end{equation*}
    from which we conclude \(L_vw_n \to L_v\tilde{w}\). The continuity of \(R_v\) is shown similarly.
\end{proof}

With the above definition, we may use the norm for estimating certain quantities.
\begin{lemma}{Estimates for the difference of powers}{estimate_difference_power}
    Let \(\algebra{A}\) be a normed algebra. Then,
    \begin{equation*}
        \norm{x^{n+1} - y^{n+1}}\leq \left(\norm{x} + \norm{y}\right)^n \norm{x-y}
    \end{equation*}
    for all \(n \in \mathbb{N}_0\) and all \(x,y \in \algebra{A}\). Moreover, for all \(x, y \in \algebra{A}\) with \(\norm{x} \leq 1\) and \(\norm{y} \leq 1\),
    \begin{equation*}
        \norm{x^{n+1} - y^{n+1}} \leq (n+1)\norm{x+y}
    \end{equation*}
    for all \(n \in \mathbb{N}_0\).
\end{lemma}
\begin{proof}
    Let \(S\subset \mathbb{N}_0\) be the set of integers for which the first inequality holds for all \(x, y \in \algebra{A}\) and let \(R \subset \mathbb{N}_0\) be the set analogously defined for the second inequality. Quite trivially, \(0 \in S \cap R\), so neither set is empty. Let \(k \in S\), then for all \(a \in \algebra{A}\) we have \(\norm{a^{k+1}} \leq \norm{a}^k\). Then, by \cref{prop:difference_powers} we have
    \begin{align*}
        \norm{x^{k+2} - y^{k+2}} &\leq \frac12 \norm{(x^{k+1} + y^{k+1})(x - y)} + \frac12\norm{(x^{k+1} - y^{k+1})(x+y)}\\
                                 &\leq \frac12 \norm{x^{k+1} + y^{k+1}} \norm{x - y}\\
                                 &\leq \frac12\left(\norm{x^{k+1}} + \norm{y^{k+1}}\right)\norm{x - y}\\
                                 &\leq \frac12\left(\norm{x}^{k+1} + \norm{y}^{k+1}\right)\norm{x - y},
    \end{align*}
    that is, \(k + 1 \in S\). By the principle of finite induction, we have \(S = \mathbb{N}_0\). Let \(x, y \in \setc{a \in \algebra{A}}{\norm{a} \leq 1}\). Then for all \(n \in \mathbb{N} \subset S\), \(\norm{x^n + y^n} \leq \norm{x^n} + \norm{y^n} \leq \norm{x}^n + \norm{y}^n\leq 2\) and \(\norm{x + y}^n \leq (\norm{x} + \norm{y})^n \leq 2^n\). Since \(\algebra{A}\) is a normed algebra, \cref{prop:difference_powers} yields
    \begin{equation*}
        \norm{x^{n+1} - y^{n+1}} \leq \norm{x - y} + \sum_{k=1}^{n-1} \norm{x-y} + \norm{x-y} = (n+1)\norm{x-y},
    \end{equation*}
    hence \(R = \mathbb{N}_0\).
\end{proof}

We now abstract the adjoint operation on \(\bounded(\hilbert)\) to associative algebras.
\begin{definition}{Involution and *-algebra}{involutive_algebra}
    Let \(\algebra{A}\) be an associative algebra. An \emph{involution} is a map
    \begin{align*}
        * : \algebra{A} &\to \algebra{A}\\
                      a &\mapsto a^*
    \end{align*}
    satisfying
    \begin{enumerate}[label=(\alph*)]
        \item involutivity: \((a^*)^* = a\), for all \(a \in \algebra{A}\);
        \item antidistributivity: \((ab)^* = b^* a^*\), for all \(a, b \in \algebra{A}\);
        \item antilinearity: \((\alpha a + \beta b)^* = \conj{\alpha}a^* + \conj{\beta}b^*\), for all \(a,b \in \algebra{A}\) and all \(\alpha, \beta \in \mathbb{C}\); and
        \item if \(\algebra{A}\) is unital, then \(\unity^* = \unity\).
    \end{enumerate}
    If \(\algebra{A}\) has an involution, we say \(\algebra{A}\) is a \emph{*-algebra} or \emph{involutive algebra}.
\end{definition}
\begin{remark}
    Note the antilinearity of the involution implies \(0^* = (x-x)^* = x^* - x^* = 0\), that is \(0^* = 0\).
\end{remark}
\begin{remark}
    We may denote an involution by \(\inv_{\algebra{A}} : \algebra{A} \to \algebra{A}\) when composing maps.
\end{remark}

The following example uses the adjoint operation to define another involution on \(\bounded(\hilbert)\).
\begin{example}{Involution on bounded operators of a Hilbert space}{involution_adjoint}
    Let \(\hilbert\) be a Hilbert space and let \(d \in \bounded(\hilbert)\) such that \(d\) is self-adjoint and unitary. The map
    \begin{align*}
        \dag : \bounded(\hilbert) &\to \bounded(\hilbert)\\
                                a &\mapsto a^\dag,
    \end{align*}
    where \(a^\dag = d^* a^* d\) defines an involution on \(\bounded(\hilbert)\), where \(a^*\) is the adjoint operator of \(a\).
\end{example}
\begin{proof}
    Let \(a \in \bounded(\hilbert)\), then
    \begin{equation*}
        (a^\dag)^\dag = d^* (a^\dag)^* d = d^* (d^* a^* d)^* d = d^* d^* a d d = a,
    \end{equation*}
    since \(d\) is unitary and self-adjoint, \(d^* d = d d^* = \unity\) and \(d^* = d\). Let \(b \in \bounded(\hilbert)\), then
    \begin{equation*}
        (ab)^\dag = d^*(ab)^* d = d^* b^* a^* d = d^* b^* d d^* a^* d = b^\dag a^\dag.
    \end{equation*}
    Let \(\alpha, \beta \in \mathbb{C}\), then
    \begin{equation*}
        (\alpha a + \beta b)^\dag = d^* (\alpha a + \beta b)^* d = \conj{\alpha} d^* a^* d + \conj{\beta} d^* b^* d = \conj{\alpha} a^\dag + \conj{\beta} b^\dag.
    \end{equation*}
    Finally, we have
    \begin{equation*}
        \unity^\dag = d^* \unity d = d^* d = \unity,
    \end{equation*}
    hence \(\dag\) is an involution on \(\bounded(\hilbert)\).
\end{proof}

Our main object of study will be involutive algebras, in particular the ones which are complete with respect to the norm.
\begin{definition}{Banach algebras and C*-algebras}{c_algebra}
    A Banach algebra \(\algebra{A}\) is a normed algebra that is a complete metric space with respect to its norm. If \(\algebra{A}\) has an involution that satisfies \(\norm{a} = \norm{a^*}\) for all \(a \in \algebra{A}\), we say \(\algebra{A}\) is a Banach *-algebra. If, in addition, the norm and the involution satisfy \(\norm{a^*a} = \norm{a}^2\) for all \(a \in \algebra{A}\), then \(\algebra{A}\) is a C*-algebra.
\end{definition}
\begin{remark}
    It should be clear the C*-property implies \(\norm{a} = \norm{a^*}\) for all \(a \in \algebra{A}\). Indeed, we have
    \begin{equation*}
        \norm{a}^2 = \norm{a^*a} \leq \norm{a^*}\norm{a}
        \quad\text{and}\quad
        \norm{a^*}^2 = \norm{aa^*} \leq \norm{a}\norm{a^*},
    \end{equation*}
    from which we conclude \(\norm{a} = \norm{a^*}\) for all \(a \in \algebra{A}\).
\end{remark}
The requirement that \(\norm{a^*} = \norm{a}\) for all \(a \in \algebra{A}\) guarantees that the involution is a continuous map in the uniform topology.
\begin{proposition}{Involution is continuous with respect to the uniform topology}{involution_uniform}
    Let \(\algebra{A}\) be a Banach *-algebra. Then the involution is continuous in the uniform topology.
\end{proposition}
\begin{proof}
    Let \(\family{a_n}{n \in \mathbb{N}} \subset \algebra{A}\) be a sequence that converges against \(\tilde{a} \in \algebra{A}\). Then
    \begin{equation*}
        \norm{\tilde{a}^* - a_n^*} = \norm{\left(\tilde{a} - a_n\right)^*} = \norm{\tilde{a} - a_n}
    \end{equation*}
    for all \(n \in \mathbb{N}\). Hence \(a_n^* \to \tilde{a}^*\) shows continuity of the involution.
\end{proof}

The C*-property implies the maps \(a \mapsto L_a\) and \(a \mapsto R_a\) are isometries.
\begin{proposition}{Norm in a C*-algebra}{norm_cstar}
    Let \(\algebra{A}\) be a C*-algebra. Then
    \begin{equation*}
        \norm{a} = \sup_{\substack{d \in \algebra{A}\\\norm{d} = 1}}{\norm{da}}= \sup_{\substack{d \in \algebra{A}\\\norm{d} = 1}}{\norm{ad}}
    \end{equation*}
    for all \(a \in \algebra{A}\)
\end{proposition}
\begin{proof}
    The identities hold trivially for \(0\in \algebra{A}\). Let \(a \in \algebra{A} \setminus \set{0}\), then for all \(d \in \algebra{A}\) we have \(\norm{ad} \leq \norm{a} \norm{d}\) and \(\norm{da} \leq \norm{a} \norm{d}\), hence
    \begin{equation*}
        \sup_{\substack{d \in \algebra{A}\\\norm{d} = 1}}{\norm{da}} \leq \norm{a}
        \quad\text{and}\quad
        \sup_{\substack{d \in \algebra{A}\\\norm{d} = 1}}{\norm{ad}} \leq \norm{a}.
    \end{equation*}
    We also have
    \begin{equation*}
        \sup_{\substack{d \in \algebra{A}\\\norm{d} = 1}}{\norm{da}} \geq \norm{\tilde{d}a}
        \quad\text{and}\quad
        \sup_{\substack{d \in \algebra{A}\\\norm{d} = 1}}{\norm{ad}} \geq \norm{a\tilde{d}}.
    \end{equation*}
    for any \(\tilde{d} \in \setc{d \in \algebra{A}}{\norm{d} = 1}.\) In particular, this means that
    \begin{equation*}
        \sup_{\substack{d \in \algebra{A}\\\norm{d} = 1}}{\norm{da}} \geq \norm*{\left(\frac{1}{\norm{a}}a^*\right)a} = \norm{a}
        \quad\text{and}\quad
        \sup_{\substack{d \in \algebra{A}\\\norm{d} = 1}}{\norm{ad}} \geq \norm*{a\left(\frac{1}{\norm{a}}a^*\right)} = \norm{a},
    \end{equation*}
    since \(\norm{aa^*} = \norm{a}^2 = \norm{a^*a}\).
\end{proof}
\begin{corollary}
    The map
    \begin{align*}
        L : \algebra{A} &\to \bounded(\algebra{A})\\
        a &\mapsto L_a
    \end{align*}
    is an isometric homomorphism and the map
    \begin{align*}
        R : \algebra{A} &\to \bounded(\algebra{A})\\
        a &\mapsto R_a
    \end{align*}
    is an isometric anti-homomorphism.
\end{corollary}
\begin{proof}
    The previous result shows \(\norm{L_a}_{\bounded(\algebra{A})} = \norm{a}_{\algebra{A}} = \norm{R_a}_{\bounded(\algebra{A})}\) for all \(a \in \algebra{A}\), hence \(L\) and \(R\) are isometric maps. Let \(\alpha, \beta \in \mathbb{C}\) and \(x, y \in \algebra{A}\), then for all \(z \in \algebra{A}\) we have
    \begin{equation*}
        L_{\alpha x + \beta y}z = (\alpha x + \beta y)z = \alpha L_xz + \beta L_yz = \left(\alpha L_x + \beta L_y\right)z,
    \end{equation*}
    \begin{equation*}
        R_{\alpha x + \beta y}z = z(\alpha x + \beta y) = \alpha R_xz + \beta R_yz = \left(\alpha R_x + \beta R_y\right)z,
    \end{equation*}
    \begin{equation*}
        L_x \circ L_y z = L_xyz =xyz = L_{xy}z,
    \end{equation*}
    and
    \begin{equation*}
        R_x \circ R_y z = R_xzy = zyx = R_{yx}z,
    \end{equation*}
    thus \(\alpha L_x + \beta L_y = L_{\alpha x + \beta y}\), \(\alpha R_x + \beta R_y = R_{\alpha x + \beta y}\), \(L_x \circ L_y = L_{xy}\), and \(R_x \circ R_y = R_{yx}\).
\end{proof}

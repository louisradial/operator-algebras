% vim: spl=en_us
\section{Construction of representations}
As an application of the Hahn-Banach theorem, we can show the existence of states and, thus, of positive linear functionals on a C*-algebra.
\begin{proposition}{Existence of states}{states_existence}
    Let \(\algebra{A}\) be a C*-algebra and let \(a \in \algebra{A}\) be an operator. There exists a state \(\omega_a \in P_{\algebra{A}}\) such that \(\omega_a(a^*a) = \norm{a}^2\).
\end{proposition}
\begin{remark}
    In fact, such a state is a pure state.
\end{remark}
\begin{proof}
    If \(\algebra{A}\) does not have an identity, adjoin one. Consider the linear subspace \(B = \lspan\set{\unity, a^*a}\) of normal operators and its linear functional
    \begin{align*}
        f : B &\to \mathbb{C}\\
        \alpha \unity + \beta a^*a &\mapsto \alpha + \beta \norm{a}^2,
    \end{align*}
    which satisfies \(f(a^*a) = \norm{a}^2\) and \(f(\unity) = 1.\) Let \(\alpha \unity + \beta a^*a \in B\), then
    \begin{equation*}
        \abs{f(\alpha \unity + \beta a^*a)} = \abs{\alpha + \beta \norm{a}^2} \leq \sup_{\lambda \in \sigma(a)}{\abs{\alpha + \beta \lambda}} = r(\alpha \unity + \beta a^*a) = \norm{\alpha \unity + \beta a^*a},
    \end{equation*}
    that is, there exists \(m \in [0,1]\) such that \(\abs{f(b)}\leq m \norm{b}\) for all \(b \in B\). The \nameref{thm:Hahn_Banach_normed} guarantees the existence of a bounded linear functional \(\omega : \algebra{A} \to \mathbb{C}\) that extends \(f\) with \(\norm{\omega} = m\). As \(\omega(\unity) = f(\unity) = 1\) and \(\omega\) is continuous, then \(\norm{\omega} = 1\) and \(\omega\) is a state by \cref{prop:state_continuous}.
\end{proof}

Moreover, given a representation of a C*-algebra, there exists a \emph{vector state} which uses the structure of the representation. Its definition is shown in the
\begin{proposition}{States defined by a representation of a C*-algebra}{vector_state}
    Let \(\algebra{A}\) be a C*-algebra. If \((\hilbert, \pi)\) is a representation and \(\Omega \in \hilbert\) is a non-zero vector, then
    \begin{align*}
        \omega_{\Omega} : \algebra{A} &\to \mathbb{C}\\
                                    a &\mapsto \inner{\Omega}{\pi(a)\Omega}
    \end{align*}
    is a positive linear functional. This linear functional is a state if \(\norm{\Omega} = 1\) and \(\pi\) is non-degenerate, called a \emph{vector state}. 
\end{proposition}
\begin{proof}
    It is clear that \(\omega_{\Omega}\) is a linear functional, but it is positive since
    \begin{equation*}
        \omega_{\Omega}(a^*a) = \inner{\Omega}{\pi(a^*a)\Omega} = \inner{\pi(a)\Omega}{\pi(a)\Omega} = \norm{\pi(a)\Omega}^2 \geq 0
    \end{equation*}
    for all \(a \in \algebra{A}\). If \(e : \Lambda \to \algebra{A}_+\) is an approximate identity on \(\algebra{A}\) we have by \cref{prop:state_continuous} that
    \begin{equation*}
        \norm{\omega_\Omega} = \lim_{\lambda \in \Lambda}{\omega_\Omega(e_\lambda^2)} = \lim_{\lambda \in \Lambda}{\norm{\pi(e_{\lambda})\Omega}^2}.
    \end{equation*}
    If \(\pi\) is non-degenerate we know by \cref{prop:representation_approximate_identity} that \(\pi \circ e\) is an approximate identity in \(\bounded(\hilbert)\) with respect to the strong operator topology, hence \(\pi(e_\lambda)\Omega\) converges against \(\Omega\). Hence \(\omega_\Omega\) is a state if \(\pi\) is non degenerate and \(\norm{\Omega} = 1\).
\end{proof}

We have not shown the existence of representations, and we will prove by concluding every state of a C*-algebra is a vector state of some cyclic representation. 
\begin{theorem}{Canonical cyclic representation}{gns_construction}
    Let \(\algebra{A}\) be a C*-algebra. If \(\omega \in E_\algebra{A}\) is a state, then there exists a cyclic representation \((\hilbert_\omega, \pi_\omega, \Omega_\omega)\) such that 
    \begin{equation*}
        \omega(a) = \inner{\Omega_\omega}{\pi_{\omega}(a)\Omega_\omega}_{\hilbert_{\omega}}
    \end{equation*}
    for all \(a \in \algebra{A}\). Up to unitary equivalence, this cyclic representation is unique.
\end{theorem}
\begin{remark}
    This representation is referred to as \emph{GNS construction}, due to Gelfand, Naimark, and Segal. The triple \((\hilbert_{\omega}, \pi_{\omega}, \Omega_{\omega})\) is referred to as the \emph{GNS triple} associated with the state \(\omega\) over the C*-algebra \(\algebra{A}\).
\end{remark}
\begin{proof}
    We consider the set \(\algebra{J}_\omega = \setc{a \in \algebra{A}}{\omega(a^*a) = 0} \subset \setc{a \in \algebra{A}}{\forall b \in \algebra{A} : \omega(b^*a) = 0}\). Let \(a \in \algebra{J}_\omega\) and \(b \in \algebra{A}\), then \(\abs{\omega(b^*a)}^2 \leq \omega(a^*a) \omega(b^*b) = 0,\) hence \(\omega(b^*a) = 0\) and we conclude
    \begin{equation*}
        \algebra{J}_\omega = \setc{a \in \algebra{A}}{\forall b \in \algebra{A} : \omega(b^*a) = 0}.
    \end{equation*}
    Let \(a, b \in \algebra{J}_\omega\) and \(\mu \in \mathbb{C}\), then
    \begin{equation*}
        \forall c \in \algebra{A} : \omega(c^*a) + \mu \omega(c^*b) = 0 \implies \forall c \in \algebra{A} : \omega(c^*(a + \mu b)) = 0 \implies a + \mu b \in \algebra{J}_\omega,
    \end{equation*}
    hence \(\algebra{J}_\omega\) is a linear subspace of \(\algebra{A}\). Let \(a : \mathbb{N} \to \algebra{J}_\omega\) be a convergent sequence to some \(\tilde{a} \in \algebra{A},\) then
    \begin{equation*}
        \omega(b^*\tilde{a}) = \lim_{n \to \infty}{\omega(b^*a_n)} = 0
    \end{equation*}
    for all \(b \in \algebra{A},\) hence \(\tilde{a} \in \algebra{J}_\omega\) and we conclude \(\algebra{J}_\omega\) is closed. Furthermore, \(\algebra{J}_\omega\) is a left ideal of \(\algebra{A}\) since 
    \begin{equation*}
        \omega((ba)^*(ba)) = \omega((a^*b^*b)a) = 0
    \end{equation*}
    for all \(b \in \algebra{A}\) and \(a \in \algebra{J}_\omega\). 

    We now consider the linear space on the coset \(\algebra{A}/\algebra{J}_\omega\). Let \([a], [b] \in \algebra{A}/\algebra{J}_\omega\) and let \(\tilde{a} \in [a]\) and \(\tilde{b} \in [b]\), then there exist \(j_a, j_b \in \algebra{J}_\omega\) such that \(\tilde{a} = a + j_a\) and \(\tilde{b} = b + j_b\), hence
    \begin{equation*}
        \omega(\tilde{a}^*\tilde{b}) = \omega(\tilde{a}^*(b + j_b)) = \omega(\tilde{a}^*b) = \omega((a + j_a)^*b) = \omega(a^*b).
    \end{equation*}
    We have thus shown the map
    \begin{align*}
        \inner{\noarg}{\noarg} : \algebra{A}/\algebra{J}_\omega \times \algebra{A}/\algebra{J}_\omega &\to \mathbb{C}\\
        ([a],[b]) &\mapsto \omega(a^*b)
    \end{align*}
    is well-defined. We have already established this is a positive sesquilinear form, so \((\algebra{A}/\algebra{J}_\omega, \inner{\noarg}{\noarg})\) is an inner product space as we have
    \begin{equation*}
        \inner{[a]}{[a]} = 0 \implies \omega(a^*a) = 0 \implies a \in \algebra{J}_\omega \implies [a] = [0].
    \end{equation*}
    As \(\algebra{A}/\algebra{I}_{\omega}\) is a pre-Hilbert space, we let \(\hilbert_\omega\) be its canonical completion, which is unique up to isomorphism, and let
    \begin{align*}
        \Psi : \algebra{A}/\algebra{I}_\omega &\to \hilbert_\omega\\
                                          [a] &\mapsto \Psi_{[a]}
    \end{align*}
    be the linear isometry whose image is dense in \(\hilbert_\omega\).

    Let us consider the action of \(\algebra{A}\) in \(\Psi(\algebra{A}/\algebra{I}_\omega)\) with the map
    \begin{align*}
        p : \algebra{A} \times \Psi(\algebra{A}/\algebra{I}_\omega) &\to \Psi(\algebra{A}/\algebra{I}_\omega)\\
        \left(a, \Psi_{[b]}\right) &\mapsto \Psi_{[ab]},
    \end{align*}
    which we denote by \(p_a\Psi_{[b]} = \Psi_{[ab]}\), and is well-defined since \(\algebra{I}_\omega\) is a left ideal. Indeed, let \(\tilde{b} \in [b]\), then \(\Psi_{[\tilde{b}]} = \Psi_{[b]}\) and \(a(b - \tilde{b}) \in \algebra{I}_\omega\), hence \(\Psi_{[ab]} = \Psi_{[a\tilde{b}]}\). Notice for all \(a \in \algebra{A}\) the map \(p_a : \Psi(\algebra{A}/\algebra{I}_\omega) \to \Psi(\algebra{A}/\algebra{I}_\omega)\) is linear, as we have \(p_a(0) = p_a(\Psi_{[0]}) = \Psi_{[0]} = 0\) and
    \begin{equation*}
        p_a(\Psi_{[b_1]} + \beta \Psi_{[b_2]}) = p_a(\Psi_{[b_1 + \beta b_2]}) = \Psi_{[ab_1 + \beta ab_2]} = \Psi_{[ab_1]} + \beta \Psi_{[ab_2]} = p_a(\Psi_{[b_1]}) + \beta p_a(\Psi_{[b_2]}),
    \end{equation*}
    for all \([b_1], [b_2] \in \algebra{A}/\algebra{I}_\omega\) and \(\beta \in \mathbb{C}\), as \(\Psi\) is linear. Moreover, \(p_a\) is bounded for all \(a \in \algebra{A}\) as we have
    \begin{equation*}
        \norm{p_a\Psi_{[b]}}^2 = \norm{\Psi_{[ab]}}^2 = \norm{[ab]}^2 = \omega(b^*a^*ab) \leq \omega(b^*b) \norm{a}^2 = \norm{a}^2 \norm{[b]}^2 = \norm{a}^2\norm{\Psi_{[b]}}^2
    \end{equation*}
    for all \(\Psi_{[b]} \in \Psi(\algebra{A}/\algebra{I}_\omega)\). We thus define the map
    \begin{align*}
        \pi_{\omega} : \algebra{A} &\to \bounded(\hilbert_\omega)\\
                                 a &\mapsto \pi_\omega(a),
    \end{align*}
    where \(\pi_\omega(a)\) is the unique bounded linear extension of \(p_a\) to \(\hilbert_\omega\) defined in \cref{thm:blt}, that is,
    \begin{align*}
        \pi_\omega(a) : \hilbert_\omega &\to \hilbert_\omega\\
                                   \phi &\mapsto \lim_{n\to\infty}{p_a(\phi_n)},
    \end{align*}
    where \(\family{\phi_n}{n\in \mathbb{N}} \subset \Psi(\algebra{A}/\algebra{I}_\omega)\) is any sequence that converges against \(\phi \in \hilbert_\omega\). Let \(a_1, a_2 \in \algebra{A}\), \(\mu\in \mathbb{C}\) and \(\phi, \psi \in \hilbert_\omega\), then there exists sequences \([b], [c]: \mathbb{N} \to \algebra{A}/\algebra{I}_\omega\) such that \(\Psi \circ [b]\) and \(\Psi \circ [c]\) converge against \(\phi\) and \(\psi\), and we have
    \begin{align*}
        \pi_\omega(a_1 + \mu a_2)\phi &= \lim_{n\to\infty}{p_{a_1 + \mu a_2}(\Psi_{[b_n]})}\\
                                          &= \lim_{n\to\infty}{\Psi_{[(a_1 + \mu a_2)b_n]}}\\
                                          &= \lim_{n\to\infty}{\left(\Psi_{[a_1b_n]} + \mu \Psi_{[a_2 b_n]}\right)}\\
                                          &= \lim_{n\to\infty}{p_{a_1}\Psi_{[b_n]}} + \mu \lim_{n\to\infty}{p_{a_2}\Psi_{[b_n]}}\\
                                          &= \pi_\omega(a_1)\phi + \mu\pi_\omega(a_2)\phi,
    \end{align*}
    \begin{align*}
        \pi_\omega(a_1)\pi_\omega(a_2)\phi &= \pi_\omega(a_1) \lim_{n\to\infty}{p_{a_2}\Psi_{[b_n]}}\\
                                           &= \lim_{n\to\infty}{\pi_\omega(a_1)\Psi_{[a_2b_n]}}\\
                                           &= \lim_{n\to\infty}{p_{a_1}\Psi_{[a_2b_n]}}\\
                                           &= \lim_{n\to\infty}{\Psi_{[a_1a_2b_n]}}\\
                                           &= \lim_{n\to\infty}{p_{a_1a_2}\Psi_{[b_n]}}\\
                                           &= \pi_\omega(a_1a_2)\phi,
    \end{align*}
    and
    \begin{align*}
        \inner{\psi}{\pi_\omega(a_1^*)\phi} &= \lim_{n\to\infty}{\inner{\Psi_{[c_n]}}{p_{a_1^*}\Psi_{[b_n]}}}\\
                                            &= \lim_{n\to\infty}{\inner{\Psi_{[c_n]}}{\Psi_{[a_1^*b_n]}}}\\
                                            &= \lim_{n\to\infty}{\omega(c_n^*a_1^*b_n)}\\
                                            &= \lim_{n\to\infty}{\inner{\Psi_{[a_1c_n]}}{\Psi_{[b_n]}}}\\
                                            &= \lim_{n\to\infty}{\inner{p_{a_1}\Psi_{[c_n]}}{\Psi_{[b_n]}}}\\
                                            &= \inner{\pi_{\omega}(a_1)\psi}{\phi}\\
                                            &= \inner{\psi}{\pi_{\omega}(a_1)^*\phi},
    \end{align*}
    hence \(\pi_\omega\) is a *-morphism, and thus \((\hilbert_\omega, \pi_\omega)\) is a representation.

    If \(\algebra{A}\) is unital, we set \(\Omega_\omega = \Psi_{[\unity]}\), which yields
    \begin{equation*}
        \inner{\Omega_\omega}{\pi_\omega(a)\Omega_\omega} = \inner{\Psi_{[\unity]}}{\Psi_{[a]}} = \omega(\unity^*a) = \omega(a),
    \end{equation*}
    for all \(a \in \algebra{A}\) and \(\pi_\omega(\algebra{A})\Omega_\omega = \setc{\Psi_{[a]}}{a \in \algebra{A}} = \Psi(\algebra{A}/\algebra{I}_\omega)\), hence \((\hilbert_\omega, \pi_\omega, \Omega_\omega)\) is a cyclic representation. If \(\algebra{A}\) is not unital, let \((\hilbert_{\hat{\omega}},\pi_{\hat{\omega}}, \Omega_{\hat{\omega}})\) be the cyclic representation of \(\mathbb{C} \ltimes \algebra{A}\) constructed for the extended state \(\hat{\omega} : \mathbb{C} \ltimes \algebra{A} \to \mathbb{C}\) as above and let \(\imath : \algebra{A} \to \mathbb{C} \ltimes \algebra{A}\) be the *-isomorphism that lets us isometrically identify \(\algebra{A}\) with a C*-subalgebra of \(\mathbb{C} \ltimes \algebra{A}\), then \((\hilbert_{\hat{\omega}}, \pi_{\hat{\omega}} \circ \imath)\) is a representation of \(\algebra{A}\). Let \((\alpha, a) \in \mathbb{C} \ltimes \algebra{A}\), then
    \begin{equation*}
        \pi_{\hat{\omega}}(\alpha,a)\Omega_{\hat{\omega}} = \Psi_{[(\alpha,a)]} = \alpha \Psi_{[\unity]} + \Psi_{\imath(a)} = \alpha \Omega_{\hat{\omega}} + \pi_{\hat{\omega}}\circ \imath(a) \Omega_{\hat{\omega}},
    \end{equation*}
    so as \(\pi_{\hat{\omega}}(\mathbb{C}\ltimes\algebra{A})\Omega_{\hat{\omega}}\) is dense in \(\hilbert_{\hat{\omega}}\), then \(\pi_{\hat{\omega}}\circ\imath(\algebra{A}) \Omega_{\hat{\omega}}\) is dense in \(\hilbert_{\hat{\omega}}\) if \(\Omega_{\hat{\omega}}\) lies in the closure of \(\pi_{\hat{\omega}}\circ \imath(\algebra{A}) \Omega_{\hat{\omega}}\). Let \(e : \Lambda \to \algebra{A}_+\) be an approximate identity on \(\algebra{A}\) and let us denote \(\tilde{e} = \imath \circ e\), then
    \begin{align*}
        \norm{\pi_{\hat{\omega}}(\tilde{e}_{\lambda})\Omega_{\hat{\omega}} - \Omega_{\hat{\omega}}}^2 
        &= \norm{\Omega_{\hat{\omega}}}^2 + \norm{\pi_{\hat{\omega}}(\tilde{e}_{\lambda})\Omega_{\hat{\omega}}}^2 - 2 \inner{\Omega_{\hat{\omega}}}{\pi_{\hat{\omega}}(\tilde{e}_{\lambda})\Omega_{\hat{\omega}}}\\
        &= \norm{\unity}^2 + \inner{\Omega_{\hat{\omega}}}{\pi_{\hat{\omega}}(\tilde{e}_\lambda^2)\Omega_{\hat{\omega}}} - 2 \hat{\omega}(\tilde{e}_\lambda)\\
        &= 1 + \hat{\omega}(\tilde{e}_{\lambda}^2) - 2 \omega(e_\lambda)\\
        &= 1 + \omega(e_{\lambda}^2) - 2\omega(e_\lambda)
    \end{align*}
    for all \(\lambda \in \Lambda\), hence \(\pi_{\hat{\omega}}(\tilde{e}_{\lambda})\Omega_{\hat{\omega}} \to \Omega_{\hat{\omega}}\)\footnote{Since \(\pi_{\hat{\omega}}\) is a cyclic representation, we could have inferred this result from \cref{prop:representation_approximate_identity}.} and we conclude \((\hilbert_{\hat{\omega}}, \pi_{\hat{\omega}} \circ \imath, \Omega_{\hat{\omega}})\) is a cyclic representation of \(\algebra{A}\).

    Let \((\tilde{\hilbert}_{\omega}, \tilde{\pi}_{\omega}, \tilde{\Omega}_{\omega})\) be another cyclic representation of \(\algebra{A}\) and define the linear map
    \begin{align*}
        U : \pi_\omega(\algebra{A})\Omega_{\omega} &\to \tilde{\pi}_{\omega}(\algebra{A}) \tilde{\Omega}_{\omega}\\
        \pi_\omega(a)\Omega_{\omega} &\mapsto \tilde{\pi}_{\omega}(a) \tilde{\Omega}_{\omega}.
    \end{align*}
    Let \(a, b \in \algebra{A}\), then
    \begin{align*}
        \inner{U\pi_\omega(a)\Omega_{\omega}}{U\pi_\omega(b)\Omega_\omega} 
        &= \inner{\tilde{\pi}_\omega(a)\tilde{\Omega}_\omega}{\tilde{\pi}_\omega(b)\tilde{\Omega}_\omega}\\ 
        &= \inner{\tilde{\Omega}_\omega}{\tilde{\pi}_{\omega}(a^*b) \tilde{\Omega}_\omega}\\
        &= \omega(a^*b)\\
        &= \inner{\Omega_\omega}{\pi_\omega(a^*b)\Omega_\omega}\\
        &= \inner{\pi_\omega(a)\Omega_\omega}{\pi_\omega(b)\Omega_\omega},
    \end{align*}
    hence \(U\) preserves the inner product, and thus, it is a linear isometry. In particular, \(U\) is bounded and densely defined in \(\hilbert_\omega\), hence the bounded linear map
    \begin{align*}
        \hat{U} : \hilbert_\omega &\to \tilde{\hilbert}_\omega\\
                             \psi &\mapsto \lim_{n\to\infty} U \psi_n,
    \end{align*}
    where \(\family{\psi_n}{n\in \mathbb{N}} \subset \pi_\omega(\algebra{A})\Omega_\omega\) is any sequence that converges against \(\psi,\) is well defined. Let \(\psi \in \hilbert_\omega\) and let \(a : \mathbb{N} \to \algebra{A}\) be a sequence such that \(\pi_\omega(a_n)\Omega_\omega \to \psi\), then
    \begin{align*}
        \norm{\hat{U}\psi}
        &= \norm*{\lim_{n\to\infty}{U \pi_{\omega}(a_n)\Omega_\omega}}\\
        &= \lim_{n\to\infty}{\norm{U\pi_\omega(a_n)\Omega_\omega}}\\
        &= \lim_{n\to\infty}{\norm{\pi_{\omega}(a_n)\Omega_\omega}}\\
        &= \norm*{\lim_{n\to\infty}{\pi_\omega(a_n)\Omega_\omega}}\\
        &= \norm{\psi},
    \end{align*}
    hence \(\hat{U}\) is an isometry. Let \(\tilde{\phi} \in \tilde{\hilbert}_\omega,\) and let \(b : \mathbb{N} \to \algebra{A}\) be a sequence such that \(\tilde{\pi}_\omega(b_n)\tilde{\Omega}_\omega \to \tilde{\phi}\), then \(U\pi_\omega(b_n)\Omega_\omega \to \tilde{\phi}\), and as \(U\) is an isometry, it follows that \(\pi_\omega(b_n)\Omega_\omega\) is a Cauchy sequence, thus convergent against some \(\phi \in \hilbert_\omega\), which satisfies \(\hat{U} \phi = \tilde{\phi},\) hence \(\hat{U}\) is surjective. By \cref{thm:unitary_iff}, we conclude \(\hat{U}\) is a unitary map. Let \(e : \Lambda \to \algebra{A}_+\) be an approximate identity on \(\algebra{A}\), then \(\pi_\omega \circ e\) and \(\tilde{\pi}_\omega \circ e\) are approximate identities on \(\hilbert_\omega\) and \(\tilde{\hilbert}_\omega\) with respect to the strong operator topology, hence 
    \begin{equation*}
        \hat{U} \Omega_\omega = \lim_{\lambda \in \Lambda}{U \pi_\omega(e_\lambda) \Omega_\omega =}\lim_{\lambda \in \Lambda}{\tilde{\pi}_\omega(e_\lambda)\tilde{\Omega}_\omega}= \tilde{\Omega}_\omega.
    \end{equation*}
    Let \(\tilde{\phi} \in \tilde{\hilbert}_\omega\), then there exists a sequence \(b : \mathbb{N} \to \algebra{A}\) such that \(\tilde{\pi}_\omega(b_n)\tilde{\Omega}_\omega \to \tilde{\phi}\) and hence
    \begin{align*}
        \tilde{\pi}_\omega(a)\tilde{\phi} &= \lim_{n\to\infty}{\tilde{\pi}_{\omega}(a)\tilde{\pi}_\omega(b_n)\tilde{\Omega}_\omega}\\
                                          &= \lim_{n\to\infty}{\tilde{\pi}_{\omega}(ab_n)\tilde{\Omega}_\omega}\\
                                          &= \lim_{n\to\infty}{\hat{U}\pi_{\omega}(ab_n)\Omega_\omega}\\
                                          &= \hat{U}\pi_{\omega}(a)\hat{U}^*\lim_{n\to\infty}{\hat{U}\pi_{\omega}(b_n)\Omega_\omega}\\
                                          &= \hat{U}\pi_\omega(a)\hat{U}^* \lim_{n\to\infty}{\tilde{\pi}_\omega(b_n)\tilde{\Omega}_\omega}\\
                                          &= \hat{U}\pi_\omega(a)\hat{U}^* \tilde{\phi},
    \end{align*}
    for all \(a \in \algebra{A}\). We have thus shown \((\hilbert_\omega,\pi_\omega, \Omega_\omega)\) and \((\tilde{\hilbert}_\omega, \tilde{\pi}_\omega, \tilde{\Omega}_\omega)\) are unitarily equivalent, as there exists a unitary map \(\hat{U} : \hilbert_\omega \to \tilde{\hilbert}_\omega\) that satisfies \(\tilde{\pi}_\omega(a) = \hat{U}\pi_\omega(a)\hat{U}^*\) for all \(a \in \algebra{A}\) and \(\tilde{\Omega}_\omega = \hat{U}\Omega_\omega\).
\end{proof}

\begin{corollary}
    Let \(\algebra{A}\) be a C*-algebra and let \(\omega \in E_\algebra{A}\) be a state over \(\algebra{A}\). If \(\tau : \algebra{A} \to \algebra{A}\) is a *-automorphism such that \(\omega \circ \tau = \omega\), then there exists a unique unitary operator \(U_\omega : \hilbert_\omega \to \hilbert_\omega\) such that
    \begin{equation*}
        U_\omega \pi_\omega(a) U_\omega^* = \pi_\omega\circ\tau(a)
    \end{equation*}
    for all \(a \in \algebra{A}\) and \(U_\omega\Omega_\omega = \Omega_\omega\), where \((\hilbert_\omega, \pi_\omega, \Omega_\omega)\) is the cyclic representation of \(\algebra{A}\) constructed from \(\omega\).
\end{corollary}
\begin{proof}
    Notice \((\hilbert_\omega, \pi_\omega \circ \tau, \Omega_\omega)\) is another cyclic representation with \(\pi_\omega \circ \tau(\algebra{A})\Omega_\omega = \pi_\omega(\algebra{A})\Omega_\omega\) and
    \begin{equation*}
        \inner{\Omega_\omega}{\pi_\omega(\tau(a)) \Omega_\omega} = \omega(\tau(a)) = \omega(a),
    \end{equation*}
    hence the corollary follows from the uniqueness of the canonical cyclic representation.
\end{proof}

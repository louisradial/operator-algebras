% vim: spl=en_us
\section{Square root of operators in Banach algebras}
Consider the function \(f : \mathbb{C} \to \mathbb{C}\) defined by \(f(z) = \sqrt{1 - z}\), which is analytic in the unit disk with
\begin{equation*}
    f(z) = 1 - \frac12z - \sum_{n = 2}^\infty \frac{(2n - 3)!!}{(2n)!!} z^n,
\end{equation*}
for all \(\abs{z} < 1\).
\begin{lemma}{Property of Taylor coefficients for \(z \mapsto \sqrt{1 - z}\) in the unit disk}{coefficients_square_root}
    Denote the Taylor coefficients of the map \(z \mapsto \sqrt{1-z}\) with \(\abs{z} < 1\) by
    \begin{equation*}
        c_0 = 1,\quad c_1 = -\frac12,\quad\text{and}\quad c_{n+1} = -\frac{(2n-1)!!}{(2n+2)!!}
    \end{equation*}
    for all \(n \in \mathbb{N}\). Then
    \begin{equation*}
        \sum_{k = 0}^\infty \abs{c_k} \leq 2
    \end{equation*}
    and the Taylor series for \(z \mapsto \sqrt{1-z}\) is absolutely convergent for all \(\abs{z} \leq 1\). Moreover,
    \begin{equation*}
        \sum_{\substack{k + \ell = m\\k,\ell \in \mathbb{N}_0}}{c_k c_\ell} = 0
    \end{equation*}
    for all \(m \geq 2\).
\end{lemma}
\begin{proof}
    Notice \(\abs{c_k} = - c_k\) for all \(k \in \mathbb{N}\), then
    \begin{equation*}
        \sum_{k = 0}^n (\abs{c_k} + c_k) = 2c_0 = 2 \implies \sum_{k = 0}^n \abs{c_k} = 2 - \sum_{k = 0}^n c_k
    \end{equation*}
    for all \(n \in \mathbb{N}_0\). Let \(t \in (0,1)\), then
    \begin{equation*}
        \sum_{k=0}^n c_k t^k = \sqrt{1 - t} - \sum_{k = {n+1}}^\infty c_k t^k = \sqrt{1 - t} + \sum_{k = n+1} \abs{c_k}t^k \geq \sqrt{1 - t}
    \end{equation*}
    for all \(n \in \mathbb{N}_0\). We then have
    \begin{equation*}
        \sum_{k = 0}^n \abs{c_k} = 2 - \sum_{k = 0}^n c_k = 2 - \lim_{t \to 1^{-}}\sum_{k = 0}^{n} c_k t^k \leq 2 - \lim_{t \to 1^{-}} \sqrt{1 - t} = 2
    \end{equation*}
    for all \(n \in \mathbb{N}_0\), which guarantees
    \begin{equation*}
        \sum_{k = 0}^\infty \abs{c_k} \leq 2.
    \end{equation*}
    Then, for all \(z \in \mathbb{C}\) with \(\abs{z} \leq 1\), we have
    \begin{equation*}
        \sum_{k = 0}^n \abs{c_k}\abs{z}^k \leq \sum_{k = 0}^n \abs{c_k} \leq 2
    \end{equation*}
    for all \(n \in \mathbb{N}_0\), that is, \(\sum_{k = 0}^\infty c_k z^k\) is absolutely convergent for all \(\abs{z} \leq 1\).

    Let \(z \in \mathbb{C}\) with \(\abs{z} < 1\), then
    \begin{equation*}
        1 - z = \left(\sum_{k = 0}^\infty c_k z^k\right)^2 = \sum_{k = 0}^\infty \sum_{\ell = 0}^\infty c_k c_\ell z^{k + \ell} = \sum_{m = 0}^\infty \sum_{\substack{k + \ell = m\\k,\ell \in \mathbb{N}_0}} c_kc_\ell z^m = 1 - z + \sum_{m = 2}^\infty z^m\sum_{\substack{k + \ell = m\\k,\ell \in \mathbb{N}_0}}c_k c_\ell.
    \end{equation*}
    From the uniqueness of the Taylor series we must have \(\sum_{\substack{k+\ell=m\\k,\ell \in \mathbb{N}_0}} c_k c_\ell = 0\) for all \(m \geq 2\).
\end{proof}

\begin{theorem}{Square root of an operator}{square_root}
    Let \(\algebra{A}\) be a unital Banach algebra, let \(u \in \algebra{A}\) with \(\norm{u} \leq 1\), and let \(c_n\) be the coefficients defined in \cref{lem:coefficients_square_root} with \(n \in \mathbb{N}_0\). There exists \(v \in \algebra{A}\) such that \(v^2 = \unity - u\). The operator defined by the convergent series
    \begin{equation*}
        v = \unity + \sum_{n = 1}^\infty c_n u^n
    \end{equation*}
    satisfies \(v^2 = \unity - u\).
\end{theorem}
\begin{proof}
    Let \(\family{v_n}{n \in \mathbb{N}}\subset \algebra{A}\) and \(\family{s_n}{n \in \mathbb{N}}\) be the sequences defined by the partial sums
    \begin{equation*}
        v_n = \unity + \sum_{k=1}^n c_k u^k\quad\text{and}\quad s_n = \sum_{k = 0}^n \abs{c_k}
    \end{equation*}
    for all \(n \in \mathbb{N}\). \cref{lem:coefficients_square_root} guarantees the sequence \(s_n\) is convergent, as it is a bounded and monotonic sequence of real numbers, hence it is a Cauchy sequence. Then, for all \(n,m \in \mathbb{N}\) with \(n \geq m\) we have
    \begin{equation*}
        \norm{v_n - v_m} = \norm*{\sum_{k=m+1}^n c_k u^k} \leq \sum_{k = m+1}^n \abs{c_k} \norm{u^k} \leq \sum_{k = m+1}^n \abs{c_k} = \abs{s_n - s_m},
    \end{equation*}
    that is, \(v_n\) is a Cauchy sequence. By completeness, there exists \(v \in \algebra{A}\) such that \(v_n \to v\).

    Let us write \(N_p = \setc{j \in \mathbb{N}_0}{j \leq p}\) for any \(p \in \mathbb{N}_0\) and use the convention \(u^0 = \unity\). Then
    \begin{equation*}
        v_n^2 = \sum_{k = 0}^n \sum_{\ell = 0}^n c_k c_\ell u^{k + \ell} = \sum_{m = 0}^{2n}u^m\sum_{\substack{k + \ell = m\\k, \ell \in N_n}} c_k c_\ell
    \end{equation*}
    for all \(n \in \mathbb{N}\) and, in particular, for \(n \geq 2\) we have
    \begin{align*}
        v_n^2 &= \unity - u + \sum_{m = 2}^{2n} u^m \sum_{\substack{k + \ell = m\\k, \ell \in N_n}}c_k c_\ell + \sum_{m = n + 1}^{2n} u^m \sum_{\substack{k + \ell = m\\k, \ell \in N_n}} c_k c_\ell\\
              &= \unity - u + \sum_{m = 2}^n u^m \sum_{\substack{k + \ell = m\\k,\ell \in \mathbb{N}_0}} c_k c_\ell + \sum_{m = n+1}^{2n} u^m \sum_{\substack{k + \ell = m\\k,\ell \in N_n}} c_k c_\ell\\
              &= \unity - u + \sum_{m = n+1}^{2n} u^m \sum_{\substack{k + \ell = m\\k,\ell \in N_n}} c_k c_\ell
    \end{align*}
    by \cref{lem:coefficients_square_root}. Notice we have for \(n + 1 \leq m \leq 2n\) and \(n \geq 2\) that
    \begin{equation*}
        \sum_{\substack{k + \ell = m\\k,\ell \in N_n}} c_k c_\ell = \sum_{k = m - n}^n c_k c_{m - k} = \sum_{k = 0}^n c_k c_{m - n} - \sum_{k = 0}^{m - n - 1} c_k c_{m - k} = \sum_{\substack{k + \ell = m\\k, \ell \in \mathbb{N}_0}} c_k c_\ell - \sum_{k = 0}^{m - n - 1} c_k c_{m - k} = - \sum_{k = 0}^{m - n - 1} c_k c_{m - k},
    \end{equation*}
    then
    \begin{align*}
        \norm{v_n^2 - (\unity - u)} &\leq \sum_{m = n + 1}^{2n} \norm{u}^m \abs*{\sum_{k = 0}^{m - n - 1} c_k c_{m-k}}\\
                                    &\leq \sum_{m = 0}^{n-1} \sum_{k = 0}^{m} \abs{c_k}\abs{c_{m+n+1-k}}\\
                                    &= \sum_{k = 0}^{n-1} \abs{c_k} \sum_{m = k}^{n-1} \abs{c_{m+n+1-k}}\\
                                    &= \sum_{k = 0}^{n-1} \abs{c_k} \sum_{m = n+1}^{2n - k} \abs{c_m}\\
                                    &\leq \sum_{k = 0}^{n-1} \abs{c_k} \sum_{m = n+1}^{2n} \abs{c_m}\\
                                    &\leq 2 \sum_{m = n+1}^{2n} \abs{c_m}\\
                                    &= 2 \abs{s_{2n} - s_{n+1}}.
    \end{align*}
    Since \(s_n\) is Cauchy, we may take \(n\) sufficiently big such that the right hand side becomes arbitrarily small, that is, \(v_n^2\) converges against \(\unity - u\). From the continuity of the product with respect to the uniform topology, this yields \(v^2 = \unity - u\).
\end{proof}
\begin{corollary}
    Let \(u \in \algebra{A}\) with \(\norm{\unity - u} \leq 1\), then there exists \(v \in \algebra{A}\) such that \(v^2 = u\). The operator defined by
    \begin{equation*}
        v = \unity + \sum_{n=1}^\infty c_n (\unity - u)^n
    \end{equation*}
    satisfies \(v^2 = u\).
\end{corollary}
\begin{proof}
    The operator \(\unity - u\) satisfies the hypothesis in \cref{thm:square_root}, then there exists
    \begin{equation*}
        v = \unity + \sum_{n = 1}^\infty c_n (\unity - u)^n
    \end{equation*}
    which satisfies \(v^2 = \unity - (\unity - u) = u\).
\end{proof}
\begin{corollary}
    If \(\algebra{A}\) is a unital Banach *-algebra and \(u \in \algebra{A}\) is self-adjoint with \(\norm{\unity - u} \leq 1\), then
    \begin{equation*}
        v = \unity + \sum_{n = 1}^{\infty}c_n(\unity - u)^n
    \end{equation*}
    is self adjoint.
\end{corollary}
\begin{proof}
    Let \(\family{v_n}{n \in \mathbb{N}} \subset \algebra{A}\) be defined by
    \begin{equation*}
        v_n = \unity + \sum_{k = 1}^n c_k (\unity - u)^k
    \end{equation*}
    for all \(n \in \mathbb{N}\). Since \(c_k \in \mathbb{R}\) for all \(k \in \mathbb{N}\), we have
    \begin{equation*}
        v_n^* = \unity^* + \sum_{k=1}^n c_k\left[(\unity - u)^k\right]^* = \unity + \sum_{k=1}^n c_k (\unity - u)^k = v_n,
    \end{equation*}
    for all \(n \in \mathbb{N}\), since \(\unity - u\) is self-adjoint. From the continuity of the adjoint operation, it follows that \(\displaystyle v = \lim_{n\to\infty} v_n\) is self-adjoint.
\end{proof}
\begin{corollary}
    If \(\algebra{A}\) is a unital Banach *-algebra and if \(u \in \algebra{A}\) with \(\norm{u} \leq 1\), then there exists \(v \in \algebra{A}\) self-adjoint such that \(v^2 = \unity - u^*u\).
\end{corollary}
\begin{proof}
    We have \(\norm{u^*u} \leq \norm{u^*}\norm{u} = \norm{u}^2 \leq 1\), then the result follows by \cref{thm:square_root} and by the previous corollary.
\end{proof}
\begin{corollary}
    If \(u \in \algebra{A} \setminus\set{0}\) satisfies \(\norm{\unity - \norm{u}^{-1} u} \leq 1\), then there exists \(v \in \algebra{A}\) such that \(v^2 = u\). The operator defined by
    \begin{equation*}
        v = \norm{u}^{\frac12} \left[\unity + \sum_{n = 1}^\infty c_n\left(\unity - \norm{u}^{-1}u\right)^n\right]
    \end{equation*}
    satisfies \(v^2 = u\). If, in addition, \(\algebra{A}\) is a Banach *-algebra and \(u\) is self-adjoint, then \(v\) is self-adjoint.
\end{corollary}
\begin{proof}
    Using \cref{thm:square_root} to the operator \(\norm{u}^{-1}u\) yields \(v_0 \in \algebra{A}\) defined by
    \begin{equation*}
        v_0  = \unity + \sum_{n = 1}^\infty c_n\left(\unity - \norm{u}^{-1}u\right)^n
    \end{equation*}
    satisfying \(v_0^2 = \norm{u}^{-1} u\). Then \(v = \norm{u}^{\frac12}v_0\) satisfies \(v^2 = \norm{u} v_0^2 = u\). If \(u\) is self-adjoint, then \(v_0\) is self-adjoint, and the result follows.
\end{proof}

Let \(D = \setc{a \in \algebra{A}}{\norm{a} \leq 1}\) be the unit closed disk around the zero operator. We now show the map
\begin{align*}
    \psi:D&\to \algebra{A}\\
        u &\mapsto \unity + \sum_{k = 1}^\infty c_k u^k,
\end{align*}
satisfying \(\psi(u)^2 = \unity - u\) for all \(u \in D\), is continuous with respect to the uniform topology.
\begin{proposition}{Uniform continuity of the square root of an operator}{square_root_continuous}
    Let \(\algebra{A}\) be a unital Banach algebra and let \(\psi : D \to \algebra{A}\) be as above. Then \(\psi\) is a continuous map with respect to the uniform topology.
\end{proposition}
\begin{proof}
    For convenience, we write
    \begin{equation*}
        \psi_n(w) = \unity + \sum_{k = 1}^n c_k w^k
    \end{equation*}
    for all \(n \in \mathbb{N}\) and \(w \in D\). Let \(\varepsilon > 0\), then \cref{lem:coefficients_square_root} shows us there exists \(N \in \mathbb{N}\) such that
    \begin{equation*}
        \norm*{\psi(w) - \psi_n(w)} \leq \sum_{k = n + 1}^\infty \abs{c_k} \norm{w}^k \leq \sum_{k = n + 1}^\infty \abs{c_k} < \frac13\varepsilon
    \end{equation*}
    for all \(n \geq N\) and all \(w \in D\).

    Let \(\family{u_n}{n\in \mathbb{N}} \subset D\) be a sequence that converges against \(u \in D\), then it follows from \cref{lem:estimate_difference_power} that
    \begin{equation*}
        \norm*{\psi_N(u_m) - \psi_N(u)} \leq \sum_{k = 1}^N \abs{c_k} \norm{u_m^k-u^k} \leq \sum_{k = 1}^N k\abs{c_k}\norm{u_m - u}
    \end{equation*}
    for all \(m \in \mathbb{N}\). Since \(u_n \to u\), there exists \(M \in \mathbb{N}\) such that \(\norm{u_m - u} < \frac{\varepsilon}{6N}\) for all \(m \geq M\), then
    \begin{equation*}
        \norm*{\psi_N(u_m) - \psi_N(u)} < \sum_{k = 1}^N \frac{k \varepsilon}{6N} \abs{c_k} \leq \frac16 \varepsilon\sum_{k = 1}^N \abs{c_k} \leq \frac13\varepsilon
    \end{equation*}
    for all \(m \geq M\). Then, for all \(m \geq \max\set{N, M}\) we have by the triangle inequality that
    \begin{equation*}
        \norm{\psi(u) - \psi(u_m)} \leq \norm*{\psi(u) - \psi_N(u)} + \norm*{\psi_N(u) - \psi_N(u_m)} + \norm*{\psi_N(u_m) - \psi(u_m)} < \varepsilon,
    \end{equation*}
    that is, \(\psi\) is continuous with respect to the uniform topology.
\end{proof}

% vim: spl=en_us
\section{Representations of C*-algebras}
We'll refer to *-homomorphisms between involutive algebras simply as *-morphisms. We show now a *-(iso)morphism preserves the order relation defined by positivity and that a *-morphism is always continuous with respect to the uniform topology, with a *-isomorphism even being a isometry.
\begin{proposition}{*-morphism preserves positivity}{morphism_positivity}
    Let \(\pi : \algebra{A} \to \algebra{B}\) be a *-morphism between C*-algebras \(\algebra{A}\) and \(\algebra{B}\). Then \(\pi\) preserves positivity, \(\pi(\algebra{A}_+)\subset \algebra{B}_+\) and it is continuous with respect to the uniform topology, enjoying the property \(\norm{\pi(a)} \leq \norm{a}\) for all \(a \in \algebra{A}\).
\end{proposition}
\begin{remark}
    It is clear that \(\pi(\algebra{A}_+) \subset \algebra{B}_+\) implies that if \(a \geq b\) then \(\pi(a) \geq \pi(b)\) as \(\pi\) is linear and \(\pi(a-b) \in \algebra{B}_+\).
\end{remark}
\begin{proof}
    Let \(a \in \algebra{A}_+\), then there exists \(b \in \algebra{A}\) such that \(a = b^*b\), hence \(\pi(a) = \pi(b)^* \pi(b) \in \algebra{B}_+\), that is, \(\pi(\algebra{A}_+)\subset \algebra{B}_+\).

    It is clear that if \(a \in \ker\pi\), then \(\norm{\pi(a)} \leq \norm{a}\). Recall \cref{prop:properties_order,thm:positive_cstar}, then for all \(a \in \algebra{A}\setminus\ker{\pi}\) we have
    \begin{align*}
        \norm{a^*a} a^*a \geq (a^*a)^2 &\implies \norm{a^*a} \pi(a^*a) \geq \pi(a^*a)^2\\
                                       &\implies \norm{a}^2 \pi(a^*a) \geq \pi(a^*a)^2\\
                                       &\implies \norm*{\norm{a}^2 \pi(a^*a)} \geq \norm*{\pi(a^*a)^2}\\
                                       &\implies \norm{a}^2 \norm{\pi(a^*a)} \geq \norm*{\pi(a^*a)^* \pi(a^*a)}\\
                                       &\implies \norm{a}^2 \norm{\pi(a^*a)} \geq \norm*{\pi(a^*a)}^2\\
                                       &\implies \norm{a}^2 \geq \norm{\pi(a^*a)}\\
                                       &\implies \norm{a}^2 \geq \norm{\pi(a)^* \pi(a)}\\
                                       &\implies \norm{a}^2 \geq \norm{\pi(a)}^2\\
                                       &\implies \norm{a} \geq \norm{\pi(a)},
    \end{align*}
    that is, for all \(a \in \algebra{A}\) it is true that \(\norm{a} \geq \norm{\pi(a)}\). This is enough to guarantee \(\pi\) is continuous with respect to the uniform topology.
\end{proof}
\begin{corollary}
    If, in addition, \(\pi\) is bijective, then \(\pi\) is an isometry.
\end{corollary}
\begin{proof}
    As \(\pi\) and \(\pi^{-1}\) are *-isomorphisms, then
    \begin{equation*}
        \norm{\pi(a)} \leq \norm{a} = \norm{\pi^{-1}\circ \pi(a)} \leq \norm{\pi(a)},
    \end{equation*}
    that is, \(\norm{\pi(a)} = \norm{a}\) for all \(a \in \algebra{A}\).
\end{proof}

It is an elementary result that the kernel of a group morphism \(\pi : G \to H\) is a normal subgroup of \(G\), and that the coset \(G / \ker\pi\) is isomorphic to the subgroup \(\ran\pi \subset H\). We will now show a similar result with the additional structure of C*-algebras.
\begin{proposition}{Kernel of a *-morphism is a closed two-sided ideal}{kernel_ideal}
    Let \(\pi : \algebra{A} \to \algebra{B}\) be a *-morphism between the C*-algebras \(\algebra{A}\) and \(\algebra{B}\). Then \(\ker \pi\) is a closed two-sided ideal of \(\algebra{A}\).
\end{proposition}
\begin{proof}
    As \(\pi\) is continuous, we know from \cref{prop:kernel_closed} that its kernel is a linear subspace that is closed in the uniform topology. Let \(a \in \ker\pi\) and \(b \in \algebra{A}\), then \(\pi(ab) = \pi(a)\pi(b) = 0\pi(b) = 0\) and \(\pi(ba) = \pi(b)\pi(a) = 0\), hence \(\ker\pi\) is a two-sided ideal.
\end{proof}
\begin{lemma}{Projector in a Banach *-algebra defines a unital Banach *-subalgebra}{projector_star_subalgebra}
    Let \(\algebra{A}\) be a Banach *-algebra. If \(p \in \algebra{A}\) is an orthogonal projector, then
    \begin{equation*}
        \algebra{B} = \setc{b \in \algebra{A}}{\exists a \in \algebra{A} : b = pap}
    \end{equation*}
    is a Banach *-algebra. If \(p \neq 0\), then \(p\) is the identity element for \(\algebra{B}\).
\end{lemma}
\begin{proof}
    We may assume without loss of generality that \(p \neq 0\), for if it were, then \(\algebra{B} = \set{0}\) is obviously a Banach *-algebra. The set \(\algebra{B}\) is not empty as it clearly contains at least \(0\) and \(p\). Let \(b_1,b_2 \in \algebra{B}\) and \(\alpha \in \mathbb{C}\), then there exist \(a_1,a_2 \in \algebra{A}\) such that \(b_i = pa_ip\), which yields
    \begin{equation*}
        b_1 + \alpha b_2 = p(a_1 + \alpha a_2)p \in \algebra{B},
        \quad
        b_1 b_2 = pa_1 p^2 a_2 p \in \algebra{B},
        \quad\text{and}\quad
        b_1^* = (pa_1p)^* = pa_1^*p \in \algebra{B},
    \end{equation*}
    hence \(\algebra{B}\) is a *-subalgebra of \(\algebra{A}\). Since \(p\) is idempotent, then \(p = p^3 \in \algebra{B}\) and it is the identity element of \(\algebra{B}\) as
    \begin{equation*}
        pb = p^2ap = pap = b
        \quad\text{and}\quad
        bp = pap^2 = pap = b
    \end{equation*}
    for all \(b = pap \in \algebra{B}\).

    Let \(x : \mathbb{N} \to \algebra{B}\) be a sequence of elements in \(\algebra{B}\) that converges to some \(\tilde{x} \in \algebra{A}\). \cref{prop:norm_projector} guarantees \(\norm{p} = 1\), then
    \begin{equation*}
        \norm{x_n - p\tilde{x}p} = \norm{px_np - p\tilde{x}p} =
        \norm{p(x_n - \tilde{x})p} \leq \norm{p}^2 \norm{x_n - \tilde{x}} = \norm{x_n - \tilde{x}},
    \end{equation*}
    for all \(n \in \mathbb{N}\), hence \(\tilde{x} = p \tilde{x} p \in \algebra{B}\). As a closed *-subalgebra of \(\algebra{A}\), \(\algebra{B}\) is a unital Banach *-algebra.
\end{proof}

\begin{lemma}{*-morphism maps identity to an orthogonal projector}{morphism_projector}
    Let \(\algebra{A}\) be a unital *-algebra and \(\algebra{B}\) a *-algebra. If \(\pi : \algebra{A} \to \algebra{B}\) is a *-morphism, then the following statements hold
    \begin{enumerate}[label=(\alph*)]
        \item \(p = \pi(\unity_{\algebra{A}})\) is an orthogonal projector;
        \item \(p = 0\) if and only if \(\ker{\pi} = \algebra{A}\);
        \item \(\ran{\pi}\) is a *-subalgebra contained in \(\algebra{C} = \setc{c \in \algebra{B}}{\exists b \in \algebra{B} : c = pbp}\).
    \end{enumerate}
\end{lemma}
\begin{proof}
    Notice
    \begin{equation*}
        p = \pi(\unity_\algebra{A}) = \pi(\unity_\algebra{A}^2) = \pi(\unity_\algebra{A})^2 = p^2
    \end{equation*}
    \begin{equation*}
        p^* = \pi(\unity_\algebra{A})^* = \pi(\unity_\algebra{A}^*) = \pi(\unity_\algebra{A}) = p,
    \end{equation*}
    and
    \begin{equation*}
        p = 0 \iff \forall a \in \algebra{A}: p \pi(a) = 0 \iff a \in \algebra{A} : \pi(a) = 0 \iff \ker{\pi} = \algebra{A}
    \end{equation*}
    hence \(p\) is an orthogonal projector and it is the zero operator if and only if \(\pi\) is identically zero. Clearly \(\ran{\pi}\) is a *-subalgebra of \(\algebra{B}\), so it remains to show it is contained in \(\algebra{C}\). Let \(b \in \ran\pi\), then there exists \(a \in \algebra{A}\) such that \(b = \pi(a)\). As such, we have \(b = \pi(\unity_{\algebra{A}} a \unity_{\algebra{A}}) = p\pi(a)p = pbp \in \algebra{C}\), as desired.
\end{proof}

\begin{proposition}{*-morphism is non-expansive}{morphism_non_expansive}
    Let \(\algebra{A}\) be a *-subalgebra with identity \(\unity\) of a unital Banach *-algebra, and let \(\algebra{B}\) be a C*-algebra. If \(\pi : \algebra{A} \to \algebra{B}\) is a *-morphism, then \(\pi\) is non-expansive, that is, for all \(a \in \algebra{A}\) it is true that \(\norm{\pi(a)} \leq \norm{a}\).
\end{proposition}
\begin{proof}
    It is clear that \(\pi\) is non-expansive with respect to its kernel, so we may assume without loss of generality that \(\ker\pi \neq \algebra{A}\), then by \cref{lem:projector_star_subalgebra,lem:morphism_projector} we know \(\algebra{C} = \setc{c \in \algebra{B}}{\exists b \in \algebra{B}: c = \pi(\unity)b\pi(\unity)}\) is a C*-algebra with identity \(p = \pi(\unity)\) that contains the *-subalgebra \(\ran\pi\). By \cref{prop:norm_cstar_spectral_radius}, we know the restriction of the norm for \(\algebra{B}\) to the C*-subalgebra \(\algebra{C}\) is equal to the spectral radius norm, that is, \(\norm{c} = \sqrt{r_{\algebra{C}}(c^*c)}\) for all \(c \in \algebra{C}\), where the spectral radius is defined with the identity \(p\).

    Let \(a \in \algebra{A}\) be a self-adjoint operator, then
    \begin{equation*}
        \lambda \notin \sigma_{\algebra{A}}(a) \implies a - \lambda \unity \in \invertible{\algebra{A}} \implies \pi(a) - \lambda p \in \invertible{\algebra{C}} \implies \lambda \notin \sigma_{\algebra{C}}(\pi(a)),
    \end{equation*}
    that is, \(\sigma_{\algebra{C}}(\pi(a))\subset \sigma_{\algebra{A}}(a)\). \cref{thm:spectral_radius_cstar} then yields
    \begin{equation*}
        \norm{\pi(a)} = r_{\algebra{C}}(\pi(a)) \leq r_{\algebra{A}}(a) \leq \norm{a}.
    \end{equation*}
    If we now consider \(a \in \algebra{A}\) not self-adjoint, then
    \begin{equation*}
        \norm{\pi(a)}^2 = \norm{\pi(a)^*\pi(a)} = \norm{\pi(a^*a)} \leq \norm{a^*a} \leq \norm{a^*}\norm{a} = \norm{a}^2,
    \end{equation*}
    hence \(\norm{\pi(a)} \leq \norm{a}\).
\end{proof}
\begin{remark}
    This result shows that a *-morphism between C*-algebras is continuous with respect to the uniform topology.
\end{remark}



\begin{proposition}{Range of *-morphism is a C*-subalgebra of its codomain}{range_cstar}
    Let \(\pi : \algebra{A} \to \algebra{B}\) be a *-morphism between the unital C*-algebra \(\algebra{A}\) and the C*-algebra \(\algebra{B}\). Then \(\ran\pi\) is a C*-subalgebra of \(\algebra{B}\) and the map
    \begin{align*}
        \hat{\pi} : \algebra{A}/\ker\pi &\to \ran\pi\\
                                    [a] &\mapsto \pi(a)
    \end{align*}
    is a *-isomorphism.
\end{proposition}
\begin{proof}
    We claim the map \(\tilde{\pi} : \algebra{A}/\ker\pi \to \algebra{B}\) defined by \([a] \mapsto \pi(a)\) is well-defined, since if \(a \in \algebra{A}\) and \(\tilde{a} \in [a]\), then \(\tilde{a} - a \in \ker\pi\), hence \(\tilde{\pi}([\tilde{a}]) = \pi(\tilde{a}) = \pi(a) = \tilde{\pi}([a])\). Let \(a_1, a_2 \in \algebra{A}\) and \(\alpha \in \mathbb{C}\), then \(\tilde{\pi}([0]) = \pi(0) = 0\),
    \begin{equation*}
        \tilde{\pi}([a_1] + \alpha[a_2]) = \tilde{\pi}([a_1 + \alpha a_2]) = \pi(a_1 + \alpha a_2) = \pi(a_1) + \alpha \pi(a_2) = \tilde{\pi}([a_1]) + \alpha \tilde{\pi}([a_2]),
    \end{equation*}
    \begin{equation*}
        \tilde{\pi}([a_1][a_2]) = \tilde{\pi}([a_1a_2]) = \pi(a_1a_2) = \pi(a_1)\pi(a_2) = \tilde{\pi}([a_1])\tilde{\pi}([a_2]),
    \end{equation*}
    and
    \begin{equation*}
        \tilde{\pi}([a_1])^* = \pi(a_1)^* = \pi(a_1)^* = \pi(a_1^*) = \tilde{\pi}([a_1^*]) = \tilde{\pi}([a_1]^*),
    \end{equation*}
    thus showing \(\tilde{\pi}\) is a *-morphism.

    By construction \(\tilde{\pi}\) is injective, hence the map with the codomain restricted to its range, \(\hat{\pi} : \algebra{A}/\ker{\pi} \to \ran\pi\), is bijective, thus a *-isomorphism. In particular, the inverse map \(\hat{\pi}^{-1} : \ran\pi \to \algebra{A}/\ker{\pi}\) is a *-morphism from the *-subalgebra with identity \(\ran\pi\) of the unital Banach *-algebra \(\algebra{C} = \setc{\pi(\unity_{\algebra{A}})b\pi(\unity_{\algebra{A}})}{b \in \algebra{B}}\) to the C*-algebra \(\algebra{A}/\ker\pi\), hence \(\hat{\pi}^{-1}\) is non-expansive by \cref{prop:morphism_non_expansive}. As a result and since \(\tilde{\pi}\) is also non-expansive as a *-morphism between C*-algebras, we have
    \begin{equation*}
        \norm{[a]} = \norm{\hat{\pi}^{-1}\circ \hat{\pi}([a])} \leq \norm{\hat{\pi}([a])} = \norm{\tilde{\pi}([a])} \leq \norm{[a]},
    \end{equation*}
    for all \(a \in \algebra{A}\), that is, the map \(\hat{\pi}\) is an isometry. From \cref{prop:isometry_Banach}, \(\ran{\pi}\) is a C*-algebra.
\end{proof}

\begin{definition}{C*-algebra representation}{representation}
    Let \(\algebra{A}\) be a C*-algebra. A \emph{representation \((\hilbert, \pi)\) of \(\algebra{A}\) on the Hilbert space \(\hilbert\)} is a *-morphism \(\pi : \algebra{A} \to \bounded(\hilbert)\). If \(\pi\) is injective, the representation is \emph{faithful}.
\end{definition}
\begin{remark}
    In an abuse of language, we refer to the representation simply to the *-morphism.
\end{remark}


\begin{proposition}{Faithful representation is isometric}{faithful_isometric}
    Let \(\pi : \algebra{A} \to \algebra{B}\) be a *-morphism between C*-algebras \(\algebra{A}\) and \(\algebra{B}\). The following statements are equivalent
    \begin{enumerate}[label=(\alph*)]
        \item \(\pi\) is injective;
        \item \(\pi\) is an isometry;
        \item if \(a \in \algebra{A}_+ \setminus\set{0}\), then \(\pi(a) \in \algebra{B}_+\setminus\set{0}\).
    \end{enumerate}
\end{proposition}
\begin{proof}
    Suppose \(\pi\) is injective, then the map \(\pi^{-1} : \ran\pi \to \algebra{A}\) is a *-isomorphism, and so must be \(\pi : \algebra{A} \to \ran\pi\), hence \(\pi\) is isometric.

    Suppose \(\pi\) is an isometry, then if \(a > 0\), we have \(\pi(a) \in \algebra{B}_+\) with \(a \neq 0\), and as a result \(\norm{\pi(a)} = \norm{a} > 0\), hence \(\pi(a) \neq 0\).

    Suppose \(\pi\) is not injective, then there exists \(a \in \ker{\algebra{A}}\setminus\set{0}\) and as a result \(\pi(a^*a) = 0\). As \(\norm{a^*a} = \norm{a}^2 > 0\), we have \(a^*a > 0\) with \(\pi(a^*a) = 0\).
\end{proof}
\begin{definition}{Stable subspace under a representation}{stable_under_representation}
    Let \(\algebra{A}\) be a C*-algebra and \((\hilbert, \pi)\) a representation. A linear subspace \(\mathscr{V} \subset \hilbert\) is \emph{stable}, or \emph{invariant}, \emph{under the representation \(\pi\)} if \(\pi(\algebra{A})\mathscr{V} \subset \mathscr{V}\).
\end{definition}

\begin{proposition}{Orthogonal decomposition stable under a representation}{orthogonal_stable_representation}
    Let \(\algebra{A}\) be a C*-algebra and \(\hilbert, \pi\) a representation. A closed linear subspace \(\mathscr{V}\subset \hilbert\) is stable under \(\pi\) if and only if \(\mathscr{V}^{\perp}\) is stable under \(\pi\).
\end{proposition}
\begin{proof}
    We have
    \begin{align*}
        \pi(\algebra{A})\mathscr{V} \subset \mathscr{V}
        &\iff
        \forall a \in \algebra{A}, \forall \psi \in \mathscr{V} : \pi(a)\psi \in \mathscr{V}\\
        &\iff
        \forall a \in \algebra{A}, \forall \psi \in \mathscr{V}, \forall \phi \in \mathscr{V}^{\perp} : \inner{\phi}{\pi(a)\psi} = 0\\
        &\iff
        \forall a \in \algebra{A}, \forall \psi \in \mathscr{V}, \forall \phi \in \mathscr{V}^{\perp}: \inner{\psi}{\pi(a^*)\phi} = 0\\
        &\iff
        \forall a \in \algebra{A}, \forall \psi \in \mathscr{V}, \forall \phi \in \mathscr{V}^{\perp}: \inner{\psi}{\pi(a)\phi} = 0\\
        &\iff
        \forall a \in \algebra{A},\forall \phi \in \mathscr{V}^{\perp} : \pi(a)\phi \in \mathscr{V}^{\perp}\\
        &\iff
        \pi(\algebra{A})\mathscr{V}^{\perp} \subset \mathscr{V}^{\perp}
    \end{align*}
    as desired.
\end{proof}

\begin{proposition}{Necessary and sufficient condition for a closed stable subspace}{subspace_stable}
    Let \(\algebra{A}\) be a C*-algebra and \((\hilbert, \pi)\) be a representation. The closed linear subspace \(\mathscr{V} \subset \hilbert\) is stable under \(\pi\) if and only if the orthogonal projector \(p_{\mathscr{V}}\) onto \(\mathscr{V}\) lies in the commutant \(\pi(\algebra{A})'\).
\end{proposition}
\begin{proof}
    Suppose \(p_{\mathscr{V}}\) commutes with every representant of \(\algebra{A}\), then for all \(\psi \in \mathscr{V}\) we have
    \begin{equation*}
        \pi(a) \psi = \pi(a)p_\mathscr{V} \psi = p_{\mathscr{V}} \pi(a) \psi \subset \mathscr{V},
    \end{equation*}
    hence \(\mathscr{V}\) is stable under \(\pi\).

    If \(\mathscr{V}\) is stable under \(\pi\), then \(p_\mathscr{V} \pi(a) p_{\mathscr{V}} = \pi(a) p_{\mathscr{V}}\) for all \(a \in \algebra{A}\), since \(\pi(a) p_{\mathscr{V}}\psi \in \mathscr{V}\) for all \(\psi \in \hilbert\). This yields
    \begin{equation*}
        \pi(a)p_{\mathscr{V}} = (p_{\mathscr{V}}\pi(a^*)p_{\mathscr{V}})^* = (\pi(a^*)p_{\mathscr{V}})^* = p_{\mathscr{V}} \pi(a),
    \end{equation*}
    hence \(p_{\mathscr{V}} \in \pi(\algebra{A})'\).
\end{proof}

\begin{proposition}{Subrepresentation}{subrepresentation}
    Let \(\algebra{A}\) be a C*-algebra and \((\hilbert, \pi)\) be a representation. If \(\mathscr{V}\) is a closed linear subspace stable under \(\pi\), then the map
    \begin{align*}
        \pi_{\mathscr{V}} : \algebra{A} &\to \bounded(\mathscr{V})\\
                                      a &\mapsto p_\mathscr{V} \pi(a) p_{\mathscr{V}}
    \end{align*}
    is a *-morphism, and the representation \((\mathscr{V}, \pi_{\mathscr{V}})\) is referred to as a \emph{subrepresentation} of \((\hilbert, \pi)\).
\end{proposition}
\begin{proof}
    The map is clearly linear, as we have similarly shown in \cref{lem:projector_star_subalgebra}. Let \(a, b \in \algebra{A}\), then
    \begin{equation*}
        \pi_{\mathscr{V}}(a)\pi_{\mathscr{V}}(b) = p_{\mathscr{V}} \pi(a) p_{\mathscr{V}}^2 \pi(b) p_{\mathscr{V}} = p_{\mathscr{V}}^2 \pi(a)\pi(b) p_{\mathscr{V}}^2 = p_{\mathscr{V}} \pi(ab) p_\mathscr{V} = \pi_{\mathscr{V}}(ab)
    \end{equation*}
    and
    \begin{equation*}
        \pi_{\mathscr{V}}(a)^* = p_{\mathscr{V}} \pi(a)^* p_{\mathscr{V}} = p_{\mathscr{V}}\pi(a^*) p_{\mathscr{V}} = \pi_{\mathscr{V}}(a^*),
    \end{equation*}
    hence \(\pi_\mathscr{V}\) is a *-morphism.
\end{proof}


A set \(\algebra{M}\subset \bounded(\hilbert)\) of bounded operators on a Hilbert space \(\hilbert\) is said to act \emph{nondegenerately} if \(\algebra{M}\hilbert \neq \set{0}\). An important class of nondegenerate representations is the one of \emph{cyclic representations}.
\begin{definition}{Cyclic vector and cyclic representation}{cyclic_representation}
    Let \(\algebra{A}\) be a C*-algebra and \(\hilbert\) be a Hilbert space. A vector \(\psi \in \hilbert\) is \emph{cyclic} for a set of bounded operators \(\algebra{M} \subset \bounded(\hilbert)\) if \(\algebra{M}\psi\) is dense in \(\hilbert\). A \emph{cyclic representation \((\hilbert, \pi, \Omega)\) of \(\algebra{A}\)} is a representation \((\hilbert, \pi)\) for which \(\Omega \in \hilbert\) is a cyclic vector.
\end{definition}

% TODO: study direct sum of hilbert spaces, including uncountable ones
A subrepresentation allows for a decomposition of a representation. If \(\hilbert_1 \subset \hilbert\) is a closed linear subspace of a Hilbert space \(\hilbert\), then we set \(\hilbert_2 = \hilbert_1^{\perp}\), and then \(\hilbert = \hilbert_1 \oplus \hilbert_2\). By the previous propositions, \(\pi_1 = p_{\hilbert_1} \pi p_{\hilbert_1}\) and \(\pi_2 = p_{\hilbert_2} \pi p_{\hilbert_2}\) are representations and we may consider \(\pi = \pi_1 \oplus \pi_2\) and write \((\hilbert, \pi) = (\hilbert_1, \pi_1) \oplus (\hilbert_2, \pi_2)\). We extend this definition to general direct sums of representations as follows: if \(\family{(\hilbert_{\lambda}, \pi_{\lambda})}{\lambda \in \Lambda}\) is a family of representations of a C*-algebra \(\algebra{A}\), then \(\hilbert = \bigoplus_{\lambda \in \Lambda} \hilbert_{\lambda}\) is \todo[defined in the usual manner] and \(\pi = \bigoplus_{\lambda \in \Lambda} \pi_{\lambda}\) is defined by setting \(\pi(a)\) equal to \(\pi_{\lambda}(a)\) in each component \(\hilbert_{\lambda}\), for all \(a \in \algebra{A}\). It follows that \(\pi : \algebra{A} \to \hilbert\) is a representation since each \(\pi_{\lambda}\) is non-expansive by \cref{prop:morphism_non_expansive}.
\begin{proposition}{Non-degenerate representation is a direct sum of cyclic representations}{direct_sum_representation}
    Let \((\hilbert, \pi)\) be a representation of the C*-algebra \(\algebra{A}\). If \(\pi\) is non-degenerate, then there exists a family of cyclic subrepresentations \family{(\hilbert_{\lambda}, \pi_{\lambda}, \Omega_\lambda)}{\lambda \in \Lambda} such that \((\hilbert, \pi) = \bigoplus_{\lambda \in \Lambda}(\hilbert_{\lambda}, \pi)\).
\end{proposition}
\begin{proof}
    We consider the set
    \begin{equation*}
        \mathfrak{O} = \setc*{\mathscr{V} \in \mathbb{P}(\hilbert\setminus\set{0})}{\forall \psi,\phi \in \mathscr{V} : \psi \neq \phi \implies \pi(\algebra{A})\psi \perp \pi(\algebra{A})\phi},
    \end{equation*}
    partially ordered by inclusion. Notice that if \(\phi \in \lspan\set{\psi}\setminus \set{0,\psi}\), then \(\set{\phi,\psi} \notin \mathfrak{O}\). If \(\hilbert\) is unidimensional, then there exists \(\xi \in \hilbert\setminus\set{0}\) such that \(\hilbert = \lspan\set{\xi}\) and as \(\pi\) is nondegenerate, we have \(\pi(\algebra{A})\xi \neq \set{0}\), and as a result \(\set{\xi}\) is a maximal element of \(\mathfrak{O}\). Suppose \(\hilbert\) has dimension greater than one. Let \(\mathcal{F} \subset \mathfrak{O}\) be a non-empty linearly ordered subset of \(\mathfrak{O}\), then its union 
    \begin{equation*}
        \bigcup \mathcal{F} = \bigcup_{\mathscr{V} \in \mathcal{F}} \mathscr{V} = \setc{\psi \in \hilbert}{\exists \mathscr{X} \in \mathcal{F} : \psi \in \mathscr{X}}
    \end{equation*}
    is an upper bound for \(\mathcal{F}\). Let \(\psi, \phi \in \bigcup \mathcal{F}\) with \(\psi \neq \phi\), then there exist \(\mathscr{U},\mathscr{V} \in \mathcal{F}\) such that \(\psi \in \mathscr{U}\) and \(\phi \in \mathscr{V}\). By linear order, we may assume without loss of generality that \(\mathscr{U} \subset \mathscr{V}\), hence \(\psi \in \mathscr{V}\), and we conclude \(\pi(\algebra{A})\psi \perp \pi(\algebra{A}\phi)\) as \(\mathscr{V} \in \mathfrak{O}\), therefore \(\bigcup \mathcal{F} \in \mathfrak{O}\). 

    We have shown every linearly ordered subset of \(\mathfrak{O}\) has an upper bound in \(\mathfrak{O}\), hence by \nameref{thm:zorn} there exists a maximal family of nonzero vectors \(\family{\Omega_{\lambda}}{\lambda \in \Lambda} \subset \hilbert \setminus \set{0}\) such that
    \begin{equation*}
        \alpha, \beta \in \Lambda : \alpha \neq \beta \implies \forall a, b \in \algebra{A} : \inner{\pi(a)\Omega_{\alpha}}{\pi(b)\Omega_{\beta}} = 0.
    \end{equation*}
    Then \(\hilbert_{\lambda} = \cl_{\hilbert}\left(\pi(\algebra{A})\Omega_{\lambda}\right)\) defines a closed linear subspace that is stable under \(\pi\). Indeed, if \(\tilde{\psi} \in \hilbert_{\lambda}\) then there exists a convergent sequence \(\psi : \mathbb{N} \to \hilbert_{\lambda}\) such that \(\psi \to \tilde{\psi}\), and as a result, for all \(a \in \algebra{A}\) we have \(\pi(a)\tilde{\psi} = \lim_{n\to\infty}{\pi(a)\psi_n} \in \hilbert_{\lambda}\) as \(\pi(a)\) is continuous. Following \cref{prop:subrepresentation}, we define the subrepresentation \((\hilbert_{\lambda}, \pi_{\lambda})\) with
    \begin{align*}
        \pi_{\lambda} : \algebra{A} &\to \bounded(\hilbert)\\
                                  a &\mapsto p_{\hilbert_{\lambda}}\pi(a)p_{\hilbert_{\lambda}}
    \end{align*}
    where \(p_{\hilbert_{\lambda}} \in \bounded(\hilbert)\) is the orthogonal projector onto \(\hilbert_{\lambda}\), and for which \(\Omega_\lambda\) is the cyclic vector by construction. Defining the Hilbert space \(\tilde{\hilbert} = \bigoplus_{\lambda \in \Lambda} \hilbert_\lambda\) and its representation \(\tilde{\pi} : \algebra{A} \to \bounded(\hilbert)\) defined by
    \begin{align*}
        \tilde{\pi}(a) : \hilbert &\to \hilbert\\
                             \psi &\mapsto \bigoplus_{\lambda \in \Lambda} \pi_{\lambda}(a)\psi
    \end{align*}
    for all \(a \in \algebra{A}\), we aim to show that \((\hilbert, \pi) = (\tilde{\hilbert}, \tilde{\pi})\).

    Notice that if \(\lambda, \mu \in \Lambda\) with \(\lambda \neq \mu\), then \(\hilbert_{\lambda} \perp \hilbert_{\mu}\) by construction, as the inner product is continuous. Suppose, by contradiction, there exists \(\tilde{\Phi} \in \hilbert\) that does not lie in the direct sum \(\tilde{\hilbert} = \bigoplus_{\lambda \in \Lambda} \hilbert_\lambda\), hence \(\tilde{\Phi} \in \hilbert_\lambda^\perp\) for all \(\lambda \in \Lambda\), and in particular, \(\tilde{\Phi} \in \algebra{A}\Omega_{\lambda}\) for all \(\lambda \in \Lambda\). That is, we have
    \begin{align*}
        \tilde{\Phi} \notin \tilde{\hilbert} &\implies \forall \lambda \in \Lambda, \forall a \in \algebra{A} : \inner{\tilde{\Phi}}{\pi(a^{\ast})\pi(b)\Omega_\lambda} = 0\\
                                             &\implies \forall \lambda \in \Lambda, \forall a,b\in \algebra{A} : \inner{\tilde{\Phi}}{\pi(a)^{\ast}\pi(b)\Omega_{\lambda}} = 0\\
                                             &\implies \forall \lambda \in \Lambda, \forall a,b \in \algebra{A}: \inner{\pi(a)\tilde{\Phi}}{\pi(b)\Omega_\lambda} = 0,
    \end{align*}
    therefore \(\set{\tilde{\Phi}}\cup \family{\Omega_\lambda}{\lambda \in \Lambda}\) is an upper bound for the maximal family. By definition, this means \(\tilde{\Phi} \in \family{\Omega_\lambda}{\lambda \in \Lambda}\), which is absurd as \(\tilde{\Phi} \in \hilbert_{\lambda}^\perp\) for all \(\lambda \in \Lambda\). This contradiction shows us \(\hilbert = \tilde{\hilbert}\).

    Notice for all \(a \in \algebra{A}\) and \(\lambda \in \Lambda\) the operators \(\pi(a)\) and \(p_{\hilbert_\lambda}\) commute. Indeed, since \(\pi(a) \hilbert_\lambda \subset \hilbert_\lambda\) we know \(\pi(a)p_{\hilbert_\lambda} = p_{\hilbert_\lambda}\pi(a)p_{\hilbert_\lambda}\), then taking the adjoint and using the fact that \(\algebra{A}\) is self-adjoint yields \(\pi(a)p_{\hilbert_\lambda} = p_{\hilbert_\lambda}\pi(a)\) for all \(a \in \algebra{A}\) and \(\lambda \in \Lambda\). Let \(\psi \in \hilbert\), then we may decompose it as \(\psi = \oplus_{\lambda \in \Lambda}p_{\hilbert_\lambda}\psi\), yielding
    \begin{align*}
        \tilde{\pi}(a)\psi &= \tilde{\pi}(a)\bigoplus_{\lambda \in \Lambda} p_{\hilbert_\lambda}\psi\\
                           &= \bigoplus_{\lambda \in \Lambda} \pi_{\lambda}(a) p_{\hilbert_\lambda}\psi\\
                           &= \bigoplus_{\lambda \in \Lambda} p_{\hilbert_\lambda} \pi(a) p_{\hilbert_\lambda}^2 \psi\\
                           &= \bigoplus_{\lambda \in \Lambda} p_{\hilbert_\lambda} \pi(a) \psi\\
                           &= \pi(a) \psi
    \end{align*}
    for all \(a \in \algebra{A}\), hence \(\tilde{\pi} = \pi\).
\end{proof}

\begin{definition}{Topologically Irreducible set of bounded operators}{irreducible}
    Let \(\hilbert\) be a Hilbert space. A set \(\algebra{M}\) of bounded operators on \(\hilbert\) is \emph{topologically irreducible} if the only closed subspaces of \(\hilbert\) which are invariant under the action of \(\algebra{M}\) are the trivial subspaces \(\set{0}\) and \(\hilbert\). A representation \((\hilbert, \pi)\) of a C*-algebra \(\algebra{A}\) is \emph{topologically irreducible} if \(\pi(\algebra{A})\) is irreducible.
\end{definition}
\begin{remark}
    We will refer to topological irreducibility simply by irreducibility. The notion of \emph{algebraic} irreducibility requires that the only invariant subspaces are the trivial ones. 
    %TODO: find proof of this
    In fact, the both notions coincide for representations of C*-algebras.
\end{remark}

\begin{proposition}{Irreducible self-adjoint set of bounded operators}{irreducible_self_adjoint}
    Let \(\hilbert\) be a Hilbert space and let \(\algebra{M} \subset \bounded(\hilbert)\) be a self-adjoint set of bounded operators on \(\hilbert\). The following statements are equivalent:
    \begin{enumerate}[label=(\alph*)]
        \item \(\algebra{M}\) is irreducible;
        \item \(\psi \in \hilbert \setminus \set{0}\) is cyclic for \(\algebra{M}\), or \(\algebra{M} = \set{0}\) and \(\hilbert = \mathbb{C}\); and
        \item \(\algebra{M}' = \mathbb{C} \unity\).
    \end{enumerate}
\end{proposition}
\begin{proof}
    Assume \(\algebra{M}\) is irreducible. If \(\algebra{M} = \set{0}\), then the only closed subspaces of \(\hilbert\) are \(\set{0}\) and itself, hence \(\hilbert\) is unidimensional and we conclude \(\hilbert = \mathbb{C}\), up to isomorphism. If \(\algebra{M} \neq \set{0}\), then we suppose, by contradiction, there exists \(\psi \in \hilbert \setminus \set{0}\) such that \(\algebra{M}\psi\) is not dense in \(\hilbert\). Then \((\algebra{M}\psi)^\perp\) contains a non-zero vector and it is invariant by the action of \(\algebra{M}\) as we have
    \begin{align*}
        \phi \in (\algebra{M}\psi)^{\perp}\setminus\set{0} &\iff \forall a,b \in \algebra{M} : \inner{\phi}{a^{\ast}b\psi} = 0\\
                                                           &\iff \forall a,b \in \algebra{M} : \inner{a\phi}{b\psi} = 0\\
                                                           &\iff \forall a \in \algebra{M} : a\phi \in (\algebra{M}\psi)^\perp,
    \end{align*}
    since \(\algebra{M}\) is self-adjoint. Notice we must have \((\algebra{M}\psi)^\perp \neq \hilbert\), otherwise this would imply \(\algebra{M}\psi = \set{0}\), and as a result \(\lspan\set{\psi}\) would be an invariant subspace under \(\algebra{M}\), contradicting its irreducibility. The only other possibility is that \((\algebra{M}\psi)^\perp\) is a non-trivial closed subspace invariant under \(\algebra{M}\), which also contradicts the hypothesis. The contradiction thus shows every non-zero vector of \(\hilbert\) must be cyclic for \(\algebra{M}\).

    %TODO: try and come up with a more elementary proof
    Clearly if \(\algebra{M} = \set{0}\) and \(\hilbert = \mathbb{C}\), we must have \(\algebra{M}' = \mathbb{C} \unity\), so we may assume \(\algebra{M} \neq \set{0}\) and \(\dim\hilbert > 1\). Suppose every non-zero vector of \(\hilbert\) is cyclic for \(\algebra{M}\). \todo[spectral projectors?]

    Suppose \(\algebra{M}\) is not irreducible, then there exists a non-trivial closed subspace \(\mathscr{V}\subset \hilbert\) such that \(\algebra{M}\mathscr{V} \subset \mathscr{V}\). Let \(p_{\mathscr{V}}\in \bounded(\hilbert)\) be the orthogonal projector onto \(\mathscr{V}\), then it follows that \(ap_{\mathscr{V}} = p_{\mathscr{V}}a p_{\mathscr{V}}\) for all \(a \in \algebra{A}\) and taking the adjoint yields \(ap_{\mathscr{V}} = p_{\mathscr{V}} a\), that is, \(p_{\mathscr{V}} \in \algebra{M}'\). As \(\mathscr{V}\) is non-trivial, it follows that \(p_{\mathscr{V}} \notin \lspan\set\unity\), hence \(\algebra{M}' \neq \mathbb{C} \unity\).
\end{proof}

Unitary operators in \(\bounded(\hilbert)\) may be used together with a representation to define yet another representation.
\begin{proposition}{Representations and unitary operators}{representation_unitary}
    Let \(\algebra{A}\) be a C*-algebra and let \(u : \hilbert_1 \to \hilbert_2\) be a unitary operator between the Hilbert spaces \(\hilbert_1\) and \(\hilbert_2\). If \((\hilbert_1, \pi)\) is a representation of \(\algebra{A}\), then the *-morphism \(\pi_u : \algebra{A} \to \bounded(\hilbert_2)\) defined by \(\pi_u(a) = u\pi(a)u^*\) for all \(a \in \algebra{A}\) establishes a representation of \(\algebra{A}\) in \(\hilbert_2\).
\end{proposition}
\begin{proof}
    Clearly \(\pi_u\) is linear and for all \(a, b \in \algebra{A}\) we have
    \begin{equation*}
        \pi_u(ab) = u \pi(ab) u^* = u \pi(a) \pi(b) u^* = u \pi(a) u^* u \pi(b) u^* = \pi_u(a)\pi_u(b)
    \end{equation*}
    and 
    \begin{equation*}
        \pi_u(a)^* = (u \pi(a) u^*)^* = u \pi(a)^* u^* = u \pi(a^*)u^* = \pi_u(a^*),
    \end{equation*}
    hence \(\pi_u\) is a *-morphism.
\end{proof}
In fact, this construction establishes an equivalence relation on the set of representations of a C*-algebra.
\begin{proposition}{Unitarily equivalent representations of a C*-algebra}{unitary_equivalence_representation}
    Let \(\algebra{A}\) be a C*-algebra and let \(\mathfrak{R}_{\algebra{A}}\) denote the set of representations of \(\algebra{A}\). If \((\hilbert_1, \pi_1), (\hilbert_2, \pi_2) \in \mathfrak{R}_{\algebra{A}}\) are such that there exists a unitary operator \(u : \hilbert_1 \to \hilbert_2\) that satisfies \(\pi_2(a) = u\pi_1(a) u^*\) for all \(a \in \algebra{A}\), then the representations are \emph{unitarily equivalent} and we write \(\pi_1 \simeq \pi_2\). Unitary equivalence is an equivalent relation on \(\mathfrak{R}_{\algebra{A}}\).
\end{proposition}
\begin{proof}
    Let \((\hilbert_1, \pi_1), (\hilbert_2, \pi_2), (\hilbert_3, \pi_3) \in \mathfrak{R}_{\algebra{A}}\) be representations. Clearly, \(\pi(a) = \id{\hilbert_1} \pi(a) \id{\hilbert_1}^*\) for all \(a \in \algebra{A},\) hence unitary equivalence is reflexive. If \((\hilbert_1, \pi_1) \simeq (\hilbert_2, \pi_2\), then there exists a unitary operator \(u_{12} : \hilbert_1 \to \hilbert_2\) such that \(\pi_2(a) = u_{12}\pi_1(a)u_{12}^*\) for all \(a \in \algebra{A}\), hence the unitary map \(u_{21} = u_{12}^*\) satisfies \(\pi_1(a) = u_{21}\pi_2(a) u_{21}^*\) for all \(a \in \algebra{A}\), hence unitary equivalence is symmetric. If \((\hilbert_1, \pi_1) \simeq (\hilbert_2, \pi_2)\) and \((\hilbert_2, \pi_2) \simeq (\hilbert_3, \pi_3)\), then with the notation used so far we have 
    \begin{equation*}
        \pi_3(a) = u_{23}\pi_2(a) u_{23}^* = u_{23} u_{12} \pi_1(a) u_{12}^* u_{23}^* = u_{13} \pi_1(a) u_{13}^*,
    \end{equation*}
    for all \(a \in \algebra{A}\), where \(u_{13} = u_{23} u_{12}\) is a composition of unitary maps, thus unitary, that is, unitary equivalence is transitive.
\end{proof}

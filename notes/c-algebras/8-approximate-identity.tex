% vim: spl=en_us
\section{Approximate identity in C*-algebras}
Even though it is always possible to adjoin an identity to a C*-algebra, it is sometimes more worthwhile to intrinsically work with the original algebra, using \emph{approximate identities.}

\begin{definition}{Directed sets and convergence}{directed_set}
    A \emph{directed set} \((\lambda, \preceq)\) is a set \(\Lambda\) with a partial order \(\preceq\) with the property that for all \(\lambda, \lambda' \in \Lambda\), there exists \(\mu \in \Lambda\) such that \(\lambda \preceq \mu\) and \(\lambda' \preceq \mu\).

    Let \(X\) be a non-empty set. If \(x : \Lambda \to X\) is a map, we say \(x\) is a \emph{net} on \(X\) based on the directed set \(\Lambda\). As with sequences, we may denote a net by \(\family{x_{\lambda}}{\lambda \in \Lambda}\), if the partial order is understood by context. If \(\tau\) is a topology on \(X\), we say \(\tilde{x} \in X\) is a limit point of \(x\) with respect to \(\tau\) if for any open neighborhood \(U \in \tau\) of \(\tilde{x}\) there exists \(\kappa\in \Lambda\) such that \(x_{\lambda} \in U\) for all \(\lambda \succeq \kappa\). If \(\tilde{x}\) is a limit point of \(x\) with respect to \(\tau\) we write \(\lim_{\lambda} x_{\lambda} = \tilde{x}\) if the topology and the directed set are understood by context.
\end{definition}

\begin{definition}{Approximate identity}{approximate_identity}
    Let \(\algebra{A}\) be a C*-algebra and let \(\algebra{I}\) be a right ideal of \(\algebra{A}\). An \emph{approximate identity on \(\algebra{A}\) by elements of the right ideal \(\algebra{I}\)} is a net \(e : \Lambda \to \algebra{I}\) satisfying
    \begin{enumerate}[label=(\alph*)]
        \item \(e_{\lambda} \in \algebra{A}_+\) for all \(\lambda \in \Lambda\);
        \item \(\norm{e_{\lambda}}\leq 1\) for all \(\lambda \in \Lambda\);
        \item if \(\lambda, \lambda' \in \Lambda\) with \(\lambda \succeq \lambda'\), then \(e_{\lambda} \geq e_{\lambda'}\); and
        \item \(\lim_{\lambda}{\norm{a - e_{\lambda}a}}= 0\) for all \(a \in \algebra{I}\).
    \end{enumerate}
    If \(\algebra{I} = \algebra{A}\), then we say \(e : \Lambda \to \algebra{A}\) is an \emph{approximate identity on \(\algebra{A}\)}.
\end{definition}

\begin{theorem}{Existence of approximate identity on a ideal of a unital C*-algebra}{approximate_identity}
    Let \(\algebra{A}\) be a unital C*-algebra. If \(\algebra{I} \subset \algebra{A}\) is a right ideal, then there exists an approximate identity on \(\algebra{A}\) by elements of \(\algebra{I}\).
\end{theorem}
\begin{proof}
    Consider the set
    \begin{equation*}
        \Lambda = \setc{\lambda \in \mathbb{P}(\algebra{I})}{\lambda\text{ is finite}}
    \end{equation*}
    partially ordered by inclusion, that is, \(\lambda \preceq \lambda'\) if \(\lambda' \subset \lambda \subset \algebra{I}\). It is clear that \((\Lambda, \preceq)\) is a directed set, since for all \(\lambda, \lambda' \in \Lambda\) we have \(\lambda \preceq \lambda \cup \lambda'\) and \(\lambda' \preceq \lambda \cup \lambda'\), with \(\lambda \cup \lambda'\) finite.

    We consider the net \(f\) on \(\algebra{I}\) based on \(\Lambda\) by
    \begin{equation*}
        f_{\lambda} = \sum_{u \in \lambda} uu^*
    \end{equation*}
    As \(f_{\lambda}\) is a finite sum of positive elements of \(\algebra{A}\), we have \(f_{\lambda} \in \algebra{A}_+\), which leads us to conclude \(\unity + x f_{\lambda} \in \invertible{\algebra{A}}\) and \((\unity + x f_{\lambda})^{-1} \in \algebra{A}_+\) for all \(x \in \mathbb{R}_+\) by \cref{prop:positive_resolvent}. In particular, we consider \(\abs{\lambda} \in \mathbb{R}^+\), where \(\abs{\lambda} \in \mathbb{N}_0\) denotes the number of elements of \(\lambda \in \Lambda\), and define the net
    \begin{align*}
        e : \Lambda &\to \algebra{I}\\
            \lambda &\mapsto \abs{\lambda} f_{\lambda} (\unity + \abs{\lambda} f_{\lambda})^{-1}
    \end{align*}
    with \(e_{\lambda} \in \algebra{A}_+\) for all \(\lambda \in \Lambda\) by \cref{lem:positive_commute,prop:resolvent_commute}. Notice
    \begin{equation*}
        e_{\lambda} = (\unity + \abs{\lambda} f_{\lambda} - \unity)(\unity + \abs{\lambda}f_{\lambda})^{-1} = \unity - (\unity + \abs{\lambda} f_{\lambda})^{-1},
    \end{equation*}
    then \(\unity - e_{\lambda} \in \algebra{A}_+\), that is, \(e_{\lambda} \leq \unity\), which yields \(\norm{e}_{\lambda} \leq 1\) by \cref{prop:properties_order}. If \(\lambda \succeq \lambda'\), then \(f_{\lambda} \geq f_{\lambda'}\),  since we have
    \begin{equation*}
        f_{\lambda} - f_{\lambda'} = \sum_{u \in \lambda \setminus \lambda'} uu^* \geq 0,
    \end{equation*}
    which yields
    \begin{equation*}
        e_{\lambda} - e_{\lambda'} = \left[\unity - (\unity + \abs{\lambda} f_{\lambda})^{-1}\right] - \left[\unity - (\unity + \abs{\lambda'} f_{\lambda'})^{-1}\right] = (\unity + \abs{\lambda'}f_{\lambda'})^{-1} - (\unity + \abs{\lambda}f_{\lambda})^{-1} \geq 0
    \end{equation*}
    by \cref{prop:positive_resolvent}, that is, \(e_{\lambda} \geq e_{\lambda'}\).

    Let \(a \in \algebra{I}\), then
    \begin{equation*}
    (a - e_{\lambda}a)(a - e_{\lambda}a)^* = (\unity - e_{\lambda})a a^* (\unity - e_{\lambda})^* = (\unity + \abs{\lambda}f_{\lambda})^{-1}aa^* (\unity + \abs{\lambda}f_{\lambda})^{-1}
    \end{equation*}
    for all \(\lambda \in \Lambda\). For all \(\lambda \succeq \set{a} \in \Lambda\), we have \(f_{\lambda} \geq f_{\set{a}} = aa^*\) and \(\abs{\lambda} \in \mathbb{N}\), therefore \cref{prop:congruence_order} yields
    \begin{align*}
        (a - e_{\lambda}a)(a - e_{\lambda}a)^* &\leq (\unity + \abs{\lambda}f_{\lambda})^{-1} f_{\lambda} (\unity + \abs{\lambda}f_{\lambda})^{-1}\\
                                               &= \frac{1}{\abs{\lambda}} (\unity + \abs{\lambda}f_{\lambda})^{-1} (\unity + \abs{\lambda}f_{\lambda} - \unity) (\unity + \abs{\lambda}f_{\lambda})^{-1}\\
                                               &= \frac{1}{\abs{\lambda}} \left[\unity - (\unity + \abs{\lambda}f_{\lambda})^{-1}\right](\unity + \abs{\lambda}f_{\lambda})^{-1} = \frac{1}{\abs{\lambda}} g_{\lambda},
    \end{align*}
    where \(g_{\lambda} = \left[\unity - (\unity + \abs{\lambda}f_{\lambda})^{-1}\right](\unity + \abs{\lambda}f_{\lambda})^{-1}\). \cref{prop:properties_order} ensures
    \begin{equation*}
        \norm{a - e_{\lambda}a}^2 = \norm{(a - e_{\lambda}a)(a - e_{\lambda}a)^*} \leq \frac{1}{\abs{\lambda}} \norm{g_{\lambda}}
    \end{equation*}
    for all \(\lambda \succ \set{a}\). \cref{thm:spectral_mapping} yields
    \begin{equation*}
        \sigma(g_{\lambda}) = \setc*{\left[1 - \frac{1}{1 + \abs{\lambda}z}\right]\frac{1}{1 + \abs{\lambda}z}}{z \in \sigma(f_{\lambda})} \subset \setc*{\frac{\abs{\lambda}z}{(1 + \abs{\lambda}z)^2}}{z \in \mathbb{R}^+} = \setc{\xi(\abs{\lambda}z)}{z \in \mathbb{R}_+},
    \end{equation*}
    where we defined the real function \(\xi(x) = \frac{x}{(1 + x)^2}\) for \(x\in\mathbb{R}_+\), with range in \(\mathbb{R}_+\). Notice \(\xi'(x) = \frac{1 - x}{(1 + x)^3}\), then \(\xi'\) changes sign from positive to negative at \(x = 1\), hence it is a local maximum and \(\xi\) is decreasing in \((1, \infty)\). That is, \(\xi(1) = \frac14\) is a global maximum of \(\xi\) and we conclude \(\norm{g_{\lambda}} \leq \frac14\) for all \(\lambda \succ \set{a}\) by \cref{thm:spectral_radius_cstar}. We have thus shown
    \begin{equation*}
        \norm{a - e_{\lambda}a}^2 \leq \frac{1}{4\abs{\lambda}}
    \end{equation*}
    for all \(\lambda \succ \set{a}\), that is, \(\lim_{\lambda}\norm{a - e_{\lambda}a} = 0\).
\end{proof}
\begin{corollary}
    Let \(\algebra{A}\) be a C*-algebra. Then there exists an approximate identity on \(\algebra{A}\).
\end{corollary}
\begin{proof}
    We may assume \(\algebra{A}\) has no identity, then we adjoin one. \cref{thm:adjoin_unity} guarantees \(\algebra{A}\) is isometrically *-isomorphic to the *-ideal \(\pi(\algebra{A})\) of \(\mathbb{C} \ltimes \algebra{A}\). Let \(e : \Lambda \to \pi(\algebra{A})\) be an approximate identity on \(\mathbb{C} \ltimes \algebra{A}\) by elements of \(\pi(\algebra{A})\), then \(\pi^{-1} \circ e : \Lambda \to \algebra{A}\) is an approximate identity on \(\algebra{A}\).
\end{proof}

\begin{lemma}{}{}
    Let \(\algebra{A}\) be a unital C*-algebra and let \(\algebra{I} \subset \algebra{A}\) be a right ideal. If \(e : \Lambda \to \algebra{I}\) is an approximate identity on \(\algebra{A}\) by elements of \(\algebra{I}\), then \(\norm{\unity - e_{\lambda}} \leq 1\) for all \(\lambda \in \Lambda\).
\end{lemma}
\begin{proof}

\end{proof}

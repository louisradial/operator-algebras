% vim: spl=en_us
\section{An introduction to von Neumann algebras}
In this section, we consider the C*-algebra of bounded operators \(\bounded(\hilbert)\) on a Hilbert space \(\hilbert\).
\begin{definition}{Commutant of a set of operators}{}
    Let \(\hilbert\) be a Hilbert space. The \emph{commutant} \(\algebra{M}'\) of a subset \(\algebra{M} \subset \bounded(\hilbert)\) is the set
    \begin{equation*}
        \algebra{M}' = \setc{b \in \bounded(\hilbert)}{\forall a \in \algebra{M} : ab = ba}.
    \end{equation*}
    The \emph{bicommutant} \(\algebra{M}''\) of \(\algebra{M}\) is the commutant of \(\algebra{M}'\).
\end{definition}
\begin{remark}
    If \(\algebra{M} \subset \algebra{N} \subset \bounded(\hilbert)\) are sets, then
    \begin{equation*}
        b \in \algebra{N}' \implies \forall a \in \algebra{N} : ab = ba \implies \forall a \in \algebra{M} : ab = ba \implies b \in \algebra{M}',
    \end{equation*}
    that is, \(\algebra{N}' \subset \algebra{M}'\).
\end{remark}

\begin{proposition}{Commutant is a Banach subalgebra}{commutant_Banach}
    Let \(\hilbert\) be a Hilbert space and \(\algebra{M} \subset \bounded(\hilbert)\) a subset of bounded operators. The commutant \(\algebra{M}'\) is a unital Banach subalgebra of \(\bounded(\hilbert)\). If \(\algebra{M}\) is self-adjoint, then \(\algebra{M}'\) is a unital C*-subalgebra of \(\bounded(\hilbert)\).
\end{proposition}
\begin{proof}
    The commutant of the subset \(\algebra{M}\) of \(\bounded(\hilbert)\) is a subalgebra since if \(\algebra{M}\) is the empty set, then its commutant is \(\bounded(\hilbert)\), and if it is not empty, then for all \(\lambda \in \mathbb{C}\) and all \(a,b \in \algebra{M}'\), then
    \begin{equation*}
        c(ab) = acb = (ab)c,\quad
        c(a+b) = ca + cb = ac + bc = (a+b)c,\quad\text{and}\quad
        c(\lambda a) = \lambda ca = (\lambda a)c,
    \end{equation*}
    for all \(c \in \algebra{M}\). Moreover, it is trivial to check \(\algebra{M}'\) contains the identity.

    Let \(x : \mathbb{N} \to \algebra{M}'\) be a convergent sequence to some \(\tilde{x} \in \bounded(\hilbert)\). If \(a \in \algebra{M}\), then for all \(n \in \mathbb{N}\) we have
    \begin{align*}
        \norm{\tilde{x}a - a\tilde{x}} &\leq \norm{(\tilde{x} - x_n)a - a(\tilde{x} - x_n) + x_na - a x_n} \\
                                       &\leq \norm{a(\tilde{x} - x_n)} + \norm{(\tilde{x}- x_n)a} \\
                                       &\leq 2\norm{a} \norm{\tilde{x} - x_n},
    \end{align*}
    therefore \(\norm{\tilde{x} a - a\tilde{x}} = 0\), that is, \(\tilde{x} \in \algebra{M}'\). As a closed subalgebra of a complete algebra, we have shown \(\algebra{M}'\) is a unital Banach subalgebra of \(\bounded(\hilbert)\).

    Suppose \(\algebra{M}\) is self-adjoint and let \(b \in \algebra{M}'\), then
    \begin{align*}
        b \in \algebra{M}' \implies \forall a \in \algebra{M} : a^*b = ba^* \implies \forall a \in \algebra{M} : b^*a = ab^* \implies b^* \in \algebra{M}',
    \end{align*}
    that is, the commutant \(\algebra{M}'\) is self-adjoint, therefore a C*-subalgebra of \(\bounded(\hilbert)\).
\end{proof}

\begin{proposition}{Higher order commutant of a set}{motivation_von_neumann}
    Let \(\hilbert\) be a Hilbert space and \(\algebra{M} \subset \bounded(\hilbert)\) a subset of bounded operators. For \(n \in \mathbb{N}\), let us briefly denote the commutant of \(\algebra{M}^{(n)}\) as \(\algebra{M}^{(n+1)}\), where \(\algebra{M}^{(1)} = \algebra{M}'\). Then
    \begin{enumerate}[label=(\alph*)]
        \item \(\algebra{M} \subset \algebra{M}'',\)
        \item \({(\algebra{M}^{(n)})}^{(m)} = \algebra{M}^{(n+m)},\)
        \item \(\algebra{M}^{(2n)} = \algebra{M}^{(2m)},\) and
        \item \(\algebra{M}^{(2n+1)} = \algebra{M}^{(2m+1)}\),
    \end{enumerate}
    for all \(n, m \in \mathbb{N}\).
\end{proposition}
\begin{proof}
    It is clear a set is contained in its bicommutant, as \(a \in \algebra{M}\) commutes with every operator of \(\algebra{M}'\), hence \(a \in \algebra{M}''\) and we conclude (a).

    Let \(n \in \mathbb{N}\), and we show (b) by induction on \(m \in \mathbb{N}\). First \({(\algebra{M}^{(n)})}^{(1)} = {(\algebra{M}^{(n)})}' = \algebra{M}^{(n+1)}\) follows by definition. Suppose it holds for some \(m \in \mathbb{N}\), then
    \begin{equation*}
        {(\algebra{M}^{(n)})}^{(m+1)} = {\left({(\algebra{M}^{(n)})}^{(m)}\right)}' = {(\algebra{M}^{(n+m)})}' = \algebra{M}^{n+m+1},
    \end{equation*}
    hence it also holds for \(m + 1\). By the principle of finite induction, it holds for all \(m \in \mathbb{N}\) and we conclude (b) since \(n\) is arbitrary.

    In order to show (c) and (d) we must only prove \(\algebra{M}' = \algebra{M}^{(3)}\), since by (b) this would imply \(\algebra{M}^{(k)} = \algebra{M}^{(k + 2)}\) for all \(k \in \mathbb{N}\). Taking the commutant of (a) yields \(\algebra{M}^{(3)} \subset \algebra{M}'\), and since \(\algebra{M}^{(3)}\) is the bicommutant of \(\algebra{M}'\), we have from (a) that \(\algebra{M}' \subset \algebra{M}^{(3)}\), concluding the proof.
\end{proof}

\begin{definition}{von Neumann algebras}{von_neumann}
    A \emph{von Neumann algebra \(\algebra{M}\) on a Hilbert space \(\hilbert\)} is a *-subalgebra \(\algebra{M}\) of \(\bounded(\hilbert)\) such that \(\algebra{M}'' = \algebra{M}\). The \emph{center of a von Neumann algebra \(\algebra{M}\)} is the intersection \(\mathfrak{Z}(\algebra{M}) = \algebra{M} \cap \algebra{M}'\). A von Neumann algebra \(\algebra{M}\) is a \emph{factor} if its center is trivial, that is, \(\mathfrak{Z}(\algebra{M}) = \mathbb{C} \unity\), where \(\mathbb{C} \unity\) denotes the algebra generated by the identity.
\end{definition}

\begin{proposition}{Trivial factors}{trivial_factors}
    Let \(\hilbert\) be a Hilbert space. Then \(\bounded(\hilbert)\) and \(\mathbb{C} \unity\) are factors with \(\bounded(\hilbert)' = \mathbb{C} \unity\) and \(\mathbb{C} \unity' = \bounded(\hilbert)\).
\end{proposition}
\begin{proof}
    From \cref{prop:motivation_von_neumann}, we have \(\bounded(\hilbert) \subset \bounded(\hilbert)'' \subset \bounded(\hilbert)\), hence \(\bounded(\hilbert)\) is a von Neumann algebra. As \(\bounded(\hilbert)\) is self-adjoint, then its commutant \(\bounded(\hilbert)'\) is self-adjoint and contains \(\mathbb{C} \unity\) by \cref{prop:commutant_Banach}. We must show \(\bounded(\hilbert)' \subset \mathbb{C} \unity\) in order to conclude \(\bounded(\hilbert)' = \mathbb{C} \unity\) and \(\mathbb{C} \unity' = \bounded(\hilbert)'' = \bounded(\hilbert)\), with \(\mathfrak{Z}(\mathbb{C}\unity) = \mathfrak{Z}(\bounded(\hilbert)) = \mathbb{C} \unity\).

    Let \(a \in \bounded(\hilbert)'\), then \(a^* \in \bounded(\hilbert)'\). We aim to show there exists a constant \(\alpha \in \mathbb{C}\) such that \(a\psi = \alpha \psi\) for all \(\psi \in \hilbert,\) that is, that \(a = \alpha \unity\). For \(\phi \in \hilbert\) with \(\norm{\phi} = 1\), let us denote by \(p_{\phi} \in \bounded(\hilbert)\) the orthogonal projector onto the linear subspace spanned by \(\phi\). If \(a \in \bounded(\hilbert)'\), then for all unitary vectors \(\phi \in \hilbert\) we have \(a p_\phi = p_\phi a\) and \(a^* p_\phi = p_\phi a^*\), hence
    \begin{equation*}
        \inner{\phi}{a \psi}\phi = p_\phi a\psi = ap_\phi \psi  = \inner{\phi}{\psi}a\phi
    \end{equation*}
    for all \(\psi \in \hilbert\). Setting \(\psi = \phi\) yields \(a\phi = \inner{\phi}{a\phi}\phi\) and, analogously, \(a^*\phi = \inner{\phi}{a^*\phi}\phi\).

    Let \(\tilde{\phi} \in \hilbert\) with \(\norm{\tilde{\phi}}=1\), then
    \begin{equation*}
        \inner{\tilde{\phi}}{a\phi} = \inner{\tilde{\phi}}{\phi}\inner{\phi}{a\phi}
        \quad\text{and}\quad
        \inner{\tilde{\phi}}{a\phi} = \inner{a^*\tilde{\phi}}{\phi} = \inner{a^*\tilde{\phi}}{\tilde{\phi}}\inner{\tilde{\phi}}{\phi} = \inner{\tilde{\phi}}{\phi} \inner{\tilde{\phi}}{a \tilde{\phi}}.
    \end{equation*}
    We may assume without loss of generality that \(\hilbert\) is not unidimensional, for if it were, then \(\phi\) and \(\tilde{\phi}\) would not be orthogonal, yielding \(\inner{\phi}{a\phi} = \inner{\tilde{\phi}}{a\tilde{\phi}} = \alpha,\) hence \(a = \alpha \unity\). If \(\tilde{\phi} \perp \phi\), then \(\Phi = \frac{1}{\sqrt{2}}(\phi + \tilde{\phi})\) is not orthogonal to either \(\phi\) or \(\tilde{\phi}\) and satisfies \(\norm{\Phi} = 1\), therefore
    \begin{equation*}
        \inner{\phi}{a\phi} = \inner{\Phi}{a\Phi} = \inner{\tilde{\phi}}{a\tilde{\phi}} = \alpha,
    \end{equation*}
    and we conclude \(a \in \mathbb{C} \unity\).
\end{proof}

\begin{proposition}{Commutant of a set is weakly and strongly closed}{}
    Let \(\hilbert\) be a Hilbert space. If \(\algebra{M} \subset \bounded(\hilbert),\) then \(\algebra{M}'\) is weakly and strongly closed
\end{proposition}
\begin{proof}
    \todo[operator topologies and seminorms]
\end{proof}


Let \(\hilbert\) be a Hilbert space, \(\phi \in \hilbert\) be a vector, \(\mathscr{V} \subset \hilbert\) be a set of vectors, and \(\mathscr{A} \subset \bounded(\hilbert)\) be a set of bounded operators, then we denote \(\mathscr{A}\mathscr{V} = \setc{a \psi}{a \in \mathscr{A}, \psi \in \mathscr{V}}\) and \(\mathscr{A}\phi = \mathscr{A}\set{\phi}\).
\begin{enumerate}[label=(\alph*)]
    \item If \(\algebra{A} \subset \bounded(\hilbert)\) is an algebra and \(\phi \in \hilbert\), then \(\lspan(\algebra{A}\phi) = \algebra{A}\phi\) for all \(\phi \in \hilbert\). Indeed, if \(\tilde{\phi} \in \lspan(\algebra{A}\phi)\), then there exists a finite subset \(\mathscr{F} \subset \algebra{A}\) such that \(\tilde{\phi} = \sum_{a \in \mathscr{F}} a\phi = \left(\sum_{a \in \mathscr{F}} a\right)\phi \in \algebra{A}\phi\).
    \item If \(\mathscr{A} \subset \bounded(\hilbert)\) and \(\mathscr{V} \subset \hilbert\), then \(\lspan(\mathscr{AV})^{\perp} = (\mathscr{AV})^{\perp}\). By \cref{lem:complement_subsets}, we already know that \(\lspan(\mathscr{A}\mathscr{V})^{\perp} \subset (\mathscr{A}\mathscr{V})^\perp\). If \(\psi \in (\mathscr{AV})^{\perp}\), then it is orthogonal to any finite linear combination of elements of \(\mathscr{AV}\), hence \(\psi \in \lspan(\mathscr{AV})^{\perp}\).
\end{enumerate}
\begin{definition}{Non-degenerate operator algebra}{non_degenerate_algebra}
    Let \(\hilbert\) be a Hilbert space. A \emph{non-degenerate operator algebra \(\algebra{A}\)} is an algebra \(\algebra{A} \subset \bounded(\hilbert)\) such that \(\algebra{A}\hilbert\) is \emph{total}, that is, \((\algebra{A}\hilbert)^{\perp} = \set{0}\).
\end{definition}
\begin{remark}
    By the previous remarks, an algebra \(\algebra{A} \subset \bounded(\hilbert)\) is non-degenerate if \(\lspan(\algebra{A}\hilbert)\) and only if it is dense in the Hilbert space \(\hilbert\) with respect to the uniform topology.
\end{remark}

\begin{lemma}{Self-adjoint non-degenerate operator algebra}{non_degenerate_self_adjoint_algebra}
    Let \(\hilbert\) be a Hilbert space. A self-adjoint algebra \(\algebra{A} \subset \bounded(\hilbert)\) is non-degenerate if and only if \(\algebra{A}\psi = \set{0}\) implies \(\psi=0\).
\end{lemma}
\begin{proof}
    Since \(\algebra{A}\) is self-adjoint, we have
    \begin{align*}
        \algebra{A}\psi = \set{0} &\iff \forall a \in \algebra{A} : a^* \psi = 0\\
                                  &\iff \forall a \in \algebra{A}, \forall \phi \in \hilbert : \inner{a^*\psi}{\phi} = 0\\
                                  &\iff \forall a \in \algebra{A}, \forall \phi \in \hilbert : \inner{\psi}{a\phi} = 0\\
                                  &\iff \forall \phi \in \algebra{A}\hilbert : \inner{\psi}{\phi} = 0\\
                                  &\iff \psi \in (\algebra{A}\hilbert)^{\perp}.
    \end{align*}
    If \(\algebra{A}\) is non-degenerate, then \(\algebra{A}\psi = \set{0} \iff \psi = 0\). If \(\algebra{A} \psi = \set{0} \iff \psi = 0\), then \(\psi \in (\algebra{A}\hilbert)^{\perp} \iff \psi = 0,\) hence \((\algebra{A}\hilbert)^{\perp} = \set{0}\) and the algebra is non-degenerate.
\end{proof}

\begin{theorem}{Bicommutant theorem}{bicommutant}
    Let \(\hilbert\) be a Hilbert space and let \(\algebra{M} \subset \bounded(\hilbert)\) be a self-adjoint non-degenerate operator algebra. The following statements are equivalent:
    \begin{enumerate}[label=(\alph*)]
        \item \(\algebra{M}\) is a von Neumann algebra;
        \item \(\algebra{M}\) is weakly closed;
        \item \(\algebra{M}\) is \(\sigma\)-weakly closed;
        \item \(\algebra{M}\) is strongly closed; and
        \item \(\algebra{M}\) is \(\sigma\)-strongly closed.
    \end{enumerate}
\end{theorem}
\begin{proof}
    \todo[lots]
\end{proof}

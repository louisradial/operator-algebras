% vim: spl=en_us
\section{Ideals and quotient spaces}
Let \(\algebra{A}\) be an algebra. A \emph{subalgebra} \(\algebra{B}\) is a linear subspace \(\algebra{B} \subset \algebra{A}\) satisfying \(ab \in \algebra{B}\) for all \(a, b \in \algebra{B}\). If \(\algebra{A}\) has more structure we name the subalgebra accordingly if it is closed with respect to the additional properties. For example, if \(\algebra{A}\) is involutive, \(\algebra{B}\) is a *-subalgebra if \(a \in \algebra{B} \implies a^* \in \algebra{B}\).
\begin{definition}{Self-adjoint subset of an involutive algebra}{self_adjointness}
    Let \(\algebra{A}\) be an involutive algebra. A \emph{self-adjoint} element \(a \in \algebra{A}\) satisfies \(a^* = a\). If \(B \subset \algebra{A}\) satisfies \(B^* = B\), then it is called a \emph{self-adjoint} subset of \(\algebra{A}\).
\end{definition}
\begin{remark}
    It is clear that a *-subalgebra is a self-adjoint subset of an involutive algebra.
\end{remark}
\begin{proposition}{Necessary and sufficient conditions for self-adjoint subset}{self_adjoint_subset}
    Let \(\algebra{A}\) be a *-algebra and \(B \subset \algebra{A}\) a non-empty subset. The following statements are equivalent:
    \begin{enumerate}[label=(\alph*)]
        \item \(B^* \subset B\);
        \item \(B \subset B^*\);
        \item \(B\) is self adjoint.
    \end{enumerate}\
\end{proposition}
\begin{proof}
    It is evident that (c) implies (a). Suppose \(B^* \subset B\) and let \(b \in B\). Since \(b^* \in B^* \subset B\), \(b\) is the adjoint to some element in \(B\), namely \(b^*\), hence \(b \in B^*\).  That is, (a) implies (b).

    Suppose \(B \subset B^*\) and let \(a \in B^*\). There exists \(b \in B\) such that \(a = b^*\), but \(b\) is the adjoint to some \(c \in B\), by hypothesis. That is, \(a = (c^*)^* = c \in B\), and we have shown \(B^* \subset B\), hence \(B\) is self-adjoint.
\end{proof}

Let \(A\) and \(B\) be non-empty subsets of an associative algebra \(\algebra{A}\). The image of the product restricted to \(A \times B\) is denoted by \(AB\), that is,
\begin{equation*}
    AB = \setc{x \in \algebra{A}}{\exists a \in A, \exists b \in B : x = ab}.
\end{equation*}
It should be clear \(AB\) is not generally equal to \(BA\).
\begin{definition}{Ideals}{ideal}
    Let \(\algebra{A}\) be an associative algebra. A linear subspace \(B \subset \algebra{A}\) is a
    \begin{enumerate}[label=(\alph*)]
        \item \emph{left ideal of \(\algebra{A}\)} if \(\algebra{A}B \subset B;\)
        \item \emph{right ideal of \(\algebra{A}\)} if \(B\algebra{A} \subset B.\)
    \end{enumerate}
    If \(B\) is a left ideal and a right ideal of \(\algebra{A}\), then we say \(B\) is a \emph{two-sided ideal of \(\algebra{A}\)}.
\end{definition}
\begin{remark}
    It is easy to conclude every ideal is a subalgebra. Indeed, suppose, for definiteness, \(B \subset \algebra{A}\) is a left ideal of \(\algebra{A}\), then \(BB \subset \algebra{A}B \subset B\), that is, \(B\) is an algebra.
\end{remark}

\begin{proposition}{Self-adjoint ideal is two-sided ideal}{self_adjoint_ideal}
    Let \(\algebra{A}\) be a *-algebra. If \(\algebra{B} \subset \algebra{A}\) is a left (or right) ideal of \(\algebra{A}\) and self-adjoint, then \(\algebra{B}\) is a two-sided ideal.
\end{proposition}
\begin{proof}
    Let \(b \in \algebra{B}\), then for all \(a \in \algebra{A}\), we have \(ab \in \algebra{B}\). Self-adjointness yields \(a^*b^* \in \algebra{B}\), hence \(ba \in \algebra{B}\) for all \(a \in \algebra{A}\). Since \(b\) is arbitrary, \(\algebra{B}\) is a right ideal.
\end{proof}
\begin{remark}
    If an ideal is self-adjoint we'll say it is a *-ideal.
\end{remark}

Recall a linear subspace \(S\) on a linear space \(V\) defines an equivalence relation \(\sim_S\) by \(u \sim_S v\) if \(u - v \in S\). We write the quotient space \(V/S\) to denote the quotient set \(V/\sim_S\) and we call the map \(V \ni v \mapsto [v] \in V/S\) the \emph{quotient map}. Moreover, with the operations \(\alpha[v] = [\alpha v]\) and \([u]+[v] = [u+v]\), \(V/S\) is made into a linear space.

Let \(\algebra{A}\) be a Banach *-algebra and let \(\algebra{I}\subset \algebra{A}\) be a closed *-ideal of \(\algebra{A}\). We'll show that with the maps
\begin{align*}
    \cdot : \algebra{A}/\algebra{I} \times \algebra{A}/\algebra{I} &\to \algebra{A}/\algebra{I}&
    * : \algebra{A}/\algebra{I} &\to \algebra{A}/\algebra{I}&
    \norm{\noarg} : \algebra{A}/\algebra{I} &\to \mathbb{R}\\
    ([x],[y]) &\mapsto [xy]&
    [x] &\mapsto [x^*]&
    [x] &\mapsto \inf_{j \in \algebra{I}}{\norm{x + j}},
\end{align*}
the quotient space \(\algebra{A}/\algebra{I}\) is a Banach *-algebra. We'll split the proof in several lemmas showing intermediate results.

\begin{lemma}{Quotient space by a two-sided ideal is an associative algebra}{quotient_two_sided_associative_algebra}
    Let \(\algebra{A}\) be an associative algebra. If \(\algebra{I}\) is a two-sided ideal, then \(\algebra{A}/\algebra{I}\) is an associative algebra. If, furthermore, \(\algebra{A}\) is unital, then \([\unity]\) is the identity of the quotient space \(\algebra{A}/\algebra{I}\).
\end{lemma}
\begin{proof}
    We begin by showing the algebraic product is well-defined. Let \(x_1, x_2, y_1, y_2 \in \algebra{A}\) with \([x_1] = [x_2]\) and \([y_1] = [y_2]\), then there exist \(j_x = x_2 - x_1\) and \(j_y = y_2 - y_1\) with \(j_x, j_y \in \algebra{I}\). We have
    \begin{equation*}
        x_1y_1 - x_2 y_2 = x_1 y_1 - (x_1 + j_x) y_2 = x_1 (y_1 - y_2) - j_x y_2 = x_1 j_y - j_x y_2,
    \end{equation*}
    hence \([x_1 y_1] = [x_2 y_2]\) because \(\algebra{I}\) is a two-sided ideal. Therefore, the product is well-defined.

    Let \([a], [b], [c] \in \algebra{A}/\algebra{I}\) and \(\alpha \in \mathbb{C}\), then
    \begin{equation*}
        [a] \cdot ([b]+[c]) = [a]\cdot[b+c] = [ab + ac] = [ab] + [ac]
    \end{equation*}
    and
    \begin{equation*}
        ([b]+[c])\cdot[a] = [b+c]\cdot[a] = [ba + ca] = [ba] + [ca],
    \end{equation*}
    that is, the product is distributive with respect to vector addition. We also have
    \begin{align*}
        \alpha([a]\cdot[b]) = \alpha[ab] = [\alpha ab] &= ([\alpha a])\cdot[b] = (\alpha[a])\cdot[b]\\
                                                       &= [a]\cdot([\alpha b]) = [a]\cdot (\alpha[b]),
    \end{align*}
    then it follows that the product is compatible with scalar multiplication. Moreover, the product is associative since
    \begin{equation*}
        [a]\cdot([b]\cdot[c]) = [a]\cdot[bc]=[abc] = [ab] \cdot [c] = ([a]\cdot[b])\cdot[c],
    \end{equation*}
    hence \(\algebra{A}/\algebra{I}\) is an associative algebra. If, in addition, \(\algebra{A}\) is unital we have \([\unity]\cdot[x] = [\unity x] = [x]\) and \([x]\cdot[\unity] = [x \unity] = [x]\) for all \([x] \in \algebra{A}/\algebra{I}\), hence \([\unity]\) is the identity of \(\algebra{A}/\algebra{I}\).
\end{proof}
\begin{lemma}{Quotient space by a self-adjoint subspace is a *-algebra}{quotient_self_adjoint_star}
    Let \(\algebra{A}\) be a *-algebra. If \(\algebra{I}\) is a self-adjoint linear subspace, then \(\algebra{A}/\algebra{I}\) is a *-algebra.
\end{lemma}
\begin{proof}
    The well-definition of the map \([x] \mapsto [x^*]\) follows from the self-adjointness of the ideal. Indeed, let \(z_1, z_2 \in \algebra{A}\) with \([z_1] = [z_2]\), then there exists \(j \in \algebra{I}\) such that \(j = z_2 - z_1 \in \algebra{I}\), hence \(j^* = z_2^* - z_1^* \in \algebra{I}\), which shows \([z_1^*] = [z_2^*]\). Let \([a], [b] \in \algebra{A}/\algebra{I}\) and \(\lambda\in \mathbb{C}\), then * satisfies
    \begin{enumerate}[label=(\alph*)]
        \item involutivity: \(\left([a]^*\right)^* = [a^*]^* = [(a^*)^*] = [a]\);
        \item antidistributivity: \(([a]\cdot[b])^* = [ab]^* = [(ab)^*] = [b^*a^*] = [b^*]\cdot[a^*] = [b]^* \cdot [a]^*\); and
        \item antilinearity: \(([a] + \lambda[b])^* = [a + \lambda b]^* = [(a + \lambda b)^*] = [a^* + \conj{\lambda}b^*] = [a^*] + [\conj{\lambda} b^*] = [a]^* + \conj{\lambda}[b]^*\).
    \end{enumerate}
    If \(\algebra{A}\) is unital, we have \([\unity]^* = [\unity^*] = [\unity]\), hence \(*\) is an involution on \(\algebra{A}/\algebra{I}\).
\end{proof}

\begin{lemma}{Quotient space by a closed subspace is a normed linear space}{quotient_closed_normed}
    Let \(\algebra{A}\) be a normed linear space. If \(\algebra{I}\) is a closed subspace, then \(\algebra{A}/\algebra{I}\) is a normed linear space.
\end{lemma}
\begin{proof}
    Let \(w_1, w_2 \in \algebra{A}\) with \([w_1] = [w_2]\), then the sets \(\setc{w_1 + j}{j \in \algebra{I}}\) and \(\setc{w_2 + j}{j \in \algebra{I}}\) are equal, hence the map \(\norm{\noarg}\) is well-defined.

    Let \(\lambda \in \mathbb{C}\setminus\set{0}\), then
    \begin{equation*}
        \norm{\lambda [y]} = \norm{[\lambda y]} = \inf_{j \in \algebra{I}}{\norm{\lambda y + \lambda j}}= \abs{\lambda} \inf_{j \in \algebra{I}}{\norm{y} + j} = \abs{\lambda}\cdot \norm{[y]}
    \end{equation*}
    for all \([y] \in \algebra{A}/\algebra{I}\). We also have \(\norm{0 [y]} = \norm{[0]} = \inf_{j \in \algebra{I}} \norm{j} = 0,\) for all \([y] \in \algebra{A}/\algebra{I}\), hence \(\norm{\noarg}\) is absolute homogeneous.

    Let \([a], [b] \in \algebra{A}/\algebra{I}\), then
    \begin{equation*}
        \norm{[a] + [b]} = \norm{[a+b]} = \inf_{n,m \in \algebra{I}}{\norm{a+b+n+m}} \leq \inf_{n,m \in \algebra{I}}{\left(\norm{a + n} + \norm{b+m}\right)} = \norm{[a]} + \norm{[b]},
    \end{equation*}
    hence \(\norm{\noarg}\) is subadditive.

    From the nonnegativity of the norm defined on \(\algebra{A}\), it follows that this map is non-negative. Let \([x] \in \algebra{A}/\algebra{I}\) be such that \(\norm{[x]} = 0\), then there exists no \(\epsilon > 0\) such that \(\norm{j - x} \geq \epsilon > 0\) for all \(j \in \algebra{I}\). As a result, every open ball centered at \(x\) has non-empty intersection with \(\algebra{I}\), then \(x \in \cl_{\algebra{A}}\algebra{I}\). Since \(\algebra{I}\) is closed, we have \(x \in \algebra{I}\). The positive-definiteness of the norm and the previous result yield
    \begin{equation*}
        \norm{[x]} = 0 \iff x \in \algebra{I} \iff [x] = [0],
    \end{equation*}
    hence \(\norm{\noarg}\) is positive-definite. That is, \(\norm{\noarg}\) defines a norm on \(\algebra{A}/\algebra{I}\).
\end{proof}
\begin{lemma}{Quotient space by a closed two-sided ideal is a normed algebra}{quotient_closed_ideal_normed}
    Let \(\algebra{A}\) be a normed algebra. If \(\algebra{I}\) is a closed two-sided ideal, then \(\algebra{A}/\algebra{I}\) is a normed algebra.
\end{lemma}
\begin{proof}
    Let \([u],[v] \in \algebra{A}/\algebra{I},\) then
    \begin{align*}
        \norm{[u]\cdot[v]} = \norm{[uv]} = \inf_{j\in \algebra{I}}{\norm{uv + j}}
        &\leq \inf_{k,\ell \in \algebra{I}}{\norm{uv + \ell k}}\\
        &\leq \inf_{k,\ell \in \algebra{I}}{\norm{uv + k\ell + u \ell + kv}}
    \end{align*}
    since \(\algebra{I}\) is a two-sided ideal and, in particular, a subalgebra. Notice \(uv + k\ell + u \ell + kv = (u + k)(v + \ell)\), then
    \begin{equation*}
        \norm{[u]\cdot[v]} \leq \inf_{k,\ell \in \algebra{I}}{\norm{(u+k)(v + \ell)}}
                           \leq \inf_{k\in\algebra{I}}{\norm{u+k}}\inf_{\ell \in \algebra{I}}{\norm{v+\ell}}
                           = \norm{[u]}\cdot\norm{[v]}.
    \end{equation*}
    If \(\algebra{A}\) is unital we have
    \begin{equation*}
        \norm{[x]} = \norm{[\unity]\cdot[x]} \leq \norm{[\unity]}\cdot\norm{[x]},
    \end{equation*}
    for all \([x] \in \algebra{A}/\algebra{I}\). In particular, we have \(\norm{[\unity]}\left(\norm{[\unity]} - 1\right) \geq 0\), hence either \(\norm{[\unity]} = 0\) or \(\norm{[\unity]}\geq 1\). However we have
    \begin{equation*}
        \norm{[\unity]} = \inf_{j\in \algebra{I}}{\norm{\unity + j}} \leq \norm{\unity}+\inf_{j\in\algebra{I}}{\norm{j}} = 1,
    \end{equation*}
    hence \(\norm{[\unity]} = 1\) or \(\norm{[\unity]} = 0\). In the second case we have \(\norm{[x]} = 0\) for all \([x] \in \algebra{A}/\algebra{I}\), that is, the quotient space is a trivial linear space, which is trivially a normed algebra. We have thus shown that \(\algebra{A}/\algebra{I}\) is a normed algebra.
\end{proof}

\begin{lemma}{Quotient space by a closed subspace is a Banach space}{quotient_closed_banach}
    Let \(\algebra{A}\) be a Banach space. If \(\algebra{I}\) is a closed subspace, then \(\algebra{A}/\algebra{I}\) is a Banach space.
\end{lemma}
\begin{proof}
    Let \(\family{[x_n]}{n \in \mathbb{N}} \subset \algebra{A}/\algebra{I}\) be a sequence in \(\algebra{A}/\algebra{I}\). For all \(n,m,p \in \mathbb{N}\) and \(\eta > 0\) there exists \(\ell(n,m,p, \eta) \in \algebra{I}\) satisfying
    \begin{equation*}
        \norm{x_m - x_n + \ell(m,n,p,\eta)} \leq \norm{[x_m] - [x_n]} + \frac{\eta}{2^{p+1}},
    \end{equation*}
    otherwise for some \(\tilde{m}, \tilde{n}, \tilde{p} \in \mathbb{N}\) and \(\tilde{\eta} > 0\) we would have
    \begin{equation*}
        \forall \ell \in \algebra{I}: \norm{[x_{\tilde{m}}] - [x_{\tilde{n}}]} + \frac{\eta}{2^{\tilde{p} + 1}} < \norm{x_{\tilde{m}} - x_{\tilde{n}} + \ell},
    \end{equation*}
    contradicting the fact that \(\norm{[x_{\tilde{m}}] - [x_{\tilde{n}}]}\) is the greatest lower bound for \(\setc{\norm{x_{\tilde{m}} - x_{\tilde{n}} + \ell}}{\ell \in \algebra{I}}\).
    We now suppose the sequence is Cauchy, then for all \(\varepsilon> 0\) there exists \(N_{\varepsilon} > 0\) such that for all \(m, n > N_{\varepsilon}\) we have \(\norm{[x_m] - [x_n]} < \varepsilon\). Then, we may choose a subsequence \(\family{[x_{i_j}]}{j \in \mathbb{N}}\) satisfying
    \begin{equation*}
        \norm{[x_{i_j}] - [x_{i_k}]} \leq \frac{1}{2^{k+1}}\varepsilon
    \end{equation*}
    for all \(j > k\). As a result we have for all \(j > k\) and \(p \in \mathbb{N}\) that
    \begin{equation*}
        \norm{x_{i_j} - x_{i_k} + \ell(i_j, i_k, p)} \leq \left(\frac{1}{2^{k+1}} + \frac{1}{2^{p+1}}\right)\varepsilon \leq \frac{1}{2^{\max\set{k,p}}}\varepsilon.
    \end{equation*}

    For all \(m \in \mathbb{N}\) set
    \begin{equation*}
        y_m = x_{i_m} + \sum_{p = 1}^m \ell(i_p, i_{p-1}, p),
    \end{equation*}
    then \(y_m \in [x_{n_m}]\). For \(m > n\), we have
    \begin{equation*}
        y_m - y_n = \sum_{k = n+1}^m (y_k - y_{k-1}) = \sum_{k=n+1}^m \left[x_{i_k} - x_{i_{k-1}} + \ell(i_k, i_{k-1},k)\right],
    \end{equation*}
    which yields
    \begin{equation*}
        \norm{y_m - y_n} \leq \sum_{k=n+1}^m \norm{x_{i_k} - x_{i_{k-1}} + \ell(i_k, i_{k-1},k)} \leq \left(\sum_{k=n+1}^m \frac{1}{2^k}\right) \varepsilon< \left(\sum_{k=n+1}^\infty \frac{1}{2^k}\right) \varepsilon = \frac{1}{2^n}\varepsilon.
    \end{equation*}
    Hence \(\family{y_n}{n \in \mathbb{N}}\) is a Cauchy sequence in the Banach space \(\algebra{A}\), and it must converge against some \(\tilde{y} \in \algebra{A}\). It is also the case that \([x_n] \to [\tilde{y}]\), since
    \begin{equation*}
        \norm{[x_n] - [\tilde{y}]} = \norm{[y_n - \tilde{y}]} = \inf_{a\in \algebra{I}}{\norm{y_n - \tilde{y} + a}} \leq \norm{y_n - \tilde{y}},
    \end{equation*}
    which goes to zero as we take \(n\) sufficiently large.
\end{proof}

\begin{theorem}{Quotient space of a Banach *-algebra by a closed *-ideal}{closed_ideal_Bstar}
    Let \(\algebra{A}\) be a Banach *-algebra. If \(\algebra{I}\) is a closed *-ideal of \(\algebra{A}\), then \(\algebra{A}/\algebra{I}\) is a Banach *-algebra.
\end{theorem}
\begin{proof}
    \cref{lem:quotient_two_sided_associative_algebra,lem:quotient_self_adjoint_star,lem:quotient_closed_ideal_normed,lem:quotient_closed_normed,lem:quotient_closed_banach} show \(\algebra{A}/\algebra{I}\) is a Banach algebra, so it remains to show the Banach *-algebra property. Let \([x] \in \algebra{A}/\algebra{I}\), then
    \begin{equation*}
        \norm{[x]^*} = \inf_{j\in\algebra{I}}{\norm{x^* + j}}=\inf_{j\in\algebra{I}}{\norm{(x + j)^*}} = \inf_{j\in\algebra{I}}{\norm{x+j}} = \norm{[x]}
    \end{equation*}
    follows from self adjointness of \(\algebra{I}\). That is, \(\algebra{A}/\algebra{I}\) is a Banach *-algebra.
\end{proof}

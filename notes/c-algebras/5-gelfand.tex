% vim: spl=en_us
\section{Gelfand representation}
Let \(\algebra{A}\) be a unital C*-algebra and let \(a \in \algebra{A}\) be a self-adjoint operator. We'll denote by \(\mathcal{C}(\sigma(a), \mathbb{C})\) the set of \todo[continuous] complex-valued maps from defined on the spectrum of \(a\) and by \(\norm{\noarg}_{\infty}\) the supremum norm, that is, \(\left(\mathcal{C}(\sigma(a), \mathbb{C}), \norm{\noarg}_{\infty}\right)\) is a Banach *-algebra. Since the set of polynomials defined on \(\sigma(a)\) is a subalgebra of \(\mathcal{C}(\sigma(a), \mathbb{C})\), we may improve on the spectral mapping theorem and show the supremum norm of a such defined polynomial \(\varphi\) is equal to the norm \(\norm{\varphi(a)}\) in the C*-algebra.
\begin{proposition}{Norm of a polynomial defined on the spectrum of an operator}{norm_polynomial}
    Let \(\algebra{A}\) be a unital C*-algebra and let \(a \in \algebra{A}\) be a self-adjoint operator. Then \(\norm{\varphi}_\infty = \norm{\varphi(a)}\) for all polynomials \(\varphi : \sigma(a) \to \mathbb{C}\).
\end{proposition}
\begin{proof}
    Let us write \(\varphi(z) = \sum_{k = 0}^{n} c_k z^k\) for all \(z \in \sigma(a)\). It follows from self-adjointness of \(a\) that
    \begin{equation*}
        \varphi(a)^*\varphi(a) = \sum_{k=0}^n \conj{c_k} a^k \sum_{\ell = 0}^n c_{\ell} a^\ell = \sum_{k = 0}^n \sum_{\ell = 0}^n \conj{c_k} c_k a^{\ell + k} = (\conj{\varphi}\varphi)(a).
    \end{equation*}
    \cref{thm:spectral_radius_cstar,thm:spectral_mapping} yield
    \begin{equation*}
        \norm{\varphi(a)^*\varphi(a)} = \norm{(\conj{\varphi}\varphi)(a)} = \sup_{\lambda \in \sigma\left[(\conj{\varphi}\varphi)(a)\right]}{\abs{\lambda}} = \sup_{\lambda \in \sigma(a)}{\abs*{(\conj{\varphi}\varphi)(\lambda)}} = \sup_{\lambda\in \sigma(a)}{\abs{\varphi(\lambda)}^2} = \norm{\varphi}_{\infty}^2,
    \end{equation*}
    then the proposition follows from the C* property, \(\norm{\varphi(a)}^2 = \norm{\varphi(a)^*\varphi(a)} = \norm{\varphi}_{\infty}^2\).
\end{proof}

\begin{proposition}{Norm of a power of a normal operator}{norm_normal}
    Let \(\algebra{A}\) be a unital C*-algebra. If \(a \in \algebra{A}\) is a normal operator, then \(\norm{a}^n = \norm{a^n}\) for all \(n \in \mathbb{N}\).
\end{proposition}
\begin{proof}
    Let \(n \in \mathbb{N}\), then for a normal operator we have \(\norm{a^n}^2 = \norm{(a^*)^n a^n} = \norm{(a^*a)^n}\). Since \(a^*a\) is self-adjoint, we have by \cref{prop:norm_polynomial} that \(\norm{(a^*a)^n} = \norm{a^*a}^n = \norm{a}^{2n}\). This yields \(\norm{a^n} = \norm{a^n}\) as desired.
\end{proof}

We recall the set of polynomials is dense in the set of continuous functions defined on a compact set with respect to the topology induced by the supremum norm. Indeed, we show the standard \nameref{thm:weierstrass_polynomial} for maps defined in the compact interval \([0,1]\), which may be generalized to continuous maps defined in any compact subset of \(\mathbb{R}\).
\begin{theorem}{Weierstrass polynomial approximation theorem}{weierstrass_polynomial}
    Let \(f \in \mathcal{C}\left([0,1], \mathbb{C}\right)\) be a continuous function. Then the sequence \(\family{\varphi_n}{n \in \mathbb{N}}\subset\mathcal{C}\left([0,1], \mathbb{C}\right)\) of polynomials defined by
    \begin{equation*}
        \varphi_n(x) = \sum_{k = 0}^n \binom{n}{k} f\left(\frac{k}{n}\right) x^k (1 - x)^{n - k}
    \end{equation*}
    converges uniformly to \(f\), that is, for all \(\varepsilon > 0\) there exists \(N \in \mathbb{N}\) such that \(\norm{f - \varphi_n}_\infty < \epsilon\) for all \(n \geq N\).
\end{theorem}
\begin{proof}
    Let \(x, y \in [0,1]\) and \(n \in \mathbb{N}_0\), then
    \begin{equation*}
        nx(x + y)^{n-1} = x\diffp*{(x + y)^n}{x} = x\diffp*{\sum_{k = 0}^n\binom{n}{k} x^k y^{n-k}}{x} = \sum_{k=0}^n \binom{n}{k} k x^{k}y^{n - k}
    \end{equation*}
    and
    \begin{equation*}
        n(n-1)x^2(x + y)^{n-2} = x^2\diffp*[2]{(x + y)^n}{x} = x^2\diffp*{ \sum_{k=0}^n \binom{n}{k} k x^{k-1}y^{n - k}}{x} =  \sum_{k = 0}^n \binom{n}{k} k (k-1) x^{k} y^{n - k}.
    \end{equation*}
    Writing \(b^n_k(x) = \binom{n}{k} x^k (1 - x)^{n - k}\), we get
    \begin{equation*}
        \sum_{k = 0}^n b^n_k(x) = 1,
        \quad
        \sum_{k=0}^n k b^n_k(x) = n x,
        \quad\text{and}\quad
        \sum_{k=0}^n k(k-1) b^n_k(x) = n(n-1)x^2
    \end{equation*}
    for all \(x \in [0,1]\). This yields
    \begin{align*}
        \sum_{k = 0}^n (k - nx)^2 b^n_k(x) &= \sum_{k = 0}^n k^2 b^n_k(x) - 2nx\sum_{k=0}^n kb^n_k(x) + n^2 x^2 \sum_{k = 0}^n b^n_k(x)\\
                                           &= \sum_{k = 0}^n k(k-1) b^n_k(x) + \sum_{k=0}^n k b^n_k(x) - n^2 x^2 \\
                                           &= n(n-1)x^2 + nx - n^2x^2\\
                                           &= nx(1 - x).
    \end{align*}

    \todo[Since \(f\) is defined on a compact set and continuous, it is uniformly continuous.] Let \(\varepsilon > 0\), then there exists \(\delta > 0\) such that \(\abs{f(x) - f(y)} < \frac12\epsilon\) whenever \(\abs{x - y} < \delta\). Let \(x \in [0,1]\), and we consider \(S_n = \setc{m \in \mathbb{N}_0}{m \leq n \land \abs{m - xn} < \delta n}\) and \(R_n = \setc{m \in \mathbb{N}_0}{m \leq n \land \abs{m - xn} \geq \delta n}\), then
    \begin{align*}
        \abs*{f(x) - \varphi_n(x)} &= \abs*{\sum_{k = 0}^n \left[f(x) - f\left(\frac{k}{n}\right)\right]b^n_k(x)}\\
                                  &\leq \sum_{k = 0}^n \abs*{f(x) - f\left(\frac{k}{n}\right)}b^n_k(x)\\
                                  &= \sum_{k \in S_n} \abs*{f(x) - f\left(\frac{k}{n}\right)}b^n_k(x) + \sum_{k \in R_n} \abs*{f(x) - f\left(\frac{k}{n}\right)}b^n_k(x)\\
                                  &< \frac12\varepsilon \sum_{k \in S_n} b^n_k(x) + 2 \norm{f}_\infty \sum_{k \in R_n} b^n_k(x)\\
                                  &\leq \frac12 \varepsilon \sum_{k = 0}^n b^n_k(x) + \frac{2\norm{f}_\infty}{n^2 \delta^2}\sum_{k=0}^n (k - nx)^2 b^n_k(x)\\
                                  &= \frac12 \varepsilon + \frac{2 \norm{f}_{\infty}x(1 - x)}{n \delta^2}\\
                                  &\leq \frac12 \varepsilon + \frac{\norm{f}_\infty}{2 \delta^2 n}.
    \end{align*}
    If \(\norm{f}_\infty\), we are done, so we may assume \(\norm{f}_\infty > 0\). We set \(N = \ceil*{\frac{\norm{f}_\infty}{\delta^2 \varepsilon}}\), then
    \begin{equation*}
        n \geq N \implies \forall x \in [0,1]: \abs{f(x) - \varphi_n(x)} < \varepsilon,
    \end{equation*}
    that is, \(\norm{f - \varphi_n}_\infty < \varepsilon\) for all \(n \geq N\).
\end{proof}

Let us denote by \(\mathcal{P}(\sigma(a)) \subset \mathcal{C}(\sigma(a), \mathbb{C})\) the subalgebra of polynomials defined on the spectrum of the self-adjoint operator \(a \in \algebra{A}\). Then by the generalized Weierstrass approximation theorem, we have \(\cl{\mathcal{P}(\sigma(a))} = \mathcal{C}(\sigma(a))\). \cref{prop:norm_polynomial} shows us the linear map
\begin{align*}
    \tilde{\Phi}_a : \mathcal{P}(\sigma(a)) &\to \algebra{A}\\
                                    \varphi &\mapsto \varphi(a)
\end{align*}
is isometric, thus bounded. \cref{thm:blt} guarantees us the uniqueness and existence of a linear map
\begin{align*}
    \Phi_a : \mathcal{C}\left(\sigma(a), \mathbb{C}\right) &\to \algebra{A}\\
                                                         f &\mapsto f(a)
\end{align*}
that isometrically extends \(\tilde{\Phi}_a\) to \(\mathcal{C}\left(\sigma(a), \mathbb{C}\right)\) by defining
\begin{equation*}
    f(a) = \lim_{n\to\infty} \varphi_n(a),
\end{equation*}
where \(\family{\varphi_n}{n\in \mathbb{N}} \subset \mathcal{P}(\sigma(a))\) is \emph{any} sequence of polynomials that uniformly converges against \(f\).

\begin{theorem}{Gelfand's homomorphism in C*-algebras}{gelfand_homomorphism}
    Let \(\algebra{A}\) be unital C*-algebra, let \(a \in \algebra{A}\) be a self-adjoint operator and consider the above defined map \(\Phi_a : \mathcal{C}\left(\sigma(a), \mathbb{C}\right) \to \algebra{A}\). Then \(\Phi_a\) is a *-homomorphism and for all \(f \in \mathcal{C}\left(\sigma(a), \mathbb{C}\right)\) we have
    \begin{enumerate}[label=(\alph*)]
        \item if \(f(\lambda) \geq 0\) for all \(\lambda \in \sigma(a)\), then \(\sigma(\Phi_a(f)) \subset [0, \infty)\);
        \item if \(\family{f_n}{n\in \mathbb{N}}\subset \mathcal{C}\left(\sigma(a), \mathbb{C}\right)\) is a sequence that uniformly converges against \(f\), then \(\family{\Phi_a(f_n)}{n\in \mathbb{N}} \subset \algebra{A}\) converges against \(\Phi_a(f)\); and
        \item \(f(\sigma(a)) = \sigma(\Phi_a(f))\).
    \end{enumerate}
\end{theorem}
\begin{proof}

\end{proof}

\begin{corollary}
    The set \(\algebra{F} = \setc{\Phi_a(f)}{f \in \mathcal{C}\left(\sigma(a), \mathbb{C}\right)}\) is a unital abelian C*-subalgebra of \(\algebra{A}\).
\end{corollary}
\begin{proof}

\end{proof}

% vim: spl=en_us
\section{Positive elements of a C*-algebra}
We now show every non-zero self-adjoint operator in a C*-algebra with a positive spectrum admits a unique square root. Let us denote the real half-lines by \(\mathbb{R}_+ = [0,\infty)\) and \(\mathbb{R}_- = (-\infty, 0]\).
\begin{definition}{Positive element of an involutive algebra}{positive}
    Let \(\algebra{A}\) be a *-algebra. A \emph{positive element} \(a \in \algebra{A}\) is self-adjoint and its spectrum lies in the positive half-line, \(\sigma(a) \subset \mathbb{R}_+\). The set of all positive elements is denoted by \(\algebra{A}_+\).
\end{definition}

\begin{proposition}{Norm of a positive operator is in its spectrum}{norm_positive}
    Let \(\algebra{A}\) be a unital C*-algebra. If \(a \in \algebra{A}_+\), then \(\norm{a} \in \sigma(a)\).
\end{proposition}
\begin{proof}
    By \cref{thm:spectra_self_adjoint} we have \(\sigma(a) \subset [0, \norm{a}]\). Suppose, by contradiction, there exists \(M \in [0, \norm{a})\) such that \(\sigma(a) \subset [0, M]\). Then by \cref{thm:spectral_radius_cstar}, we have
    \begin{equation*}
        \norm{a} = r(a) = \sup_{\lambda \in \sigma(a)}{\abs{\lambda}} \leq \sup_{\lambda \in [0, M]} \abs{\lambda} = M < \norm{a},
    \end{equation*}
    which shows there exists no such \(M\).
\end{proof}

\begin{lemma}{Positive elements of a C*-algebra has no pair of distinct opposite operators}{positive_salient}
    Let \(\algebra{A}\) be a unital C*-algebra. Then \(\algebra{A}_+ \cap (-\algebra{A}_+) = \set{0}\).
\end{lemma}
\begin{proof}
    Notice \(0 \in \algebra{A}\) is a positive element, as it is self-adjoint with \(\sigma(0) = \set{0} \subset \mathbb{R}_+\). As \(-0 = 0\), we have \(0 \in \algebra{A}_+ \cap (-\algebra{A}_+)\). Let \(a \in \algebra{A}_+ \cap (-\algebra{A}_+)\), then \(\sigma(a) \subset \mathbb{R}_+\) and there exists \(b \in \algebra{A}_+\) such that \(a = -b\). \cref{thm:spectral_mapping}, yields \(\sigma(a) = -\sigma(b) \subset \mathbb{R}_-\). As \(\sigma(a)\in \algebra{A}_+\), we have \(\sigma(a) \subset \mathbb{R}_+\), then it follows that \(\sigma(a) = \set{0}\). That is, \(r(a) = 0\), and we conclude from \cref{thm:spectral_radius_cstar} that \(a = 0\).
\end{proof}
\begin{corollary}
    Let \(\algebra{A}\) be a unital C*-algebra. If \(a, b\in \algebra{A}_+\) such that \(a + b = 0\), then \(a = b = 0\).
\end{corollary}
\begin{proof}
    If \(a + b = 0\), then \(\algebra{A}_+ \ni a = -b \in (-\algebra{A}_+)\), that is, \(a \in \algebra{A}_+ \cap (-\algebra{A}_+)\). We conclude \(a = 0\) from \cref{lem:positive_salient}, hence \(b = 0\).
\end{proof}

\begin{lemma}{If two positive operators commute, then their product is positive}{positive_commute}
    Let \(\algebra{A}\) be a unital C*-algebra. If \(a,b \in \algebra{A}_+\) such that \(ab = ba\), then \(ab \in \algebra{A}_+\).
\end{lemma}
\begin{proof}
    Gelfand homomorphism yields \(c,d \in \algebra{A}\) such that \(c^2 = a\) and \(d^2 = d\), where \(c\) and \(d\) are obtained by the limit of sequences of polynomials of \(a\) and \(b\), hence \(cd = dc\) as \(a\) and \(b\) commute. Then \(ab = c^2 d^2 = (cd)^2\), and we conclude \(\sigma(ab) \subset [0, \norm{cd}^2] \subset \mathbb{R}_+\) from \cref{thm:spectral_mapping}.
\end{proof}

\begin{theorem}{Square root in C*-algebra}{square_root_cstar}
    Let \(\algebra{A}\) be a unital C*-algebra. If \(u \in \algebra{A}\setminus\set{0}\) is self-adjoint, then the following statements are equivalent:
    \begin{enumerate}[label=(\alph*)]
        \item \(u \in \algebra{A}_+\);
        \item \(\norm*{\unity - \norm{u}^{-1}u} \leq 1\); and
        \item there exists \(v \in \algebra{A}\) self-adjoint such that \(v^2 = u\).
    \end{enumerate}
    In addition, if \(u\) is positive, then there exists a unique \(w \in \algebra{A}_+\) such that \(w^2 = u\), and we say \(w = \sqrt{u}\) is \emph{the positive square root of \(u\)}.
\end{theorem}
\begin{proof}
    Suppose (a) and consider the polynomial \(\varphi(z) = 1 - \frac{z}{\norm{u}}\). By \cref{thm:spectral_mapping}, we have \(\sigma(\varphi(u)) = \varphi(\sigma(u)) \subset \varphi([0, \norm{u}]) = [0,1]\). Then, (b) follows from \cref{thm:spectral_radius_cstar}. Supposing (b), (c) follows from \cref{thm:square_root}. Supposing (c), we have \(\sigma(u) = \sigma(v^2) = \sigma(v)^2 \subset [0, \norm{v}^2]\) by \cref{thm:spectral_mapping,thm:spectral_radius_cstar}, hence \(u \in \algebra{A}_+\), and we conclude (a).

    If \(u \in \algebra{A}_+\setminus\set{0}\), then \(\sigma(u) \subset [0, \norm{u}]\). The map \(\id{\sigma(u)} : \sigma(u) \to \sigma(u)\) is continuous and so is the map \(\sqrt{\noarg} : \mathbb{R} \to \mathbb{R}\), then the map \(\psi = \restrict{\sqrt{\noarg}}{\sigma(u)} \circ \id{\sigma(u)} : \sigma(u) \to \mathbb{R}\) is continuous by \cref{prop:restriction_map} and satisfies \(\psi(\lambda) \geq 0\) for all \(\lambda \in \sigma(u)\). By \cref{thm:gelfand_homomorphism}, we know \(w = \Phi_u(\psi)\) is self-adjoint and satisfies both \(\sigma(w) \subset \mathbb{R}_+\) and
    \begin{equation*}
        w^2 = \Phi_u(\psi)^2 = \Phi_u(\id{\sigma(u)}) = u.
    \end{equation*}
    It remains to show \(w \in \algebra{A}_+\) is the only such positive element of \(\algebra{A}\).

    Let \(v \in \algebra{A}_+\) such that \(v^2 = u\), then \(v\) commutes with \(u\), as \(uv = v^3 = vu\). As \(w\) is the limit of polynomials of \(u\), it commutes with \(u\), and therefore it commutes with any operator that commutes with \(u\) therefore, in particular, we have \(wv = vw\). This yields
    \begin{equation*}
        0 = (u - u)(v - w) = (v^2 - w^2)(v - w) = (v + w)(v - w)^2 = v(v - w)^2 + w(v - w)^2.
    \end{equation*}
    Notice \((v - w)^2 \in \algebra{A}_+\), then \cref{lem:positive_commute} guarantees \(v(v - w)^2\) and \(w(v - w)^2\) are positive. We have then written \(0\) as the sum of two positive operators, and we conclude by \cref{lem:positive_salient} that both must be equal to the zero operator. Then,
    \begin{equation*}
        0 = v(v - w)^2 - w(v - w)^2 = (v - w)^3 \implies (v - w)^4 = 0,
    \end{equation*}
    hence
    \begin{equation*}
        \norm{(v - w)}^4 = \norm{(v - w)^2}^2 = \norm{(v - w)^4} = 0
    \end{equation*}
    follows from the self-adjointness of \(v - w\) and the C*-property. That is, \(v - w = 0\), which shows the uniqueness of the positive square root.
\end{proof}
\begin{corollary}
    Let \(\algebra{A}\) be a unital C*-algebra. If \(u \in \algebra{A}\) is self-adjoint and \(\norm{u} \leq 1\), then there exists a unique \(v \in \algebra{A}_+\) such that \(v^2 = \unity - u\), and we write \(v = \sqrt{\unity - u}\).
\end{corollary}
\begin{proof}
    We consider \(w = \unity - u\), then \(\norm{\unity - w} = \norm{u} \leq 1\), then it follows from \cref{thm:square_root} that there exists \(v \in \algebra{A}\) such that \(v^2 = w\). By \cref{thm:square_root_cstar}, we know \(w \in \algebra{A}_+\) and we may take \(v\) as the unique element of \(\algebra{A}_+\) such that \(v^2 = w\).
\end{proof}


\begin{theorem}{Set of positive elements of a C*-algebra is a closed convex salient cone}{positive_cone}
    Let \(\algebra{A}\) be a unital C*-algebra. Then
    \begin{enumerate}[label=(\alph*)]
        \item \(\algebra{A}_+\) is a salient cone, that is, if \(u \in \algebra{A}_+\) and \(\lambda \in \mathbb{R}_+\), then \(\lambda u \in \algebra{A}_+\) and it has the property \(\algebra{A}_+ \cap (-\algebra{A}_+) = \set{0}\);
        \item \(\algebra{A}_+\) is convex, that is, if \(u, v \in \algebra{A}_+\) and \(\lambda \in [0,1]\), then \(\lambda u + (1 - \lambda)v \in \algebra{A}_+\); and
        \item \(\algebra{A}_+\) is closed in the uniform topology;
    \end{enumerate}
\end{theorem}
\begin{proof}
    Let \(u \in \algebra{A}_+\) and \(\lambda \in \mathbb{R}_+\), then \(\sigma(\lambda u) = \lambda \sigma(u) \subset [0, \lambda \norm{u}] \subset \mathbb{R}_+\) and \(\lambda u\) is self-adjoint, hence \(\lambda u \in \algebra{A}_+\), that is, \(\algebra{A}_+\) is a cone. We have shown that it is salient in \cref{lem:positive_salient}, thus (a) follows.

    Let us consider \(a \in \algebra{A}_+\) and \(\mu \geq \norm{a}\) with \(\mu \neq 0\). Then by \cref{thm:spectral_mapping} we have
    \begin{equation*}
        \sigma(\unity - \mu^{-1} a) = \setc*{1 - \frac{\lambda}{\mu}}{\lambda \in \sigma(a)} \subset \left[1 - \frac{\norm{a}}{\mu}, 1\right] \subset [0, 1],
    \end{equation*}
    and it follows from \cref{thm:spectral_radius_cstar} that \(\norm{\unity - \mu^{-1} a} \leq 1\). Let \(u,v \in \algebra{A}_+\) and let \(\lambda \in [0,1]\), then for any \(\kappa \geq \max\set{\norm{u}, \norm{v}}\) with \(\kappa \neq 0\), we have
    \begin{align*}
        \norm*{\unity - \kappa^{-1}\left[\lambda u + (1 - \lambda)v\right]}
        &= \norm*{\lambda\left(\unity - \kappa^{-1} u\right) + (1- \lambda) (\unity - \kappa^{-1}v)}\\
        &\leq \lambda \norm{\unity - \kappa^{-1}u} + (1 - \lambda) \norm{\unity - \kappa^{-1} v}\\
        &\leq \lambda + 1 - \lambda = 1,
    \end{align*}
    thus showing \(\sigma\left\{\unity - \kappa^{-1}\left[\lambda u + (1 - \lambda)v\right]\right\} \subset [-1,1]\) and as a result, \(\sigma\left[\lambda u + (1 - \lambda)v\right] \subset [0, 2 \kappa]\). That is, \(\lambda u + (1 - \lambda)v \in \algebra{A}_+\) and we conclude (b).

    Let \(\family{u_n}{n \in \mathbb{N}} \subset \algebra{A}_+\) be a sequence of positive operators that converge against \(u \in \algebra{A}\). We may assume without loss of generality that \(u_n \neq 0\) for all \(n \in \mathbb{N}\), for if the sequence were to converge against \(0 \in \algebra{A}_+\) there would be nothing to show and if the sequence converges against \(u \in \algebra{A}\setminus{0}\), we may take a convergent subsequence. For \(n \in \mathbb{N}\), we have \(u_n \in \algebra{A}_+\setminus\set{0}\), then \cref{thm:square_root_cstar} yields \(\norm*{\unity - \norm{u_n}^{-1}u_n}\leq 1\) and we have
    \begin{equation*}
        \norm*{\unity - \norm{u}^{-1}u} = \lim_{n \to \infty} \norm*{\unity - \norm{u_n}^{-1}u_n} \leq \lim_{n \to \infty} 1 = 1,
    \end{equation*}
    that is, \(u \in \algebra{A}_+\).
\end{proof}
\begin{corollary}
    Let \(\algebra{A}\) be a unital C*-algebra. If \(u,v\in \algebra{A}_+\), then \(u + v \in \algebra{A}_+\).
\end{corollary}
\begin{proof}
    Since \(\frac12 u + \frac12 v\) is a convex linear combination of positive operators, it is a positive operator. As \(\algebra{A}_+\) is a cone, \(2 \left(\frac12 u + \frac12 v\right) = u + v \in \algebra{A}_+\).
\end{proof}
\begin{corollary}
    Let \(\algebra{A}\) be a unital C*-algebra. If \(u \in \algebra{A}\) is such that \(-u^*u \in \algebra{A}_+\), then \(u = 0\).
\end{corollary}
\begin{proof}
    Since \(\sigma(u^*u) \setminus\set{0} = \sigma(uu^*)\setminus\set{0}\), we know that \(-u^*u\) is positive if and only if \(-uu^*\) is positive, then by \cref{thm:positive_cone}, we know \(\frac12 uu^* + \frac12 u^*u \in -\algebra{A}_+\). We define the self adjoint operators \(x = \frac12 (u + u^*)\) and \(y = \frac1{2i}(u - u^*)\), with
    \begin{equation*}
        x^2 + y^2 = \frac12(u^*u + uu^*) \in -\algebra{A}_+.
    \end{equation*}
    Notice \(x^2, y^2 \in \algebra{A}_+\), then by the previous corollary \(x^2 + y^2 \in \algebra{A}_+\). Since \(\algebra{A}_+\) is a salient cone, we have \(x^2 = y^2 = 0\). The C* property then yields \(x = y = 0\), thus showing \(u = x + iy = 0\).
\end{proof}

We will now show every positive element of a C*-algebra is of the form \(x^*x\). First we show the following decomposition result.
\begin{lemma}{Orthogonal decomposition of an operator}{orthogonal_decomposition_cstar}
    Let \(\algebra{A}\) be a unital C*-algebra. If \(u \in \algebra{A}\) is self-adjoint, then there exists unique positive operators \(u_+, u_- \in \algebra{A}_+\) such that \(u = u_+ - u_-\) and \(u_+u_- = u_-u_+ = 0\).
\end{lemma}
\begin{proof}
    We consider the continuous maps
    \begin{align*}
        f_+ : \sigma(u) &\to \mathbb{R}&
        f_- : \sigma(u) &\to \mathbb{R}\\
                \lambda &\mapsto \frac12\left(\abs{\lambda} + \lambda\right)&
                \lambda &\mapsto \frac12\left(\abs{\lambda} - \lambda\right)
    \end{align*}
    which verify
    \begin{equation*}
        (f_+\cdot f_-)(\lambda) = f_+(\lambda)f_-(\lambda) = \frac14 \left(\abs{\lambda}^2 - \lambda^2\right) = 0,
    \end{equation*}
    \begin{equation*}
        (f_+ - f_-)(\lambda) = f_+(\lambda) - f_-(\lambda) = \lambda,
    \end{equation*}
    and
    \begin{equation*}
        (f_+ + f_-)(\lambda) = f_+(\lambda) + f_-(\lambda) = \abs{\lambda} = \sqrt{\lambda^2}
    \end{equation*}
    for all \(\lambda \in \sigma(u)\), that is, \(f_+ f_- = \id{\sigma(u)}\) and \(f_+ - f_- = 0\). We define \(u_+ = \Phi_u(f_+)\) and \(\Phi_u(f_-)\), then \(u_+ - u_- = u\) and \(u_+ u_- = 0\). As the image of Gelfand homomorphism is a commutative C*-subalgebra, we also have \(u_- u_+ = 0\).

    Let \(\tilde{u}_+, \tilde{u}_- \in \algebra{A}_+\) such that \(\tilde{u}_+ - \tilde{u}_- = u\) and \(\tilde{u}_+\tilde{u}_- = \tilde{u}_-\tilde{u}_+ = 0\). Then
    \begin{equation*}
        u^2 = \left(\tilde{u}_+ - \tilde{u}_-\right)^2 = \tilde{u}_+^2 + \tilde{u}_-^2 = \left(\tilde{u}_+ + \tilde{u}_-\right)^2 \implies u_+ + u_- = \sqrt{u^2} = \tilde{u}_+ + \tilde{u}_-,
    \end{equation*}
    which yields \(\tilde{u}_+ = u_+\) and \(\tilde{u}_- = u_-\).
\end{proof}

\begin{theorem}{Characterization of positive elements of a C*-algebra}{positive_cstar}
    Let \(\algebra{A}\) be a unital C*-algebra. Then \(\algebra{A}_+ = \setc{u^*u}{u \in \algebra{A}}\).
\end{theorem}
\begin{proof}
    Let \(u \in \algebra{A}_+\), then \(\sqrt{u} \in\algebra{A}_+\) satisfies \(\sqrt{u}^* \sqrt{u} = \sqrt{u}^2 = u\).

    Let \(a \in \algebra{A}\) and let \(b = a^*a\), with orthogonal decomposition \(b_+, b_- \in \algebra{A}_+\). Consider \(c = ab_-\), then \(- c^*c = - b_- a^* a b_- = b_-bb_- = (b_-)^3 \in \algebra{A}_+\). By \cref{thm:positive_cone}, we know \(c = 0\), which yields \(0 = a^*c = b b_- = -(b_-)^2\), and we conclude \(b_- = 0\) by the C* property. That is, \(a^*a = b_+ \in \algebra{A}_+\).
\end{proof}

\begin{corollary}
    Let \(\algebra{A}\) be a unital C*-algebra. If \(a \in \algebra{A}\), then \(\norm{a}^2 \in \sigma(a^*a)\).
\end{corollary}
\begin{proof}
    Since \(a^*a \in \algebra{A}_+\), then \(\norm{a}^2 = \norm{a^*a} \in \sigma(a^*a)\).
\end{proof}

Recall \cref{prop:polarization_star_algebra}, which shows we may write any operator \(a \in \algebra{A}\) of a unital *-algebra as
\begin{equation*}
    a = \frac14 \sum_{k = 0}^3 i^k(a + i^k \unity)^*(a + i^k \unity).
\end{equation*}
In a C*-algebra, this shows every operator can be written as a linear combination of four positive operators.
\begin{proposition}{Unitary decomposition}{unitary_decomposition}
    Let \(\algebra{A}\) be a unital C*-algebra. If \(a \in \algebra{A}\) is self-adjoint, then there exist \(u_+, u_- \in \algebra{A}\) unitary such that \(a = \frac{\norm{a}}{2}(u_+ + u_-)\). If \(b \in \algebra{A}\), then for \(k \in \set{1,2,3,4}\) there exist \(u_k \in \algebra{A}\) unitary and \(\beta_k \in \setc{\lambda \in \mathbb{C}}{\abs{\lambda} \leq \frac12 \abs{b}}\) such that \(b = \sum_{k = 1}^4 \beta_k u_k\).
\end{proposition}
\begin{proof}
    Let \(a \in \algebra{A}\) be a self-adjoint operator, then we may assume \(a \neq 0\), since \(0 = \frac{0}{2}(\unity - \unity)\). Notice \(\norm{\norm{a}^{-2}a^2} = 1\) by the C*-property, then there exists \(\sqrt{\unity - \norm{a}^{-2}a^2} \in \algebra{A}_+\). We define
    \begin{equation*}
        u_\pm = \norm{a}^{-1} a \pm i \sqrt{\unity - \norm{a}^{-2} a^2},
    \end{equation*}
    then \(a = \frac{\norm{a}}{2}(u_+ + u_-)\). By self-adjointness of \(a\) and the square root, we have \(u_\pm^* = u_\mp\), then
    \begin{equation*}
        u_\pm^*u_\pm = \left(\norm{a}^{-1} a \mp i \sqrt{\unity - \norm{a}^{-2}a^2}\right)\left(\norm{a}^{-1} a \pm i \sqrt{\unity - \norm{a}^{-2}a^2}\right) = \norm{a}^{-2}a^2 + \sqrt{\unity - \norm{a}^{-2}a^2}^2 = \unity
    \end{equation*}
    and
    \begin{equation*}
        u_\pm u_\pm^* = \left(\norm{a}^{-1} a \pm i \sqrt{\unity - \norm{a}^{-2}a^2}\right)\left(\norm{a}^{-1} a \mp i \sqrt{\unity - \norm{a}^{-2}a^2}\right) = \norm{a}^{-2}a^2 + \sqrt{\unity - \norm{a}^{-2}a^2}^2 = \unity,
    \end{equation*}
    that is, \(u_\pm\) is unitary.

    Let \(b \in \algebra{A}\) be an operator, then \(b = a_1 + ia_2\), where \(a_1 = \frac{1}{2}(b + b^*)\) and \(a_2 = \frac{1}{2i}(b - b^*)\) are self-adjoint. By the previous result, we may decompose \(a_1\) and \(a_2\) with two unitary operators, \(a_j = \frac12\norm{a_j}(u_j^+ + u_i^-)\) for \(j \in \set{1,2}\), then
    \begin{equation*}
        b = \frac14\norm{b + b^*} (u_1^+ + u_1^-) + \frac14\norm{b - b^*} (u_2^+ + u_2^-),
    \end{equation*}
    with \(\frac14\norm{b \pm b^*} \leq \frac14\norm{b} + \frac14\norm{b^*} = \frac12 \norm{b}\).
\end{proof}

% In \cref{lem:orthogonal_decomposition_cstar}, we have used the Gelfand homomorphism to define the modulus of a self-adjoint operator. In fact, we may generalize such a map for any operator in a C*-algebra.
% \begin{proposition}{Polar decomposition}{polar_decomposition_cstar}
%     Let \(\algebra{A}\) be a unital C*-algebra. The modulus of an operator is defined by the map
%     \begin{align*}
%         \abs{\noarg} : \algebra{A} &\to \algebra{A}_+\\
%                                  a &\mapsto \sqrt{a^*a}.
%     \end{align*}
%     Then for every operator \(a \in \algebra{A}\), there exists an unitary operator \(u \in \algebra{A}\) in the C*-subalgebra generated by \(a\) and \(a^*\) such that \(a = u\abs{a}\).
% \end{proposition}
% \begin{proof}
%     The map is well-defined as \(a^*a \in \algebra{A}_+\) by \cref{thm:positive_cstar} and there exists a unique positive square root for every positive operator. Notice
% \end{proof}

\subsection{Partial order induced by positive elements of a C*-algebra}
The set of positive elements \(\algebra{A}_+\) of a C*-algebra \(\algebra{A}\) induces naturally a partial order on \(\algebra{A}\) defined as \(a \preceq b\) if \(b - a \in \algebra{A}_+\). Reflexivity follows from \(0 \in \algebra{A}_+\), anti-symmetry follows from \cref{lem:positive_salient}, and transitivity follows from \(\algebra{A}_+\) being a convex cone. In fact, the set of self-adjoint operators is linearly ordered by this partial order, and we write \(a \leq b\) if \(a\) and \(b\) are self-adjoint with \(a \preceq b\). If \(b \leq a\), we may also write \(a \geq b\), and in addition to \(b \neq a\), we write \(a > b\) and, analogously, \(b < a\).

\begin{proposition}{Congruence transformations are order preserving}{congruence_order}
    Let \(\algebra{A}\) be a unital C*-algebra. If \(a,b \in \algebra{A}\) are self-adjoint with \(a \geq b\), then \(c^*ac \geq c^* b c\) for all \(c \in \algebra{A}\).
\end{proposition}
\begin{proof}
    If \(u\in \algebra{A}\) is self-adjoint, then \((c^*uc)^* = c^* u^* c = c^* u c\) for any \(c \in \algebra{A}\). Then, \(c^*ac - c^*bc = c^*(a - b)c = c^*\sqrt{a - b} \sqrt{a - b}c = (\sqrt{a - b} c)^* (\sqrt{a - b}c) \in \algebra{A}_+\).
\end{proof}

\begin{proposition}{Properties of the order induced by positive elements}{properties_order}
    Let \(\algebra{A}\) be a unital C*-algebra.
    \begin{enumerate}[label=(\alph*)]
        \item If \(u\in\algebra{A}_+\), then \(\norm{u} \unity \geq u \geq 0\).
        \item If \(u \in \algebra{A}_+\), then \(\norm{u} u \geq u^2 \geq 0\).
        \item If \(u,v \in \algebra{A}_+\) with \(u \geq v\), then \(\norm{u}\geq \norm{v}\).
    \end{enumerate}
\end{proposition}
\begin{proof}
    If \(u \in \algebra{A}_+\), then \(u - 0 \in \algebra{A}_+\), that is, \(u \geq 0\). \cref{thm:spectral_mapping} yields
    \begin{equation*}
        \sigma(\norm{u} \unity - u) = \setc{\norm{u} - \lambda}{\lambda \in \sigma(u)} \subset [0, \norm{u}],
    \end{equation*}
    and
    \begin{equation*}
        \sigma(\norm{u} u - u^2) = \setc{\norm{u}\lambda - \lambda^2}{\lambda \in \sigma(u)} \subset \setc{\norm{u}\lambda - \lambda^2}{\lambda \in [0, \norm{u}]} \subset \left[0, \frac{\norm{u}^2}{4}\right],
    \end{equation*}
    that is, \(\norm{u} \unity - u \in \algebra{A}_+\) and \(\norm{u} u - u^2 \in \algebra{A}_+\), and we conclude (a) and (b).

    If \(0 \leq v \leq u\), then by transitivity we have \(v \leq \norm{u} \unity\), then
    \begin{equation*}
        \sigma(\norm{u}\unity - v) = \setc{\norm{u} - \lambda}{\lambda \in \sigma(v)} \subset \mathbb{R}_+,
    \end{equation*}
    which implies \(\norm{u} - \lambda \geq 0\) for all \(\lambda \in \sigma(v)\). In particular, by \cref{prop:norm_positive}, we have thus shown \(\norm{u} - \norm{v} \geq 0\), and we conclude (c).
\end{proof}

\begin{proposition}{Positive resolvent and order relations}{positive_resolvent}
    Let \(\algebra{A}\) a unital C*-algebra. If \(a \in \algebra{A}_+\) and \(\nu \in \mathbb{R}_+\), then \(\unity + \nu a \in \invertible{\algebra{A}}\) and \((\unity + \nu a)^{-1} \in \algebra{A}_+\). Moreover, if \(b \in \algebra{A}\) with \(b \leq a\), then \((\unity +\nu b)^{-1} \geq (\unity + \nu a)^{-1}\).
\end{proposition}
\begin{proof}
    We may assume \(\nu > 0\), since \(\unity = \unity^{-1} \in \algebra{A}_+\) and \(\unity \geq \unity\). Then \(-\nu^{-1} \in \mathbb{R}_-\) and we conclude from positivity that \(-\nu^{-1} \notin \sigma(a)\). That is, \(-\nu^{-1} \unity - a \in \invertible{\algebra{A}}\) which yields \(\unity + \nu a \in \invertible{\algebra{A}}\). \cref{prop:spectrum_inverse,thm:spectral_mapping} yields
    \begin{equation*}
        \sigma\left((\unity + \nu a)^{-1}\right) = \setc{\mu^{-1}}{\mu \in \sigma(\unity + \nu a)} = \setc*{\frac1{1 + \mu \nu}}{\mu \in \sigma(a)} \subset \mathbb{R}_+,
    \end{equation*}
    that is, \((\unity + \nu a)^{-1} \in \algebra{A}_+\).

    Notice \(\unity + \nu b \in \algebra{A}_+\), then there exists a unique positive square root \(\sqrt{\unity + \nu b} \in \algebra{A}_+\) and by the previous result, we have \(\sqrt{\unity + \nu b}^{-1} = \sqrt{(\unity + \nu b)^{-1}} \in \algebra{A}_+\). Since the positive square root \(\sqrt{(\unity + \nu b)^{-1}}\) is in the C*-subalgebra generated by \(\unity\) and \((\unity + \nu b)^{-1}\), it commutes with \((\unity + \nu b)^{-1}\) and therefore with \(\unity + \nu b\) by \cref{prop:resolvent_commute}. Then \cref{prop:congruence_order} yields
    \begin{equation*}
        a \geq b \implies \unity + \nu a \geq \unity + \nu b \implies  u = \sqrt{(\unity + \nu b)^{-1}} (\unity + \nu a) \sqrt{(\unity + \nu b)^{-1}} \geq \unity.
    \end{equation*}
    Notice \(u \in \invertible{\algebra{A}}\) with \(u^{-1} = \sqrt{\unity + \nu b} (\unity + \nu a)^{-1} \sqrt{\unity + \nu b}\), then \(\sigma(u) \subset [1, \infty)\), which yields \(\sigma(u^{-1}) \subset (0, 1]\). That is, \(u^{-1} \leq \unity\) and we have
    \begin{equation*}
        \sqrt{\unity + \nu b} (\unity + \nu a)^{-1} \sqrt{\unity + \nu b} \leq \unity \implies (\unity + \nu a)^{-1} \leq (\unity + \nu b)
    \end{equation*}
    by \cref{prop:congruence_order}.
\end{proof}

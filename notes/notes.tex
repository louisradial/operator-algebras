% vim: spl=en_us
\documentclass[english]{notas}

% \usepackage[backend=biber,style=alphabetic,sorting=ynt]{biblatex}
% \addbibresource{references.bib}

\newcommand{\opnorm}[4]{\sup_{#2\in#3}{\frac{\norm*{#1(#2)}_{#4}}{\norm*{#2}_{#3}}}}
\newcommand\hilbert{\mathscr{H}}
\newcommand\riesz{\mathscr{R}}
\newcommand\eval{\operatorname{ev}}
\newcommand\domain[1]{\mathcal{D}_{#1}}
\newcommand\range[1]{\mathcal{R}_{#1}}
\newcommand\graph[1]{\mathcal{G}_{#1}}
\newcommand\bounded{\ensuremath{\mathscr{B}}}
\newcommand\algebra[1]{\mathfrak{#1}}
\newcommand\invertible[1]{\operatorname{Inv}(#1)}
\DeclareMathOperator*{\slim}{s-lim}
\DeclareMathOperator*{\wlim}{w-lim}
\newcommand\wto{\rightharpoonup}
\DeclareMathOperator{\adjoint}{adj}
\DeclareMathOperator{\cl}{cl}
\DeclareMathOperator{\inte}{int}

\title{Operator Algebras}
\author{Louis Bergamo Radial}
% \allowdisplaybreaks
\begin{document}
\maketitle
\tableofcontents
% preliminaries
% vim: spl=en_us
\chapter{Topological spaces and Metric spaces}

\section{Topological spaces}
Throughout these notes, the power set \(\mathbb{P}(X)\) denotes the collection of subsets of the set \(X\).
\begin{definition}{Topological space}{topology}
    A \emph{topological space} \topology{X} is a set \(X\) with a choice of \emph{topology} \(\tau_X\), a subset \(\tau_X\) of the power set \(\mathbb{P}(X)\) satisfying
    \begin{enumerate}[label=(\alph*)]
        \item \(X \in \tau_X\) and \(\emptyset \in \tau_X\);
        \item a finite intersection of elements of \(\tau_X\) is an element of \(\tau_X\); and
        \item an arbitrary union of elements of \(\tau_X\) is an element of \(\tau_X\).
    \end{enumerate}
    An \emph{open set} \(U\) on a topological space is a subset \(U \subset X\) such that \(U \in \tau_X\) and a \emph{closed set} \(V\) on a topological space is a subset \(V \subset X\) such that \(X \setminus V \in \tau_X\). A \emph{neighborhood of a point \(x \in X\)} is an open set \(U \in \tau_X\) that contains it, \(x \in U\).
\end{definition}
\begin{remark}
    Notice \(X\) and \(\emptyset\) are both open and closed sets in any topology on \(X\).
\end{remark}

We may show results for the union and intersection of closed sets similar to those of open sets.
\begin{lemma}{Complement of an arbitrary union}{complement_union}
    Let \(J\) be an arbitrary index set and let \(\family{E_j}{j\in J}\) be a family of subsets of \(X\). Then
    \begin{equation*}
        X \setminus \left(\bigcup_{j\in J} E_j\right) = \bigcap_{j \in J} \left(X \setminus E_j\right).
    \end{equation*}
\end{lemma}
\begin{proof}
    Let \(A = X \setminus\left(\bigcup_{j \in J} E_j\right)\) and let \(B = \bigcap_{j \in J} \left(X \setminus E_j\right)\). Then, we want to show \(A = B\).

    Let \(a \in A\), then \(a \notin \bigcup_{j \in J} E_j\). That is, there is no \(i \in J\) such that \(a \in U_i\). Hence, for all \(j \in J\) we have \(a \in a \setminus E_j\). As a result, \(a \in B\), thus \(A \subset B\).

    Let \(b \in B\), then \(b \in X \setminus E_j\) for all \(j \in J\). That is, for all \(j \in J\) we have \(b \notin E_j\), and as a result, \(b \notin \bigcup_{j \in J} E_j\). Hence, \(b \in A\), and we have \(B \subset A\).
\end{proof}

\begin{proposition}{Unions and intersections of closed sets}{closed_sets}
    Let \((X, \tau)\) be a topological space. Then
    \begin{enumerate}[label=(\alph*)]
        \item a finite union of closed sets is closed; and
        \item an arbitrary intersection of closed sets is closed.
    \end{enumerate}
\end{proposition}
\begin{proof}
    Let \(\ffamily{V_i}{i=1}{N}\) be a finite family of closed sets and let \(V = \bigcup_{i=1}^N V_i\). By \cref{lem:complement_union}, we have \(X \setminus V = \bigcap_{i = 1}^N \left(X \setminus V_i\right).\) That is, \(X \setminus V\) is the finite intersection of the open sets \(X \setminus V_i\), therefore it is open. As such, \(V\) is closed.

    Let \(\family{U_i}{i \in I}\) be a family of closed sets, for some index set \(I\), and let \(U = \bigcap_{i \in I} U_i\). By \cref{lem:complement_union}, we have \(X \setminus U = \bigcup_{i \in I} X \setminus U_i\). That is, \(X \setminus U\) is the arbitrary union of the open sets \(X \setminus U_i\), therefore it is open. We conclude \(U\) is closed.
\end{proof}
\begin{proposition}{Subspace topology is a topology}{subspace_topology}
    Given a topological space \topology{X} and a subset \(S\) of \(X\), we define the \emph{subspace topology} \restrict{\tau_X}{S} as
    \begin{equation*}
        \restrict{\tau_X}{S} = \set{U \cap S : U \in \tau_X}.
    \end{equation*}
    Then \((S, \restrict{\tau_X}{S})\) is a topological space.
\end{proposition}
\begin{proof}
    We must show the conditions (a), (b), and (c) of \cref{def:topology} are satisfied.
    \begin{enumerate}[label=(\alph*)]
        \item Since \(S = X \cap S\) and \(\emptyset = \emptyset \cap S\), we have \(S \in \restrict{\tau_X}{S}\) and \(\emptyset \in \restrict{\tau_X}{S}\).
        \item Let \(U, V \in \restrict{\tau_X}{S}\). Then, there exists \(\tilde{U}, \tilde{V} \in \tau_X\) such that \(U = \tilde{U} \cap S\) and \(V = \tilde{V} \cap S\).Then, \(U \cap V = (\tilde{U}\cap S) \cap (\tilde{V} \cap S) = (\tilde{U}\cap\tilde{V})\cap S\). Since \(\tilde{U} \cap \tilde{V} \in \tau_X\), we have \(U \cap V \in \restrict{\tau_X}{S}\).
        \item Let \family{U_\alpha}{\alpha \in J} be a family of open sets in \(\restrict{\tau_X}{S}\). For each \(\alpha \in J\), there exists a \(\tilde{U}_\alpha\in\tau_X\) such that \(U_\alpha = \tilde{U}_\alpha \cap S\). Then
            \begin{align*}
                \bigcup_{\alpha \in J} U_\alpha &= \bigcup_{\alpha \in J} \tilde{U}_\alpha \cap S\\
                                                &= \setc{s \in S }{\exists \alpha \in J : s \in \tilde{U}_\alpha}\\
                                                &= \setc{x \in X}{\exists \alpha \in J : x \in \tilde{U}_\alpha} \cap S\\
                                                &= S\cap\bigcup_{\alpha\in J}\tilde{U}_\alpha.
            \end{align*}
        Since arbitrary unions of open sets is an open set, it follows that \(\bigcup_{\alpha\in J}U_\alpha \in \restrict{\tau_X}{S}\).
    \end{enumerate}
    We have thus shown \(\restrict{\tau_X}{S}\) is a topology on \(S\).
\end{proof}
\begin{remark}
    It should be noted that if \(S\) is an open set in \(X\), then every open set in the subspace topology is an open set in \(X\). Indeed, let \(V \in \restrict{\tau_X}{S}\), then there exists \(\tilde{V} \in \tau_X\) such that \(V = \tilde{V} \cap S\), that is, \(V\) is an intersection of two open sets in \(X\), hence open.
\end{remark}

\begin{proposition}{Product topology}{product_topology}
    Let \topology{X} and \topology{Y} be topological spaces. Define the \emph{product topology} \(\tau_{X\times Y}\) as the collection of subsets \(U \subset X \times Y\) such that for all \((x,y) \in U\), there exists neighborhoods \(S \subset X\) and \(T \subset Y\) of \(x \in X\) and \(y\in Y\) such that \(S \times T \subset U\). Then \topology{X\times Y} is a topological space.
\end{proposition}
\begin{proof}
    Clearly, \(X\times Y\) and \(\emptyset\) are open sets in the product topology.

    Next, we consider open sets \(U, V \in \tau_{X\times Y}\) and an element \(p \in U \cap V\). Let \(p = (x, y) \in X \times Y\), then there exists neighborhoods \(S_U, S_V\subset X\) of \(x\) and \(T_U, T_V \subset Y\) of \(y\) such that \(S_U \times T_U \subset U\) and \(S_V \times T_V \subset V\). Let \(S = S_U \cap S_V\) and \(T = T_U \cap T_V\), then \(S \in \tau_X\) and \(T \in \tau_Y\) are neighborhoods of \(x\) and \(y\), respectively. Moreover, \(S \times T \subset U \cap V\) is a neighborhood of \(p\), from which follows \(U \cap V \in \tau_{X\times Y}\).

    Let \family{U_\alpha}{\alpha\in J} be a family of open sets in the product topology. Let \(p\in \bigcup_{\alpha\in J}U_\alpha\), then there exists \(\beta \in J\) such that \(p \in U_{\beta}\). By definition, there exists open sets \(S \in \tau_X\) and \(T \in \tau_Y\) such that \(S \times T \subset U_\beta \subset \bigcup_{\alpha\in J} U_\alpha\). Therefore, \(\bigcup_{\alpha\in J}U_\alpha\) is an open set.
\end{proof}

Along with the axioms of topological spaces described in \cref{def:topology} one might add further restrictions to specify the space considered.
\begin{definition}{Hausdorff space}{hausdorff}
    A topological space \topology{X} is called a \emph{Hausdorff space} if for any \(p,q\in X\) with \(p\neq q\), there exists a neighborhood \(U\) of \(p\), i.e. \(p \in U \in \tau_X\), and a neighborhood \(V\) of \(q\) such that \(U \cap V = \emptyset\).
\end{definition}

\subsection{Continuity and Homeomorphisms}
With the notion of topological spaces, we may ask ourselves whether certain maps between topological spaces can preserve the topology. One requirement is certainly that such a map has to associate open sets in its domain to open sets in its codomain, with respect to the topologies considered.
\begin{definition}{Open mapping}{open_mapping}
    Let \topology{X} and \topology{Y} be topological spaces. An \emph{open mapping} \(f : X \to Y\) is a map such that \(f(U) \in \tau_Y\) if \(U \in \tau_X\), that is, it maps open sets to open sets.
\end{definition}

Closely related and important to preserving the topological structure is that of a \emph{continuous map}, familiar from standard analysis on the real line.
\begin{definition}{Continuous map}{continuity}
    Let \topology{X} and \topology{Y} be topological spaces. Then a map \(f : X \to Y\) is \emph{continuous} (with respect to \(\tau_X\) and \(\tau_Y\)) if, for all \(V \in \tau_Y\), the preimage \(\preim{f}{V}\) is an open set in \(\tau_X\).
\end{definition}

In short, a map is continuous if and only the preimages of (all) open sets are open sets. Another useful characterization of continuity can be made with closed sets.
\begin{proposition}{Continuity and closed sets}{continuity_closed}
    Let \topology{X} and \topology{Y} be topological spaces. A map \(f : X \to Y\) is continuous if and only if for all closed sets \(V \subset Y\), the preimage \(\preim{f}{V}\) is a closed set in \(\tau_X\).
\end{proposition}
\begin{proof}
    Notice \(\preim{f}{Y\setminus V} = X \setminus \preim{f}{V}\) for any subset \(V \subset Y\). Indeed, we have
    \begin{align*}
        \preim{f}{Y \setminus V} &= \setc{x \in X}{f(x) \in Y \setminus V}\\&= \setc{x \in X}{f(x) \notin V}\\&=\setc{x \in X}{x \notin \preim{f}{V}}\\&= X \setminus \preim{f}{V}.
    \end{align*}
    In particular, we also have \(\preim{f}{V} = X \setminus \preim{f}{Y\setminus V}\).

    Suppose \(f\) is continuous. Let \(V \subset Y\) be a closed set in \(\tau_Y\), then \(Y\setminus V \in \tau_Y\). By continuity \(\preim{f}{Y \setminus V} = X \setminus\preim{f}{V} \in \tau_X\), then \(\preim{f}{V}\) is closed.

    Suppose the preimage of any closed set in \topology{Y} is closed in \topology{X}. Let \(U \in \tau_Y\) be an open set, then \(Y\setminus U\) is closed and so is \(\preim{f}{Y\setminus U}\). Since \(\preim{f}{U} = X \setminus \preim{f}{Y\setminus U}\), the preimage of \(U\) is open, hence \(f\) is continuous.
\end{proof}

We may now define the structure preserving map of a topological space.
\begin{definition}{Homeomorphism}{homeomorphism}
    Let \topology{X} and \topology{Y} be topological spaces. A \emph{homeomorphism} is a continuous bijection \(f : X \to Y\) that is an open mapping.
\end{definition}

An equivalent definition is that a homeomorphism is a continuous map with a continuous inverse map.
\begin{theorem}{Homeomorphism and continuity of inverse}{homeomorphism_inverse}
    Let \topology{X} and \topology{Y} be topological spaces. A bijection \(f : X \to Y\) is a homeomorphism if and only if \(f : X \to Y\) and \(f^{-1} : Y \to X\) are continuous maps.
\end{theorem}
\begin{proof}
    Suppose \(f\) is a continuous and open bijection. Let \(U \in \tau_X\) be an open set, then \(f(U) \in \tau_Y\) is open. Since \(f\) is a bijection, \(\preim{f^{-1}}{U} = f(U)\), hence \(f^{-1}\) is continuous.

    Suppose \(f\) and \(f^{-1}\) are continuous maps. Let \(U \in \tau_X\) be an open set. By continuity, the preimage \(\preim{f^{-1}}{U} \in \tau_Y\) is an open set. Since \(f\) is bijective, it follows that \(f(U) = \preim{f^{-1}}{U} \in \tau_Y\), hence \(f\) is an open mapping.
\end{proof}

Before showing homeomorphisms map closed sets to closed sets we recall a result from set theory.
\begin{lemma}{Image of a union}{union}
    Let \(f : X \to Y\) be a mapping. If \(A, B \in \mathbb{P}(X)\), then \(f(A) \cup f(B) = f(A \cup B)\).
\end{lemma}
\begin{proof}
    Clearly \(A \subset A\cup B\) and \(B \subset A\cup B\), hence by \(f(A) \subset f(A \cup B)\) and \(f(B) \subset f(A \cup B)\) we conclude \(f(A) \cup f(B) \subset f(A \cup B)\).

    If either \(A = \emptyset\) or \(B = \emptyset\), then the equality is trivially satisfied, so we may assume \(A \cup B \neq \emptyset\). Let \(y \in f(A \cup B)\), then there exists \(x \in A \cup B\) such that \(y = f(x)\). That is, \(x \cup A \lor x \cup B\), hence \(y \in f(A) \lor y \in f(B)\). This shows \(y \in f(A) \cup f(B)\), hence \(f(A \cup B) \subset f(A) \cup f(B)\).
\end{proof}

\begin{proposition}{Homeomorphism maps closed sets to closed sets}{homeomorphism_closed}
    Let \topology{X} and \topology{Y} be topological spaces. If \(f : X \to Y\) is a homeomorphism, then \(f\) maps closed sets in \(\tau_X\) into closed sets in \(\tau_Y\).
\end{proposition}
\begin{proof}
    Let \(F\subset X\) be a closed set in \(\tau_X\), then \(f(X \setminus F) \in \tau_Y\) since \(f\) is an open mapping. Notice \(f(X \setminus F) \cup f(F) = Y\) since \(f\) is surjective. As \(f\) is injective, we must have \(f(F) = Y \setminus f(X\setminus F)\), hence \(f(F)\) is closed in \(\tau_Y\).
\end{proof}

If there exists a homeomorphism between two topological spaces, they are said to be \emph{homeomorphic} to each other. In fact, this establishes an equivalence relation on topological spaces, thanks to the continuity o the composition of continuous maps.
\begin{theorem}{Composition of continuous maps}{continuous_composition}
Let \topology{X}, \topology{Y}, and \topology{Z} be topological spaces. If the maps \(f: X \to Y\) and \(g : Y \to Z\) are continuous (with respect to the appropriate topologies), then the map \(g \circ f : X \to Z\) is continuous with respect to \(\tau_X\) and \(\tau_Z\).
\end{theorem}
\begin{proof}
    Let \(V\) be an open set of \topology{Z}. We must show the preimage \((g \circ f)^{-1}(V)\) is an open set of \topology{X}. We have
    \begin{align*}
        (g\circ f)^{-1}(V) &= \setc{x \in X }{g\circ f(x) \in V}\\
                           &= \setc{x \in X }{f(x) \in \preim{g}{V}}\\
                           &= \preim{f}{\preim{g}{V}}.
    \end{align*}
    Since the map \(g\) is continuous and \(V\) is an open set in \topology{Z}, it follows that \(\preim{g}{V}\) is open in \topology{Y}. By the same argument, \(\preim{f}{\preim{g}{V}}\) is an open set in \topology{X}.
\end{proof}

\begin{corollary}
    If two topological spaces \topology{X} and \topology{Y} are homeomorphic, we write \(\topology{X} \cong \topology{Y}\). Then, \(\cong\) is an equivalence relation on topological spaces.
\end{corollary}
\begin{proof}
    It is clear \(\topology{X} \cong \topology{X}\) for any topological space \topology{X}. In particular, the map \(\id{X}\) is a homeomorphism. Moreover, it is clear that \(\topology{X} \cong \topology{Y} \iff \topology{Y} \cong \topology{X}\), after all homeomorphisms are continuous bijections with continuous inverses.

    Let \topology{X}, \topology{Y}, and \topology{Z} be topological spaces, where \(X \cong Y\) and \(Y \cong Z\). Let \(f : X \to Y\) and \(g : Y \to Z\) be homeomorphisms from \topology{X} to \topology{Y} and \topology{Y} to \topology{Z}, respectively. Consider the composition \(g\circ f : X \to Z\).
    \begin{equation*}
        \begin{tikzcd}[column sep = normal, row sep = large]
            X \arrow{r}{f} \arrow[swap]{dr}{g\circ f} & Y \arrow{d}{g} \\
                                                      & Z
        \end{tikzcd}
    \end{equation*}
    By \cref{thm:continuous_composition}, the map \(g\circ f\) is a homeomorphism from \topology{X} to \topology{Z}.
\end{proof}

As was done for the subspace topology, we prove a similar result for continuous maps.
\begin{proposition}{Restriction of a continuous map}{restriction_map}
    Let \topology{X} and \topology{Y} be topological spaces and let \(f : X \to Y\) be a continuous map. Let \(S\) be a subset of \(X\) and let \topology{S} be the subspace topology, then \(\restrict{f}{S} : S \to Y\) is a continuous map with respect to \(\tau_S\) and \(\tau_Y\).
\end{proposition}
\begin{proof}
    Let \(V \in \tau_Y\). Then, by the definition of preimage, we have
    \begin{align*}
        \preim{\restrict{f}{S}}{V} &= \setc{s \in S}{\restrict{f}{S}(s) \in V}\\
                                &= \setc{s \in S}{f(s) \in V}\\
                                &= \preim{f}{V} \cap S.
    \end{align*}
    By hypothesis, the preimage \(\preim{f}{V}\) is an open set in \topology{X}, so \(\preim{\restrict{f}{S}}{V}\) is an open set in the subspace topology.
\end{proof}

\subsection{Initial and final topologies}
Given a collection of maps, it is possible to define a topology on a set such that every map in that collection is continuous. Before stating how such a topology can be defined, we prove two results from set theory related to intersection and union of preimages.
\begin{lemma}{Union and intersection of preimages}{union_intersection_preimage}
    Let \(X\) and \(Y\) be non-empty sets. If \(\mathcal{V} \subset \power{Y}\) is a non-empty collection of subsets of \(Y\), then
    \begin{equation*}
        \preim{f}{\bigcup \mathcal{V}} = \bigcup_{V \in \mathcal{V}} \preim{f}{V}
        \quad\text{and}\quad
        \preim{f}{\bigcap \mathcal{V}} = \bigcap_{V \in \mathcal{V}} \preim{f}{V}
    \end{equation*}
    for any mapping \(f : X \to Y\).
\end{lemma}
\begin{proof}
    We have
    \begin{align*}
        \preim{f}{\bigcup \mathcal{V}} &= \setc*{x \in X}{f(x) \in \bigcup_{V \in \mathcal{V}} V}&
        \preim{f}{\bigcap \mathcal{V}} &= \setc*{x \in X}{f(x) \in \bigcap_{V \in \mathcal{V}} V}\\
                                        &= \setc*{x \in X}{\exists V \in \mathcal{V} : f(x) \in V}&
                                        &= \setc*{x \in X}{\forall V \in \mathcal{V} : f(x) \in V}\\
                                        &= \bigcup_{V \in \mathcal{V}} \setc{x \in X}{f(x) \in V}&
                                        &= \bigcap_{V \in \mathcal{V}} \setc{x \in X}{f(x) \in V}\\
                                        &= \bigcup_{V \in \mathcal{V}} \preim{f}{V}&
                                        &= \bigcap_{V \in \mathcal{V}} \preim{f}{V},
    \end{align*}
    as claimed.
\end{proof}

With this result we may show a map can induce a topology on its domain (codomain), provided the codomain (domain) already has a topology.
\begin{proposition}{Initial topology with respect to a map}{initial_topology_map}
    Let \(f : X \to Y\) be a mapping, where \(X\) is a non-empty set and \topology{Y} is a topological space. The set
    \begin{equation*}
        \tau_X = \setc{U \in \power{X}}{\exists V \in \tau_{Y} : U = \preim{f}{V}}
    \end{equation*}
    defines a topology on \(X\) with respect to which \(f\) is continuous. This topology is called the \emph{initial topology on \(X\) with respect to \(f\) and \topology{Y}}.
\end{proposition}
\begin{proof}
    Clearly, we have \(X = \preim{f}{Y}\) and \(\emptyset = \preim{f}{\emptyset}\), hence \(X, \emptyset \in \tau_X\). Let \(\mathcal{U} \subset \tau_X\) be a non-empty collection of sets in \(\tau_X\). For each \(U \in \mathcal{U}\), there exists \(V_U \in \tau_{Y}\) such that \(U = \preim{f}{V_U}\), hence
    \begin{align*}
        \bigcup \mathcal{U} &= \bigcup_{U \in \mathcal{U}} \preim{f}{V_U}&
        \bigcap \mathcal{U} &= \bigcap_{U \in \mathcal{U}} \preim{f}{V_U}\\
                             &= \preim{f}{\bigcup_{U \in \mathcal{U}} V_U}&
                             &= \preim{f}{\bigcap_{U \in \mathcal{U}} V_U}
    \end{align*}
    by \cref{lem:union_intersection_preimage}. Since \(\tau_Y\) is a topology, we immediately have \(\bigcup_{U \in \mathcal{U}} V_U \in \tau_Y\), therefore \(\bigcup \mathcal{U} \in \tau_X\). If, in addition, \(\mathcal{U}\) is finite, we have \(\bigcap \mathcal{U} \in \tau_X\) by the same argument.

    By construction, the map \(f\) is continuous. Indeed, let \(V \in \tau_Y\), then \(\preim{f}{V} \in \tau_X\) is open.
\end{proof}

\begin{proposition}{Final topology with respect to a map}{final_topology_map}
    Let \(f : X \to Y\) be a mapping, where \topology{X} is topological space and \(Y\) is a non-empty set. The set
    \begin{equation*}
        \tau_Y = \setc{V \in \power{Y}}{\preim{f}{V} \in \tau_X}
    \end{equation*}
    defines a topology on \(Y\) with respect to which \(f\) is continuous. This topology is called the \emph{final topology on \(Y\) with respect to \(f\) and \topology{X}}.
\end{proposition}
\begin{proof}
    Since \(X = \preim{f}{Y}\) and \(\emptyset = \preim{f}{\emptyset}\), we have \(Y,\emptyset \in \tau_Y\). Let \(\mathcal{V} \subset \tau{Y}\) be a non-empty collection of sets in \(\tau_Y\), then by \cref{lem:union_intersection_preimage} we have \(\preim{f}{\bigcup \mathcal{V}}= \bigcup_{V \in \mathcal{V}}\preim{f}{V}\) and \(\preim{f}{\bigcap \mathcal{V}} = \bigcap_{V \in \mathcal{V}} \preim{f}{V}\). For each \(V \in \mathcal{V}\), we have \(\preim{f}{V} \in \tau_X\), therefore \(\preim{f}{\bigcup \mathcal{V}} \in \tau_X\) and we conclude \(\bigcup \mathcal{V} \in \tau_Y\). If \(\mathcal{V}\) is finite, we also have \(\preim{f}{\bigcap \mathcal{V}} \in \tau_X\), hence \(\bigcap \mathcal{V} \in \tau_Y\).
\end{proof}

\begin{proposition}{Initial topology}{initial_topology}
    Let \(X\) be a non-empty set and let \(\mathcal{Y} = \family{\topology{Y_i}}{i\in I}\) be a collection of topological spaces, with some arbitrary indexing set \(I\), where we consider the collection \(\mathcal{F}\) of maps \(f_i : X \to Y_i\), for all \(i \in I\). For each \(i \in I\), the set
    \begin{equation*}
        \tau_{i} = \setc{U \in \power{X}}{\exists V \in \tau_{Y_i} : U = \preim{f_i}{V}}
    \end{equation*}
    is a topology on \(X\). The \emph{initial topology \(\tau_X\) on \(X\) with respect to the collections \(\mathcal{Y}\) and \(\mathcal{F}\)} is the topology on \(X\) defined by
    \begin{equation*}
        \tau_X = \bigcap_{i \in I} \tau_i,
    \end{equation*}
    and has the property that every map \(f \in \mathcal{F}\) is continuous.
\end{proposition}
\begin{proof}
    \todo[Not sure if this is it.]
\end{proof}

\subsection{Closure and interior of a set}
Every subset on a topological space defines a closed set that contains it and an open set that is contained in it. Let's begin with the closure of a set.
\begin{definition}{Closure of a set}{closure}
    Let \topology{X} be a topological space and let \(S \subset X\) be a subset. A \emph{point of closure of \(S\)} is a point \(x \in X\) such that every neighborhood \(U\in\tau_X\) of \(x\) has non-empty intersection with \(S\). The \emph{closure \(\cl_{(X, \tau_X)}{S}\) of \(S\)} is the set of all points of closure of \(S\).
\end{definition}
\begin{remark}
    We may simply write \(\cl_X S\) or \(\cl S\) if the topological space is understood.
\end{remark}
\begin{remark}
    It should be clear that every point of \(S\) is a point of closure, that is \(S \subset \cl{S}\). Indeed, let \(s \in S\), then every neighborhood of \(s\) contains \(s,\) a point in \(S\). That is, \(s\) is a point of closure of \(S\), and we may conclude \(S \subset \cl S\).
\end{remark}

We motivate the name \emph{closure} by showing that every closure is closed. Furthermore, we will show a closed set is already its closure and conclude the closure is the smallest (in the sense of inclusion) closed set that contains a set.
\begin{lemma}{Closure of a set is closed}{closure_is_closed}
    Let \((X, \tau)\) be a topological space and let \(S\subset X\) be a subset. Then \(\cl S\) is closed.
\end{lemma}
\begin{proof}
    Let \(x \notin \cl S\). Then there is a neighborhood \(U_x\) of \(x\) that does not intersect with \(S\). Since \(U_x\) is a neighborhood for each of its points, we also have \(U_x \cap \cl S = \emptyset\), hence \(U_x \subset X \setminus \cl S\). Then \(X \setminus \cl{S} = \bigcup_{x \in X\setminus \cl S} U_x\), thus \(X \setminus \cl{S}\) is open.
\end{proof}
\begin{lemma}{A closed set is its closure}{closure_closed}
    Let \((X, \tau)\) be a topological space. A subset \(S \subset X\) is closed if and only if \(S = \cl S\).
\end{lemma}
\begin{proof}
    Suppose \(S = \cl S\). By \cref{lem:closure_is_closed}, the closure is closed, then \(S\) is closed.

    Suppose \(S\) is closed, then \(X \setminus S\) is open. Then \(X \setminus S\) is a neighborhood of any of its points that does not contain any elements of \(S\), that is, \(x \in X\setminus S \implies x \notin \cl S\). Then we conclude \(S = \cl S\).
\end{proof}
\begin{theorem}{Closure as the smallest closed set}{closure_smallest}
    Let \((X, \tau)\) be a topological space. If \(S\subset X\) is a subset, there exists no closed subset \(A \subset X\) such that \(S \subset A \subsetneq \cl S\).
\end{theorem}
\begin{proof}
    Let \(A\) be a closed subset of \(X\) that contains \(S\). Let \(x \in \cl{S}\), then for every neighborhood of \(x\) has non-empty intersection with \(S\), and as a result, with \(A\) too. Hence, \(x \in \cl{A} = A\), thus showing \(\cl{S} \subset A\).
\end{proof}
\begin{corollary}
    Let \((X, \tau)\) be a topological space. If \(S \subset X\) is a subset, its closure \(\cl{S}\) is the intersection of every closed set in \(X\) that contains \(S\).
\end{corollary}
\begin{proof}
    Let \(F = \bigcap \setc{A \in \mathbb{P}(X)}{X \setminus A \in \tau \land S \subset A}\) be the intersection of all closed sets that contain \(S\). In the previous proof, we have shown the closure is contained in every closed set that contains \(S\), that is, \(\cl{S} \subset F\). Let \(x \in F\), then \(x\) is an element of every closed set that contains \(S\). In particular, \(x \in \cl{S}\), hence \(F \subset \cl{S}\).
\end{proof}
\begin{corollary}
    Let \((X, \tau)\) be a topological space. If \(S \subset X\) is a closed subset of \(X\) and \(T \subset S\) is a subset of \(S\), then \(\cl{T} \subset S\).
\end{corollary}
\begin{proof}
    Since \(T \subset S\) and \(T \subset \cl{T}\), we must have \(\cl{T} \subset S\).
\end{proof}

We may use closures to provide an additional characterization of continuity.
\begin{theorem}{Closure and continuity}{closure_continuity}
    Let \topology{X} and \topology{Y} be topological spaces. The map \(f : X \to Y\) is continuous if and only if \(f(\cl_XS) \subset \cl_Y(f(S))\) for all \(S \subset X\).
\end{theorem}
\begin{proof}
    Suppose \(f\) is continuous and let \(S \subset X\). Notice \(S \subset \preim{f}{f(S)}\) and \(f(S) \subset \cl_Y(f(S))\), then
    \begin{equation*}
        S \subset \preim{f}{f(S)} \subset \preim{f}{\cl_Y{f(S)}}.
    \end{equation*}
    By continuity, \(\preim{f}{\cl_Y(f(S))}\) is closed in \(X\), hence \(\cl_XS \subset \preim{f}{\cl_Y{f(S)}}\). This shows \(f(\cl_X(S)) \subset \cl_Y(f(S))\).

    Suppose for all \(S \in \mathbb{P}(X)\) we have \(f(\cl_XS) \subset \cl_Y(f(S))\) and let \(V\) be a closed set in \(Y\). Let \(U = \preim{f}{V} \in \mathbb{P}(X)\), then \(V = f(U) \subset f(\cl_X U)\). By hypothesis, \(f(\cl_XU) \subset \cl_Y(f(U)) = V\), since \(V\) is closed. We have thus shown \(f(\cl_X(U)) = V\), hence \(\cl_X(U) \subset \preim{f}{V} = U\) implies \(U\) is closed.
\end{proof}

We now move on to the interior of a set.
\begin{definition}{Interior of a set}{interior}
    Let \topology{X} be a topological space and let \(S \subset X\) be a subset. An \emph{interior point of \(S\)} is a point \(x \in X\) such that there exists a neighborhood \(U \in \tau_X\) of \(x\) that is contained in \(S\). The \emph{interior \(\inte_{\topology{X}}S\) of \(S\)} is the set of all interior points of \(S\).
\end{definition}
\begin{remark}
    We may simply write \(\inte_X S\) or \(\inte S\) if the topological space is understood.
\end{remark}
\begin{remark}
    It should be clear that every interior point of \(S\) is a point of \(S\), that is \(\inte S \subset S\). Indeed, let \(x \in \inte S\), then there exists a neighborhood \(U_x\) of \(x\) that is contained in \(S\). That is, \(x\) belongs to \(S\), and we may conclude \(\inte S \subset S\). Moreover, since \(U_x\) is a neighborhood for its elements, we conclude \(U_x \subset \inte S\).
\end{remark}
Similarly to the closure, the interior of a set is an open set. Moreover, it is the largest open set contained in it.
\begin{lemma}{Interior of a set is open}{interior_is_open}
    Let \topology{X} be a topological space and let \(S \subset X\) be a subset. Then \(\inte S\) is open.
\end{lemma}
\begin{proof}
    Let \(x \in \inte S\) be an interior point. Then consider the union of every neighborhood of \(x\) that is contained in \(S\), \(U_x = \bigcup \setc{U \in \tau_X}{x \in U \land U \subset S}\). This set is open, hence \(\inte S = \bigcup_{x \in \inte S} U_x\) is open.
\end{proof}
\begin{lemma}{An open set is its interior set}{interior_open}
    Let \topology{X} be a topological space. A subset \(S \subset X\) is open if and only if \(S = \inte S\).
\end{lemma}
\begin{proof}
    Suppose \(S = \inte S\). By \cref{lem:interior_is_open}, \(S\) is open.

    Suppose \(S\) is open. Then every point of \(S\) is an interior point, as \(S \subset S\), hence \(S \subset \inte S\). Since \(\inte S \subset S\), we have \(S = \inte S\).
\end{proof}
\begin{theorem}{Interior as the largest open set}{interior_largest}
    Let \topology{X} be a topological space. If \(S \subset X\) is a subset, there exists no open subset \(A \subset X\) such that \(\inte S \subsetneq A \subset S\).
\end{theorem}
\begin{proof}
    Let \(A\) be an open subset of \(X\) that is contained in \(S\). Then every point of \(A\) is an interior point of \(S\), hence \(A \subset \inte S\).
\end{proof}
\begin{corollary}
    Let \topology{X} be a topological space. If \(S \subset X\) is a subset, its interior \(\inte S\) is the union of every open set in \(X\) that is contained in \(S\).
\end{corollary}
\begin{proof}
    Let \(U = \bigcup\setc{A \in \tau_X}{A \subset S}\) be the union of all open sets that are contained in \(S\). The previous result shows \(U \subset \inte S\). Since \(\inte S\) is an open set that is contained in \(S\), we have \(\inte S \subset U\), and the result follows.
\end{proof}
\begin{corollary}
    Let \topology{X} be a topological space. If \(S \subset X\) is an open subset of \(X\) and \(T \supset S\) is a subset that contains \(S\), then \(S \subset \inte{T}\).
\end{corollary}
\begin{proof}
    Since \(S \subset T\) and \(\inte T \subset T\), we must have \(S \subset \inte{T}\).
\end{proof}

Furthermore, we may relate the closure and interior of a set by means of the complement of the interior and closure of the complement.
\begin{theorem}{Closure and interior}{closure_interior}
    Let \topology{X} be a topological space. Then
    \begin{equation*}
        \inte S = X \setminus \cl(X \setminus S)\quad\text{and}\quad \cl{S} = X \setminus \inte(X \setminus S)
    \end{equation*}
    for any subset \(S \subset X\).
\end{theorem}
\begin{proof}
    Let \(x \in X\setminus\cl(X\setminus S)\), then \(x\) is not a point of closure of \(X \setminus S\). As a result, there exists a neighborhood \(U\) of \(x\) that has empty intersection with \(X \setminus S\), hence \(U \subset S\). We conclude \(X\setminus\cl(X \setminus S) \subset \inte S\).

    Let \(s \in \inte S\), then there exists a neighborhood \(V\) of \(s\) that is contained in \(S\). Since \(V \subset \inte S \subset S\), we have \(V \cap (X \setminus S) = \emptyset\), hence \(s\) is not a point of closure of \(X \setminus S\). We have thus shown \(\inte S = X \setminus \cl(X \setminus S)\).

    Finally, we replace \(S \mapsto X \setminus S\), obtaining \(\inte(X\setminus S) = X \setminus \cl S\). The complement yields \(\cl{S} = X \setminus \inte(X\setminus S)\) as desired.
\end{proof}
\begin{remark}
    This theorem makes it clear that descriptions that rely on closures can be restated as descriptions that rely on interiors and vice-versa.
\end{remark}

Finally, we show how closures and interiors are related with homeomorphisms. Before, we recall a couple of results from set theory.
\begin{lemma}{Image of a set difference}{difference_injection}
    Let \(f : X \to Y\) be a mapping. For every pair of subsets \(A, B \subset X\), \(f(A) \setminus f(B) \subset f(A\setminus B)\). Moreover, \(f(A \setminus B) = f(A) \setminus f(B)\) for all \(A, B \subset X\) if and only if \(f\) is injective.
\end{lemma}
\begin{proof}
    If \(A \cap B = \emptyset\), the result follows trivially. If \(A \subset B\), then \(f(A) \subset f(B)\), hence \(f(A) \setminus f(B) = \emptyset\) and \(f(A \setminus B) = \emptyset\), so the inclusion holds. If \(B \subset A\), then \(f(A \setminus B)\) and \(f(A) \setminus f(B)\) are non-empty. Let \(y \in f(A) \setminus f(B)\), then there exists \(a \in A\) such that \(y = f(a)\) and there exists no \(b \in B\) such that \(y = f(b)\). Then \(a \in A \setminus B\), hence \(y \in f(A \setminus B)\).

    Suppose \(f\) is injective. Let \(A, B \subset X\), then if \(A \setminus B = \emptyset\) or if \(A \setminus B = A\), the equality holds trivially, so we may assume \(A \setminus B \neq \emptyset\) and \(B \subset A\). Let \(y \in f(A \setminus B)\), then there exists \(x \in A \setminus B\) such that \(f(x) = y\). Since \(f\) is one-to-one, there exists no \(b \in B\) such that \(f(b) = f(x)\), hence \(y \notin f(B)\). Since \(f(A\setminus B) \subset f(A)\), we have \(y \in f(A) \setminus f(B)\).

    Suppose \(f(A\setminus B) = f(A) \setminus f(B)\) for all \(A, B \subset X\). If \(X\) is a singleton, then \(f\) is injective, so we may assume \(X\) has at least two elements. Suppose by contradiction \(f\) is not injective, then there exist \(x_1, x_2 \in X\) with \(x_1 \neq x_2\) such that \(f(x_1) = f(x_2)\). By hypothesis, we have \(f(\set{x_1}) = f(\set{x_1,x_2}) \setminus f(\set{x_2})\). Notice the right hand side is the empty set, since \(f(\set{x_1, x_2}) = \set{f(x_2)}\), hence \(f(\set{x_1}) = \emptyset\). This contradiction shows \(f\) is injective.
\end{proof}
\begin{lemma}{Image of an intersection}{intersection_injection}
    Let \(f : X \to Y\) be a mapping. For every pair of subsets \(A, B \subset X\), \(f(A \cap B) \subset f(A) \cap f(B)\). Moreover, \(f(A \cap B) = f(A) \cap f(B)\) for all \(A, B \subset X\) if and only if \(f\) is injective.
\end{lemma}
\begin{proof}
    Notice \(A \cap B \subset A\) and \(A \cap B \subset B\), then \(f(A \cap B) \subset f(A)\) and \(f(A \cap B) \subset f(B)\) since relations preserve inclusions. That is, \(f(A\cap B) \subset f(A) \cap f(B)\).

    Suppose \(f\) is injective. Let \(A, B \subset X\). If at least one is the empty set, the equality trivially holds, so we may assume neither subset is the empty set. Moreover, if \(f(A) \cap f(B) = \emptyset\) we have by the previous result \(f(A \cap B) = \emptyset\), so the equality holds, therefore we may assume \(f(A) \cap f(B) \neq \emptyset.\) Let \(y \in f(A) \cap f(B)\), then there exists \(a \in A\) and \(b \in B\) such that \(y = f(a)\) and \(y = f(b)\). As \(f\) is one-to-one, we must have \(a = b\), hence \(a \in A \cap B\). Then we conclude \(y \in f(A \cap B)\).

    Suppose \(f(A \cap B) = f(A) \cap f(B)\) for all \(A, B \subset X\). If \(X\) is a singleton, then \(f\) is vacuously injective, so we may assume \(X\) has at least two elements. Suppose, by contradiction, \(f\) is not injective, then there exist \(x_1, x_2 \in X\) with \(x_1 \neq x_2\) such that \(y = f(x_1) = f(x_2)\). That is, \(y \in f(\set{x_1}) \cap f(\set{x_2})\), hence \(y \in f(\set{x_1} \cap \set{x_2})\) by hypothesis. Since \(\set{x_1} \cap \set{x_2}\), we have \(y \in \emptyset\). This contradiction shows \(f\) is injective.
\end{proof}

\begin{theorem}{Image of closure and interior under homeomorphism}{closure_interior_homeomorphism}
    Let \topology{X} and \topology{Y} be topological spaces. If \(f : X \to Y\) is a homeomorphism and \(A \subset X\) is a subset, then \(\cl_Y(f(A)) = f(\cl_XA)\) and \(\inte_Y(f(A)) = f(\inte_XA)\).
\end{theorem}
\begin{proof}
    Let \(y \in f(\cl_XA)\), then \(f^{-1}(y) \in \cl_X A\). Let \(V \in \tau_Y\) be a neighborhood of \(y\), then \(f^{-1}(V) \in \tau_X\) is a neighborhood for \(f^{-1}(y)\) such that \(f^{-1}(V) \cap A \neq \emptyset\). Since \(f\) is injective, we have \(V \cap f(A) \neq \emptyset\), that is, \(y \in \cl_Y(f(A))\).

    Let \(y \in \cl_Y(f(A))\), then every neighborhood of \(y\) has a non-empty intersection with \(f(A)\). Since \(f^{-1}\) is injective, \(f^{-1}(y)\) is a point of closure of \(A,\) hence \(f^{-1}\left(\cl_Y(f(A))\right) \subset \cl_XA\). We conclude \(\cl_Y(f(A)) \subset f(\cl_XA)\), thus showing \(\cl_Y(f(A)) = f(\cl_XA)\).

    \cref{thm:closure_interior} yields \(f(\inte_XA) = f(X \setminus \cl_X(X\setminus A))\). Using \cref{lem:difference_injection} and surjectivity, we have \(f(\inte_XA) = Y \setminus f(\cl_X(X\setminus A)).\) By the previous result, we have \(f(\inte_XA) = Y \setminus \cl_Y(Y \setminus f(A))\). Using \cref{thm:closure_interior} again shows \(f(\inte_XA) = \inte_Y(f(A))\).
\end{proof}
\begin{remark}
    We could have used \cref{thm:closure_continuity} to show this result. Indeed, the hypothesis shows \(f(\cl_XA) \subset \cl_Y(f(A))\) and \(f^{-1}(\cl_Y(f(A))) \subset \cl_X(f^{-1}\circ f(A)) = \cl_XA\), hence \(\cl_Y(f(A)) \subset f(\cl_XA)\).
\end{remark}

\subsection{Dense and nowhere dense sets}
Closely related to the notions of closure and interior of a set is the concept of a dense set.
\begin{definition}{Dense set}{dense_set}
    Let \topology{X} be a topological space. A set \(S\) is \emph{dense in \(X\)} if \(\cl_X S = X\). Moreover, a set \(S\) is \emph{dense in a non-empty subset \(U \subset X\)} if \(S \cap U\) is dense in the subspace topology on \(U\).
\end{definition}
\begin{remark}
    It should be noted that if \(S\) is dense in \(X\), then it is dense in any non-empty open subset \(U \in \tau_X\). Indeed, \(U \subset \cl_X S\), hence every neighborhood of \(x \in U\) has a non-empty intersection with \(S\). In particular, for every \(V \in \tau_U\) that contains \(x \in U\) satisfies \(V \cap S \neq \emptyset\), hence \(U \subset \cl_U(S \cap U)\). This shows \(U = \cl_U(S\cap U)\) since \(\cl_U(A) \subset U\) for any \(A \subset U\).
\end{remark}

An equivalent description is given in terms of the interior of a dense set.
\begin{theorem}{Interior of the complement of a dense set is empty}{interior_dense}
    Let \topology{X} be a topological space. A set \(S\subset X\) is dense in \(X\) if and only if \(\inte{(X \setminus S)} = \emptyset\).
\end{theorem}
\begin{proof}
    Suppose \(S\) is dense in \(X\), then \(\cl S = X\). By \cref{thm:closure_interior}, it follows that \(X = X \setminus \inte(X\setminus S)\). Hence, \(\inte(X\setminus S) = \emptyset\).

    Suppose \(\inte(X\setminus S) = \emptyset\). By \cref{thm:closure_interior}, we have \(\cl{S} = X\), hence \(S\) is dense in \(X\).
\end{proof}

With the notion of a set dense in some subset, we may categorize topological spaces.
\begin{definition}{Nowhere dense sets}{nowhere_dense}
    Let \topology{X} be a topological space. A set \(S\) is \emph{nowhere dense in \(X\)} if \(S\) is not dense in any non-empty open subset of \(X\).
\end{definition}

\begin{theorem}{Interior of the closure of a nowhere dense set}{interior_closure_nowhere_dense}
    Let \topology{X} be a topological space. A subset \(S \subset X\) is nowhere dense in \(X\) if and only if \(\inte_X\left(\cl_X S\right) = \emptyset\).
\end{theorem}
\begin{proof}
    Suppose \(U = \inte_X\left(\cl_X S\right)\) is not the empty set. Let \(x \in U\), there exists \(V \in \tau_X\) containing \(x\) with \(V \subset U\), since \(U\) is open in \(X\). Moreover, \(V \in \tau_U\) because \(U \in \tau_X\). As \(U\) is the interior of \(\cl_XS\), we have \(U \subset \cl_X S\), hence \(x \in \cl_X S\). Then, every neighborhood of \(x\) has a non-empty intersection with \(S\). This implies \(V \cap (S \cap U) \neq \emptyset\), which yields \(x \in \cl_U(S \cap U)\). That is, \(\cl_U(S\cap U) = U\), hence \(S \cap U\) is dense in \(U\). We have thus shown \(\inte_X\left(\cl_X S\right) \neq \emptyset\) implies that \(S\) is dense in some non-empty open subset of \(X\), hence the contrapositive gives us that if \(S\) is nowhere dense in \(X\), then \(\inte_X(\cl_XS) = \emptyset\).

    Suppose \(S\) is dense in some non-empty open set \(U \in \tau_X\), that is, \(\cl_U(S \cap U) = U\). Let \(x \in U\), and let \(V \in \tau_U\) be some neighborhood of \(x\), then \(V \cap S \neq \emptyset\). Since \(U\) is open in \(X\), we have \(x \in \cl_X S\). Hence, \(U \subset \cl_X S\). We have shown if \(S\) is somewhere dense, then the interior of its closed is non-empty, thus showing that \(\inte_X (\cl_X S) = \emptyset\) implies \(S\) is nowhere dense by contrapositive.
\end{proof}
\begin{corollary}
    If \(S\) is nowhere dense in \(X\), then \(\cl_X{S}\) is nowhere dense in \(X\).
\end{corollary}
\begin{proof}
    Since \(\cl_XS\) is closed, we have \(\inte_X(\cl_X(\cl_X S)) = \inte_X(\cl_X S) = \emptyset\) by hypothesis.
\end{proof}

\begin{proposition}{Nowhere density preserved under homeomorphism}{nowhere_dense_homeomorphism}
    Let \topology{X} and \topology{Y} be topological spaces and let \(f : X \to Y\) be a homeomorphism. The subset \(S\subset X\) is nowhere dense in \(X\) if and only if \(f(S)\) is nowhere dense in \(Y\).
\end{proposition}
\begin{proof}
    \cref{thm:closure_interior_homeomorphism} yields \(\inte_Y(\cl_Y f(S)) = f(\inte_X(\cl_X(S)))\). \cref{thm:interior_closure_nowhere_dense} shows \(f(S)\) is nowhere dense in \(Y\) if and only if \(S\) is nowhere dense in \(X\).
\end{proof}

\begin{definition}{First and second Baire categories}{Baire_categories}
    Let \topology{X} be a topological space. A subset \(S \subset X\) is said to be of \emph{the first category} if \(S\) is expressible as the countable union of nowhere dense sets in \(X\). Otherwise, \(S\) is of \emph{the second category.}
\end{definition}

% vim: spl=en_us
\section{Metric spaces}
We now move on to metric spaces, which can be seen as a specialization of topological spaces, in the sense that every metric space is a topological space, but there are notions particular to the metric that can't be expressed topologically.
\begin{definition}{Metric space}{metric_space}
    A \emph{metric space} \((X, d)\) is a non-empty set \(X\) equipped with a map \(d : X \times X \to \mathbb{R}\), called \emph{metric} or \emph{distance function}, satisfying
    \begin{enumerate}[label=(\alph*)]
        \item \(d(x, y) \geq 0\), for all \(x, y \in X\), with \(d(x, y) = 0 \iff x = y\);
        \item \(d(x, y) = d(y, x)\), for all \(x, y \in X\); and
        \item \(d(x, y) = d(x, z) + d(z, y)\), for all \(x,y,z \in X\).
    \end{enumerate}
\end{definition}
\begin{example}{Supremum metric for continuous complex-valued functions}{continuous_complex_ab}
    Let \(\mathcal{C}([a,b];\mathbb{C})\) denote the set of continuous functions from the interval \([a,b]\subset \mathbb{R}\) to the complex plane \(\mathbb{C}\) and let the \emph{supremum metric}, or \emph{sup metric}, be the map
    \begin{align*}
        d_\infty : \mathcal{C}([a,b];\mathbb{C}) \times \mathcal{C}([a,b];\mathbb{C}) &\to \mathbb{R}\\
        (f, g) &\mapsto \sup_{x\in[a,b]}\abs*{f(x) - g(x)}.
    \end{align*}
    Then, \((\mathcal{C}([a,b];\mathbb{C}), d_\infty)\) is a metric space.
\end{example}
\begin{proof}
    We first note that the range of the map \(d_\infty\) is contained in the ray \([0, \infty)\), that is,
    \begin{equation*}
        d_\infty(f,g) \geq 0
    \end{equation*}
    for all \(f,g \in \mathcal{C}([a,b];\mathbb{C})\).

    For all \(f, g \in \mathcal{C}([a,b]; \mathbb{C})\) we have \(f = g\) if and only if \(f(x) = g(x)\) for all \(x \in [a,b]\). Therefore,
    \begin{align*}
        f = g &\iff \forall x \in [0,1] : \abs{f(x) - g(x)} = 0\\
              &\iff \sup_{x \in [0,1]}  \abs{f(x) - g(x)} = 0\\
              &\iff d_\infty(f,g) = 0.
    \end{align*}

    It is clear the map \(d_\infty\) is symmetric with respect to its arguments, that is,
    \begin{align*}
        d_\infty(g,f) = \sup_{x\in[0,1]}\abs*{g(x) - f(x)}= \sup_{x\in[0,1]} \abs*{f(x) - g(x)} = d_\infty(f,g).
    \end{align*}

    Finally, we consider \(f,g,h \in \mathcal{C}([a,b];\mathbb{C})\), then
    \begin{align*}
        d_\infty(f,g) &= \sup_{x\in[0,1]}\abs*{f(x) - h(x) + h(x) - h(x)}\\\
                      &\leq \sup_{x\in[0,1]} \abs{f(x) - h(x)} + \abs{h(x) - g(x)}\\
                      &\leq d_\infty(f,h) + d_\infty(h,g).
    \end{align*}
    Thus, we have shown the sup norm \(d_\infty\) is a metric in \(\mathcal{C}([a,b];\mathbb{C})\).
\end{proof}


Now we show how a metric space may be understood as a topological space. The following construction will be referred to as the \emph{metric topology}.
\begin{theorem}{Metric space induces a topology}{metric_space_topology}
    Let \((X, d)\) be a metric space and define the \emph{open ball \(B_r(x)\) of radius \(r > 0\) centered at a point \(x \in X\)} as the subset
    \begin{equation*}
        B_r(x) = \setc*{y \in X}{d(x,y) < r}.
    \end{equation*}
    Let \(\tau\) be a subset of the power set of \(X\) with the property that \(U \in \tau\) if and only if for every point \(x \in U\) there exists \(r > 0\) such that \(B_r(x) \subset U\). Then \((X, \tau)\) is a topological space.
\end{theorem}
\begin{proof}
    Vacuously, \(\emptyset \in \tau\). Trivially, \(B_r(x) \subset X\) for all \(x \in X\) and all \(r > 0\).

    We consider a finite family \(\ffamily{V_i}{i=1}{N}\subset \tau\) for some integer \(N \geq 2\), and its intersection \(V = \bigcap_{i=1}^{N} V_i\). Let \(x \in V\), then \(x \in V_i\), for all \(i \in \set{1,2,\dots, N}\). Since \(V_i \in \tau\), there exists \(r_i > 0\) such that \(B_{r_i}(x) \subset V_i\). Let \(r = \min\setc{r_i}{1 \leq i \leq N}\), then \(B_r(x) \subset B_{r_i}(x) \subset V_i\) for all \(i\), that is, \(B_r(x) \subset V\). Hence, \(V \in \tau\).

    We consider a family \(\family{U_\alpha}{\alpha \in A} \subset \tau\), for some indexing set \(A\), and its union \(U = \bigcup_{\alpha \in A} U_{\alpha}\). Let \(x \in U\), then there exists \(\beta \in A\) such that \(x \in U_\beta\). Since \(U_\beta \in \tau\), there exists \(r_\beta > 0\) such that \(B_{r_\beta}(x) \subset U_\beta\). Since \(U_\beta \subset U\), we have shown \(U \in \tau\).
\end{proof}


\begin{lemma}{Open balls are open sets in the metric topology}{open_balls}
    Let \((X,d)\) be a metric space. For all \(x \in X\) and all \(r > 0\), the open ball \(B_r(x)\) is open in the metric topology.
\end{lemma}
\begin{proof}
    We claim for all \(y \in B_r(x)\) the open ball \(B_{r - d(x,y)}(y)\) is contained in \(B_r(x)\). Indeed, if \(y \in B_r(x)\), \(d(x,y) < r\), then for all \(z \in B_{r - d(x,y)}(y)\) we have \(d(x,z) \leq d(x,y) + d(y, z) < r,\) hence \(z \in B_r(x)\) as claimed.
\end{proof}

\begin{proposition}{Closed ball contained in an open set}{closed_ball}
    Let \((X, d)\) be a metric space and let \(\tau\) be the metric topology. Define the closed ball \(\bar{B}_r(x)\) of radius \(r > 0\) centered at a point \(x \in X\) as the subset
    \begin{equation*}
        \bar{B}_r(x) = \setc{y \in X}{d(x,y) \leq r},
    \end{equation*}
    then \(\bar{B}_r(x)\) is closed. If \(U \in \tau\), then for every point \(x \in U\) there exists \(r > 0\) such that \(\bar{B}_r(x) \subset U\).
\end{proposition}
\begin{proof}
    Let \(r > 0\) and \(x \in X\) be fixed. Suppose \(y \in X \setminus \bar{B}_r(x)\), then \(d(x,y) = r + 2R\) for some \(R > 0\). For every \(z \in B_R(y)\) we have by the triangle inequality
    \begin{equation*}
        d(x,y) \leq d(x,z) + d(z,y) \implies d(x,y) \geq r + R,
    \end{equation*}
    that is, \(z \in X\setminus \bar{B}_r(x)\). Therefore, \(X \setminus \bar{B}_r(x)\) is open, hence \(\bar{B}_r(x)\) is closed.

    Let \(U \in \tau\), then for every \(x \in U\) there exists \(\varepsilon > 0\) such that \(B_{2\varepsilon}(x) \subset U\). Then \(\bar{B}_{\varepsilon}(x) \subset B_{2 \varepsilon}(x) \subset U\), as claimed.
\end{proof}

\begin{proposition}{Metric spaces are Hausdorff}{metric_hausdorff}
    If \((X, d)\) be a metric space, then \topology{X} is a Hausdorff space, where \(\tau_X\) is the metric topology.
\end{proposition}
\begin{proof}
    Let \(p, q \in X\), with \(p \neq q\). Let \(r = \frac12d(p, q)\) and consider the neighborhoods of \(p\) and \(q\) defined by the open balls \(U = B_r(p)\) and \(V = B_r(q)\). Suppose, by contradiction, there exists \(x \in U \cap V\), then
    \begin{equation*}
        d(p,q) \leq d(p, x) + d(x, q) < d(p, q)
    \end{equation*}
    by the triangle inequality. This shows \(p = q\), contradicting the hypothesis \(p \neq q\), hence the neighborhoods must satisfy \(U \cap V = \emptyset\).
\end{proof}

\subsection{Convergence of sequences}
As a convention, we denote the set of positive integers by \(\mathbb{N}\) and the set of non-negative integers by \(\mathbb{N}_0\). A sequence is a map \(x : \mathbb{N} \to X\), where we may denote \(x_n = x(n)\) for \(n \in \mathbb{N}\). Sometimes we may denote \(\family{x_n}{n\in \mathbb{N}}\) as the range of the map, often calling it a \emph{family}, or with abuse of notation as the map itself.
\begin{definition}{Convergent sequence}{converge}
    A sequence \family{x_n}{n\in \mathbb{N}} is said to \emph{converge to \(x \in X\) with respect to the metric space \((X, d)\)} if for all \(\varepsilon > 0\), there exists \(N \in \mathbb{N}\) such that
    \begin{equation*}
        n \geq N \implies d(x_n,x) < \varepsilon.
    \end{equation*}
    In this case, we write \(x_n \to x\) or \(\displaystyle\lim_{n\to \infty}x_n = x\), and may refer to \family{x_n}{n\in \mathbb{N}} as a \emph{convergent sequence in \((X, d)\)} and say that \(x \in X\) is the \emph{limit} of the sequence in \((X, d)\).
\end{definition}
\begin{remark}
    It follows from the triangle inequality that a sequence on a metric space converges to at most one element. Indeed, let the sequence \family{x_n}{n\in \mathbb{N}} converge to \(x\) and \(\tilde{x}\) with respect to the metric space \((X, d)\), then
    \begin{equation*}
        d(x, \tilde{x}) \leq d(x, x_n) + d(x_n, \tilde{x}),
    \end{equation*}
    for all \(n \in \mathbb{N}\). For all \(\varepsilon > 0\), there exist \(N, \tilde{N} \in \mathbb{N}\) such that
    \begin{equation*}
        n \geq \max(N, \tilde{N}) \implies d(x, x_n) < \frac12\varepsilon\quad\text{and}\quad d(\tilde{x}, x_n) < \frac12\varepsilon,
    \end{equation*}
    therefore \(d(x, \tilde{x}) < \varepsilon\). This justifies calling \(x \in X\) \emph{the} limit of a sequence.
\end{remark}

A subsequence is a composition \(x \circ n : \mathbb{N} \to X\), where \(n : \mathbb{N} \to \mathbb{N}\) is an increasing sequence of natural numbers, and we may denote it as \family{x_{n_k}}{k \in \mathbb{N}}.
\begin{proposition}{Every subsequence of a convergent sequence is convergent}{convergence_subsequence}
    Let \(\family{x_n}{n\in \mathbb{N}} \subset X\) be a sequence that converges to \(x\in X\) with respect to the metric space \((X, d)\). Then every subsequence \(\family{x_{n_k}}{k\in \mathbb{N}}\subset X\) converges to the same \(x\in X\) with respect to the metric space \((X, d)\).
\end{proposition}
\begin{proof}
    Let \(\varepsilon > 0\). Then, there exists \(N \in \mathbb{N}\) such that
    \begin{equation*}
        n \geq N \implies d(x_n, x) < \varepsilon.
    \end{equation*}
    Since \family{n_k}{k \in \mathbb{N}} is an increasing sequence of natural numbers, there exists \(K \in \mathbb{N}\) such that \(k \geq K \implies n_k \geq N\). Then,
    \begin{equation*}
        k \geq K \implies d(x_{n_k}, x) < \varepsilon,
    \end{equation*}
    that is, \(x_{n_k} \to x\) as desired.
\end{proof}

We may express convergence of a sequence in terms of the metric topology.
\begin{proposition}{Convergent sequence in the topological sense}{convergent_topology}
    Let \((X, d)\) be a metric space and let \(\tau\) be the metric topology. A sequence \(\family{x_n}{n\in \mathbb{N}}\subset X\) converges to some \(x \in X\) if and only if for every neighborhood \(U \in \tau\) of \(x\) there exists \(N\in \mathbb{N}\) such that \(n \geq N \implies x_n \in U\).
\end{proposition}
\begin{proof}
    Suppose the sequence converges to \(x\) in the metric space sense. Let \(U \in \tau\) be a neighborhood of \(x\). Then there exists \(\varepsilon > 0\) such that \(B_{\varepsilon}(x) \subset U\). From convergence, there exists \(N \in \mathbb{N}\) such that \(n \geq N \implies d(x_n, x) < \varepsilon\), that is, \(n \geq N \implies x_n \in B_{\varepsilon}(x) \subset U\).

    Suppose for every neighborhood \(U \in \tau\) of \(x\) there exists \(N_U \in \mathbb{N}\) such that \(x_n \in U\) for all \(n \geq N_U\). By \cref{lem:open_balls}, for all \(\varepsilon > 0\), the open ball \(B_{\varepsilon}(x)\) is a neighborhood of \(x\). By hypothesis, for all \(\varepsilon > 0\), there exists \(N_\varepsilon \in \mathbb{N}\) such that \(x_n \in B_{\varepsilon}(x)\) for all \(n \geq N_\varepsilon\). Hence, \(n \geq N_\varepsilon \implies d(x_n, x) < \varepsilon\).
\end{proof}

The convergence of sequences is related to dense sets in the metric topology. We first show how density is expressible in terms of the metric.
\begin{theorem}{Dense set in a metric space}{dense_metric}
    Let \((X, d)\) be a metric space and let \(\tau\) be the metric topology. A subset \(S \subset X\) is dense in \(X\) if and only if for all \(x \in X\) and all \(\varepsilon > 0\), there exists \(y \in S\) such that \(d(x,y) < \varepsilon\).
\end{theorem}
\begin{proof}
    Suppose \(S\) is dense in \(X\), in the metric space sense. Let \(x\in X\) and let \(U \in \tau\) be a neighborhood of \(x\). Since \(U\) is open, there exists \(\varepsilon > 0\) such that \(B_\varepsilon(x) \subset U\). By hypothesis, there exists \(y \in S\) such that \(y \in B_\varepsilon(x)\), hence \(y \in U \cap S\). That is, \(x\) is a point of closure of \(S\).

    Suppose \(\cl S = X\), then every point of \(X\) is a point of closure of \(S\). Let \(x \in \cl S\), then every neighborhood of \(x\) contains a point in \(S\). In particular, in the light of \cref{lem:open_balls}, for every \(\varepsilon > 0\), the open ball \(B_\varepsilon(x)\) contains a point in \(S\). That is, for all \(x \in X\) and for all \(\varepsilon > 0\) there exists \(y \in S\) such that \(d(x,y) < \varepsilon\).
\end{proof}
\begin{remark}
    If the metric space is understood from context, we will simply say \(S \subset X\) is dense in \(X\) if it is dense with respect to metric topology.
\end{remark}

\begin{proposition}{Sequences in a dense subset}{sequence_dense}
    Let \((X, d)\) be a metric space. A non-empty subset \(Y \subset X\) is a dense subset of \(X\) if and only if for each \(x \in X\) there exists a sequence \(\family{y_n}{n\in \mathbb{N}} \subset Y\) that converges to \(x\).
\end{proposition}
\begin{proof}
    Suppose \(Y\) is dense in \((X, d)\) and let \(x \in X\). For each \(n \in \mathbb{N}\), there exists \(y_n \in Y\) such that \(d(y_n, x) < \frac{1}{n}\). Thus, there is a sequence \(\family{y_n}{n\in \mathbb{N}} \subset Y\) that converges to \(x \in X\). Indeed, let \(\varepsilon > 0\) and set \(N = 1+ \ceil{\varepsilon^{-1}}\), then \(n \geq N \implies d(y_n, x) < \varepsilon\).

    Suppose there exists a sequence \(\family{y_n}{n\in \mathbb{N}}\subset Y\) that converges to \(x \in X\). For all \(\varepsilon > 0\), there exists \(N \in \mathbb{N}\) such that \(n \geq N \implies d(y_n, x) < \varepsilon\). In particular, \(y_N \in Y\) and \(d(y_N, x) < \varepsilon\), that is, there exists and element in \(Y\) that is arbitrarily close to each element in \(X\).
\end{proof}

The following results concern convergent sequences and their relation to closed sets.
\begin{lemma}{Convergent sequences and closure of a subset}{convergent_closure}
    Let \((X, d)\) be a metric space and let \(\tau\) be the metric topology. A point \(x \in X\) is a point of closure of \(S \subset X\) if and only if there exists a convergent sequence \(\family{s_n}{n\in \mathbb{N}}\subset S\) that converges to \(x\).
\end{lemma}
\begin{proof}
    Suppose \(x \in \cl S\), then every neighborhood of \(x\) has a non-empty intersection with \(S\). In particular, the family of open balls of radius \(\frac1{n}\) centered at \(x\) has an intersection at \(s_n \in S\) for all \(n \in \mathbb{N}\). That is, \family{s_n}{n\in \mathbb{N}} is a convergent sequence of elements of \(S\) that converges to \(x \in \cl S\).

    Suppose there exists a convergent sequence \family{s_n}{n\in \mathbb{N}} in \(S\) that converges to \(x \in X\). Let \(U\) be a neighborhood of \(x\), then there exists \(\varepsilon > 0\) such that \(B_\varepsilon(x) \subset U\). By convergence, there exists \(N \in \mathbb{N}\) such that \(s_n \in B_{\varepsilon}(x)\), that is \(s_n \in B_{\varepsilon}(x) \cap S \subset U \cap X\). Hence, \(x \in \cl S\).
\end{proof}

\begin{theorem}{Convergent sequences in a closed set}{convergent_closed}
    Let \((X, d)\) be a metric space and let \(\tau\) be the metric topology. A subset \(S \subset X\) is closed if and only if every convergent sequence of elements in \(S\) converges to some element in \(S\).
\end{theorem}
\begin{proof}
    Suppose there exists a convergent sequence of elements in \(S\) that converges to some \(x \in X \setminus S\). By \cref{lem:convergent_closure}, \(x \in \cl S\). That is, \(S \neq \cl S\), then by \cref{lem:closure_closed}, \(S\) is not closed.

    Suppose every convergent sequence of elements in \(S\) converges to some element in \(S\). In particular, there is no convergent sequence of elements in \(S\) that converges to some element in \(X \setminus S\). By \cref{lem:convergent_closure}, there is no point of closure of \(S\) in \(X \setminus S\), therefore \(\cl S = S\). By \cref{lem:closure_is_closed}, \(S\) is closed.
\end{proof}

\subsection{Continuity in metric spaces}
We may specialize the topological definition of continuity and state whether a map is continuous at a given point, rather than on the entire space. Moreover, we show that continuity at any point of the space implies continuity in the topological sense.
\begin{definition}{Continuity at a point}{continuity_metric}
    Let \((X,d_X)\) and \((Y, d_Y)\) be metric spaces. A map \(f : X \to Y\) is \emph{continuous at \(x_0 \in X\)} if for all \(\varepsilon > 0\), there exists \(\delta > 0\) such that
    \begin{equation*}
        d_X(x, x_0) < \delta \implies d_Y(f(x), f(x_0)) < \varepsilon.
    \end{equation*}
\end{definition}
% \begin{remark}
%     Even though it should be clear from the fact that the definition makes explicit use of the metrics in each metric space, we stress that the notion of continuity depends on the choice of metric on a set.
% \end{remark}
% \begin{remark}
%     It is easy to see isometries are continuous. Indeed, let \(f : X \to Y\) be a distance-preserving map, then for all \(\varepsilon > 0\), we have \(d_X(x, x_0) < \varepsilon \implies d_Y(f(x), f(x_0)) < \varepsilon\).
% \end{remark}

\begin{proposition}{Composition of continuous maps is continuous}{continuous_composition}
    Let \((X, d_X)\), \((Y, d_Y)\) and \((Z, d_Z)\) be metric spaces. If \(f : X \to Y\) is continuous at \(x_0 \in X\) and \(g : Y \to Z\) is continuous at \(f(x_0) \in Y\), then the composition \(g\circ f : X \to Z\) is continuous.
\end{proposition}
\begin{proof}
    We consider the continuity of \(g\circ f\) at \(x_0 \in X\). Let \(\varepsilon > 0\). From continuity of \(g\), there exists \(\eta > 0\) such that
    \begin{equation*}
        d_Y(y, f(x_0)) < \eta \implies d_Z(g(y), g\circ f(x_0)) < \varepsilon.
    \end{equation*}
    From continuity of \(f\), there exists \(\delta > 0\) such that
    \begin{equation*}
        d_X(x, x_0) < \delta \implies d_Y(f(x), f(x_0)) < \eta.
    \end{equation*}
    Then,
    \begin{equation*}
        d_X(x, x_0) < \delta \implies d_Z(g\circ f(x), g\circ f(x_0)) < \varepsilon,
    \end{equation*}
    that is, \(g\circ f\) is continuous at \(x_0\).
\end{proof}

\begin{theorem}{Continuity in metric spaces}{continuity_topology}
    Let \((X, d_X), (Y, d_Y)\) be metric spaces and let \(\tau_X,\tau_Y\) be the respective metric topologies. A map \(f : X \to Y\) is continuous if and only if \(f\) is continuous at every \(x_0 \in X\).
\end{theorem}
\begin{proof}
    Let \(f\) be continuous in the topological space sense. Let \(\varepsilon > 0\), \(x_0 \in X\), \(V = B_{\varepsilon}(f(x_0)) \in \tau_Y\) and \(U = \preim{f}{V} \neq \emptyset\). By hypothesis, \(U \in \tau_X\), and since \(x_0 \in U\), there exists \(\delta > 0\), such that \(B_\delta(x_0) \subset U\). We have thus shown that for all \(\varepsilon > 0\), there exists \(\delta > 0\) such that \(d_X(x,x_0) < \delta \implies d_Y(f(x), f(x_0)) < \varepsilon\).

    Let \(f\) be continuous in the metric space sense at every point of \(X\). If an open set in \(\tau_Y\) has empty intersection with the range of \(f\), then its preimage is the empty set, which is open. Let \(V \in \tau_Y\) be an open set containing at least one element of the range of \(f\). Let \(x_0 \in \preim{f}{V}\), then there exists \(\varepsilon > 0\) such that \(B_{\varepsilon}(f(x_0)) \subset V\). By continuity, there exists \(\delta > 0\) such that \(x \in B_\delta(x_0) \implies f(x) \in B_\varepsilon(f(x_0))\), therefore \(x \in B_\delta(x_0) \implies x \in \preim{f}{V}\), that is, \(\preim{f}{V}\) is open, since \(x_0\) is an interior point.
\end{proof}

The structure preserving maps of metric spaces, the bijective \emph{isometries} or \emph{distance-preserving functions}, are homeomorphisms.
\begin{definition}{Isometric metric spaces}{isometry}
    Let \((X, d_X), (Y, d_Y)\) be metric spaces. An \emph{isometry} is a map \(f : X \to Y\) that is \emph{distance-preserving}, that is, \(d_X(x_1, x_2) = d_Y(f(x_1), f(x_2))\) for all \(x_1,x_2 \in X\). If there exists a bijective isometry, we say \((X, d_X)\) and \((Y, d_Y)\) are \emph{isometric metric spaces}.
\end{definition}
\begin{proposition}{Surjective isometries are homeomorphisms}{isometry_continuous}
    If \(f: X \to Y\) is a surjective isometry between metric spaces \((X, d_X)\) and \((Y, d_Y)\), then \(f\) is a homeomorphism with respect to the metric topologies.
\end{proposition}
\begin{proof}
    Let \(x_0 \in X\) and \(\varepsilon > 0\). As \(f\) is distance-preserving, we have
    \begin{equation*}
        d_X(x, x_0) < \varepsilon \implies d_Y(f(x), f(x_0)) < \varepsilon,
    \end{equation*}
    hence \(f\) is continuous at \(x_0\). Since \(x_0\) is arbitrary, \(f\) is continuous. We've shown distance-preserving maps are continuous.

    Let \(x_1, x_2 \in X\) be such that \(f(x_1) = f(x_2)\). Then \(d_Y(f(x_1), f(x_2)) = 0\) implies \(d_X(x_1, x_2) = 0\) by hypothesis. Hence \(x_1 = x_2\), and \(f\) is injective. That is, \(f\) is a bijection and admits an inverse map \(f^{-1} : Y \to X\).

    To show the inverse map is continuous, we prove \(f^{-1} : Y \to X\) is distance-preserving. For all \(y, \tilde{y} \in Y\) we have
    \begin{equation*}
        d_X(f^{-1}(y), f^{-1}(\tilde{y})) = d_Y\left(f \circ f^{-1} (y), f \circ f^{-1} (\tilde{y})\right) = d_Y(y, y_0),
    \end{equation*}
    since \(f\) is a bijection and \(f\) is distance-preserving.
\end{proof}
\begin{remark}
    We've shown every isometry is necessarily injective and continuous with respect to the metric topology.
\end{remark}
\begin{proposition}{Composition of isometries is an isometry}{composition_isometry}
    Let \((X, d_X),(Y, d_Y),(Z, d_Z)\) be metric spaces. If \(f : X \to Y\) and \(g : Y \to Z\) are isometries, then \(g \circ f : X \to Z\) is an isometry.
\end{proposition}
\begin{proof}
    Let \(x_1, x_2 \in X\), then
    \begin{equation*}
        d_Z(g \circ f(x_1), g \circ f(x_2)) = d_Y(f(x_1), f(x_2)) = d_X(x_1, x_2),
    \end{equation*}
    hence \(g \circ f\) is an isometry.
\end{proof}

We may define a stronger notion of continuity with the additional structure of the metric space.
\begin{definition}{Uniform continuity}{uniform_continuity}
    Let \((X, d_X), (Y, d_Y)\) be metric spaces. A map \(f : X \to Y\) is \emph{uniformly continuous in a non-empty subset \(A\)}, if for all \(\varepsilon > 0\) there exists \(\delta > 0\) such that for every \(x,y \in A\)
    \begin{equation*}
        d_X(x, y) < \delta \implies d_Y(f(x), f(y)) < \varepsilon.
    \end{equation*}
    Moreover, if \(A = X\), we say \(f\) is uniformly continuous.
\end{definition}

\begin{proposition}{Uniform continuity implies continuity}{uniform_continuity}
    If \(f : X \to Y\) is a uniformly continuous map between the metric spaces \((X, d_X)\) and \((Y, d_Y)\), then \(f\) is continuous.
\end{proposition}
\begin{proof}
    Let \(x_0 \in X\) and let \(\varepsilon > 0\). Since \(f\) is uniformly continuous, there exists \(\delta > 0\) such that for all \(x,\tilde{x} \in X\), we have \(d_X(x, \tilde{x}) < \delta \implies d_Y(f(x), f(\tilde{x})) < \varepsilon\). In particular, fixing \(\tilde{x} = x_0\), we have for all \(x \in B_\delta(x_0)\) that \(f(x) \in B_\varepsilon(f(x_0))\), hence \(f\) is continuous at \(x_0\).
\end{proof}

Finally, we relate continuity with convergence of sequences. We note that the theorem could be weakened to continuity on a subset or a single point as in definition \cref{def:continuity_metric}.
\begin{theorem}{Convergent sequence definition of continuity}{convergence_continuity}
    Let \((X, d_X)\) and \((Y, d_Y)\) be metric spaces. A map \(f : X \to Y\) is continuous if and only if for every \(x \in X\) and for every sequence \(\family{x_n}{n \in \mathbb{N}} \subset X\) that converges to \(x\) with respect to \((X, d_X)\) we have
    \begin{equation*}
        \lim_{n \to \infty} f(x_n) = f(x)
    \end{equation*}
    with respect to \((Y, d_Y)\).
\end{theorem}
\begin{proof}
    Suppose \(f\) is continuous and let \(\varepsilon > 0\). Let \(\tilde{x} \in X\) and let \(\family{\tilde{x}_n}{n\in \mathbb{N}} \subset X\) be a sequence that converges to \(\tilde{x}\) with respect to \((X, d_X)\). From continuity there exists \(\delta > 0\) such that
    \begin{equation*}
        d_X(x, \tilde{x}) < \delta \implies d_Y(f(x), f(\tilde{x})) < \varepsilon.
    \end{equation*}
    Since the sequence is convergent, there exists \(N \in \mathbb{N}\) such that
    \begin{equation*}
        n \geq N \implies d_X(\tilde{x}_n, \tilde{x}) < \delta,
    \end{equation*}
    thus
    \begin{equation*}
        n \geq N \implies d_Y(f(\tilde{x}_n), f(\tilde{x})) < \varepsilon.
    \end{equation*}
    Since \(\tilde{x}\) and \family{\tilde{x}_n}{n\in \mathbb{N}} were arbitrary, we have shown that \(f(x_n) \to f(x)\) for all \(x \in X\).

    To show the converse, we prove its contrapositive: we will show that if \(f\) is not continuous, then there exists a convergent sequence \(x_n\to \tilde{x} \in X\) such that \(f(x_n)\) does not converge to \(f(\tilde{x})\), for some \(\tilde{x} \in X\). If \(f\) is not continuous, then there exists \(\eta > 0\) such that for a given \(\delta > 0\), there exists \(x \in X\) such that \(d_X(x, \tilde{x}) < \delta\) but \(d_Y(f(x), f(\tilde{x})) \geq \eta\). We consider a family \(\family{\delta_n}{n\in \mathbb{N}}\) with \(\delta_n = 2^{-n}\), for all \(n \in \mathbb{N}\). Let \(\family{x_n}{n\in \mathbb{N}}\subset X\) be a sequence of elements that make \(f\) fail to be continuous, that is,
    \begin{equation*}
        d_X(x_n ,\tilde{x}) < \delta_n \implies d_Y(f(x_n), f(\tilde{x})) \geq \eta
    \end{equation*}
    for all \(n \in \mathbb{N}\). We have thus constructed a convergent sequence \(x_n \to \tilde{x}\) such that \(f(x_n)\) does not converge to \(f(\tilde{x})\).
\end{proof}
\begin{remark}
    This formulation of continuity makes it evident that the image of a convergent sequence under a continuous map is itself a convergent sequence.
\end{remark}

\subsection{Completeness and the Cauchy property}
It is important to note that the convergence of a sequence cannot be understood alone from properties of the sequence itself: one must provide an element of the metric space.
\begin{definition}{Cauchy sequence}{cauchy}
    A sequence \family{x_n}{n \in \mathbb{N}} is said to be a \emph{Cauchy sequence}, or simply to be \emph{Cauchy}, with respect to the metric space \((X, d)\), if it has the \emph{Cauchy property}: for all \(\varepsilon > 0\), there exists an \(N \in \mathbb{N}\) such that
    \begin{equation*}
        n,m \geq N \implies d(x_n, x_m) < \varepsilon.
    \end{equation*}
\end{definition}

\begin{proposition}{Convergent subsequence of a Cauchy sequence}{subsequence_Cauchy}
    Let \((X, d)\) a metric space and let \(\family{x_n}{n \in \mathbb{N}}\subset X\) be a Cauchy sequence in \((X,d)\). If there exists a subsequence \(\family{x_{n_j}}{j \in \mathbb{N}}\) that converges to \(\tilde{x}\) in \((X, d)\), then the Cauchy sequence converges against \(\tilde{x}\).
\end{proposition}
\begin{proof}
    Let \(\varepsilon > 0\), then there exists \(N > 0\) such that for all \(m, n > N\)
    \begin{equation*}
        d(x_n, x_m) < \frac12 \varepsilon,
    \end{equation*}
    since the sequence is Cauchy with respect to the metric \(d\). Moreover, there exists \(M > 0\) such that for all \(n_j > J\)
    \begin{equation*}
        d(x_{n_j}, \tilde{x}) < \frac12 \varepsilon.
    \end{equation*}

    Set \(L = \max\set{M,N}\), then for all \(m, n_j > M\) it follows that
    \begin{align*}
        d(x_m, \tilde{x}) \leq d(x_m, x_{n_j}) + d(x_{n_j}, \tilde{x}) < \varepsilon,
    \end{align*}
    that is, the Cauchy sequence converges against \(\tilde{x} \in X\) with respect to \((X,d)\).
\end{proof}

The convergent sequences we have considered so far are Cauchy sequences.
\begin{proposition}{Every convergent sequence has the Cauchy property}{convergent_cauchy}
    Let \((X, d)\) be a metric space. If \family{x_n}{n\in \mathbb{N}} is a convergent sequence with respect to \((X, d)\), then it is a Cauchy sequence with respect to \((X, d)\).
\end{proposition}
\begin{proof}
    Let \(x \in X\) be the unique element in \(X\) such that \(x_n \to x\), with respect to the metric \(d\). Then, for all \(\varepsilon > 0\), there exists \(N \in \mathbb{N}\) such that \(n \geq N \implies d(x_n, x) < \frac12\varepsilon.\) In particular, let \(\ell, m \geq N\), then \(d(x_\ell, x) < \frac12 \varepsilon\) and \(d(x_m, x) < \frac12 \varepsilon\), hence
    \begin{equation*}
        d(x_\ell, x_m) \leq d(x_\ell, x) + d(x, x_m) < \varepsilon,
    \end{equation*}
    which shows the sequence is Cauchy.
\end{proof}

As isometries are distance preserving, the image of a Cauchy sequence under an isometry is Cauchy. In fact, uniform continuity preserves the Cauchy property.
\begin{proposition}{Uniform continuity preserves the Cauchy property}{uniformly_continuous_cauchy}
    Let \(f : X \to Y\) be a uniformly continuous map with respect to the metric spaces \((X, d_X)\) and \((Y, d_Y)\). If \(\family{x_n}{n\in \mathbb{N}} \subset X\) is a Cauchy sequence with respect to \((X, d_X)\), then \(\family{f(x_n)}{n\in \mathbb{N}} \subset Y\) is a Cauchy sequence with respect to \((Y, d_Y)\).
\end{proposition}
\begin{proof}
    Let \(\varepsilon > 0\). From uniform continuity, there exists \(\delta > 0\) such that for all \(x, \tilde{x} \in X\) we have \(d_X(x, \tilde{x}) < \delta \implies d_Y(f(x), \tilde{x}) < \varepsilon.\) Since the sequence is Cauchy, there exists \(N \in \mathbb{N}\) such that for all \(n, m \geq N\) we have \(d_X(x_n, x_m) < \delta\). That is,
    \begin{equation*}
        n,m\geq N \implies d_Y(f(x_n), f(x_m)) < \varepsilon,
    \end{equation*}
    hence \(\family{f(x_n)}{n\in \mathbb{N}}\) is Cauchy.
\end{proof}

\begin{proposition}{Cauchy sequences are bounded}{Cauchy_bounded}
    Let \((X, d)\) be a metric space. If \(\family{x_n}{n\in \mathbb{N}}\subset X\) is a Cauchy sequence with respect to \((X,d)\), then it is bounded, that is, there exists \(M > 0\) such that  \(\sup\setc{d(x_n, x_m)}{n,m \in \mathbb{N}} \leq M\).
\end{proposition}
\begin{proof}
    Let \(\varepsilon > 0\), then by the Cauchy property there exists \(N \in \mathbb{N}\) such that
    \begin{equation*}
        m,n \geq N \implies d(x_n, x_m) < \varepsilon.
    \end{equation*}
    Let \(\mathbb{N}_{< N} = \setc*{j \in \mathbb{N}}{j < N}\) be the set of the first \(N\) natural numbers and let
    \begin{equation*}
        M_0 = \max\setc*{d(x_n, x_m)}{n,m \in \mathbb{N}_{< N}},
    \end{equation*}
    which is well-defined since \(\mathbb{N}_{< N}\) is clearly finite. Next, we let \(\ell \geq N\) and define
    \begin{equation*}
        M_1 = \max\setc*{d(x_n, x_\ell)}{n \in \mathbb{N}_{< N}},
    \end{equation*}
    then for all \(n \in \mathbb{N}_{<N}\) and all \(m \geq N\), we have
    \begin{align*}
        d(x_n, x_m) &\leq d(x_n, x_\ell) + d(x_\ell, x_m)\\
                    &\leq M_1 + \varepsilon.
    \end{align*}
    Finally, we take \(M = \max\set{\varepsilon, M_0, M_1 + \varepsilon}\), and we have
    \begin{equation*}
        d(x_n, x_m) \leq M,
    \end{equation*}
    for all \(n, m \in \mathbb{N}\). That is, \(M\) is an upper bound for the distance between elements of \family{x_n}{n\in \mathbb{N}}.
\end{proof}

As opposed to convergence, whether a sequence is Cauchy and properties that follow from it can be studied solely from the sequence itself. It would be desirable, then, if one could decide if a sequence converges based on whether it is Cauchy. However, it is not always the case: take for example the metric space \((\mathbb{Q}, \abs{\noarg})\) and the Cauchy sequence \(n \mapsto \frac{1}{n!}\), which converges in \((\mathbb{R}, \abs{\noarg})\) to \(e\), but it does not converge in \((\mathbb{Q}, \abs{\noarg})\).

\begin{definition}{Complete metric space}{completeness}
    A metric space \((X, d)\) is \emph{complete} if every Cauchy sequence \(\family{x_n}{n \in \mathbb{N}} \subset X\) converges to some \(x \in X\) with respect to \((X, d)\).
\end{definition}

\begin{example}{\((\mathcal{C}([a,b];\mathbb{C}), d_\infty)\) is a complete metric space}{sup_norm_complete}
    The metric space \((\mathcal{C}([a,b];\mathbb{C}), d_\infty)\) is a complete metric space.
\end{example}
\begin{proof}
    Let \(\family{f_n}{n \in \mathbb{N}} \subset \mathcal{C}([a,b];\mathbb{C})\) be a Cauchy sequence of continuous functions in \([a,b]\). Then, for some \(y \in [a,b]\), we have, by the definition of the supremum,
    \begin{equation*}
        \abs*{f_n(y) - f_m(y)} \leq \sup_{x \in [a,b]} \abs*{f_n(x) - f_m(x)} = d_\infty(f_n, f_m).
    \end{equation*}
    For all \(\varepsilon > 0\), there exists \(N \in \mathbb{N}\) such that
    \begin{align*}
        n,m \geq N &\implies d_\infty(f_n, f_m) < \varepsilon\\
                   &\implies \abs*{f_n(y) - f_m(y)} < \varepsilon
    \end{align*}
    then as a result the sequence \(\family{f_n(y)}{n \in \mathbb{N}} \subset \mathbb{C}\) is Cauchy with respect to the metric space \((\mathbb{C}, d)\), where
    \begin{align*}
        d : \mathbb{C} \times \mathbb{C} &\to \mathbb{R}\\
        (z,w) &\mapsto \abs{z-w}
    \end{align*}
    is the usual metric in the complex plane. From completeness of the complex plane with respect to the usual metric, this sequence converges to some \(\xi_y \in \mathbb{C}\). Since \(y\) is arbitrary and from the uniqueness of convergence, we may define the map
    \begin{align*}
        f : [a,b] &\to \mathbb{C}\\
                y &\mapsto \xi_y.
    \end{align*}
    We will show \(f\) is continuous in \([a,b]\) and that \(f_n \to f\).

    Let \(\varepsilon > 0\) and let \(x_0 \in [a,b]\). From repeatedly using the triangle inequality in \((\mathbb{C}, d)\), we have
    \begin{align*}
        d(f(x), f(x_0)) \leq d(f(x), f_n(x)) + d(f_n(x), f_n(x_0)) + d(f_n(x_0), f(x_0)),
    \end{align*}
    for all \(n \in \mathbb{N}\) and \(x \in [a,b]\). From the convergence of \(\family{f_n(x)}{n\in \mathbb{N}}, \family{f_n(x_0)}{n\in \mathbb{N}} \subset \mathbb{C}\) with respect to \((\mathbb{C}, d)\), there exists \(M \in \mathbb{N}\) such that
    \begin{equation*}
        n \geq M \implies d(f(x), f_n(x)) < \frac13 \varepsilon \quad\text{and}\quad d(f(x_0), f_n(x_0)) < \frac13 \varepsilon.
    \end{equation*}
    From the continuity of \(f_n\), there exists \(\delta > 0\) such that
    \begin{equation*}
        \abs{x - x_0} < \delta \implies \abs*{f_n(x) - f_n(x_0)} = d(f_n(x), f_n(x_0)) < \frac13\varepsilon.
    \end{equation*}
    Hence, we have shown there exists \(\delta > 0\) such that
    \begin{equation*}
        \abs{x - x_0} < \delta \implies d(f(x), f(x_0)) < \varepsilon,
    \end{equation*}
    that is, \(f\) is continuous at \(x_0\). Since \(x_0\) was arbitrary, \(f \in \mathcal{C}([a,b]; \mathbb{C})\).

    Let \(\eta > 0\). We consider an increasing sequence of natural numbers \family{N_k}{k\in \mathbb{N}} such that \(N_{k + 1} > N_k\) for all \(k \in \mathbb{N}\) and
    \begin{equation*}
        n, m > N_k \implies d_\infty(f_m, f_n) < \frac{\eta}{2^{k+1}},
    \end{equation*}
    which are guaranteed to exist since the sequence of functions is Cauchy. From this sequence we choose another increasing sequence sequence of natural numbers \family{n_k}{k \in \mathbb{N}} such that \(n_{k+1} > n_k\) and \(n_k > N_k\). We turn our attention to the subsequence \family{f_{n_k}}{k\in \mathbb{N}} of continuous functions. For a given \(\ell \in \mathbb{N}\) we have
    \begin{equation*}
        d_\infty(f_{n_{\ell + 1}}, f_{n_{\ell}}) < \frac{\eta}{2^{\ell + 1}},
    \end{equation*}
    since \(n_{\ell+1} > n_{\ell} > N_\ell\). For all \(x \in [a,b]\)  and \(k \in \mathbb{N}\), we can use the telescoping sum
    \begin{equation*}
        f_{n_k}(x) - f_{n_1}(x) = \sum_{\ell = 1}^{k - 1} [f_{n_{\ell + 1}}(x) - f_{n_{\ell}}(x)]
    \end{equation*}
    to estimate
    \begin{align*}
        \abs*{f_{n_k}(x) - f_{n_1}(x)} &\leq \sum_{\ell = 1}^{k - 1} \abs*{f_{n_{\ell + 1}}(x) - f_{n_\ell}(x)}\\
                                       &\leq \sum_{\ell = 1}^{k - 1} \sup_{x \in [a,b]}\abs*{f_{n_{\ell + 1}}(x) - f_{n_\ell}(x)}\\
                                       &\leq \sum_{\ell = 1}^{k - 1} d_\infty(f_{n_{\ell+1}}, f_{n_\ell})\\
                                         &< \frac12 \eta \sum_{\ell = 1}^{k-1}\frac{1}{2^\ell} = \frac12 \eta \left(1 - \frac{1}{2^{k - 1}}\right).
    \end{align*}
    Then, for each \(x \in [a,b]\), we have
    \begin{align*}
        \abs*{f(x) - f_{n_1}(x)} &\leq \abs*{f(x) - f_{n_k}(x)} + \abs*{f_{n_k}(x) - f_{n_1}(x)}\\
                                 &< \abs*{f(x) - f_{n_k}(x)} + \frac12 \eta\left(1 - \frac1{2^{k-1}}\right),
    \end{align*}
    then since the left hand side does not depend on \(k\), we have, after taking the limit \(k \to \infty\) and recalling \(f_{n_k}(x) \to f(x)\), that
    \begin{equation*}
        \abs*{f(x) - f_{n_1}(x)} \leq \frac12 \eta.
    \end{equation*}
    We have shown that for all \(n > N_1\) and for all \(x \in [a,b]\),
    \begin{equation*}
        \abs*{f(x) - f_n(x)} \leq \abs*{f(x) - f_{n_1}(x)} + \abs*{f_{n_1}(x) - f_n(x)} \leq \frac34 \eta.
    \end{equation*}
    Hence, \(\frac34\eta\) is an upper bound for \(\setc{\abs{f(x)-f_n(x)}}{x \in [a,b]}\), therefore \(d_\infty(f, f_n) < \eta\), proving the convergence of the Cauchy sequence.
\end{proof}

Completeness of a subset of a complete metric space is equivalent with the subset being closed in the metric topology.
\begin{theorem}{Completeness and closed sets}{complete_closed}
    Let \((X, d)\) be a complete metric space and let \(\tau\) be the metric topology. A subset \(S \subset X\) is closed with respect to \(\tau\) if and only if \(S\) is complete, in the sense that the metric space \((S, \restrict{d}{S})\) is complete.
\end{theorem}
\begin{proof}
    Suppose \(S\) is closed. By \cref{thm:convergent_closed}, every convergent sequence of elements in \(S\) converges to some element in \(S\). In particular, since \(X\) is complete, every Cauchy sequence of elements in \(S\) is convergent, therefore they must converge to some element in \(S\). Hence, \(S\) is complete.

    Suppose \(S\) is complete. Then every Cauchy sequence of elements in \(S\) converges to some element in \(S\). Recalling that every convergent sequence is Cauchy, we have by \cref{thm:convergent_closed} that \(S\) is closed.
\end{proof}

Finally, we give Baire's category argument.
\begin{theorem}{Complete metric spaces are of the second category}{Baire_Hausdorff}
    A complete metric space is of the second category with respect to the metric topology.
\end{theorem}
\begin{proof}
    Let \((X, d)\) be a complete metric space and let \(\tau\) be the metric topology. Suppose, by contradiction, \(X\) is of the first category. Then there exists a countable collection of nowhere dense sets \(\family{M_n}{n \in \mathbb{N}} \subset \mathbb{P}(X)\) such that \(X = \bigcup_{n \in \mathbb{N}} M_n\). Without loss of generality, we may assume \(M_n\) closed, since \(\cl_X(M_n)\) is nowhere dense in \(X\). That is, \(X \setminus M_n \in \tau\) and, by \cref{thm:closure_interior,thm:interior_closure_nowhere_dense}, \(\cl_X(X \setminus M_n) = X\), hence \(X \setminus M_n\) is dense in \(X\).

    Let \(x_0 \in X\), then every neighborhood of \(x_0\) has non-empty intersection with every \(X\setminus M_n\). Let \(U\) be a neighborhood of \(x_0\), then \(U \cap (X\setminus M_n)\) is dense and open in \(X\). Then, there exists a closed ball \(S_1 = \bar{B}_{r_1}(x_1)\) contained in \(U \cap (X \setminus M_1)\), where we assume \(0 < r_1 < 1\) and \(x_1\) is an arbitrary point of \(U \cap (X \setminus M_1)\). Recursively for \(n > 1\), there exists a closed ball \(S_n = \bar{B}_{r_n}(x_n)\) contained in \(S_{n-1} \cap (X \setminus M_n)\) such that \(0 < r_n < \frac1n\). Then, for all \(n,m \in \mathbb{N}\) we have \(n < m \implies x_m \in S_n\) and \(S_n \cap M_m = \emptyset\).

    This yields a sequence \(x : \mathbb{N} \to X\) which has the Cauchy property. Indeed, for all \(n,m \in \mathbb{N}\) with \(n < m\) we have \(d(x_n, x_m) \leq r_n < \frac1n\). Hence, for all \(\varepsilon > 0\), for all \(n,m \geq \ceil{\frac{1}{\varepsilon}}\) we have \(d(x_n,x_m) < \varepsilon\).


    As \(X\) is complete, there exists \(\tilde{x} \in X\) such that \(x_n \to \tilde{x}\). By the triangle inequality, we have have
    \begin{equation*}
        d(\tilde{x}, x_n) \leq d(x_n, x_m) + d(x_m, \tilde{x}) \leq r_n + d(x_m, \tilde{x})
    \end{equation*}
    for all \(n,m \in \mathbb{N}\) with \(n < m\). By convergence, we have \(d(\tilde{x}, x_n) \leq r_n\), hence \(\tilde{x} \in \cl_X(B_{r_n}(x_n))\) for all \(n\in \mathbb{N}\). By construction, \(\tilde{x} \in \bigcap_{n\in \mathbb{N}} (X\setminus M_n)\), then by \cref{lem:complement_union}, we have \(\tilde{x} \in X \setminus \bigcup_{n\in \mathbb{N}} M_n\). This contradicts the hypothesis the collection \(\family{M_n}{n\in \mathbb{N}}\) covers \(X\), thus showing there is no such collection.
\end{proof}

% normed linear spaces
% vim spl=en_us
\section{Topological linear spaces}
Throughout these notes we will consider linear spaces equipped with some topology.
\begin{definition}{Topological linear space}{tvs}
    Let \((X, +, \cdot)\) be a linear space over a field \(\mathbb{K}\) (either \(\mathbb{R}\) or \(\mathbb{C}\)). A \emph{vector topology} \(\tau\) is a topology on \(X\) such that
    \begin{enumerate}[label=(\alph*)]
        \item for all \(x \in X\), \(X \setminus \set{x} \in \tau\); and
        \item the linear space operations are continuous with respect to \(\tau\);
    \end{enumerate}
    and we say \((X, \tau)\) is a \emph{topological linear space}.
\end{definition}

Some basic results follow from requiring the linear space operations to be continuous. First, let \(S,T \subset X\) be a subset of a topological linear space \(X\), let \(x \in X\), and let \(\alpha \in \mathbb{K}\), then we denote
\begin{align*}
    S + T &= \setc{v \in X}{\exists s \in S, \exists t \in T : v = s + t} = \setc{s + t}{s \in S, t \in T},\\
    x + S &= \setc{v \in X}{\exists s \in S : v = x + s} = \setc{x + s}{s \in S},\quad\text{and}\\
    \alpha S &= \setc{v \in X}{\exists s \in S : v = \alpha s} = \setc{\alpha s}{s \in S}
\end{align*}
as the images of these subsets under the basic linear space operations.

\begin{proposition}{Characterization of open sets in topological linear spaces}{tvs}
    Let \(X\) be a linear space and let \(\tau\) be a topology on \(X\) such that the linear space operations are continuous. Let us denote the translation and multiplication maps by
    \begin{align*}
        T_x : X &\to X&
        M_\alpha : X &\to X\\
        y &\mapsto x + y&
              y &\mapsto \alpha y,
    \end{align*}
    where \(x \in X\) and \(\alpha \in \mathbb{K}\). Then
    \begin{enumerate}[label=(\alph*)]
        \item for all \(x \in X,\) the translation \(T_x\) is a homeomorphism;
        \item a subset \(S \subset X\) is open if and only if \(x + S\) is open for all \(x \in X\);
        \item for all \(\alpha \in \mathbb{K}\setminus\set{0}\), the multiplication \(M_\alpha\) is a homeomorphism; and
        \item a subset \(S \subset X\) is open if and only if \(\alpha S\) is open for all \(\alpha \in \mathbb{K} \setminus\set{0}\).
    \end{enumerate}
\end{proposition}
\begin{proof}
    Let \(x \in X\), then \(T_x\) is a bijection as it has the inverse \(T_{-x}\). Let \(\jmath_{x} : X \to \set{x} \times X\) be the continuous map defined by \(y \mapsto (x,y)\), then \(T_x = \restrict{+}{\set{x}\times X} \circ \jmath_{x}\) is continuous. As \(x\) is arbitrary, then the inverse \(T_{-x}\) is also continuous, therefore translations are homeomorphisms. Let \(S \subset X\), then
    \begin{equation*}
        S = \preim{T_{x}}{S + x} = T_{-x}(S + x)
    \end{equation*}
    hence \(S\) is open if and only if \(S + x\) is open for all \(x \in X\). An analogous argument shows (c) and (d).
\end{proof}


Since a set is open if and only if its translates are open, we may characterize a vector topology with a system of neighborhoods of a vector.
\begin{definition}{Local base}{local_basis}
    Let \((X,\tau)\) be a topological space. A \emph{local basis at a point \(x \in X\)} is a collection \(\mathcal{B} \subset \tau\) such that if \(U \in \tau\) is a neighborhood of \(x\), then there exists \(B \in \mathcal{B}\) such that \(B \subset U\).
\end{definition}
In the context of linear topological spaces, we will use only local basis at the zero vector.
\begin{proposition}{Open sets and local basis}{local_basis}
    Let \((X, \tau)\) be a topological linear space and let \(\mathcal{B} \subset \tau\) be a local basis. Then the translates of the elements of \(\mathcal{B}\) form a basis for \(\tau,\) that is, if \(U \in \tau,\) then there exists \(\family{x_{\lambda}}{\lambda \in \Lambda} \subset X\) and \(\family{B_\lambda}{\lambda \in \Lambda} \subset \mathcal{B}\) such that \(U = \bigcup_{\lambda \in \Lambda}{x_\lambda + B_\lambda}\).
\end{proposition}
\begin{proof}
    Let \(U\in \tau,\) then for every \(u \in U\) there exists \(V_u \in \tau\) with \(u \in V_u\) and \(V_u \subset U,\) hence \(U = \bigcup_{u \in U} V_u\). Moreover, each \(V_u - u\) is a neighborhood of zero, hence there exists \(B_u \in \mathcal{B}\) such that \(B_u \subset V_u - u,\) thus \(u \in B_u + u \subset V_u\) and we conclude \(U = \bigcup_{u \in U} u + B_u,\) as desired.
\end{proof}

In a linear space \(X,\) a subset \(S \subset X\) is \emph{balanced} if \(\alpha S \subset S\) for all \(\alpha \in \mathbb{K}\) with \(\abs{\alpha} \leq 1\); and it is \emph{convex} if \(\alpha S + (1 - \alpha) S \subset S\) for all \(\alpha \in [0,1].\) 
\begin{proposition}{Properties}{}
    Let \(X\) be a linear space and let \(\tau\) be a topology such that the linear space operations are continuous. Then
    \begin{enumerate}[label=(\alph*)]
        \item if \(S \subset X,\) then \(\cl_XS = \bigcap_{V \in \mathcal{V}} (S + V),\) where \(\mathcal{V} \subset \tau\) is the set of neighborhoods of zero;
        \item if \(S, T \subset X,\) then \(\cl_XS + \cl_XT \subset \cl_X(S + T)\);
        \item if \(S \subset X\) is a linear subspace, then so is \(\cl_XS\);
        \item if \(S \subset X\) is convex, then so are \(\cl_XS\) and \(\inte_XS\);
        \item if \(S \subset X\) is balanced, then so is \(\cl_XS\); and
        \item if \(S \subset X\) is balanced and \(0 \in S,\) then \(\inte_XS\) is balanced.
    \end{enumerate}
\end{proposition}
\begin{proof}
    Notice
    \begin{equation*}
        x \in \cl_XS \iff \forall V \in \mathcal{V} : (x + V) \cap S \neq \emptyset \iff \forall V \in \mathcal{V} : x \in S - V \iff x \in \bigcap_{V \in \mathcal{V}}(S + V),
    \end{equation*}
    and we conclude (a). From \cref{thm:closure_continuity} and the continuity of addition, (b) follows. 

    Suppose \(S\) is a linear subspace. As multiplication is a homeomorphism, we have \(\cl_X(\lambda S) = \lambda\cl_XS\) for all \(\lambda \in \mathbb{K}\), hence the continuity of addition yields
    \begin{equation*}
        \alpha \cl_XS + \beta \cl_XS = \cl_X(\alpha S) + \cl_X(\beta S) \subset \cl_X(\alpha S + \beta S) \subset \cl_X(S),
    \end{equation*}
    for all \(\alpha, \beta \in \mathbb{K}\). If we suppose instead that \(S\) is convex, we have similarly
    \begin{equation*}
        \alpha \cl_XS + (1 - \alpha) \cl_XS \subset \cl_X\left(\alpha S + (1 - \alpha) S\right) \subset \cl_X S
    \end{equation*}
    for all \(\alpha \in [0,1]\). If we suppose that \(S\) is balanced
\end{proof}

\subsection{Separation properties of a topological linear space}
Let us now show requiring the singletons to be closed is equivalent to requiring the topology to be Hausdorff.
\begin{lemma}{Every neighborhood of zero contains a symmetric neighborhood of zero}{symmetric_neighborhood}
    Let \(X\) be a linear space and let \(\tau\) be a topology on \(X\) such that the linear space operations are continuous. If \(U \in \tau\) is a neighborhood of zero, then there exists an open set \(W \in \tau\) such that
    \begin{enumerate}[label=(\alph*)]
        \item \(W\) is a neighborhood of zero;
        \item \(W + W \subset U\); and
        \item \(W = -W\).
    \end{enumerate}
\end{lemma}
\begin{proof}
    Notice \((0,0) \in \preim{+}{U}\), hence there exists neighborhoods of zero \(V_1, V_2 \in \tau\) such that \(V_1 + V_2 \subset U.\) As \(0 \in V_1 \cap V_2\) and \(0 \in (-V_1) \cap (-V_2)\), the non-empty set \(W = V_1 \cap V_2 \cap (-V_1) \cap (-V_2)\) is an open neighborhood of \(0\). It is clear this set satisfies the desired properties.
\end{proof}

\begin{lemma}{Existence of disjoint open sets that contain disjoint compact and closed sets}{tvs_hausdorff}
    Let \(X\) be a linear space and let \(\tau\) be a topology on \(X\) such that the linear space operations are continuous. Suppose \(K \subset X\) is compact, \(C \subset X\) is closed, and that \(K \cap C = \emptyset\). Then there exists a neighborhood of zero \(V \in \tau\) such that \((K + V) \cap (C + V) = \emptyset\) and \(\cl_X(K+V) \cap C = \emptyset\).
\end{lemma}
\begin{proof}
    We may assume without loss of generality that \(K\) is non-empty, for \(\empty + V = \emptyset\) and the lemma follows trivially. Let \(x \in K\), then there exists a neighborhood of \(x\) that does not intersect \(C,\) hence there exists a symmetric neighborhood of zero \(V_x \in \tau\) such that \(C \cap (x + V_x + V_x + V_x) = \emptyset\), by \cref{lem:symmetric_neighborhood}.

    Clearly the sets \(x + V_x\) form an open cover of \(K\), hence there exists \(N \in \mathbb{N}\) and a finite family \(\family{x_n}{n = 1}{N} \subset K\) such that \(\family{x_n + V_{x_n}}{n = 1}{N}\) is an open cover of \(K\). Let \(V = \bigcap_{n = 1}^{N} V_n\), then
    \begin{equation*}
        K + V \subset V + \bigcup_{n = 1}^N (x_n + V_{x_n}) = \bigcup_{n = 1}^N (x_n + V_{x_n} + V \subset \bigcup_{n = 1}^N (x_n + V_{x_n} + V_{x_n}),
    \end{equation*}
    hence
    \begin{equation*}
        (K + V)\cap(C + V) \subset (C + V) \cap \bigcup_{n = 1}^N (x_n + V_{x_n} + V_{x_n}) \subset \bigcup_{n = 1}^{N} C \cap (x_n + V_{x_n} + V_{x_n} + V_{x_n}) = \emptyset.
    \end{equation*} 
    Note \(C + V\) is open as we have \(C + V = \bigcup_{x \in C} (x + V)\), then we conclude \((C + V) \cap \cl_{X}(K + V) = \emptyset\), otherwise \(C + V\) would be a neighborhood for a point of closure of \(K + V\) that does not intersect \(K + V\). Furthermore, \(C \subset C + V\) does not intersect \(\cl_{X}(K+V)\).
\end{proof}

\begin{proposition}{Necessary and sufficient conditions for a vector topology}{vector_topology}
    Let \(X\) be a linear space with a topology \(\tau\) such that the linear space operations are continuous. The following statements are equivalent:
    \begin{enumerate}[label=(\alph*)]
        \item \((X, \tau)\) is Hausdorff;
        \item \((X, \tau)\) is a topological linear space; and
        \item for every \(x \in X\setminus\set{0}\), there exists an neighborhood of zero \(V \in \tau\) such that \(x \notin \cl_XV.\)
    \end{enumerate}
\end{proposition}
\begin{proof}
    Suppose \(\tau\) is Hausdorff. Let \(v \in X,\) and consider the set \(X \setminus \set{v}\). Let \(u \in X\setminus \set{v}\), then there exists \(S, T \in \tau\) such that \(u \in S\), \(v \in T\), and \(S \cap T = \emptyset\), hence \(u \in S \subset X \setminus \set{v}\), and we conclude \(X \setminus \set{v}\) is open, as desired.

    Suppose \(\tau\) is a vector topology and let \(x \in X\setminus \set{0}\). As \(\set{x}\) is closed and \(\set{0}\) is compact, then by \cref{lem:tvs_hausdorff} there exists a neighborhood \(V \in \tau\) of zero such that \(\set{x} \cap (\set{0} + \cl_XV) = \emptyset,\) hence \(x \notin \cl_XV.\)

    Suppose for every nonzero vector there exists an open neighborhood of zero that does not contain it and let \(u, v \in X\) with \(u \neq v\). Then there exists \(U \in \tau\) such that \(v - u \notin \cl_XU\) and \(0 \in U\), hence \(v \notin u + \cl_XU \subset \cl_X(u + U)\) by \cref{prop:tvs,thm:closure_continuity}. In particular, there exists a neighborhood \(V \in\ \tau\) of \(v\) that does not intersect \(u + U\), which is a neighborhood of \(U\), therefore \(\tau\) is Hausdorff.
\end{proof}

The \cref{lem:tvs_hausdorff} also lets us infer a result about local basis.
\begin{proposition}{Every element of a local basis contains the closure of some basic element}{tvs_local_basis}
    Let \((X, \tau)\) be a topological linear space and let \(\mathcal{B} \subset \tau\) be a local basis. If \(U \in \mathcal{B}\), then there exists \(V \in \mathcal{B}\) such that \(\cl_X V \subset U\).
\end{proposition}
\begin{proof}
    Let \(U \in \mathcal{B}\), then by \cref{lem:tvs_hausdorff} there exists \(W \in \tau\) such that \(0 \in W\) and \(\cl_XW \cap (X\setminus U) = \emptyset\). That is, \(\cl_X W \subset U\). As \(W\) is a neighborhood of \(0,\) there exists \(V \in \mathcal{B}\) such that \(V \subset W,\) therefore \(\cl_X V \subset \cl_X W \subset U\).
\end{proof}

\subsection{Topology generated by a family of semi-norms on a linear space}
We begin by studying semi-norms defined on a linear space.
\begin{definition}{Seminorm on a linear space}{seminorm}
    Let \(V\) be a linear space over \(\mathbb{K}\). A map \(p : V \to \mathbb{R}\) satisfying
    \begin{enumerate}[label=(\alph*)]
        \item subadditivity: \(p(u + v) \leq p(u) + p(v)\);
        \item absolute homogeneity: \(p(\alpha v) = \abs{\alpha} p(v)\);
    \end{enumerate}
    for all \(u,v \in V\) and \(\alpha \in \mathbb{K}\), is a \emph{semi-norm on \(V\)}.
\end{definition}

\begin{proposition}{Semi-norms are positive}{seminorm_positive}
    Let \(V\) be a linear space. A semi-norm \(p : V \to \mathbb{R}\) satisfies 
    \begin{enumerate}[label=(\alph*)]
        \item \(p(0) = 0\);
        \item \(p(u - v) \geq \abs{p(u) - p(v)}\) for all \(u, v \in V\);
        \item \(v \in V \implies p(v) \geq 0\); and
        \item \(N = \setc{v \in V}{p(v) = 0}\) is a linear subspace of \(V\).
    \end{enumerate}
\end{proposition}
\begin{proof}
    It is clear (a) follows from absolute homogeneity and that (c) follows from (b). Let \(u, v \in V\), then by subadditivity we have
    \begin{equation*}
        p(u) \leq p(u - v) + p(v) \implies p(u - v) \geq p(u) - p(v)
    \end{equation*}
    and thus
    \begin{equation*}
        p(u - v) = \abs{-1} p(v - u) = p(v - u) \geq p(v) - p(u),
    \end{equation*}
    yielding (b). From (a), we know \(0 \in N\), then for all \(u, v \in N\) and \(\lambda \in \mathbb{K}\) we have
    \begin{equation*}
        0 \leq p(u + \alpha v) \leq p(u) + \abs{\alpha} p(v) = 0,
    \end{equation*}
    hence \(N\) is a linear subspace.
\end{proof}

\begin{proposition}{Open ball with respect to a semi-norm}{ball_seminorm}
    Let \(V\) be a linear space. An \emph{open ball \(B_p(x, r)\) with respect to the semi-norm \(p : V \to \mathbb{R}\) of radius \(r > 0\) centered at \(x \in V\)} is the set
    \begin{equation*}
        B_p(x, r) = \setc{v \in V}{p(v - x) < r}.
    \end{equation*}
    Such a set enjoys the following properties:
    \begin{enumerate}[label=(\alph*)]
        \item \(x \in B_p(x, r)\);
        \item \(B_p(0, r)\) is convex: \(u, v \in B_p(0,r), \alpha \in (0,1) \implies \alpha u + (1 - \alpha)v \in B_p(0,r)\);
        \item \(B_p(0, r)\) is balanced: \(v \in B_p(0,r), \alpha \in \mathbb{K} : \abs{\alpha} \leq 1\implies \alpha v \in B_p(0,r)\);
        \item \(B_p(0, r)\) is absorbing: \(\forall v \in V : \exists \alpha > 0 : \alpha^{-1}v\in B_p(0,r)\); and
        \item \(p(v) = \inf\setc{\alpha r}{\alpha > 0 : \alpha^{-1} v \in B_p(0, r)}\) for all \(v \in V\).
    \end{enumerate}
\end{proposition}
\begin{proof}
    Property (a) follows from \cref{prop:seminorm_positive}. Let \(u, v \in B_p(0,r)\) and \(\alpha \in (0,1)\), then 
    \begin{equation*}
        p(\alpha u + (1 - \alpha)v) \leq p(\alpha u) + p((1 - \alpha)v) = \alpha p(u) + (1 - \alpha) p(v) < \alpha r + (1 - \alpha) r = r,
    \end{equation*}
    hence \(B_p(0,r)\) is convex. It is clear (c) follows from absolute homogeneity. Let \(x \in V,\) then if \(p(x) < r\), we have \(1^{-1}x \in B_p(0,r)\), and if \(p(x) \geq r\), we may set \(\alpha = \frac{2p(x)}{r} > 2\) such that \(p(\alpha^{-1} x) = \frac12r,\) that is, \(B_p(0,r)\) is absorbing. Let \(v \in V\), then for all \(\alpha > 0\), we have
    \begin{equation*}
        \alpha^{-1}v \in B_p(0,r) \iff p(\alpha^{-1}v) < r \iff p(v) < \alpha r,
    \end{equation*}
    hence \(p(v)\) is the greatest lower bound of \(\setc{\alpha r}{\alpha > 0 : \alpha^{-1} v \in B_p(0,r)}\), and (e) follows.
\end{proof}

\begin{proposition}{Topology generated by a family of semi-norms}{topology_seminorms}
    Let \(V\) be a linear space and let \(\mathcal{P}\) be a family of semi-norms on \(V\). The set
    \begin{equation*}
        \mathcal{S} = \bigcup_{x \in V}{\bigcup_{p \in \mathcal{P}}{\bigcup_{r > 0}{\setc{v \in V}{p(v - x) < r}}}}
    \end{equation*}
    is a subbasis for a topology on \(V\) relative to which the vector operations are continuous and every semi-norm \(p \in \mathcal{P}\) is continuous.
\end{proposition}
\begin{proof}
    Notice we have \(\bigcup_{x \in V}\bigcup_{p \in \mathcal{P}} B_p(x, 1) = V,\) then \(\mathcal{S}\) is a subbasis for a topology, \(\tau\) say. Notice
    \begin{equation*}
        \mathcal{S}_\mathcal{P} = \bigcup_{p \in \mathcal{P}}\bigcup_{r > 0} \setc{v \in V}{p(v) < r} = \bigcup_{p \in \mathcal{P}}\bigcup_{r > 0} B_p(0, r) = \bigcup_{p \in \mathcal{P}}\bigcup_{r > 0} \preim{p}{(-\infty,r)} \subset \mathcal{S}
    \end{equation*}
    is also a subbasis for a topology, namely the initial topology \(\tau_{\mathcal{P}}\) relative to \(\mathcal{P}\). By \cref{prop:subbasis_topology} we have \(\tau_{\mathcal{P}} \subset \tau\), hence every \(p \in \mathcal{P}\) is continuous with respect to \(\tau\).

    Let \(S \in \mathcal{S},\) then there exists \(x \in V,\) \(p \in \mathcal{P}\), and \(\varepsilon > 0\) such that \(S = B_p(x, \varepsilon)\). We consider the preimage of \(S\) under scalar multiplication, \(U = \setc{(\alpha, v) \in \mathbb{K} \times V}{p(\alpha v - x) < \varepsilon}\). Let \((\alpha, v) \in U\) and set \(r = p(\alpha v - x) < \varepsilon\), then notice for all \((\tilde{\alpha}, \tilde{v}) \in \mathbb{K} \times V\) we have
    \begin{equation*}
        p(\tilde{\alpha}\tilde{v} - x) \leq p(\tilde{\alpha}\tilde{v} - \tilde{\alpha} v) + p(\tilde{\alpha} v -  \alpha v) + p(\alpha v - x) = \abs{\tilde{\alpha}} p(\tilde{v} - v) + \abs{\tilde{\alpha} - \alpha} p(v) + r.
    \end{equation*}
    If \(p(v) = 0,\) we set \(A = \setc{\tilde{\alpha} \in \mathbb{K}}{\abs{\tilde{\alpha}} < 1}\) and \(B = B_p(v, \varepsilon - r)\), then \(A \times B \subset U\), hence \(U\) is open. If \(p(v) \neq 0\), we set \(A = \setc{\tilde{\alpha} \in \mathbb{K}}{\abs{\tilde{\alpha} - \alpha} < \frac{\varepsilon - r}{2p(v)}}\) and \(B = B_p(v, \frac{\varepsilon - r}{2\abs{a} + 2\frac{\varepsilon - r}{p(v)}}),\) then for all \(\tilde{\alpha} \in A\) and \(\tilde{v} \in B\) we have
    \begin{equation*}
        \abs{\tilde{\alpha}} p(\tilde{v} - v) \leq \left(\abs{\tilde{\alpha} - \alpha} + \abs{\alpha}\right) p(\tilde{v} - v) < \frac{\varepsilon - r}{2}
        \quad\text{and}\quad
        \abs{\tilde{\alpha} - \alpha} p(v) < \frac{\varepsilon - r}{2},
    \end{equation*}
    hence \(p(\tilde{\alpha} \tilde{v} - x) < \varepsilon\) and we conclude \(A \times B \subset U,\) hence \(U\) is open. Let us consider now the preimage of \(S\) under addition, \(\tilde{U} = \setc{(u,v) \in V\times V}{p(u + v - x) < \varepsilon}\). Let \((u,v) \in \tilde{U}\) and set \(\tilde{r} = p(u + v - x) < \varepsilon,\) then for all \(\tilde{u} \in C = B_p(u, \frac{\varepsilon - \tilde{r}}{2}\) and \(\tilde{v} \in D = B_p(v, \frac{\varepsilon - \tilde{r}}{2})\) we have
    \begin{equation*}
        p(\tilde{u} + \tilde{v} - x) \leq p(\tilde{u} - u) + p(\tilde{v} - v) + p(u + v - x) < \varepsilon,
    \end{equation*}
    hence \(C \times D \in \tilde{U}\), and we conclude \(\tilde{U}\) is open. By \cref{prop:continuity_subbasis}, it follows that addition and scalar multiplication is continuous.
\end{proof}

% vim: spl=en_us
\chapter{Normed linear spaces}
A particular case of a metric space is a \emph{normed linear space}. In fact, we will show that there is a certain equivalence of a metric on a linear space with normed linear spaces. We begin by defining a normed linear space.
\begin{definition}{Normed linear space}{norm}
    Let \((V, +, \cdot)\) be a linear space over a field \(\mathbb{K}\) (either \(\mathbb{R}\) or \(\mathbb{C}\)). A \emph{norm} is a map \(\norm{\noarg} : V \to \mathbb{R}\) satisfying
    \begin{enumerate}[label=(\alph*)]
        \item absolute homogeneity: \(\norm{\lambda f} = \abs{\lambda}\cdot \norm{f}\) for all \(\lambda \in \mathbb{K}\) and for all \(f \in V\);
        \item subadditivity: \(\norm{f + g} \leq \norm{f} + \norm{g}\), for all \(f, g \in V\);
        \item nonnegativity: \(\norm{f} \geq 0\) for all \(f \in V\); and
        \item positive-definiteness: \(\norm{f} = 0 \iff f = 0\).
    \end{enumerate}
    A \emph{normed linear space} \((V,+, \cdot, \norm{\noarg})\) is a linear space equipped with a distinguished norm \(\norm{\noarg}\).
\end{definition}
\begin{remark}
    From now on we will suppress the linear space operations in notation, denoting \((V, \norm{\noarg})\) as a normed linear space.
\end{remark}

We now give a possible construction that turns a normed linear space into a metric space. This construction is the usual way to define such a metric space, albeit non-unique.
\begin{definition}{Metric induced by a norm}{metric_norm}
    Let \((V, \norm{\noarg})\) be a normed linear space. The map
    \begin{align*}
        d : V \times V &\to [0, \infty)\\
        (x,y) &\mapsto \norm{x - y}
    \end{align*}
    is called a metric induced by the norm \(\norm{\noarg}\).
\end{definition}
Let us verify this map is indeed a metric. From positive-definiteness and nonnegativity of the norm, it follows that \(d(x, y) \geq 0\) for all \(x,y \in V\) and \(d(x,y) = 0\) if and only if \(x = y\). We show the triangle inequality by using the subadditivity property:
\begin{equation*}
    d(x,y) = \norm{x - y} = \norm{x - z + z - y} \leq \norm{x - z} + \norm{z - y} = d(x,z) + d(z, y),
\end{equation*}
for all \(x,y,z \in V\). Thus, every normed linear space is a metric space, and we may express the convergence of sequences, the Cauchy property, and completeness in terms of a norm, by way of the construction in \cref{def:metric_norm}. Precisely, we say the normed linear space \((V, \norm{\noarg})\) is complete if the metric space \((V,d)\) is complete, where \(d\) is the metric induced by the norm \(\norm{\noarg}\).

The \cref{prop:metric_norm} shows a property of such a metric, but also determines conditions under which a metric space \((V, d)\), where \(V\) is a linear space, is a normed linear space.
\begin{proposition}{Metric induced by a norm}{metric_norm}
    Let \(V\) be a linear space over a field \(\mathbb{K}\). A metric \(d : V \times V \to [0, \infty)\) is induced by a norm in \(V\) if and only if \(d\) is
    \begin{enumerate}[label=(\alph*)]
        \item translation invariant, that is, \(d(u + t, v + t) = d(u, v)\) for all \(u,v,t \in V\); and
        \item homogeneous, that is, \(d(\alpha u , \alpha v) = \abs{\alpha}d(u, v)\) for all \(u,v \in V\) and \(\alpha \in \mathbb{K}\).
    \end{enumerate}
\end{proposition}
\begin{proof}
    Suppose \(d\) is a metric induced by the norm \(\norm{\noarg}\). For all \(u,v,t\in V\) and \(\alpha \in \mathbb{K}\), we have
    \begin{equation*}
        d(u + t, v + t) = \norm{(u+t) - (v + t)} = \norm{u - v} = d(u,v)
    \end{equation*}
    and
    \begin{equation*}
        d(\alpha u, \alpha v) = \norm{\alpha (u - v)} = \abs{\alpha} \norm{u-v} = \abs{\alpha} d(u,v).
    \end{equation*}
    Thus, if \(d\) is induced by a norm, then \(d\) satisfies (a) e (b).

    Now suppose \(d\) has properties (a) e (b). We now show the map
    \begin{align*}
        \norm{\noarg} : V &\to [0, \infty)\\
                                 v &\mapsto d(v, 0)
    \end{align*}
    is a norm in \(V\). Note that
    \begin{align*}
        v = 0 &\iff d(v, 0) = 0\\
              &\iff \norm{v} = 0,
    \end{align*}
    and
    \begin{equation*}
        \norm{\lambda u} = d(\lambda u, 0) = \abs{\lambda} d(u,0) = \abs{\lambda}\norm{u}
    \end{equation*}
    for all \(\lambda \in \mathbb{K}\) and \(u \in V\). From the homogeneity property it follows that
    \begin{equation*}
        \norm{x + y} = d(x + y, 0) = d(x, -y),
    \end{equation*}
    therefore by translation invariance e by the triangle inequality for \(d\), we have
    \begin{equation*}
        \norm{x+y} \leq d(x, 0) + d(0, y)
    \end{equation*}
    or, equivalently, \(\norm{x+y} \leq \norm{x} + \norm{y}\) for all \(x,y\in V\). Thus, \(\norm{\noarg}\) is a norm in \(V\). Furthermore,
    \begin{equation*}
        d(u, v) = d(u-v, 0) = \norm{u - v},
    \end{equation*}
    hence \(d\) is a metric induced by the norm \(\norm{\noarg}\). That is, if \(d\) has the properties (a) and (b), then \(d\) is a metric induced by a norm.
\end{proof}

We have thus finished defining every notion needed to define a Banach space.
\begin{definition}{Banach space}{banach_space}
    A \emph{Banach space} is a normed linear space complete with respect to its norm.
\end{definition}
Following \cref{exam:continuous_complex_ab,exam:sup_norm_complete}, we use the construction used in the proof of \cref{prop:metric_norm} to show the linear space of continuous complex-valued functions is a Banach space with respect to the sup norm.
\begin{example}{Supremum norm of continuous complex-valued functions}{sup_norm}
    The \emph{supremum norm}, or \emph{sup norm}, is the map defined by
    \begin{align*}
        \norm{\noarg}_{\infty} : \mathcal{C}([a,b];\mathbb{C}) &\to \mathbb{R}\\
                                                             f &\mapsto \norm{f}_\infty = \sup_{x \in [a,b]}\abs*{f(x)}.
    \end{align*}
    Then, \(\left(\mathcal{C}([a,b];\mathbb{C}), \norm{\noarg}_\infty\right)\) is a Banach space.
\end{example}
\begin{proof}
    We must show the sup metric is translation invariant and homogeneous. First, let \(f, g, h \in \mathcal{C}([a,b];\mathbb{C})\), then
    \begin{align*}
        d_\infty(f + h, g + h) &= \sup_{x \in [a,b]}\abs*{f(x) + h(x) - g(x) - h(x)} \\&= \sup_{x \in [a,b]} \abs*{f(x) - g(x)} \\&= d_\infty(f, g),
    \end{align*}
    that is, \(d_\infty\) is translation invariant. Let \(z \in \mathbb{C}\), then
    \begin{equation*}
        d_\infty(zf, zg) = \sup_{x\in[a,b]}\abs*{zf(x) - zg(x)} = \abs{z} \sup_{x\in[a,b]}\abs*{f(x) - g(x)} =  \abs{z} d_\infty(f,g),
    \end{equation*}
    then \(d_\infty\) is homogeneous. By \cref{prop:metric_norm}, we have shown the sup metric is induced by a norm given by \(v \mapsto d_\infty(v, 0)\), which is precisely the sup norm. Hence, \((\mathcal{C}([a,b];\mathbb{C})\) is a normed linear space. By the result from \cref{exam:sup_norm_complete}, this linear space is complete with respect to the metric induced by this norm, and we conclude it is a Banach space.
\end{proof}

Finally, we show that the usual operations of a complex normed linear space are continuous by use of \cref{thm:convergence_continuity}.
\begin{proposition}{Addition, scalar multiplication, and the norm are continuous}{norm_continuous}
    Let \((V, \norm{\noarg})\) be a normed linear space over the field \(\mathbb{C}\). For any sequences \(\family{x_n}{n\in \mathbb{N}} \subset V\), \(\family{y_n}{n \in \mathbb{N}} \subset V\) and \(\family{z_n}{n \in \mathbb{N}} \subset \mathbb{C}\) that converge to \(x, y \in V\) with respect to \((V, \norm{\noarg})\) and to \(z \in \mathbb{C}\) with respect to \((\mathbb{C}, \abs{\noarg})\), we have
    \begin{equation*}
        \lim_{n\to \infty} (x_n + y_n) = x + y, \lim_{n \to \infty} z_n x_n = zx,\quad\text{and} \lim_{n \to \infty} \norm{x_n} = \norm{x},
    \end{equation*}
    that is, addition, scalar multiplication and the norm are continuous maps.
\end{proposition}
\begin{proof}
    By the triangle inequality, we have
    \begin{equation*}
        \norm{(x_n + y_n) - (x - y)} \leq \norm{x_n - x} + \norm{y_n - y}.
    \end{equation*}
    By convergence of these two sequences, it follows that \(x_n + y_n \to x + y\). Hence, addition is continuous with respect to \((V \times V, d_{V\times V})\) and \((V, \norm{\noarg})\), where \(d_{\mathbb{V}\times V}\) is the product metric.

    Since \family{z_n}{n\in \mathbb{N}} is a convergent sequence, we have \(\abs{z_n} \leq M\) for all \(n \in \mathbb{N}\). Recalling \cref{prop:Cauchy_bounded}, we let \(M = \sup\setc{\abs{z_n - z_m}}{n,m \in \mathbb{N}} + \abs{z_0}\), then by the triangle inequality, for all \(n \in \mathbb{N}\)
    \begin{equation*}
        \abs{z_n} \leq \abs{z_n - z_0} + \abs{z_0} \leq \sup_{i,j \in \mathbb{N}}{\abs{z_i - z_j}} + \abs{z_0} = M.
    \end{equation*}
    From the triangle inequality, we have
    \begin{align*}
        \norm{z_n x_n - zx} &\leq \norm{z_n x_n - z_n x} + \norm{z_n x - zx}\\
                            &\leq \abs{z_n}\cdot \norm{x_n - x} + \abs{z_n - z}\cdot \norm{x}\\
                            &\leq M \norm{x_n - x} + \abs{z_n - z}\cdot\norm{x}
    \end{align*}
    for all \(n \in \mathbb{N}\). From the convergence \(x_n \to x\) and \(z_n \to z\), it follows that \(z_n x_n \to zx\). Hence, scalar multiplication is continuous with respect to \((\mathbb{C} \times V, d_{\mathbb{C}\times V})\) and \((V, \norm{\noarg})\), where \(d_{\mathbb{C}\times V}\) is the product metric constructed from \((\mathbb{C}, \abs{\noarg})\) and \((V, \norm{\noarg})\).

    By the triangle inequality, it follows that
    \begin{equation*}
        \norm{x_n} \leq \norm{x_n - x} + \norm{x}\quad\text{and}\quad \norm{x} \leq \norm{x - x_n} + \norm{x_n},
    \end{equation*}
    for all \(n \in \mathbb{N}\). That is,
    \begin{equation*}
        \abs*{\norm*{x_n} - \norm*{x}} \leq \norm{x_n - x}
    \end{equation*}
    for all \(n \in \mathbb{N}\). By the convergence \(x_n \to x\), it follows that \(\norm{x_n} \to \norm{x}\), hence the norm is continuous with respect to \((V, \norm{\noarg})\) and \((\mathbb{R}, \abs{\noarg})\).
\end{proof}
\begin{remark}
    This characterizes normed linear spaces as \emph{topological linear spaces}: a linear space endowed with a topology such that the singleton sets are closed and that the linear space operations are continuous.
\end{remark}

% vim: spl=en_us
\section{Linear operators in normed linear spaces}
Henceforth we consider linear spaces over the complex numbers unless otherwise stated. As linear maps, or linear operators, are the structure preserving maps of linear spaces, we will study how these maps are related to the additional structure given by the norms.
\begin{definition}{Linear operators}{linear_operators}
    Let \(V, W\) be linear spaces. A \emph{linear operator} \(T : \domain{T} \subset V \to W\) is a linear map,
    \begin{equation*}
        \forall u, v \in \domain{T}, \forall \alpha, \beta \in \mathbb{C}: %\quad
        T(\alpha u + \beta v) = \alpha T(u) + \beta T(v)
    \end{equation*}
    for all \(u,v \in \domain{T}\) and all \(\alpha, \beta \in \mathbb{C}\), where the domain of definition \(\domain{T}\) is a linear subspace of \(V\). The image, or range, of \(T\) is denoted by \(\range{T} = T(\domain{T}) \subset W\). The set
    \begin{equation*}
        \ker{T} = \setc{v \in \domain{T}}{T(v) = 0}
    \end{equation*}
    is the \emph{kernel of \(T\)}.
\end{definition}
\begin{remark}
    Recall that the range and the kernel of a linear operator are linear subspaces. Indeed, the linearity of \(T\) yields \(0 \in \range{T}\) and \(0 \in \ker{T}\), since \(\domain{T}\) is a linear space. Let \(w_1, w_2 \in \range{T}\), then there exist \(v_1, v_2 \in \domain{T}\) such that \(Tv_1 = w_1\) and \(Tv_2 = w_2\), hence \(\alpha w_1 + \beta w_2 = T(\alpha v_1 + \beta v_2) \in \range{T}\) for all \(\alpha, \beta \in \mathbb{C}\). Let \(u_1, u_2 \in \ker{T}\), then \(Tu_1 = Tu_2 = 0\), hence \(T(\alpha u_1 + \beta u_2) = 0\) for all \(\alpha, \beta \in \mathbb{C}\).
\end{remark}
\begin{remark}
    It is common to drop the parenthesis and write \(Tx = T(x)\). We will also use \enquote{linear operator} and \enquote{linear map} interchangeably.
\end{remark}

\begin{proposition}{Injective linear operator}{injective_linear}
    A linear operator is injective if and only if its kernel is the trivial subspace.
\end{proposition}
\begin{proof}
    Let \(T : \domain{T} \to W\) be a linear operator. Suppose \(T\) is injective, let \(v \in \ker{T}\), then \(T(v) = T(0)\), hence \(v = 0\). Suppose the kernel is the trivial subspace, then \(T(u) = T(v)\) implies \(T(u - v) = 0\), hence \(u = v\).
\end{proof}

\begin{proposition}{Kernel of a continuous linear operator}{kernel closed}
    Let \((V, \norm{\noarg}_V), (W, \norm{\noarg}_W)\) be normed linear spaces. If \(T : V \to W\) is a continuous linear operator, then \(\ker{T}\) is a closed linear subspace of \(V\).
\end{proposition}
\begin{proof}
    Recall \cref{prop:continuity_closed}: a map is continuous if and only if the preimage of closed sets are closed sets. Since \(\set{0} \subset W\) is a closed subset of \(W\), we must have \(\preim{T}{\set{0}} = \ker{T}\) closed.
\end{proof}

If a metric is induced by a norm, then linear maps that preserve the norm are also isometries.
\begin{proposition}{Linear isometries}{linear_isometry}
    Let \((V, \norm{\noarg}_V)\) and \((W, \norm{\noarg}_W)\) be normed linear spaces. A linear map \(T : V \to W\) is an isometry with respect to the metrics induced by the norms if and only if \(\norm{Tx}_W = \norm{x}_V\) for all \(x \in V\).
\end{proposition}
\begin{proof}
    Let \(d_V\) and \(d_W\) be the metrics induced by the norms on \(V\) and \(W\), respectively, then
    \begin{align*}
        \forall u, v \in V: d_V(u,v) = d_W(Tu, Tv) &\iff \forall u,v \in V: \norm{u-v}_V = \norm{Tu - Tv}_W\\
                                                   &\iff \forall u,v \in V: \norm{u-v}_V = \norm{T(u-v)}_W\\
                                                   &\iff \forall x \in V: \norm{x}_V = \norm{Tx}_W.
    \end{align*}
    Hence, a linear map preserves the norm if and only if it is distance-preserving.
\end{proof}
\begin{remark}
    A linear map \(T\) that preserves the norm is called a \emph{linear isometry} and we say \(T\) is \emph{isometric}.
\end{remark}

We now show a linear isometry maps a Banach space to a Banach space.
\begin{proposition}{Linear Isometry on a Banach space}{isometry_Banach}
    Let \(T : V \to W\) be a linear isometry, where \((V, \norm{\noarg}_V)\) is a Banach space and \((W, \norm{\noarg}_W)\) is a normed linear space. Then, the image \(\range{T}\) is a closed linear subspace in \(W\) and it is complete with respect to the norm \(\norm{\noarg}_W\).
\end{proposition}
\begin{proof}
    Notice the corestricted map \(T : V \to \range{T}\) is a bijection, then the normed linear spaces \((V, \norm{\noarg}_V)\) and \((\range{T}, \restrict{\norm{\noarg}_W}{\range{T}})\) are isometric, in the metric space sense. Let \(S : \range{T} \to V\) be the inverse map to the corestricted linear isometry, that is, \(S \circ T = \id{V}\) and \(T \circ S = \id{\range{T}}\). We have shown in \cref{prop:isometry_continuous} that \(S\) is distance-preserving, then by \cref{prop:linear_isometry} it is a linear isometry.

    Let \(\family{w_n}{n\in \mathbb{N}} \subset \range{T}\) be a sequence that converges to \(\tilde{w} \in W\). For all \(n, m \in \mathbb{N}\), we have
    \begin{equation*}
        \norm{w_n - w_m}_W = \norm{S(w_n - w_m)}_V = \norm{S(w_n)-S(w_m)}_V,
    \end{equation*}
    then \(\family{S(w_n)}{n\in \mathbb{N}} \subset V\) is a Cauchy sequence in \(V\). Since \(V\) is complete, there exists \(\tilde{v} \in V\) such that \(S(w_n) \to \tilde{v}\). This vector satisfies
    \begin{equation*}
        \norm{\tilde{w} - T(\tilde{v})}_W \leq \norm{\tilde{w} - w_n}_W + \norm{w_n - T(\tilde{v})}_W = \norm{\tilde{w} - w_n}_W + \norm{S(w_n) - \tilde{v}}_V
    \end{equation*}
    for all \(n \in \mathbb{N}\). By convergence of the sequences, we must have \(\tilde{w} = T(\tilde{v})\), hence \(\tilde{w} \in \range{T}\) and \(\range{T}\) is closed.

    Let \(\family{u_n}{n\in \mathbb{N}} \subset \range{T}\) be a Cauchy sequence. By the previous argument, we know \(\family{S(u_n)}{n\in \mathbb{N}} \subset V\) is a Cauchy sequence, hence it converges to some \(\tilde{s} \in V\). The subadditivity of the norm yields
    \begin{equation*}
        \norm{u_n - T(\tilde{s})}_W \leq \norm{u_n - u_m}_W + \norm{u_m - T(\tilde{s})}_W = \norm{u_n - u_m}_W + \norm{S(u_m) - \tilde{s}}_V
    \end{equation*}
    for all \(n, m \in \mathbb{N}\). Let \(\varepsilon > 0\), then by the Cauchy property and by convergence there exists \(N, M \in \mathbb{N}\) such that \(\norm{S(u_m) - \tilde{s}}_V < \frac12 \varepsilon\) for \(m \geq M\) and such that \(\norm{u_n - u_m}_W < \frac12 \varepsilon\) for \(n,m \geq N\). That is,
    \begin{equation*}
        n \geq \max\set{N,M} \implies \norm{u_n - T(\tilde{s})}_W < \varepsilon,
    \end{equation*}
    hence \(u_n \to T(\tilde{s})\). We conclude \((\range{T}, \restrict{\norm{\noarg}_W}{\range{T}})\) is a Banach space.
\end{proof}

% vim: spl=en_us
\section{Bounded linear operators}
As it usually is the case in Mathematics, one studies a given structure by studying maps between instances of that structure. We note that, when dealing with different linear spaces, norms will be simply denoted by \(\norm{\noarg}\), if the argument of the norm does not leave any ambiguity towards which norm is being used.

\begin{definition}{Bounded map}{bounded}
    Let \(V\) and \(W\) be normed linear spaces. A linear map \(A : V \to W\) is \emph{bounded} if
    \begin{equation*}
        \sup_{f\in V\setminus\set{0}} \frac{\norm{A f}}{\norm{f}} < \infty.
    \end{equation*}
    If an operator is not bounded, we say it is \emph{unbounded}.
\end{definition}
\begin{remark}
    An equivalent formulation is the condition
    \begin{equation*}
        \sup{\setc*{\norm{Af}}{f \in V : \norm{f} = 1}} < \infty
    \end{equation*}
    which follows from the homogeneity property of the norm and the linearity of the operator. Indeed, for all \(g \in V\setminus\set{0}\),
    \begin{equation*}
        \frac{\norm{Ag}}{\norm{g}} = \norm*{\frac{1}{\norm{g}}Ag} = \norm*{A\frac{g}{\norm{g}}} = \norm{A\tilde{g}},
    \end{equation*}
    where \(\tilde{g} \in \setc{f \in V}{\norm{f} = 1}\).
\end{remark}
\begin{remark}
    It is cumbersome to always write \(f \in V \setminus\set{0}\), so we will simply write
    \begin{equation*}
        \sup_{f \in V} \frac{\norm{Af}}{\norm{f}}
    \end{equation*}
    as shorthand.
\end{remark}

\begin{example}{The derivative operator of complex-valued functions is unbounded}{derivative_unbounded}
    Let \(W = \mathcal{C}([0,1]; \mathbb{C})\) be the Banach space of continuous complex-valued functions equipped with the sup norm and let the set \(V = \mathcal{C}^1([0,1];\mathbb{C})\subset W\) of continuously differentiable complex-valued functions in the interval \([0,1]\). The derivative operator
    \begin{align*}
        P : V &\to W\\
            f &\mapsto f'
    \end{align*}
    is unbounded.
\end{example}
\begin{proof}
    We consider the family of continuously differentiable functions \(\family{f_n}{n\in \mathbb{N}}\subset V\) defined by
    \begin{align*}
        f_n : [0,1] &\to \mathbb{C}\\
                  t &\mapsto \sin(2\pi n t),
    \end{align*}
    then it is clear that for all \(n \in \mathbb{N}\), \(\norm{f_n}_\infty = 1,\) that is \(\family{f_n}{n\in \mathbb{N}} \subset \setc{f \in V}{\norm{f} = 1}\). Also, we have
    \begin{equation*}
        \norm{P f_n}_\infty = \sup_{t \in [0,1]} \abs{2\pi n \cos(2\pi nt)} = 2\pi n,
    \end{equation*}
    for all \(n \in \mathbb{N}\). Finally, we show the operator \(P\) is unbounded with this sequence
    \begin{equation*}
        \sup_{f \in V} \frac{\norm{P f}_\infty}{\norm{f}_\infty} \geq \sup_{n \in \mathbb{N}\setminus\set{0}} \norm{P f_n} = \sup_{n \in \mathbb{N}} 2\pi n = \infty,
    \end{equation*}
    as claimed.
\end{proof}
\begin{remark}
    Details aside, this example bears some resemblance to the fact that the momentum operator in Quantum Mechanics is unbounded.
\end{remark}

\begin{theorem}{Bounded operators are continuous}{bounded_continuous}
    Let \(A : V \to W\) be a linear operator where \(V\) and \(W\) are normed linear spaces. The following are equivalent
    \begin{enumerate}[label=(\alph*)]
        \item \(A\) is bounded;
        \item \(A\) is uniformly continuous;
        \item \(A\) is continuous; and
        \item \(A\) is continuous at \(0\).
    \end{enumerate}
\end{theorem}
\begin{proof}
    \(\text{(a)}\implies\text{(b)}\): Suppose \(A\) is bounded and let \(\varepsilon > 0\). Then the set \(\setc*{\frac{\norm{Af}}{\norm{f}}}{f \in V \setminus\set{0}}\) has a finite upper bound, \(M > 0\) say. Then, for all \(x,y \in V\), we have \(\norm{A(x - y)} \leq M \norm{x-y}\). We set \(\delta = \frac{\varepsilon}{M + 1}\), then for all \(x, y \in V\) it follows that
    \begin{equation*}
        \norm{x - y} < \delta \implies \norm{Ax - Ay} \leq \frac{M}{M+1} \varepsilon < \varepsilon,
    \end{equation*}
    that is, \(A\) is uniformly continuous.

    \(\text{(b)}\implies\text{(c)}\): Suppose \(A\) is uniformly continuous and let \(\varepsilon > 0\). There exists \(\delta > 0\) such that for all \(x,y \in V\), we have \(\norm{x - y} < \delta \implies \norm{Ax - Ay} < \varepsilon\). In particular, for all \(y \in V\), we have \(\norm{x - y} < \delta \implies \norm{Ax - Ay} < \varepsilon\), that is, \(A\) is continuous.

    \(\text{(c)}\implies\text{(d)}\): Suppose \(A\) is continuous. Quite trivially, \(A\) is continuous at \(0\).

    \(\text{(d)}\implies\text{(a)}\): Suppose \(A\) is continuous at \(0\). We fix some \(\varepsilon > 0\), then there exists \(\delta > 0\) such that \(\norm{x} < \delta \implies \norm{Ax} < \varepsilon\), where \(\delta\) depends solely on the choice of \(\varepsilon\). It follows from continuity and linearity that
    \begin{align*}
        x \in V \setminus\set{0}&\implies\norm*{\frac{\delta}{2\norm{x}}x} < \delta\\
                                     &\implies\norm*{A\left(\frac{\delta}{2\norm{x}}x\right)} < \varepsilon\\
                                     &\implies \frac{\norm{Ax}}{\norm{x}} < \frac{2 \epsilon}{\delta}.
    \end{align*}
    We have thus shown that \(\frac{2 \varepsilon}{\delta}\) is a finite upper bound for the set \(\setc*{\frac{\norm{Af}}{\norm{f}}}{f \in V \setminus\set{0}}\), hence \(A\) is bounded.
\end{proof}

Let \(V\) be a normed linear space and let \(W\) be a Banach space, we denote \(\bounded(V,W)\) as the set of all bounded operators from \(V\) to \(W\). Defining operator addition and multiplication by a complex number pointwise,
\begin{equation*}
    (A + B)f = Af + Bf\quad\text{and}\quad (zA)f = zAf,
\end{equation*}
it is clear that \(A+B\) and \(zA\) are linear operators, and we only have to show that these operators are bounded. Indeed,
\begin{equation*}
    \sup_{f \in V} \frac{\norm{(A+B)f}}{\norm{f}} = \sup_{f\in V} \frac{\norm{Af + Bf}}{\norm{f}} \leq \sup_{f \in V} \frac{\norm{Af} + \norm{Bf}}{\norm{f}} \leq \sup_{f\in V} \frac{\norm{Af}}{\norm{f}} + \sup_{f\in V} \frac{\norm{Bf}}{\norm{f}} < \infty
\end{equation*}
and
\begin{align*}
    \sup_{f \in V} \frac{\norm{(zA)f}}{\norm{f}} = \sup_{f \in V} \frac{\norm{z(Af)}}{\norm{f}} = \sup_{f \in V} \frac{\abs{z}\norm{Af}}{\norm{f}} = \abs{z} \sup_{f \in V} \frac{\norm{Af}}{\norm{f}} < \infty
\end{align*}
therefore these operations are closed in \(\bounded(V, W)\). It follows from the linear space structure of \(W\) that \(\bounded(V, W)\) is a linear space.
\begin{definition}{Operator norm of a bounded operator}{bounded_operator_norm}
    The \emph{operator norm} or \emph{uniform norm} is the map defined as
    \begin{align*}
        \norm{\noarg} : \bounded(V, W) &\to \mathbb{R}\\
        A &\mapsto \sup{\setc*{\norm{Af}}{f \in V : \norm{f} = 1}}.
    \end{align*}
    The topology induced by this norm is called the \emph{uniform topology}.
\end{definition}
\begin{remark}
    Unsurprisingly, the operator norm of the identity operator \(\id{W}\) is unity. Indeed, let \(f \in \setc{g \in W}{\norm{g} = 1}\), then \(\norm{\id{W}f} = \norm{f} = 1\).
\end{remark}

\begin{proposition}{Isometries are bounded}{isometry_bounded}
    Let \(T : V \to W\) be a linear isometry where \(V\) and \(W\) are normed linear spaces. Then, \(T \in \bounded(V,W)\) and \(\norm{T} = 1.\)
\end{proposition}
\begin{proof}
    If \(T\) is a linear isometry, we have
    \begin{equation*}
        \sup_{v \in V} \frac{\norm{Tv}}{\norm{v}} = \sup_{v \in V} \frac{\norm{v}}{\norm{v}} = 1,
    \end{equation*}
    hence \(T\) is bounded with operator norm equal to unity.
\end{proof}

Let us check that \((\bounded(V, W), \norm{\noarg})\) is a normed linear space. We have already shown homogeneity and subadditivity when checking that \(\bounded(V, W)\) defines a linear space. Nonnegativity follows from the nonnegativity of norms in \(W\) and \(V\). Finally, positive-definiteness follows from the positive-definiteness of the norm in \(W\),
\begin{align*}
    \norm{A} = 0 &\iff \sup_{f \in V} \frac{\norm{Af}}{\norm{f}} = 0\\
                 &\iff \forall f \in V : \norm{Af} = 0\\
                 &\iff \forall f \in V : Af = 0\\
                 &\iff A = 0.
\end{align*}
We conclude \((\bounded(V, W), \norm{\noarg})\) is a normed linear space. \cref{thm:bounded_operators_Banach} shows this linear space is in fact a Banach space. The outline of the proof is very similar to \cref{exam:sup_norm_complete}: considering an arbitrary Cauchy sequence, we first construct a candidate for the limit of this sequence, then we show that it is a bounded linear map and we conclude the proof by showing that the Cauchy sequence indeed converges to this operator. In fact, this can be generalized for any set of bounded maps with image in some complete metric space.

\begin{theorem}{\(\bounded(V, W)\) is a Banach space}{bounded_operators_Banach}
    The normed linear space \((\bounded(V, W), \norm{\noarg})\) is a Banach space if \(W\) is a Banach space.
\end{theorem}
\begin{proof}
    Let \(\family{A_n}{n \in \mathbb{N}} \subset \bounded(V, W)\) be a Cauchy sequence of linear bounded operators. For any given \(f \in V\), we have
    \begin{equation*}
        \norm*{A_nf - A_mf} \leq \norm{A_n - A_m}\cdot\norm{f},
    \end{equation*}
    then \(\family{A_nf}{n\in \mathbb{N}} \subset W\) is a Cauchy sequence in the Banach space \(W\). We have thus constructed a map
    \begin{align*}
        A : V &\to W\\
        f &\mapsto \lim_{n\to \infty} A_n f,
    \end{align*}
    which is a candidate for the convergence of the family of bounded operators. We have to show \(A\) is linear and bounded before showing that \(A_n \to A\).

    Consider \(f, g \in V\) and \(z \in \mathbb{C}\), then it follows from the linearity of each element in the sequence of the operators that
    \begin{equation*}
        A(f + zg) = \lim_{n\to\infty} A_n(f + zg) = \lim_{n\to \infty} \left[A_nf + z A_n g\right].
    \end{equation*}
    Recall from \cref{prop:norm_continuous,prop:continuous_composition} that vector addition and scalar multiplication are continuous and that the composition of continuous maps is continuous, then
    \begin{equation*}
        A(f + zg) = \lim_{n \to \infty} Af + z \lim_{n\to \infty}A_ng = Af + z Ag,
    \end{equation*}
    hence, \(A\) is a linear map.

    Let us show that \(\family{\norm{A_n}}{n\in \mathbb{N}} \subset \mathbb{R}\) is a convergent sequence in the Banach space \((\mathbb{R}, \abs{\noarg})\). From the triangle inequality, we have
    \begin{equation*}
        \abs*{\norm{A_n} - \norm{A_m}} \leq \norm{A_n - A_m}.
    \end{equation*}
    From the Cauchy property, for all \(\varepsilon > 0\) there exists \(N \in \mathbb{N}\) such that
    \begin{equation*}
        n, m \geq N \implies \norm{A_n - A_m} < \frac12\varepsilon,
    \end{equation*}
    then
    \begin{equation*}
        n, m \geq N \implies \abs*{\norm{A_n} - \norm{A_m}} < \varepsilon.
    \end{equation*}
    Since \((\mathbb{R}, \abs{\noarg})\) is complete, it follows that the sequence converges to some real number \(M \geq 0\) say. Let \(f \in V \setminus\set{0}\). Then, by continuity of the norm in \(W\) and by its completeness,
    \begin{equation*}
        \norm{Af} = \norm*{\lim_{n\to \infty} A_n f} = \lim_{n\to \infty} \norm{A_nf} \leq \norm{f} \lim_{n\to \infty} \norm{A_n} = M \norm{f}.
    \end{equation*}
    We have thus a finite upper bound for \(\setc*{\frac{\norm{Af}}{\norm{f}}}{f \in V\setminus\set{0}}\), therefore \(A\) is bounded.

    Finally, we show \(A_n \to A\). For all \(n \in \mathbb{N}\) and \(f \in V\setminus\set{0}\), we have
    \begin{equation*}
        \norm{A_n f - Af} \leq \lim_{m \to \infty} \norm{A_n f - A_m f} \leq \norm{f} \lim_{m \to \infty} \norm{A_n - A_m},
    \end{equation*}
    therefore
    \begin{equation*}
        \norm{A - A_n} = \sup_{f \in V} \frac{\norm{(A - A_n)f}}{\norm{f}} \leq \lim_{m\to \infty} \norm{A_n - A_m}
    \end{equation*}
    Let \(\varepsilon > 0\). Since \family{A_n}{n\in \mathbb{N}} is Cauchy, there exists \(K \in \mathbb{N}\) such that
    \begin{equation*}
        i, j \geq K \implies \norm{A_i - A_j} < \varepsilon,
    \end{equation*}
    that is, if \(n \geq K\), then \(\displaystyle\lim_{m\to \infty}\norm{A_n - A_m} < \varepsilon\). We have shown
    \begin{equation*}
        n \geq K \implies \norm{A - A_n} < \varepsilon,
    \end{equation*}
    hence, \(A_n \to A\) with respect to \((\bounded(V, W), \norm{\noarg})\).
\end{proof}

% vim: spl=en_us
\section{Extensions of densely defined bounded operators}
So far we have considered maps whose domain are the entirety of a normed linear space. The following results concern the \emph{extensions} of linear maps defined in a subspace of a linear space.
\begin{definition}{Extension of a map}{extension}
    Let \(V, W\) be linear spaces. An \emph{extension} of a linear map \(A : \domain{A} \subset V \to W\) defined on the linear subspace \(\domain{A}\) is a linear map \(\hat{A} : V \to W\) such that
    \begin{equation*}
        \hat{A}\alpha = A \alpha,
    \end{equation*}
    for all \(\alpha \in \domain{A}\).
\end{definition}

We say a linear map \(A : \domain{A} \subset V \to W\) is \emph{densely defined} if its domain \(\domain{A}\) is a dense subspace of the linear space \(V\). An important insight about the domain of a bounded linear map is the \emph{bounded linear transformation theorem}.
\begin{theorem}{Bounded linear transformation theorem}{blt}
    Let \(V\) be a normed linear space and \(W\) be a Banach space. For any densely defined bounded linear operator \(A : \domain{A} \to W\), there exists a unique extension \(\hat{A} : V \to W\) that is a bounded linear operator. Furthermore, \(\norm{\hat{A}}_{\bounded(V,W)} = \norm{A}_{\bounded(\domain{A}, W)}\).
\end{theorem}
\begin{proof}
    Let \(f \in V\) and consider a sequence \(\family{f_n}{n\in \mathbb{N}}\subset \domain{A}\) that converges to \(f\), whose existence is guaranteed since \(\domain{A}\) is dense in \(V\). From the uniform continuity of \(A\) we know from \cref{prop:uniformly_continuous_cauchy} that \(\family{Af_n}{n\in \mathbb{N}} \subset W\) is a Cauchy sequence in the Banach space \(W\). From its completeness, we know its limit exists in \(W\), \(w\) say. We consider \(\family{\tilde{f}_n}{n\in \mathbb{N}}\subset \domain{A}\) another sequence that converges to \(f\). By the previous argument we know \(A\tilde{f}_n \to \tilde{w}\) for some \(\tilde{w} \in W\). We aim to show that \(\tilde{w} = w\), that is, we will show the map
    \begin{align*}
        \hat{A} : V &\to W\\
                  f &\mapsto \lim_{n\to \infty} Af_n,
    \end{align*}
    is well-defined. By the triangle inequality we have
    \begin{align*}
        \norm*{\tilde{w} - w} &\leq \norm*{\tilde{w} - A\tilde{f}_n} + \norm*{A\tilde{f}_n - Af_n}  + \norm*{Af_n - w}\\
                              &\leq \norm*{\tilde{w} - A\tilde{f}_n} + \norm{A}\cdot\norm*{\tilde{f}_n - f_n}  + \norm*{Af_n - w}\\
                              &\leq \norm*{\tilde{w} - A\tilde{f}_n} + \norm{A}\left(\norm*{\tilde{f}_n - f} + \norm*{f - f_n}\right)  + \norm*{Af_n - w}.
    \end{align*}
    From convergence of each sequence, it is easy to see that \(\tilde{w} = w\), thus the map is well-defined.

    This map is an extension of \(A\), which follows from the continuity of linear space operations and from linearity of \(A\). Indeed, we begin by showing it is a linear map: let \(f, g \in V\) and let \(z \in \mathbb{C}\), then
    \begin{align*}
        \hat{A}(f + zg) &= \lim_{n\to \infty} A(f_n + zg_n)\\&= \lim_{n\to\infty} Af_n + z \lim_{n\to\infty} Ag_n\\&= Af + z Ag,
    \end{align*}
    which shows linearity. To show it is an extension of \(A\), we let \(\alpha \in \domain{A}\), then one sequence in \(\domain{A}\) that converges to \(\alpha\) is the constant sequence, therefore \(\hat{A}\alpha = A\alpha\).

    Let \(f \in V\setminus\set{0}\), then from the continuity of the norm we have
    \begin{equation*}
        \norm{\hat{A} f} = \norm*{\lim_{n\to\infty} Af_n} = \lim_{n\to \infty} \norm{A f_n} \leq \lim_{n\to \infty} \norm{A} \norm{f_n} = \norm{A} \norm{f}.
    \end{equation*}
    Thus, \(\norm{A}\) is an upper bound for the set \(\setc*{\frac{\norm{\hat{A}f}}{\norm{f}}}{f \in V\setminus\set{0}}\), hence \(\hat{A}\) is bounded and \(\norm{\hat{A}} \leq \norm{A}\). Furthermore, we have
    \begin{equation*}
        \norm{A} = \sup_{f\in\domain{A}}\frac{\norm{Af}}{\norm{f}}= \sup_{f\in\domain{A}}\frac{\norm{\hat{A}f}}{\norm{f}} \leq \sup_{f \in V} \frac{\norm{\hat{A}f}}{\norm{f}} = \norm{\hat{A}},
    \end{equation*}
    therefore it follows that \(\norm{\hat{A}} = \norm{A}\).

    Suppose there exists \(\hat{B} \in \bounded(V, W)\) that is an extension of \(A\). From continuity, we have for all \(f \in V\) that
    \begin{equation*}
        (\hat{A} - \hat{B})f = \lim_{n\to\infty} (\hat{A} - \hat{B})f_n = \lim_{n\to \infty} (Af_n - Af_n) = 0,
    \end{equation*}
    hence \(\hat{A} = \hat{B}\). We have thus shown that \(\hat{A}\) is the unique bounded extension of \(A\).
\end{proof}

% vim: spl=en_us
\section{Hahn-Banach theorem}
In a linear space \(V\), a \(\mathbb{K}\)-functional is a map \(f : V \to \mathbb{K}\), where \(\mathbb{K}\) is either \(\mathbb{R}\) or \(\mathbb{C}\). As an example, a norm is a \(\mathbb{R}\)-functional (real functional) that is absolute-homogeneous and subadditive. The following definition exhibits a (non-exhaustive) list of other such possible properties of real functionals.
\begin{definition}{Real functional}{real_functionals}
    Let \(V\) be a linear space \(V\). A real functional \(f : V \to \mathbb{R}\) is
    \begin{enumerate}[label=(\alph*)]
        \item positive-homogeneous if \(f(\lambda x) = \lambda f(x)\) for all \(x \in V\) and \(\lambda \geq 0\);
        \item additive if \(f(x + y) = f(x) + f(y)\) for all \(x,y \in V\);
        \item subadditive if \(f(x + y) \leq f(x) + f(y)\) for all \(x,y \in V\);
        \item convex if \(f(\alpha x + (1 - \alpha)y) \leq \alpha f(x) + (1-\alpha)f(y)\) for all \(x,y \in V\) and \(\alpha \in [0,1]\).
        \item sublinear if \(f\) is positive-homogeneous and subadditive;
        \item \(\mathbb{R}\)-linear if \(f(\alpha x + \beta y) = \alpha f(x) + \beta f(y)\) for all \(x,y \in V\) and \(\alpha, \beta \in \mathbb{R}\).
    \end{enumerate}
\end{definition}
\begin{remark}
    It is worthwhile to note sublinearity implies convexity. Indeed let \(f : V \to \mathbb{R}\) be a sublinear real functional. Then for all \(x,y \in V\) and \(\alpha \in [0,1]\) we have
    \begin{equation*}
        f(\alpha x + (1-\alpha)y) \leq f(\alpha x) + f((1 - \alpha)y)
    \end{equation*}
    by subadditivity. Then positive-homogeneity yields
    \begin{equation*}
        f(\alpha x + (1-\alpha)y) \leq \alpha f(x) + (1-\alpha)y,
    \end{equation*}
    hence \(f\) is convex.
\end{remark}

Let \(A\) be some set and let \(f, g : A \to \mathbb{R}\). We say \(f\) is an upper bound for \(g\) in \(A\) if \(x \in A \implies f(x) \geq g(x)\). The existence of a convex functional that is an upper bound to a linear functional on some proper subspace is a sufficient condition to extend this linear functional while respecting the upper bound on the new subspace of definition.
\begin{lemma}{Existence of an extension with a convex upper bound}{extension_linear_functional}
    Let \(V\) be a linear space over \(\mathbb{R}\) and let \(f : W \to \mathbb{R}\) be a linear functional defined in the proper subspace \(W \subset V\). If there exists a convex functional \(p : V \to \mathbb{R}\) that is an upper bound for \(f\) in \(W\), then, for every \(u \notin W\), there exists a linear functional \(f_u : U \to \mathbb{R}\) defined in the subspace spanned by \(\set{u} \cup W\), such that \(f_u\) is an extension of \(f\) and \(p\) is an upper bound for \(f_u\) in \(U\).
\end{lemma}
\begin{proof}
    Let \(u \in V \setminus W\) and let \(U = \setc{\alpha u + w}{\alpha \in \mathbb{R}, w \in W}\) be the subspace spanned by \(\set{u} \cup W\). Notice we have the implication \(v \in U \implies \exists! \alpha \in \mathbb{R}, \exists! w \in W : v = \alpha u + w\). Indeed, if \(v \in U\) can be written as both \(\alpha u + w\) and \(\tilde{\alpha}u + \tilde{w}\), with \(\alpha, \tilde{\alpha}\in \mathbb{R}\) and \(w, \tilde{w} \in W\), then
    \begin{equation*}
        (\alpha - \tilde{\alpha})u = \tilde{w} - w,
    \end{equation*}
    hence \(\alpha = \tilde{\alpha}\) and \(\tilde{w} - w\) since the right hand side belongs to \(W\) and the left hand side's only intersection with \(W\) is the zero vector. This shows there is a bijective map \(\phi : U \to \mathbb{R} \times W\) that maps \(v \mapsto (\alpha_v, w_v)\), and we may compose this map with coordinate projections, obtaining the linear maps \(\phi_1 = \pi_1 \circ \phi : U \to \mathbb{R}\) and \(\phi_2 = \pi_2 \circ \phi : U \to W\) such that \(v \mapsto \alpha_v\) and \(v \mapsto w_v\), respectively. Indeed, if \(v_1, v_2 \in U\), then there exists unique \(\alpha_1, \alpha_2 \in \mathbb{R}\) and unique \(w_1, w_2 \in W\) such that \(v_1 = \alpha_1 u + w_1\) and \(v_2 = \alpha_2 u + w_2\), which yields \(\lambda_1 v_1 + \lambda_2 v_2 = \left(\lambda_1 \alpha_1 + \lambda_2 \alpha_2\right) u + \lambda_1 w_1 + \lambda_2 w_2\), hence
    \begin{equation*}
        \phi_1(\lambda_1 v_1 + \lambda_2 v_2) = \lambda_1 \phi_1(v_1) + \lambda_2 \phi_1(v_2)
        \quad\text{and}\quad
        \phi_2(\lambda_1 v_1 + \lambda_2 v_2) = \lambda_1 \phi_2(v_1) + \lambda_2 \phi_2(v_2),
    \end{equation*}
    as claimed.

    We consider the linear functional \(f_u = F\phi_1 + f\circ\phi_2\) on \(U\), where \(F\in \mathbb{R}\) is a constant to be determined. It is clear that \(\restrict{f_u}{W} = f\), since \(\phi_1(W) = \set{0}\) and \(\restrict{\phi_2}{W} = \id{W}\). Moreover, \(f_u(u) = F\). We'll choose \(F\) by requiring \(f_u(v) \leq p(v)\) for all \(v \in U\). Therefore, we want to find \(F\) such that \(\alpha F + f(w) \leq p(\alpha u + w)\) for all \(\alpha \in \mathbb{R}\) and \(w \in W\). By hypothesis, this is satisfied for \(\alpha = 0\), then for \(\alpha > 0\) this requirement implies
    \begin{equation*}
        F \leq \frac1\alpha \left[p(\alpha u + w) - f(w)\right]
    \end{equation*}
    and for \(\alpha < 0\) it implies
    \begin{equation*}
        F \geq \frac1\alpha \left[p(\alpha u + w) - f(w)\right].
    \end{equation*}
    That is, \(F\) exists and satisfies
    \begin{equation*}
        \sup_{\alpha > 0, w \in W} \frac{1}{\alpha}\left[f(w) - p(w - \alpha u)\right] \leq F \leq \inf_{\alpha > 0, w \in W} \frac{1}{\alpha}\left[p(w + \alpha u) - f(w)\right],
    \end{equation*}
    provided that supremum is less or equal to that infimum.

    Let \(\lambda, \mu > 0\) and let \(v, w \in W\), then by linearity of \(f\),
    \begin{equation*}
        \frac{1}{\lambda} f(v) + \frac{1}{\mu}f(w) = f\left(\frac{1}{\lambda}v + \frac{1}{\mu}{w}\right) = \frac{\lambda + \mu}{\lambda \mu} f\left(\frac{\mu}{\lambda + \mu}v + \frac{\lambda}{\lambda + \mu}w\right).
    \end{equation*}
    We add and subtract \(u\) in the argument of \(f\) on the right hand side, we may write
    \begin{equation*}
        \frac{1}{\lambda} f(v) + \frac{1}{\mu}f(w) = \frac{\lambda + \mu}{\lambda \mu} f\left(\frac{\mu}{\lambda + \mu}(v - \lambda u) + \frac{\lambda}{\lambda + \mu}(w + \mu u)\right),
    \end{equation*}
    noting that we are not computing \(f\) on a vector that is outside of its domain of definition. By hypothesis, \(p\) is an upper bound for \(f\) in \(W\), then
    \begin{equation*}
        \frac{1}{\lambda}f(v) + \frac{1}{\mu}f(w) \leq \frac{\lambda + \mu}{\lambda \mu} p\left(\frac{\mu}{\lambda + \mu}(v - \lambda u) + \frac{\lambda}{\lambda + \mu}(w + \mu u)\right).
    \end{equation*}
    Since \(\frac{\lambda}{\lambda + \mu} = 1 - \frac{\mu}{\lambda + \mu}\), convexity yields
    \begin{equation*}
        \frac1{\lambda}f(v) + \frac1{\mu}f(w) \leq \frac{\lambda + \mu}{\lambda \mu} \left(\frac{\mu}{\lambda + \mu} p(v - \lambda u) + \frac{\lambda}{\lambda + \mu} p(w + \mu u)\right) = \frac{1}{\lambda}p(v - \lambda u) + \frac{1}{\mu}p(w + \mu u).
    \end{equation*}
    Rearranging, we have shown that
    \begin{equation*}
        \frac{1}{\lambda} \left[f(v) - p(v - \lambda u)\right] \leq \frac{1}{\mu} \left[p(w + \mu u) - f(w)\right]
    \end{equation*}
    for all \(\lambda, \mu > 0\) and \(v,w \in W\), hence the supremum of the left hand side is less or equal to the infimum of the right hand side. That is, there exists \(F\) such that \(p\) is an upper bound for \(f_u\).
\end{proof}

This result can be further improved to show the existence of a linear functional that extends the original to the entire linear space, while respecting the upper bound condition of a convex functional. This improved result is known as the Hahn-Banach theorem for real linear spaces. Before stating and proving the theorem, we recall definitions of partially ordered sets and \nameref{thm:zorn}.
\begin{definition}{Partially ordered set and linearly ordered set}{poset_chain}
    A \emph{partially ordered set} \(X\) is a set \(X\) equipped with an \emph{ordering} \(\preceq\) that is
    \begin{enumerate}[label=(\alph*)]
        \item reflexive: \(\forall x \in X, x \preceq x\);
        \item transitive: \(\forall x,y,z \in X, x \preceq y \land y \preceq z \implies x \preceq z\); and
        \item anti-symmetric: \(\forall x, y\in X, x \preceq y \land y \preceq x \implies x = y\).
    \end{enumerate}
    If for any pair \(x, y \in X\) either \(x \preceq y\) or \(y \preceq x,\) then \(X\) is a \emph{linearly ordered set}.

    An \emph{upper bound of a subset \(Y \subset X\)} in a partially ordered set \(X\) is an element \(x \in X\) such that \(y \preceq x\) for all \(y \in Y.\) The \emph{greatest element of a partially ordered set \(X\)} is an element \(n \in X\) such that \(x \preceq n\) for all \(x \in X\). A \emph{maximal element of a partially ordered set \(X\)} is an element \(m \in X\) such that \(x \in X : m \preceq x \implies x = m\).
\end{definition}
\begin{remark}%{Power set is partially ordered under inclusion}{power_poset}
    An example of a partially ordered set is the power set \(\mathbb{P}(S)\) of some set \(S\) with the ordering given by the inclusion \(\subset\).
\end{remark}

\begin{theorem}{Zorn's lemma}{zorn}
    Let \(X\) be a non-empty partially ordered set, any non-empty linearly ordered subset of which has an upper bound in \(X\). Then some linearly ordered subset has an upper bound that is simultaneously a maximal element in \(X\).
\end{theorem}
\begin{remark}
    Within ZF-set theory, this lemma is equivalent to the axiom of choice, that is, this theorem follows from ZFC-set theory.
\end{remark}

\begin{theorem}{Hahn-Banach theorem for real linear spaces}{Hahn_Banach_real}
    Let \(V\) be a linear space over \(\mathbb{R}\) and let \(f : U \to \mathbb{R}\) be a real linear functional defined on a linear subspace \(U\subset V\). If there exists a convex functional \(p : V \to \mathbb{R}\) that is an upper bound for \(f\) in \(U\), then there exists a linear functional \(\tilde{f} : V \to \mathbb{R}\) such that \(\tilde{f}\) extends \(f\) and such that \(p\) is an upper bound for \(\tilde{f}\) in \(V\).
\end{theorem}
\begin{proof}
    Clearly, if \(U = V\), \(\tilde{f} = f\) satisfies the theorem. We assume \(U\) is a proper subspace of \(V\).

    Let \(\mathcal{F}\) be the set of linear functionals defined in subspaces of \(V\) that extend \(f\) and have \(p\) as an upper bound in their domains of definition. \cref{lem:extension_linear_functional} guarantees that \(\mathcal{F}\) contains more than just \(f\), since \(U\) is a proper subspace of \(V\). In particular, \(\mathcal{F}\) is non-empty.

    We consider the binary relation \(\preceq\) on \(\mathcal{F}\) defined by \(\ell_1 \preceq \ell_2\) if \(\ell_1\) is extended by \(\ell_2\). Let \(\ell_1, \ell_2, \ell_3 \in \mathcal{F}\), where \(\ell_1 : V_1 \to \mathbb{R}\), \(\ell_2 : V_2 \to \mathbb{R}\) and \(\ell_3 : V_3 \to \mathbb{R}\). Since \(\restrict{\ell_1}{V_1} = \ell_1\), we have \(\ell_1 \preceq \ell_1\), hence this relation is reflexive. If \(\ell_1 \preceq \ell_2\) and \(\ell_2 \preceq \ell_3\), then \(\restrict{\ell_3}{V_2} = \ell_2\) and \(\restrict{\ell_2}{V_1} = \ell_1\), hence \(\restrict{\ell_3}{V_1} = \ell_1\), that is \(\ell_1 \preceq \ell_3\) and \(\preceq\) is transitive. If \(\ell_1 \preceq \ell_2\) and \(\ell_2 \preceq \ell_1\), we must have \(V_1 = V_2\), hence \(\ell_1 = \ell_2\), thus showing the relation is anti-symmetric. Then \((\mathcal{F}, \preceq)\) is a non-empty partially ordered set.

    Let \(\family{\ell_{\lambda}}{\lambda \in \Lambda} \subset \mathcal{F}\) be a non-empty linearly ordered subset of \(\mathcal{F}\), where \(\Lambda\) is some non-empty indexing set and \(\ell_{\lambda} : V_{\lambda} \to \mathbb{R}\). We consider the union \(W = \bigcup_{\lambda \in \Lambda} V_{\lambda}\). Since each \(V_\lambda\) is a subspace, we have \(0 \in W\). Let \(v, w \in W\) and \(\alpha, \beta \in \mathbb{R}\), then there exists \(\lambda_v, \lambda_w \in \Lambda\) such that \(\alpha v \in V_{\lambda_v}\) and \(\beta w \in V_{\lambda_w}\). By linear order, we have either \(V_{\lambda_v} \subset V_{\lambda_w}\) or \(V_{\lambda_w} \subset V_{\lambda_v}\), then \(\alpha v + \beta w \in V_{\lambda_v}\) or \(\alpha v + \beta w \in V_{\lambda_w}\), hence \(\alpha v + \beta w \in W\). That is, \(W\) is a linear subspace of \(V\) and it contains every \(V_{\lambda}\). We claim the map \(\tilde{\ell} : W \to \mathbb{R}\) defined by \(\ell(v) = \ell_{\lambda}(v)\) if \(v \in V_{\lambda}\) is a linear functional on \(W\) that extends \(\ell_{\lambda}\) for \(\lambda \in \Lambda\). First, by the previous argument, \(\tilde{\ell}(\alpha v + \beta w) = \ell_{\lambda_v}(\alpha v + \beta w)\) or \(\tilde{\ell}(\alpha v + \beta w) = \ell_{\lambda_w}(\alpha v + \beta w)\), hence \(\tilde{\ell}(\alpha v + \beta w) = \alpha \ell(v) + \beta \ell(w)\), that is, \(\tilde{\ell}\) is linear. Notice the map is well defined since if \(v \in V_{\lambda}\) and \(v \in V_{\mu}\) with \(\lambda, \mu \in \Lambda\), then either \(\ell_{\mu} \preceq \ell_\lambda\) or \(\ell_{\lambda} \preceq \ell_{\lambda}\), hence \(\ell_{\mu}(v) = \ell_{\lambda}\). By construction, \(\ell_{\lambda} \preceq \tilde{\ell}\) for all \(\lambda \in \Lambda\). We have thus constructed an upper bound for \family{\ell_{\lambda}}{\lambda \in \Lambda} that lies in \(\mathcal{F}\), since \(p\) is an upper bound for \(\tilde{\ell}\) in \(W\).

    Since every non-empty linearly ordered subset of \((\mathcal{F}, \preceq)\) has an upper bound in \(\mathcal{F}\), \nameref{thm:zorn} guarantees the existence of a maximal element \(\tilde{f} \in \mathcal{F}\), that is, \(\tilde{f} : \tilde{V} \to \mathbb{R}\) is a linear functional defined on a linear subspace \(\tilde{V}\) that extends \(f\) and has \(p\) as an upper bound in \(\tilde{V}\). Suppose, by contradiction, \(\tilde{V}\) is a proper subspace of \(V\). Then, there exists \(u \in V \setminus \tilde{V}\), hence \cref{lem:extension_linear_functional} guarantees the existence of a linear functional defined on the subspace spanned by \(u\) and \(\tilde{V}\) that extends \(\tilde{f}\) and has \(p\) as an upper bound. This contradicts the maximal property of \(\tilde{f}\), hence \(\tilde{V} = V\).
\end{proof}

We may now generalize the Hahn-Banach theorem for complex linear spaces. A map \(f : A \to \mathbb{K}\) is \emph{dominated} by a map \(g : A \to \mathbb{R}\) on \(A\) if \(x \in A \implies \abs{f(x)} \leq g(x)\), where \(\mathbb{K}\) is either \(\mathbb{R}\) or \(\mathbb{C}\).
\begin{theorem}{Hahn-Banach theorem for complex linear spaces}{Hahn_Banach_complex}
    Let \(V\) be a linear space over \(\mathbb{C}\) and let \(f : U \to \mathbb{C}\) a complex linear functional defined in a linear subspace \(U \subset V\). Suppose there exists a real functional \(p : V \to \mathbb{R}\) satisfying \(p(\alpha u + \beta v) \leq \abs{\alpha} p(u) + \abs{\beta} p(v)\) for all \(u,v \in V\) and all \(\alpha,\beta \in \mathbb{C}\) with \(\abs{\alpha} + \abs{\beta} = 1\) such that \(f\) is dominated by \(p\) in \(U\). Then, there exists a complex linear functional \(\tilde{f} : V \to \mathbb{C}\) that extends \(f\) and is dominated by \(p\) on \(V\).
\end{theorem}
\begin{proof}
    We may again assume \(U\) is a proper subset of \(V\). Let \(g : U \to \mathbb{R}\) be the real functional defined by \(g = \Re \circ f\). Then, \(g\) is a real functional dominated by \(p\) on \(U\) and, in particular, \(p\) is an upper bound for \(g\) on \(U\). We consider \(\alpha, \beta \in \mathbb{R}\) and \(u,v \in V\), then the linearity of \(f\) yields
    \begin{equation*}
        g(\alpha u + \beta v) = \Re\circ f(\alpha u + \beta v) = \Re(\alpha f(u) + \beta f(v)) = \alpha g(u) + \beta g(v),
    \end{equation*}
    that is, \(g\) is \(\mathbb{R}\)-linear.

    Recall a complex linear space can be made into a real linear space by restricting the scalar multiplication to the real numbers. Observing that \(p\) is a convex function that is an upper bound for the real linear functional \(g\) defined on the subspace \(U\) of the real linear space \(V\), we have by the \nameref{thm:Hahn_Banach_real} that there exists a real linear functional \(\tilde{g} : V \to \mathbb{R}\) that extends \(g\) and has \(p\) as an upper bound. \todo[Details on decomplexification and complexification]

    We consider the map
    \begin{align*}
        \tilde{f} : V &\to \mathbb{C}\\
                    v &\mapsto \tilde{g}(v) - i \tilde{g}(iv)
    \end{align*}
    and show it is an extension of \(f\), a complex linear functional, and dominated by \(p\) on \(V\). For \(u \in U\), we have \(iu \in U\), then \(\tilde{g}(u) = g(u)\) and \(\tilde{g}(iu) = g(iu)\), hence
    \begin{align*}
        \tilde{f}(u) = g(u) - i g(iu) &= \Re\circ f(u) - i \Re\circ f(iu)\\
                                      &= \Re\circ f(u) - i \Re\circ(i f(u))\\
                                      &= \Re\circ f(u) + i \Im \circ f(u) = f(u),
    \end{align*}
    that is, \(\tilde{f}\) extends \(f\).

    To show linearity, we first show \(\tilde{f}\) is \(\mathbb{R}\)-linear. Let \(u, v \in V\), \(\alpha, \beta \in \mathbb{R}\), then the \(\mathbb{R}\)-linearity of \(\tilde{g}\) yields
    \begin{align*}
        \tilde{f}(\alpha u + \beta v) &= \tilde{g}(\alpha u + \beta v) - i \tilde{g}(i \alpha u + i \beta v)\\
                                      &= \alpha \tilde{g}(u) + \beta \tilde{g}(v) - i \alpha \tilde{g}(iu) - i \beta \tilde{g}(v)\\
                                      &= \alpha \left[\tilde{g}(u) - i \tilde{g}(iu)\right] + \beta \left[\tilde{g}(v) - i \tilde{g}(iv)\right]\\
                                      &= \alpha \tilde{f}(u) + \beta \tilde{f}(v),
    \end{align*}
    hence \(\tilde{f}\) is \(\mathbb{R}\)-linear and, in particular, additive. Next, notice
    \begin{equation*}
        \tilde{f}(iv) = \tilde{g}(iv) -i \tilde{g}(-v) = i \left[\tilde{g}(v) - i\tilde{g}(iv)\right] = i \tilde{f}(iv),
    \end{equation*}
    for all \(v \in V\). We consider \(\xi \in \mathbb{C}\), then \(\xi = \alpha + i \beta\) for real numbers \(\alpha, \beta\). By \(\mathbb{R}\)-linearity and additivity, we have
    \begin{align*}
        \tilde{f}(\xi v) = \tilde{f}(\alpha v + i \beta v) &= \alpha \tilde{f}(v) + \beta \tilde{f}(iv)\\ &= \alpha \tilde{f} (v) + i \beta \tilde{f}(v)\\&= (\alpha + i \beta) \tilde{f}(v) = \xi \tilde{f}(v)
    \end{align*}
    for all \(v \in V\). Together with additivity, this shows \(\tilde{f}\) is a complex linear functional.

    Let \(\alpha \in \mathbb{C}\) with \(\abs{\alpha} = 1\), then for every \(v \in V\), we have \(p(\alpha v) \leq p(v)\) and, in fact, we always have the equality. Indeed, we consider \(u = \alpha v\), then \(p(\alpha^{-1} u) \leq p(u)\), which shows \(p(\alpha v) \geq p(v)\) for all \(v \in V\). For all \(v \in V\), there exists \(\theta \in [0,2\pi)\) such that \(\abs{\tilde{f}(v)} = \tilde{f}(v) e^{i\theta}\), then
    \begin{equation*}
        \abs{\tilde{f}(v)} = \tilde{f}(v) e^{i\theta} = \tilde{f}(e^{i\theta}v)
    \end{equation*}
    since \(\tilde{f}\) is \(\mathbb{C}\)-linear. Taking the real part yields
    \begin{equation*}
        \abs{\tilde{f}(v)} = \tilde{g}(e^{i\theta}v) \leq p(e^{i\theta}v) = p(v),
    \end{equation*}
    because \(p\) dominates \(\tilde{g}\) on \(V\).
\end{proof}

Finally, we specialize the result for normed linear spaces over \(\mathbb{C}\).
\begin{theorem}{Hahn-Banach theorem for normed linear spaces}{Hahn_Banach_normed}
    Let \((V, \norm{\noarg})\) be a linear space over \(\mathbb{C}\) and let \(f : U \to \mathbb{C}\) be a complex linear functional defined on the linear subspace \(U\subset V\). If there exists \(M \in \mathbb{R}\) such that
    \begin{equation*}
        M = \sup \setc*{\frac{\abs{f(v)}}{\norm{v}}}{v \in U\setminus \set{0}},
    \end{equation*}
    then there exists a complex linear functional \(\tilde{f} : V \to \mathbb{C}\) that extends \(f\) and is bounded on \(V\), with \(\norm{\tilde{f}} = M\).
\end{theorem}
\begin{proof}
    Let \(p : V \to \mathbb{C}\) be the real functional defined by \(v \mapsto M \norm{v}\). Then \(p\) inherits absolute-homogeneity and subadditivity from the norm. In particular, for all \(u, v \in V\) and all \(\alpha, \beta \in \mathbb{C}\), we have
    \begin{equation*}
        p(\alpha u + \beta v) \leq p(\alpha u) + p(\beta v) = \abs{\alpha} p(u) + \abs{\beta}p(v).
    \end{equation*}
    By the definition of \(M\), we have \(\abs{f(v)} \leq p(v)\) for all \(v \in U\). \nameref{thm:Hahn_Banach_complex} ensures the existence of a complex linear functional \(\tilde{f} : V \to \mathbb{C}\) that extends \(f\) and is dominated by \(p\).

    Since \(p\) dominates \(\tilde{f}\), then \(\norm{\tilde{f}} \leq M\). However, \(\tilde{f}\) extends \(f\), then
    \begin{equation*}
        \norm{\tilde{f}} = \sup_{v \in V}{\frac{\abs{\tilde{f}(v)}}{\norm{v}}} \geq \sup_{v \in U} \frac{\abs{\tilde{f}(v)}}{\norm{v}} = \sup_{v \in U} \frac{\abs{f(v)}}{\norm{v}} = M.
    \end{equation*}
    That is, \(\norm{\tilde{f}} = M\).
\end{proof}


% vim: spl=en_us
\section{Topological dual}
Recall from Linear Algebra that the \emph{algebraic dual} \(V'\) of a complex linear space \(V\) is the linear space of linear functionals \(\ell : V \to \mathbb{C}\).
\begin{definition}{Topological dual}{topological_dual}
    Let \((V, \norm{\noarg})\) be a normed linear space. The \emph{topological dual} \((V^\dag, \norm{\noarg})\) with respect to \((V, \norm{\noarg})\) is the Banach space of bounded linear functionals \((\bounded(V, \mathbb{C}), \norm{\noarg})\) with the operator norm.
\end{definition}
\begin{remark}
    Clearly, \(V^\dag \subset V'\). Moreover, from \cref{thm:bounded_operators_Banach} we know \((V^\dag, \norm{\noarg})\) is a Banach space.
\end{remark}

We may use Hahn-Banach theorem to show the non-triviality of the topological dual of a non-trivial normed linear space, that is, it contains more than just the zero linear functional.
\begin{lemma}{Existence of a bounded linear functional}{topological_dual_nontrivial}
    Let \((V, \norm{\noarg})\) be a non-trivial normed linear space. For every \(u \in V\), there exists a bounded linear functional \(\ell_{u} : V \to \mathbb{C}\) such that \(\norm{\ell_u} = 1\) and \(\ell_u(u) = \norm{u}\).
\end{lemma}
\begin{proof}
    Let \(u \in V \setminus \set{0}\) and consider the linear subspace \(U\) spanned by \(u\),
    \begin{equation*}
        U = \setc{v \in V}{\exists \alpha \in \mathbb{C}: v = \alpha u}.
    \end{equation*}
    Since \(u \neq 0\), for every \(v \in U\), there exists a unique \(\alpha_v \in \mathbb{C}\) such that \(v = \alpha_v u\).
    Indeed, suppose \(v = \alpha_v u\) and \(v = \tilde{\alpha}_v u\), then \((\alpha_v - \tilde{\alpha}_v)u = 0\), hence \(\alpha_v = \tilde{\alpha}_v\). In other words, the linear functional
    \begin{align*}
        f : U &\to \mathbb{C}\\
            v &\mapsto \alpha_v
    \end{align*}
    exists. In particular, we consider the linear functional \(g = \norm{u} f\) defined on \(U\), then
    \begin{equation*}
        \sup_{v \in U} \frac{\abs{g(v)}}{\norm{v}} = \sup_{\alpha \in \mathbb{C}\setminus\set{0}} \frac{\abs*{\norm{u} f(\alpha u)}}{\norm{\alpha u}} = \sup_{\alpha \in \mathbb{C}\setminus\set{0}} \frac{\abs*{\alpha\norm{u}}}{\abs{\alpha}\cdot \norm{u}} = 1.
    \end{equation*}
    The \nameref{thm:Hahn_Banach_normed} ensures the existence of \(\ell_u \in \bounded(V, \mathbb{C})\) that extends \(g\) with \(\norm{\ell_v} = 1\). Moreover, \(\ell_u(u) = g(u) = \norm{u}\).

    For \(\ell_0\), we may take any bounded linear functional constructed as above and set \(\ell_0 = \ell_u\), which satisfies \(\ell_0(0) = 0\) by linearity.
\end{proof}
\begin{corollary}
    If \(V\) is a non-trivial normed linear space, then its topological dual \(V^\dag\) is non-trivial.
\end{corollary}
\begin{corollary}
    Let \((V, \norm{\noarg})\) be a normed linear space. Then
    \begin{equation*}
        \bigcap_{\ell \in V^\dag} \ker \ell = \set{0},
    \end{equation*}
    that is, if \(v \in V\) is such that for all \(\ell \in V^\dag\) we have \(\ell(v) = 0\), then \(v = 0\).
\end{corollary}
\begin{proof}
    Let \(v \in \bigcap_{\ell \in V^\dag}\ker\ell\). In particular, \(v \in \ker\ell_v\), where \(\ell_v \in \bounded(V, \mathbb{C})\) is the map guaranteed to exist by \cref{lem:topological_dual_nontrivial}. Then \(\ell_v(v) = \norm{v} = 0\), that is, \(v = 0\).
\end{proof}

In a finite dimensional linear space \(V\), we know there is a linear isomorphism between \(V\) and its \emph{algebraic bidual} \((V')'\). Similarly, we may construct an injective linear map from \(V\) to its \emph{topological bidual} \((V^\dag)^\dag\).
\begin{proposition}{Natural linear map to topological bidual}{topological_bidual}
    Let \((V, \norm{\noarg})\) be a complex normed linear space. The evaluation map \(\eval : V \to (V^\dag)^\dag\) defined by \(v\mapsto \eval_v\), where
    \begin{align*}
        \eval_v : V^\dag &\to \mathbb{C}\\
                          \ell &\mapsto \ell v,
    \end{align*}
    is a natural injective linear map. Moreover, \(\eval\) is a linear isometry.
\end{proposition}
\begin{proof}
    Let \(v, u \in V\) and \(z \in \mathbb{C}\), then for all \(\ell \in V^\dag\),
    \begin{equation*}
        \eval_{v + zu}\ell = \ell(v + zu) = \ell v + z \ell u = \eval_v \ell + z \eval_{u} \ell = (\eval_v + z \eval_u) \ell,
    \end{equation*}
    hence \(\eval(v + zu) = \eval(v) + z\eval(u)\), as desired.

    For all \(v \in V\) and \(\ell \in V^\dag \setminus\set{0}\) we have
    \begin{equation*}
        \abs{\eval_v\ell} = \abs{\ell v} \leq \norm{\ell} \norm{v}.
    \end{equation*}
    For any given \(v\), we have thus \(\norm{v}\) as an upper bound to the set \(\setc*{\frac{\abs{\eval_v\ell}}{\norm{\ell}}}{\ell \in V^\dag\setminus\set{0}}\), hence \(\eval_v\) is bounded. That is, \(\eval_v \in (V^\dag)^\dag\). Furthermore, consider the family of maps \(\family{\ell_v}{v \in V} \subset V^\dag\) guaranteed to exist by \cref{lem:topological_dual_nontrivial}. Then for every \(v \in V\), we have
    \begin{equation*}
        \norm{\eval_v} = \sup_{\ell \in V^\dag} \frac{\abs*{\eval_v\ell}}{\norm{\ell}} \geq \frac{\abs*{\ell_vv}}{\norm{\ell_v}} = \norm{v},
    \end{equation*}
    hence \(\norm{\eval_v} = \norm{v}\). That is, \(\eval\) is a isometry, so it is injective.
\end{proof}



As opposed to the case of the algebraic dual of infinite linear spaces, it may be so that a Banach space \(W\) is isomorphic to its topological bidual, that is \(\eval(W) = (W^\dag)^\dag\).
\begin{definition}{Reflexive spaces}{reflexive_spaces}
    A \emph{reflexive space} is a Banach space \(W\) with the property \(\eval(W) = (W^\dag)^\dag\), and we say \(W\) is \emph{reflexive}.
\end{definition}
\begin{remark}
    We will see later that every Hilbert space is reflexive.
\end{remark}

With the topological dual we may define another notion of convergence.
\begin{definition}{Weak convergence}{weak_convergence}
    A sequence \(\family{f_n}{n\in \mathbb{N}} \subset V\) is said to \emph{weakly converge} to \(f \in V\) if for all \(\ell \in V^\dagger\) we have a sequence \(\family{\ell f_n}{n\in \mathbb{N}} \subset \mathbb{C}\) that converges to \(\ell(f)\), and we denote \(\displaystyle\wlim_{n\to\infty} f_n = f\) or \(f_n \wto f\). Moreover, when there is a need to distinguish them, the usual notion of convergence will be called \emph{strong convergence}, and we may denote it by \(\displaystyle\slim_{n\to\infty}f_n = f\) whenever the sequence \emph{strongly converges} to \(f\).
\end{definition}

\begin{proposition}{Weak convergence of a strongly convergent sequence}{weak_strong}
    Let \((V, \norm{\noarg})\) be a normed linear space. If \(\family{f_n}{n \in \mathbb{N}}\subset V\) strongly converges to \(f\), then \(f_n \wto f\).
\end{proposition}
\begin{proof}
    If we have \(f_n \to f\), then for all \(\ell \in \bounded(V, \mathbb{C})\), we have
    \begin{equation*}
        \lim_{n\to \infty} \ell f_n = \ell\left(\lim_{n\to \infty} f_n\right) = \ell f,
    \end{equation*}
    by continuity. Hence \(f_n \wto f\).
\end{proof}

% vim: spl=en_us
\section{Uniform boundedness theorem}
A consequence of the completeness of a Banach space due to Baire's category argument is the Banach-Steinhaus theorem, also known as the uniform boundedness theorem or principle. % A collection of linear maps \(\mathfrak{T}\) defined on a linear space \(V\) is \emph{dominated} by a real functional \(p : V \to \mathbb{R}\) if \(\norm{Tv} \leq p(v)\) for all \(v \in V\) and all \(T \in \mathfrak{T}\). We'll denote by \(\mathbb{R}^+\) the set of positive real numbers and by \(\mathbb{R}^+_0\) the set of non-negative real numbers.
\begin{theorem}{Banach-Steinhaus theorem}{Banach_Steinhaus}
    Let \(\mathfrak{T} \subset \bounded{(V,W)}\) be a non-empty collection of bounded linear operators defined on the Banach space \(V\) into a normed linear space \(W\). If for all \(v \in V\) the set \(\setc{\norm{Tv}}{T \in \mathfrak{T}}\) is bounded, then the set \(\setc{\norm{T}}{T \in \mathfrak{T}}\) is bounded.
\end{theorem}
\begin{proof}
    For each \(v \in V\), the set \(\setc{\norm{Tv}}{T \in \mathfrak{T}}\) is bounded above by some natural number \(n_v \in \mathbb{N}\) and in particular by any natural number greater than \(n_v\). Consider
    \begin{equation*}
        V_n = \setc{v \in V}{\forall T \in \mathfrak{T} : \norm{Tv} \leq n}
    \end{equation*}
    for all \(n \in \mathbb{N}\), then \(V = \bigcup_{n \in \mathbb{N}} V_n\). We may rewrite
    \begin{equation*}
        V_n = \bigcap_{T \in \mathfrak{T}} \setc{v \in V}{\norm{Tv}\leq n}.
    \end{equation*}
    Moreover, for each \(T \in \mathfrak{T}\) we consider the real functional \(p_T = \norm{\noarg} \circ T\), which is continuous as a composition of continuous maps. Then,
    \begin{equation*}
        V_n = \bigcap_{T \in \mathfrak{T}} \preim{p_T}{[0,n]}
    \end{equation*}
    is manifestly closed in \(V\).

    \nameref{thm:Baire_Hausdorff}, then there exists \(m \in \mathbb{N}\) such that \(\inte_V(V_m)\) is non-empty, otherwise \(V\) would be the countable union of nowhere dense sets. Let \(u\) be an interior point of \(V_m\), then there exists \(r > 0\) such that \(B_r(u) \subset V_m\). If \(w \in B_r(0)\), then
    \begin{equation*}
        \norm{(w + u) - u} = \norm{w} < r,
    \end{equation*}
    that is \(w + u \in B_r(u)\). Since \(u, u + w \in V_m\), we have
    \begin{equation*}
        \norm{Tu} \leq m
        \quad\text{and}{\quad}
        \norm{T(u + w)} \leq m
    \end{equation*}
    for all \(T \in \mathfrak{T}\), provided \(w \in B_r(0)\). For all \(T \in \mathfrak{T}\), this yields
    \begin{equation*}
        \norm{Tw} = \norm{T(w + u) - Tu} \leq \norm{T(w + u)} + \norm{Tu} \leq 2m
    \end{equation*}
    for all \(w \in B_r(0)\).

    Let \(v \in V\setminus{0}\), then \(\frac{r}{2\norm{v}}v \in B_r(0)\). As a result
    \begin{equation*}
        \norm{Tv} = \frac{2\norm{v}}{r}\norm*{T\left(\frac{r}{2\norm{v}}v\right)} \leq \frac{4m}{r}\norm{v}
    \end{equation*}
    for every \(T \in \mathfrak{T}\). The previous inequality holds for \(v = 0\), hence
    \begin{equation*}
        \sup_{T \in \mathfrak{T}} \norm{T} \leq \frac{4m}{r},
    \end{equation*}
    since \(r\) depends only on \(m\) and the arbitrary interior point \(u\).
\end{proof}

An immediate result concerns the convergence of a sequence of bounded operators.
\begin{theorem}{Strong limit of a sequence of bounded operators}{strong_limit_operators}
    Let \(\family{T_n}{n\in \mathbb{N}} \subset \bounded(V,W)\) be a sequence of bounded operators defined on a Banach space \(V\) into a normed linear space \(W\). If for each \(v \in V\) the sequence \(\family{T_nv}{n \in \mathbb{N}}\subset W\) converges in \(W\), then the \emph{strong limit} of the sequence, defined by
    \begin{align*}
        T : V &\to W\\
            v &\mapsto \lim_{n\to \infty} T_n v,
    \end{align*}
    is a bounded linear operator and \todo[\(\norm{T} \leq \liminf_{n\to\infty}\norm{T_n}\). liminf?]
\end{theorem}
\begin{proof}
    As a sequence of bounded linear operators, each \(T_n\) is continuous, then it follows that \(T\) is linear. Recall convergent sequences are bounded, then by the \nameref{thm:Banach_Steinhaus} the set \(\setc{\norm{T_n}}{n \in \mathbb{N}}\) is bounded above by some \(M \geq 0\). For each \(n \in \mathbb{N}\), we have
    \begin{equation*}
        \norm{T_n v} \leq \norm{T_n} \cdot \norm{v} \leq M \norm{v}.
    \end{equation*}
    for all \(v \in V\). The continuity of the norm yields
    \begin{equation*}
        \norm{Tv} = \lim_{n\to\infty} \norm{T_n v} \leq M \norm{v},
    \end{equation*}
    hence
    \begin{equation*}
        \sup_{v \in V} \frac{\norm{Tv}}{\norm{v}} \leq M,
    \end{equation*}
    that is, \(T \in \bounded(V,W)\).
\end{proof}
\begin{remark}
    It is worth noting the strong limit is not the uniform limit. That is, it does not follow that \(\norm{T - T_n} \to 0\). To denote this limit, we'll write \(\displaystyle{\slim_{n\to\infty} T_n = T}\).
\end{remark}

Let \(\bounded(V) = \bounded(V,V)\) be the set of bounded linear operators that are endomorphisms on a normed linear space. We may use the previous result to express the inverse of an operator in \(\bounded(V)\). If \(A\) is an endomorphism, we write \(A^n = A^{n-1}\circ A\) with \(A^0 = \unity\) for all \(n \in \mathbb{N}\), where \(\unity\) is the identity map.
\begin{theorem}{C. Neumann series}{neumann-series}
    Let \(T \in \bounded(V)\) defined on a Banach space \(V\) such that \(\norm{\unity - T} < 1\). Then the strong limit
    \begin{equation*}
        S = \slim_{n\to \infty} \sum_{k = 0}^{n} (\unity - T)^k
    \end{equation*}
    exists. Moreover, \(S\) is the unique bounded linear inverse map for \(T\).
\end{theorem}
\begin{proof}
    We show by induction on \(n\) that \(\norm*{(\unity - T)^n} \leq \norm{\unity - T}^n\) for all \(n \in \mathbb{N}\). It trivially holds for the base case \(n = 1\), then it remains to show the inductive step. Suppose it holds for some \(k \in \mathbb{N}\), then
    \begin{align*}
        \norm{(\unity - T)^{k+1}} &= \sup_{v \in V}\frac{\norm{(\unity - T)^{k+1}v}}{\norm{v}}\\
                                  &= \sup_{v \in V} \frac{\norm{(\unity - T)^k \circ (\unity - T)v}}{\norm{v}}\\
                                  &\leq \sup_{v \in V} \frac{\norm{(\unity - T)^k}\cdot \norm{(\unity - T)v}}{\norm{v}}\\
                                  &\leq \norm{(\unity - T)}^k \norm{\unity - T},
    \end{align*}
    hence it holds for \(k + 1\). By the principle of finite induction, \(\norm*{(\unity - T)^n} \leq \norm{\unity - T}^n\) for all \(n \in \mathbb{N}\). Notice it also holds for \(n = 0\).

    Let \(S_n = \sum_{k=0}^n (\unity - T)^k\) for all \(n \in \mathbb{N}_0\). Let \(v \in V\), then for \(n, m \in \mathbb{N}_0\) with \(n < m\) we have
    \begin{align*}
        \norm{S_nv - S_mv} = \norm*{\sum_{k=n+1}^m (\unity - T)^kv} &\leq \sum_{k=n+1}^m \norm*{(\unity - T)^kv}\\
        &\leq \sum_{k = n+1}^m \norm{(\unity - T)^k}\cdot\norm{v}                        \\
        &\leq \left(\sum_{k=0}^{m-n-1} \norm{\unity - T}^k\right) \norm{\unity - T}^{n+1}\norm{v}                   \\
        &\leq \left(\sum_{k=0}^\infty \norm{\unity - T}^k\right) \norm{\unity - T}^{n+1} \norm{v}\\
        &\leq \frac{\norm{\unity - T}^{n+1}}{1 - \norm{\unity - T}} \norm{v}.
    \end{align*}
    We may take \(n\) arbitrarily large as to make \(\norm{S_nv - S_mv}\) arbitrarily small, hence \(\family{S_nv}{n\in \mathbb{N}}\) is a Cauchy sequence in \(V\). By completeness, we've shown \(\family{S_nv}{n\in \mathbb{N}}\) converges in \(V\) for all \(v \in V\). \cref{thm:strong_limit_operators} shows \(S = \displaystyle{\slim_{n\to\infty}S_n}\) is a bounded linear operator.

    Let \(v \in V\), then
    \begin{align*}
        S \circ T(v) &= S \circ \left[\unity - (\unity - T)\right](v)&
        T \circ S(v) &= \left[\unity - (\unity - T)\right]\circ S(v)\\
                     &= S(v) - S\circ(\unity - T)(v)&
                     &= S(v) - (\unity - T)\circ S(v)\\
                     &= S(v) - \slim_{n \to \infty} \sum_{k=0}^n (\unity - T)^{k+1}(v)&
                     &= S(v) - \slim_{n \to \infty} \sum_{k=0}^n (\unity - T)^{k+1}(v)\\
                     &= S(v) - \slim_{n\to\infty} \sum_{k=1}^n (\unity - T)^{k}(v)&
                     &= S(v) - \slim_{n\to\infty} \sum_{k=1}^n (\unity - T)^{k}(v)\\
                     &= \unity(v) = v&
                     &= \unity(v) = v.
    \end{align*}
    That is, \(S = T^{-1}\).
\end{proof}


% vim: spl=en_us
\section{Open mapping theorem}
Baire's category argument can also be used to show the open mapping theorem, which states surjective bounded maps defined on Banach spaces map open sets into open sets. To prove the theorem we need the following lemma.
\begin{lemma}{Image of a neighborhood of the zero vector}{neighborhood_0}
    Let \(X,Y\) be linear normed spaces and let \(T : X \to Y\) be a linear map. If the range \(\range{T}\) is a set of the second category in \(Y\), then to each neighborhood \(U \in \tau_X\) of \(0\) there corresponds some neighborhood \(V \in \tau_Y\) of \(0\) such that \(V \subset \cl_Y(T(U))\).
\end{lemma}
\begin{proof}
    Let \(U \in \tau_X\) be a neighborhood of \(0\). Then, there exists an open ball \(W = B_r(0) \in \tau_X\) with radius \(r > 0\) centered at the zero vector and is contained in \(U\) such that for all \(u, v \in W\) we have \(u + v \in U\). For every \(x \in X\), there exists \(n_x \in \mathbb{N}\) such that \(x \in n_x W\). That is, we may write \(X = \bigcup_{n\in \mathbb{N}} nW\), hence \(\range{T} = \bigcup_{n\in \mathbb{N}} T(nW)\). \(\range{T}\) is of the second category, therefore there exists \(m \in \mathbb{N}\) such that \(\cl_Y(T(mW))\) has non-empty interior.

    \cref{prop:closure_linear} shows \(\cl_Y(T(mW)) = m \cl_Y(T(W))\) as a linear map is in particular homogeneous. Since the map \(v \mapsto \frac1m v\) is a homeomorphism in any topological linear space, then \(S = \inte_Y\cl_Y(T(W))\) is non-empty by \cref{prop:nowhere_dense_homeomorphism}. Notice an element of \(S\) is also a point of closure of \(T(W)\), so \(S \cap T(W) \neq \emptyset\), as \(S\) is an open neighborhood for its points.

    Let \(\tilde{y} \in S \cap T(W)\), then there exists \(\tilde{x} \in W\) such that \(\tilde{y} = Tx\). Then the translation \(S - \tilde{y}\) is a neighborhood of \(0\) contained in the translation \(\cl_Y(T(W)) - \tilde{y}\). Let \(y \in T(W) - \tilde{y}\), then there exists \(w \in W\) such that \(y = T(w - \tilde{x})\). Notice \(-\tilde{x} \in W\), then \(w - \tilde{x} \in U\). That is, \(T(W) - \tilde{y} \subset T(U)\), hence \(\cl_Y(T(W)) - \tilde{y} \subset \cl_Y(T(U))\) by \cref{thm:closure_interior_homeomorphism}. We have shown \(V\) is a neighborhood of \(0\) in \(Y\) such that \(V \subset \cl_Y(T(U))\).
\end{proof}

We now show the main theorem.
\begin{theorem}{Open mapping theorem}{open_mapping}
    Let \(X, Y\) be Banach spaces. If \(T \in \bounded(X, Y)\) is surjective, then \(T\) is an open mapping.
\end{theorem}
\begin{proof}
    In this proof we denote an open ball centered at 0 of radius \(r > 0\) by \(X_r \subset X\) and \(Y_r \subset Y\). By surjectivity of \(T\) and completeness of \(Y\) it follows from \cref{lem:neighborhood_0} that for all \(r > 0\) there exists \(\rho > 0\) such that \(Y_\rho \subset \cl_Y(T(X_r))\).

    Let \(\varepsilon > 0\) and consider a sequence of radii \(\family{r_n}{n\in \mathbb{N}}\) defined by \(r_n = 2^{1-n}\varepsilon\), then there exists a sequence of radii \(\family{\rho_n}{n\in \mathbb{N}}\) such that \(Y_{\rho_n} \subset \cl_Y(T(X_{r_n}))\) for all \(n \in \mathbb{N}\) and \(\rho_n \to 0\). Let \(y \in Y_{\rho_1}\), then there exists \(x_1 \in X_{r_1}\) such that \(Tx_1 \in T(X_{r_1}) \cap (Y_{\rho_2} + y)\), since \(Y_{\rho_2} + y\) is manifestly a neighborhood of \(y\). As a result, \(y_1 = y - Tx_1 \in Y_{\rho_2}\) and we may repeat this procedure for \(y_1\). More precisely, we define recursively for \(n > 1\) as follows: since \(y_{n-1} \in Y_{\rho_n} \subset \cl_Y(T(X_{r_n}))\), there exists \(x_n \in X_{r_n}\) such that \(Tx_n \in Y_{\rho_{n+1}} + y_{n-1}\), hence we set \(y_n = y_{n-1} - Tx_n \in Y_{\rho_{n+1}}\).

    By construction, this yields a sequence \(\family{x_n}{n \in \mathbb{N}}\) such that \(x_n \in X_{r_n}\) and
    \begin{equation*}
        y - \sum_{k=1}^{n} Tx_k = y - T\left(\sum_{k=1}^{n} x_k\right)\in Y_{\rho_{n+1}}
    \end{equation*}
    for all \(n \in \mathbb{N}\). The sequence \(\family{\sum_{k = 1}^n x_k}{n\in \mathbb{N}}\) is Cauchy since for all \(\eta > 0\) we have for all \(n,m \in \mathbb{N}\) with \(\log_2\left(\frac{2 \varepsilon}{\eta}\right) < n < m\) that
    \begin{align*}
        \norm*{\sum_{k=1}^m x_k - \sum_{k=1}^n x_k} = \norm*{\sum_{k=n+1}^{m} x_k} &\leq \sum_{k=n+1}^m \norm{x_k}\\
                                                                                   &\leq \sum_{k=n+1}^{m} 2^{1-k}\varepsilon = 2^{-n} \varepsilon\sum_{k=0}^{m-n-1} 2^{-k}\\
                                                                                   &< 2^{-n} \varepsilon \sum_{k=0}^\infty 2^{-k} = 2^{1-n} \varepsilon\\
                                                                                   &< \eta.
    \end{align*}
    Since \(X\) is complete, there exists \(\tilde{x} \in X\) against which this Cauchy sequence converges and we have \(\tilde{x} \in X_{2 \varepsilon}\) since
    \begin{equation*}
        \norm{\tilde{x}} = \lim_{n\to\infty} \norm*{\sum_{k=1}^n x_k} \leq \lim_{n\to\infty} \sum_{k=1}^n \norm{x_k} < \varepsilon \lim_{n\to\infty} \sum_{k=0}^{n-1} 2^{-k} = 2 \varepsilon.
    \end{equation*}
    As \(\rho_n \to 0\), we have \(y = T\tilde{x}\) by continuity. We have thus shown \(T\) maps \(X_{2 \varepsilon}\) surjectively to a set containing \(Y_{\rho_1}\).

    Let \(U \in \tau_X\) be an open set. If \(U = \emptyset\), it is clear \(T(U) = \emptyset \in \tau_Y\), so we may assume \(U \neq \emptyset\). Let \(x \in U\) and let \(W \in \tau_X\) be an open ball centered at \(0\) such that \(x + W \subset U\). By the previous result, \(W\) is mapped surjectively to a subset of \(Y\) containing an open ball centered at \(0\), therefore there exists an open set \(V \in \tau_Y\) such that \(V \subset T(W)\). We consider the translation \(V + Tx\), then
    \begin{equation*}
        V + Tx \subset T(W) + Tx = T(W + x) \subset T(U).
    \end{equation*}
    That is, there exists a neighborhood of \(Tx\) contained in \(T(U)\), hence \(Tx\) is an interior point of \(T(U)\), thus showing \(T(U) \in \tau_Y\).
\end{proof}

The immediate corollary to the open mapping theorem is that every bijective bounded operator defined in a Banach space is a homeomorphism.
\begin{theorem}{Bounded inverse theorem}{bounded_inverse_theorem}
    Let \(X, Y\) be Banach spaces. If \(T \in \bounded(X, Y)\) is bijective, then \(T^{-1} \in \bounded(Y, X)\).
\end{theorem}
\begin{proof}
    Since \(T\) is bijective and continuous, \(T\) is an open mapping by \cref{thm:open_mapping}, hence \(T\) is a homeomorphism. By \cref{thm:homeomorphism_inverse}, we have \(T^{-1}\) continuous. It is clear \(T^{-1}\) is linear, hence \(T^{-1} \in \bounded(Y, X)\). Indeed, let \(y_1,y_2 \in Y\), then there exist \(x_1 = T^{-1}(y_1)\) and \(x_2 = T^{-1}(y_2)\). We thus have
    \begin{equation*}
        T^{-1}(y_1 + \alpha y_2) = T^{-1}(T(x_1) + \alpha T(x_2)) = T^{-1}\circ T \left(x_1 + \alpha x_2\right) = x_1 + \alpha x_2 = T^{-1}(y_1) + \alpha T^{-1}(y_2)
    \end{equation*}
    for all \(\alpha \in \mathbb{C}\).
\end{proof}

% vim: spl=en_us
\section{Closed graph theorem}
We recall the definition of a map and its graph.
\begin{definition}{Map and graph}{graph}
    Let \(S\) be a set and let \(T\) be a non-empty set. A \emph{map} \(f\) is a triple \((S, T, \graph{f})\), where the \emph{graph \(\graph{f} \subset S \times T\) of \(f\)} satisfies
    \begin{enumerate}[label=(\alph*)]
        \item for all \(s \in S\) and for all \(t_1, t_2 \in T\), if \((s, t_1) \in \graph{f}\) and \((s, t_2) \in \graph{f}\), then \(t_1 = t_2\); and
        \item for all \(s \in S\), there exists \(t \in T\) such that \((s, t) \in \graph{f}\).
    \end{enumerate}
    By abuse of notation we write \((s, t) \in f\) and, in view of (a), denote this by \(f(s) = t\). For this reason, by abuse of notation we denote the map by \(f : S \to T\).
\end{definition}
Note that given non-empty sets \(S, T\) and a non-empty collection \(G \subset S \times T\) satisfying the properties of a graph, then there exists a map \(f\) with graph \(G\) such that the diagram
\begin{equation*}
    \begin{tikzcd}[column sep = large, row sep = large]
        G \arrow{rd}{\pi_2} \arrow[swap]{d}{\pi_1} &\\
        S \arrow[swap]{r}{f} & T
    \end{tikzcd}
\end{equation*}
commutes, where \(\pi_1 : S \times T \to S\) and \(\pi_2 : S\times T \to T\) are the coordinate projections, that is, \(f \circ \pi_1 = \pi_2\).

An important consequence of the open mapping theorem is the closed graph theorem, which relates the continuity of a map \(T : X \to Y\) with its graph \(\graph{T}\subset X \times Y\) and a topology on \(X \times Y\).

\begin{proposition}{Topological direct sum}{direct_sum}
    Let \((X, \norm{\noarg}_X), (Y, \norm{\noarg}_Y)\) be normed linear spaces. The topological direct sum \(X \oplus Y\) is the normed linear space \((X \times Y, \norm{\noarg}_{X \times Y})\), where addition, scalar multiplication, and norm are defined by
    \begin{align*}
        +_{X \times Y} : (X \times Y) \times (X \times Y) &\to X \times Y\\
        \left((x_1,y_1),(x_2,y_2)\right)&\mapsto (x_1 +_X x_2, y_1 +_Y y_2),
    \end{align*}
    \begin{align*}
        \cdot_{X \times Y} : \mathbb{C} \times (X \times Y) &\to X \times Y\\
        \left(\alpha,(x,y)\right)&\mapsto (\alpha \cdot_X x, \alpha \cdot_Y y),
    \end{align*}
    and
    \begin{align*}
        \norm{\noarg}_{X \times Y} : (X \times Y) &\to \mathbb{R}\\
        (x,y)&\mapsto \norm{x}_X + \norm{y}_Y.
    \end{align*}
    Moreover, if \((X, \norm{\noarg}_X)\) and \((Y, \norm{\noarg}_Y)\) are Banach spaces, then so is \(X \oplus Y\).
\end{proposition}
\begin{proof}
    We first show \(X \oplus Y\) is a normed linear space. It is trivial to verify the linear space axioms are satisfied since \(X\) and \(Y\) are linear spaces, with the main insight being \((0,0) \in X\times Y\) is unsurprisingly the neutral element of addition. Let us verify \(\norm{\noarg}_{X\ \times Y}\) is indeed a norm. Since the norms on \(X\) and \(Y\) are non-negative, it is clear that \(\norm{\noarg}_{X\times Y}\) is non-negative. It is also positive-definite since
    \begin{align*}
        \norm{(x, y)}_{X \times Y} = 0 &\iff \norm{x}_X + \norm{y}_{Y} = 0\\
                                       &\iff \norm{x}_X = 0 \land \norm{y}_Y = 0\\
                                       &\iff x = 0 \land y = 0\\
                                       &\iff (x,y) = (0,0).
    \end{align*}
    Absolute homogeneity follows from the absolute homogeneity of the norms on \(X\) and \(Y\) as
    \begin{align*}
        \norm{\lambda (x, y)}_{X \times Y} = \norm{\lambda x}_X + \norm{\lambda y}_Y = \abs{\lambda} \left(\norm{x}_X + \norm{y}_Y\right) = \abs{\lambda} \norm{(x,y)}_{X \times Y}
    \end{align*}
    holds for all \(\lambda \in \mathbb{C}\) and all \((x,y) \in X \times Y\). Finally, let \((x_1, y_1), (x_2, y_2) \in X\times Y\), then
    \begin{align*}
        \norm{(x_1, y_1) + (x_2, y_2)}_{X \times Y} &= \norm{x_1 + x_2}_X + \norm{y_1 + y_2}_Y\\ &\leq \left(\norm{x_1}_X + \norm{y_1}_Y\right) + \left(\norm{x_2}_X + \norm{y_2}_Y\right) \\&= \norm{(x_1, y_1)}_{X \times Y} + \norm{(x_2, y_2)}_{X\times Y},
    \end{align*}
    hence \(\norm{\noarg}_{X\times Y}\) is subadditive, and we conclude \(X \oplus Y\) is indeed a normed linear space.

    Suppose \(X\) and \(Y\) are Banach spaces and let \(\family{v_n}{n \in \mathbb{N}} \subset X \oplus Y\) be a Cauchy sequence in \(X \oplus Y\). For each \(n \in \mathbb{N}\), there exists \(x_n \in X\) and \(y_n \in Y\) such that \(v_n = (x_n, y_n)\). Let \(\varepsilon > 0\), then there exists \(N \in \mathbb{N}\) such that for all \(n, m \geq N\) we have \(\norm{v_n - v_m}_{X \times Y} < \varepsilon\), hence \(\norm{x_n - x_m}_X + \norm{y_n - y_m}_Y < \varepsilon\). Since the norms are non-negative, we have \(\norm{x_n - x_m}_X < \varepsilon\) and \(\norm{y_n - y_m} < \varepsilon\), thus showing \(\family{x_n}{n \in \mathbb{N}} \subset X\) and \(\family{y_n}{n \in \mathbb{N}}\) are Cauchy sequences. By completeness, there exists \(\tilde{x} \in X\) and \(\tilde{y} \in Y\) such that \(x_n \to \tilde{x}\) and \(y_n \to \tilde{y}\).

    We consider \(\tilde{v} = (\tilde{x}, \tilde{y}) \in X \oplus Y\) and let \(\eta > 0\). By convergence, there exists \(M_X, M_Y \in \mathbb{N}\) such that \(\norm{\tilde{x} - x_n}_X < \frac12 \eta\) for all \(n > M_X\) and such that \(\norm{\tilde{y} - y_n}_Y < \frac12 \eta\) for all \(n > M_Y\). Then, we set \(M = \max\set{M_X, M_Y}\) such that for all \(n > M\) we have
    \begin{equation*}
        \norm{\tilde{v} - v_n}_{X\times Y} = \norm{\tilde{x} - x_n}_X + \norm{\tilde{y} - y_n}_Y < \eta,
    \end{equation*}
    that is, \(v_n \to \tilde{v}\), hence \(X \oplus Y\) is complete.
\end{proof}

\begin{proposition}{Graph of a linear operator is a linear subspace}{graph_linear_subspace}
    Let \(X, Y\) be normed linear spaces. If \(T : \domain{T} \subset X \to Y\) is a linear operator, then \(\graph{T}\) is a linear subspace of \(X \oplus Y\).
\end{proposition}
\begin{proof}
    Since \(\domain{T}\) is a linear subspace of \(X\), then the graph \(\graph{T}\) is a linear subspace of \(X \oplus Y\). Indeed, let \(v_1, v_2 \in \graph{T}\), then there exists \(x_1, x_2 \in \domain{T}\) such that \(v_1 = (x_1, Tx_1)\) and \(v_2 = (x_2, Tx_2)\), hence for all \(\alpha \in \mathbb{C}\) we have \(v_1 + \alpha v_2 = (x_1 + \alpha x_2, T(x_1 + \alpha x_2)) \in \graph{T}\).
\end{proof}
\begin{proposition}{Coordinate projections are bounded operators}{coordinate_projection_bounded}
    Let \(X_1, X_2\) be normed linear spaces. The coordinate projections \(\pi_i : X_1 \oplus X_2 \to X_i\) are bounded linear operators, with \(i \in \set{1,2}.\)
\end{proposition}
\begin{proof}
    Notice that for all \(w \in X_1 \oplus X_2\), we have \(w = (\pi_1(w), \pi_2(w))\) by the definitions of the topological direct sum and the coordinate projections. Let \(u, v \in X_1 \oplus X_2\) and \(\alpha, \beta \in \mathbb{C}\), then
    \begin{equation*}
        \pi_i(\alpha u + \beta v) = \pi_i\left(\alpha \pi_1(u) + \beta \pi_1(v), \alpha \pi_2(u) + \beta \pi_2(v)\right) = \alpha \pi_i(u) + \beta \pi_i(v),
    \end{equation*}
    hence \(\pi_i\) is linear, with \(i \in \set{1,2}\). Moreover, for all \(w \in X_1 \oplus X_2\) we have
    \begin{equation*}
        \norm{\pi_iw}_{X_i} \leq \norm{\pi_1 w}_{X_1} + \norm{\pi_2 w}_{X_2} = \norm{w}_{X_1\oplus X_2},
    \end{equation*}
    hence \(\pi_i \in \bounded(X_1 \oplus X_2, X_i)\).
\end{proof}
\begin{proposition}{Coordinate projections restricted to a graph}{coordinate_projection_graph}
    Let \(X_1, X_2\) be normed linear spaces and let \(T : X_1 \to X_2\) be a linear operator. Then the restricted coordinate projection \(\restrict{\pi_i}{\graph{T}} : X_1 \oplus X_2 \to X_i\) is a bounded linear operator for \(i \in \set{1,2}\). In addition, \(\restrict{\pi_1}{\graph{T}}\) is bijective.
\end{proposition}
\begin{proof}
    By the previous result, we conclude \(\restrict{\pi_i}{\graph{T}}\) is linear. Moreover, \(\pi_i\) clearly extends \(\restrict{\pi_i}{\graph{T}}\), then we have
    \begin{equation*}
        \sup_{w \in \graph{T}} \frac{\norm{\restrict{\pi_i}{\graph{T}}w}_{X_i}}{\norm{w}_{\graph{T}}} \leq \sup_{w \in X_1 \oplus X_2} \frac{\norm{\pi_i w}_{X_i}}{\norm{x}_{X_1 \oplus X_2}} < \infty,
    \end{equation*}
    hence \(\restrict{\pi_i}{\graph{T}} \in \bounded(\graph{T}, X_i)\). By properties (a) and (b) of \cref{def:graph}, it follows that \(\restrict{\pi_1}{\graph{T}}\) is bijective. Indeed, (b) states that \(\restrict{\pi_1}{\graph{T}}(\domain{T}) = \domain{T}\) and (a) states that \(\restrict{\pi_1}{\graph{T}}(w) = \restrict{\pi_1}{\graph{T}}(v)\) implies \(w = v\).
\end{proof}

We may describe a linear operator \(T : \domain{T} \subset X \to Y\) in terms of the standing of its graph \(\graph{T}\) in the metric topology on \(X \oplus Y\).
\begin{definition}{Closed linear operator}{closed_linear_operator}
    Let \(X, Y\) be normed linear spaces. A linear operator \(T : \domain{T} \subset X \to Y\) is \emph{closed} if its graph \(\graph{T} \subset \domain{T} \times Y\)  is a closed linear subspace of \(X \oplus Y\).
\end{definition}


Recall \cref{exam:derivative_unbounded}, where we presented the derivative operator as an unbounded linear operator on the Banach space of continuous complex-valued functions. We now show this operator is also closed.
\begin{example}{The derivative operator of complex-valued functions is closed}{derivative_closed}
    Let \(Y = \mathcal{C}([0,1]; \mathbb{C})\) be the Banach space of continuous complex-valued functions equipped with the sup norm and let the set \(X = \mathcal{C}^1([0,1];\mathbb{C})\subset Y\) of continuously differentiable complex-valued functions in the interval \([0,1]\). The derivative operator
    \begin{align*}
        P : X \subset Y &\to Y\\
            f &\mapsto f'
    \end{align*}
    is closed.
\end{example}
\begin{proof}
    Let \(\family{v_n}{n \in \mathbb{N}} \subset \graph{T}\) be a convergent sequence with \(v_n \to (\tilde{x}, \tilde{y}) \in Y \oplus Y\). Then, there exists convergent sequences \(\family{x_n}{n \in \mathbb{N}} \subset X\) and \(\family{Tx_n}{n \in \mathbb{N}} \subset Y\) such that we have \(v_n = (x_n, Tx_n)\) for all \(n \in \mathbb{N}\) and such that \(x_n \to \tilde{x}\) and \(Tx_n = x'_n \to \tilde{y}\). Notice \(x'_n\) converges uniformly to \(\tilde{y}\), then \(\tilde{x}\) must be differentiable with \(\tilde{x}' = \tilde{y}\). That is, \((\tilde{x}, \tilde{y}) \in \graph{T}\), thus showing \(\graph{T}\) is closed in \(Y \oplus Y\).
\end{proof}

\begin{theorem}{Closed graph theorem}{closed_graph}
    Let \(X, Y\) be Banach spaces. A linear map \(T : X \to Y\) is continuous if and only if it is closed.
\end{theorem}
\begin{proof}
    Suppose \(T\) is continuous and let \(\family{v_n}{n \in \mathbb{N}}\subset \graph{T}\) be a sequence that converges to \(\tilde{v} \in X \oplus Y\). Then there exist convergent sequences \(\family{x_n}{n \in \mathbb{N}} \subset X\) and \(\family{Tx_n}{n \in \mathbb{N}} \subset Y\), such that \(v_n = (x_n, Tx_n)\) for all \(n \in \mathbb{N}\) and such that \(x_n \to \tilde{x}\) and \(Tx_n \to \tilde{y}\), where \(\tilde{v} = (\tilde{x}, \tilde{y})\). Since \(T\) is continuous, we have
    \begin{equation*}
        \tilde{y} = \lim_{n\to\infty} Tx_n = T\left(\lim_{n\to\infty} x_n\right)= T\tilde{x},
    \end{equation*}
    hence \(\tilde{v} \in \graph{T}\). That is, \(\graph{T}\) is closed in \(X \oplus Y\).

    Suppose \(T\) is closed. Then \(\graph{T}\) is a closed subset of the Banach space \(X \oplus Y\), hence \(\graph{T}\) is a Banach space. Let \(S_1 = \restrict{\pi_1}{\graph{T}}\) and \(S_2 = \restrict{\pi_2}{\graph{T}}\) be the restricted coordinate projections considered in \cref{prop:coordinate_projection_graph}, then \(S_1 \in \bounded(\graph{T}, X)\) is a homeomorphism, and \(S_2 \in \bounded(\graph{T}, Y)\) is a continuous map. Then, \(T = S_2 \circ S_1^{-1}\) is a composition of continuous linear operators, hence \(T \in \bounded(X, Y).\)
\end{proof}

% hilbert spaces
% vim: spl=en_us
\chapter{Hilbert spaces}
An \emph{inner product space} \((V, \inner{\noarg}{\noarg})\) is a linear space \(V\) equipped with a distinguished \emph{sesquilinear form} \(\inner{\noarg}{\noarg} : V \times V \to \mathbb{C}\), called the \emph{inner product}, satisfying
\begin{enumerate}[label=(\alph*)]
    \item conjugate symmetry: \(\inner{x}{y} = \conj{\inner{y}{x}}\), for all \(x, y \in V\);
    \item positive-definiteness: \(\inner{x}{x} \geq 0\) for all \(x \in X\) and \(\inner{x}{x} = 0 \iff x = 0\);
    \item linearity in the second argument\footnote{In Mathematics literature, it is usually required linearity \emph{not} in the second argument, but the first one. We stick to the Physics literature convention.}: \(\inner{z}{\alpha x + \beta y} = \alpha \inner{z}{x} + \beta \inner{z}{y}\), for all \(x, y, z \in V\) and all \(\alpha, \beta \in \mathbb{C}\).
\end{enumerate}
From conjugate symmetry and linearity in the second argument, it follows that the inner product is conjugate linear in the second argument,
\begin{equation*}
    \inner{\alpha x + \beta y}{z} = \conj{\inner{z}{\alpha x + \beta z}} = \conj{\alpha\inner{z}{x}} + \conj{\beta\inner{z}{y}} = \conj{\alpha} \inner{x}{z} + \conj{\beta} \inner{y}{z},
\end{equation*}
for all \(x,y,z \in V\) and all \(\alpha, \beta \in \mathbb{C}\).

\begin{proposition}{Positive-definiteness implies non-degeneracy}{non_degeneracy}
    Let \((V, \inner{\noarg}{\noarg})\) be an inner product space. Then the inner product is non-degenerate, that is, \(\inner{x}{y} = 0\) for all \(y \in V\) if and only if \(x = 0\).
\end{proposition}
\begin{proof}
    Suppose \(\inner{x}{y} = 0\) for all \(y \in V\). In particular, \(\inner{x}{x} = 0\), then \(x = 0\) by positive definiteness.

    Suppose \(x = 0\), then for all \(y \in Y\), we have
    \begin{equation*}
        \inner{0}{y} = \inner{y-y}{y} = \inner{y}{y} - \inner{y}{y} = 0.
    \end{equation*}
    That is, \(\inner{x}{y} = 0\) for all \(y \in Y\).
\end{proof}
\begin{corollary}
    Let \((W\) be a linear space. If \(A : W \to V\) and \(B : W \to V\) are linear maps such that \(\inner{v}{Aw} = \inner{v}{Bw}\) for all \(v \in V\) and all \(w \in W\), then \(A = B\).
\end{corollary}
\begin{proof}
    For each \(w \in W\), we have \(\inner{v}{(A - B)w} = 0\) for all \(v \in V\). Since the inner product is non-degenerate, it follows that \((A-B)w = 0\). This holds for all \(w \in W,\) hence \(A = B\).
\end{proof}

In Analysis, inequalities are often very important to estimate certain quantities, as we've used many times when studying properties of bounded operators in normed linear spaces. It turns out the inner product properties are enough to define of the most important inequalities in Mathematics.
\begin{theorem}{Cauchy-Schwarz inequality}{cauchy_schwarz}
    Let \((V, \inner{\noarg}{\noarg})\) be an inner product space. Then,
    \begin{equation*}
        \abs*{\inner{x}{y}}^2 \leq \inner{x}{x} \inner{y}{y},
    \end{equation*}
    for all \(x, y \in V\).
\end{theorem}
\begin{proof}
    The inequality is trivially satisfied if at least one of the vectors is the zero vector. We may assume \(x, y \in V\setminus\set{0}\) and consider the inequality \(\inner*{\alpha x + y}{\alpha x + y} \geq 0\), which follows from the positive-definiteness of the sesquilinear form for all \(\alpha \in \mathbb{C}\). By expanding the inner product with sesquilinearity, we get
    \begin{align*}
        \inner{\alpha x + y}{\alpha x + y} &= \alpha\inner{\alpha x + y}{x} + \inner*{\alpha x + y}{y}\\
                                           &= \alpha \left(\conj{\alpha} \inner{x}{x} + \inner{y}{x}\right) + \conj{\alpha}\inner{x}{y} + \inner{y}{y}\\
                                           &= \alpha\conj{\alpha}\inner{x}{x} + \alpha \inner{y}{x} + \conj{\alpha}\inner{x}{y} + \inner{y}{y}\\
                                           &= \abs{\alpha}^2 \inner{x}{x} + 2\Re\left(\conj{\alpha}\inner{x}{y}\right) + \inner{y}{y}.
    \end{align*}
    We set \(\alpha = - \frac{\inner{x}{y}}{\inner{x}{x}}\), then from conjugate symmetry we have
    \begin{equation*}
        \inner{\alpha x + y}{\alpha x + y} = \inner{y}{y} - \frac{\abs*{\inner{x}{y}}^2}{\inner{x}{x}} \geq 0,
    \end{equation*}
    that is, \(\abs*{\inner{x}{y}}^2 \leq \inner{x}{x} \inner{y}{y}\).
\end{proof}

There is a necessary and sufficient condition for the equality in the Cauchy-Schwarz inequality. To state it, we first recall the definition of linear independence.
\begin{definition}{Linearly independent subset}{linear_independent}
    Let \(V\) be a linear space. A non-empty subset \(F \subset V\) is \emph{linearly independent} if for every finite subset \(\ffamily{v_i}{i=1}{N} \subset F\) the only resulting linear combination that results in the zero vector is the trivial linear combination, that is,
    \begin{equation*}
        \sum_{i = 1}^N \alpha_i v_i = 0 \implies \alpha_i = 0, i \in \set{1,2,\dots,N}.
    \end{equation*}
    If a set is not linearly independent, we say it is \emph{linearly dependent}.
\end{definition}
\begin{remark}
    In particular, any finite set of vectors \(\ffamily{v_i}{i=1}{N}\subset V\) is linearly dependent if there exist \(\ffamily{\alpha_i}{i=1}{N} \in \mathbb{C}\) with at least one non-zero such that \(\sum_{i=1}^N \alpha_iv_i = 0\).
\end{remark}

\begin{proposition}{Equality in the Cauchy-Schwarz inequality}{cauchy_schwarz_equality}
    Let \((V, \inner{\noarg}{\noarg})\) be an inner product space. The equality \(\abs*{\inner{x}{y}}^2 = \inner{x}{x}\inner{y}{y}\) holds if and only if the set \(\set{x,y} \subset V\) is linearly dependent.
\end{proposition}
\begin{proof}
    Suppose \(\set{x,y} \subset V\) is linearly dependent. If either vector is the zero vector, then the equality holds trivially. We may assume \(x, y \in V\setminus\set{0}\), then there exists \(\alpha \in \mathbb{C}\setminus\set{0}\) such that \(\alpha x = y\). We have
    \begin{equation*}
        \abs*{\inner{x}{y}}^2 = \inner{x}{y}\inner{y}{x} = \inner{x}{\alpha x}\inner*{y}{\frac1\alpha y} = \inner{x}{x} \inner{y}{y},
    \end{equation*}
    that is, the equality holds.

    Suppose \(\abs*{\inner{x}{y}} = \inner{x}{x} \inner{y}{y}\). Again, if either vector is the zero vector, then \(\set{x,y}\) is trivially linearly dependent, then we may assume \(x, y \in V \setminus\set{0}\). We've seen that
    \begin{equation*}
        \inner*{y - \frac{\inner{x}{y}}{\inner{x}{x}}x}{y - \frac{\inner{x}{y}}{\inner{x}{x}}x} = \inner{y}{y} - \frac{\abs*{\inner{x}{y}}}{\inner{x}{x}}.
    \end{equation*}
    By hypothesis the right hand side vanishes, then by positive-definiteness we have \(y - \frac{\inner{x}{y}}{\inner{x}{x}}x = 0\), hence \(\set{x,y}\) is linearly dependent.
\end{proof}

We verify every inner product defines a norm on a linear space, thus every inner product space is a normed linear space. As it was the case with norms and metrics, the following construction is not the only way to induce a norm from an inner product, but we will refer to the map here defined as a \emph{norm induced by an inner product}.
\begin{proposition}{Inner product induced norm}{inner_product_norm}
    Let \((V, \inner{\noarg}{\noarg})\) be an inner product space. The map
    \begin{align*}
        \norm{\noarg} : V &\to \mathbb{R}\\
                        x &\mapsto \sqrt{\inner{x}{x}}
    \end{align*}
    defines a norm on \(V\).
\end{proposition}
\begin{proof}
    Non-negativity and positive-definiteness of the map follow from the positive-definiteness of the inner product and the fact that the square root is a monotonically increasing function, that is, it is \emph{order preserving}. Absolute homogeneity follows trivially from the sesquilinearity.

    It remains to show subadditivity. Let \(x,y \in V\), then
    \begin{equation*}
        \norm{x + y}^2 = \norm{x}^2 + 2\Re(\inner{x}{y}) + \norm{y}^2.
    \end{equation*}
    From \(\Re(z) \leq \abs{z}\) for all \(z \in \mathbb{C}\), we may estimate
    \begin{equation*}
        \norm{x + y}^2 \leq \norm{x}^2 + 2\abs*{\inner{x}{y}} + \norm{y}^2.
    \end{equation*}
    Using the Cauchy-Schwarz inequality yields
    \begin{equation*}
        \norm{x+y}^2 \leq (\norm{x} + \norm{y})^2 \implies \norm{x+y} \leq \norm{x} + \norm{y},
    \end{equation*}
    thus showing subadditivity.
\end{proof}
\begin{remark}
    The Cauchy-Schwarz inequality may be expressed in terms of the induced norm:
    \begin{equation*}
        \abs*{\inner{x}{y}} \leq \norm{x} \norm{y},
    \end{equation*}
    for all \(x, y \in V\), with equality holding if and only if \(\set{x,y}\) is linearly independent.
\end{remark}

Since there is a norm on every inner product space, they have an induced metric space structure. Thus, we may study the continuity of the inner product.
\begin{proposition}{Inner product is continuous}{inner_product_continuous}
    Let \((V, \inner{\noarg}{\noarg})\) be an inner product space. Let \(\family{x_n}{n\in \mathbb{N}}, \family{y_n}{n\in \mathbb{N}} \subset V\) be sequences that converge to \(x, y \in V\) with respect to \((V, \inner{\noarg}{\noarg})\), then
    \begin{equation*}
        \lim_{n\to\infty} \inner{x_n}{y_n} = \inner{x}{y}.
    \end{equation*}
\end{proposition}
\begin{proof}
    Notice that
    \begin{equation*}
        \inner{x_n - x}{y_n} + \inner{x}{y_n -y} = \inner{x_n}{y_n} - \inner{x}{y}
    \end{equation*}
    for all \(n \in \mathbb{N}\). Recall that every convergent sequence in a normed linear space is bounded. Let \(M > 0\) such that \(\norm{y_n} \leq M\) for all \(n \in \mathbb{N}\). Then the triangle inequality and the Cauchy-Schwarz yields
    \begin{align*}
        \abs*{\inner{x_n}{y_n} - \inner{x}{y}} &\leq \abs*{\inner{x_n - x}{y_n} + \abs*{\inner{x}{y_n - y}}}\\
                                               &\leq \norm{x_n - x}\cdot \norm{y_n} + \norm{x} \cdot\norm{y_n - y}\\
                                               &\leq M\norm{x_n - x} + \norm{x} \cdot \norm{y_n - y},
    \end{align*}
    for all \(n \in \mathbb{N}\).

    Let \(\varepsilon > 0\). From convergence of the sequences in \(V\), there exists \(N_1, N_2 \in \mathbb{N}\) such that \(n \geq N_1 \implies M \norm{x_n - x} < \frac13 \varepsilon\) and \(n \geq N_2 \implies \norm{x} \cdot \norm{y_n- y} < \frac13 \varepsilon\). We have thus
    \begin{equation*}
        n \geq \max\set{N_1, N_2} \implies \abs*{\inner{x_n}{y_n} - \inner{x}{y}} < \varepsilon,
    \end{equation*}
    hence \(\inner{x_n}{y_n}\to \inner{x}{y}\).
\end{proof}

Analogously for normed linear spaces, we borrow the metric space terminology for inner product spaces, keeping in mind the metric is the one induced by the inner product.
\begin{definition}{Hilbert space}{hilbert_space}
    A Hilbert space \((\hilbert, \inner{\noarg}{\noarg})\) is a complete inner product space.
\end{definition}
It should now be obvious that a Hilbert space is a Banach space with respect to the norm induced by its inner product. Then results for Banach spaces, such as the continuity of a linear map, completeness of the Banach space of bounded operators, and the BLT theorem, all follow for Hilbert spaces.

% vim: spl=en_us
\section{Pre-Hilbert spaces and the parallelogram identity}
It is not the case that every normed linear space is an inner product space. It was in this sense that we meant that Banach spaces are a generalization of Hilbert spaces. We will now show the equivalent of inner product spaces and a subset of normed linear spaces.
\begin{definition}{Pre-Hilbert space}{pre_hilbert}
    A \emph{pre-Hilbert space} is a normed linear space \((V, \norm{\noarg})\) where the norm satisfies the \emph{parallelogram identity}, that is,
    \begin{equation*}
        \norm{x + y}^2 + \norm{x - y}^2 = 2\left(\norm{x}^2 + \norm{y}^2\right)
    \end{equation*}
    for all \(x, y \in V\).
\end{definition}

Let us verify every inner product space is a pre-Hilbert space.
\begin{proposition}{Inner product space is a pre-Hilbert space}{pre_hilbert}
    Let \((V, \inner{\noarg}{\noarg})\) be a inner product space and let \(\norm{\noarg}\) be the norm induced by the inner product on \(V\). Then, \((V, \norm{\noarg})\) is a pre-Hilbert space and the inner product satisfies the \emph{polarization identity},
    \begin{equation*}
        \inner{x}{y} = \frac14\left(\norm{x + y}^2 + i\norm{x - iy}^2 - \norm{x - y}^2 - i\norm{x + iy}^2\right)
    \end{equation*}
    for all \(x, y \in V\).
\end{proposition}
\begin{proof}
    Notice
    \begin{align*}
        \norm{x + y}^2 + \norm{x - y}^2 &= \inner{x+y}{x+y} + \inner{x - y}{x - y}\\
                                        &= \left(\norm{x}^2 + 2\Re(\inner{x}{y}) + \norm{y}^2\right) + \left(\norm{x}^2 + 2\Re(\inner{x}{y}) - \norm{y}^2\right) \\
                                        &= 2\left(\norm{x}^2 + \norm{y}^2\right)
    \end{align*}
    holds for all \(x,y \in V\). Hence, \((V, \norm{\noarg})\) is a pre-Hilbert space.

    Recall that \(\norm{x + \alpha y}^2 = \norm{x}^2 + 2\Re(\alpha \inner{x}{y}) + \abs{\alpha}^2 \norm{y}^2\) for all \(\alpha \in \mathbb{C}\) and all \(x,y \in V\). Then
    \begin{equation*}
        \norm{x + \alpha y}^2 - \norm{x - \alpha y}^2 = 4 \Re(\alpha \inner{x}{y}).
    \end{equation*}
    Using \(\alpha = 1\) and \(\alpha = -i\) yields
    \begin{equation*}
        4\Re(\inner{x}{y}) = \norm{x + y}^2 - \norm{x - y}^2\quad\text{and}\quad 4\Re(-i\inner{x}{y}) = \norm{x - iy}^2 - \norm{x + iy}^2.
    \end{equation*}
    Since for all \(z \in \mathbb{C}\) we have \(\Im(z) = \Re(-iz)\), we have
    \begin{equation*}
        \inner{x}{y} = \Re(\inner{x}{y}) + i \Re(-i\inner{x}{y}) = \frac14\left(\norm{x + y}^2 + i \norm{x - iy}^2 - \norm{x - y}^2 - i \norm{x + iy}^2\right),
    \end{equation*}
    as claimed.
\end{proof}

We now show that the polarization identity defines an inner product in a pre-Hilbert space, and that it is the inner product that induces the norm.
\begin{theorem}{Fréchet-von Neumann-Jordan theorem}{parallelogram}
    Let \((V, \norm{\noarg})\) be a pre-Hilbert space. The map given by the \emph{polarization identity}
    \begin{align*}
        \inner{\noarg}{\noarg} : V \times V &\to \mathbb{C}\\
                                      (x,y) &\mapsto \frac14\left(\norm{x + y}^2 + i \norm{x - iy}^2 - \norm{x - y}^2 - i \norm{x + iy}^2\right)
    \end{align*}
    defines an inner product in \(V\). Moreover, the norm is induced by this inner product.
\end{theorem}
\begin{proof}
    We first show conjugate symmetry and positive-definiteness. Notice that \(\norm*{e^{i\theta}x} = \norm{x}\) for all \(x \in V\) and \(\theta \in \mathbb{R}\). For all \(x, y \in V\) it follows that
    \begin{align*}
        \conj{\inner{x}{y}} &= \frac14\conj{\left(\norm{x + y}^2 + i \norm{x - iy}^2 - \norm{x - y}^2 - i \norm{x + iy}^2\right)}\\
                            &= \frac14\left(\norm{x + y}^2 - i \norm{x - iy}^2 - \norm{x - y}^2 + i \norm{x + iy}^2\right)\\
                            &= \frac14\left(\norm{y + x}^2 - i \norm{ix + y}^2 - \norm{-x + y}^2 + i \norm{-ix + y}^2\right)\\
                            &= \inner{y}{x},
    \end{align*}
    hence \(\inner{\noarg}{\noarg}\) is conjugate symmetric. For all \(x \in V\) we have
    \begin{align*}
        \inner{x}{x} &= \frac14\left(\norm{x + x}^2 + i \norm{x - ix}^2 - \norm{x - x}^2 - i \norm{x + ix}^2\right)\\
                     &= \frac14\left(\norm{2x}^2 + i \norm*{\sqrt{2}e^{-i\frac{\pi}{4}}x}^2 - i \norm*{\sqrt{2}e^{i\frac{\pi}{4}}x}^2\right)\\
                     &= \frac14\left(4\norm{x}^2 + 2i \norm{x}^2 - 2i\norm{x}^2\right)\\
                     &= \norm{x}^2.
    \end{align*}
    Thus, \(\inner{\noarg}{\noarg}\) is positive-definite. Moreover, when \(\inner{\noarg}{\noarg}\) is shown to be an inner product on \(V\), we have shown that \(\sqrt{\inner{x}{x}} = \norm{x}\) for all \(x \in V\), thus the inner product induces the norm.

    We consider the family of maps
    \begin{align*}
        F_{\alpha} : V \times V &\to \mathbb{R}\\
        (u,v) &\mapsto \frac{\norm*{u + \alpha v}^2 - \norm{u - \alpha v}^2}{4},
    \end{align*}
    with \(\alpha \in \mathbb{C}\), and notice that
    \begin{equation*}
        \inner{u}{v} = F_{1}(u,v) + iF_{-i}(u,v)\quad\text{and}\quad\inner{u}{\lambda v} = F_{\lambda}(u, v) + i F_{-i \lambda}(u, v),
    \end{equation*}
    for all \(u, v \in V\) and \(\lambda \in \mathbb{C}\). Let us show that \(F_{\alpha}\) is additive in the second argument,
    for all \(\alpha \in \mathbb{C}\). First, notice that for all \(\alpha \in \mathbb{C}\) and all \(z,x,y \in V\), we have
    \begin{align*}
        \norm{z + \alpha x}^2 + \norm{z + \alpha y}^2 &= \norm*{z + \frac{\alpha x + \alpha y}{2} + \frac{\alpha x - \alpha y}{2}}^2 + \norm*{z + \frac{\alpha x + \alpha y}{2} - \frac{\alpha x - \alpha y}{2}}^2\\
                                                      &= 2 \left(\norm*{z + \frac{\alpha x + \alpha y}{2}}^2 + \norm*{\frac{\alpha x - \alpha y}{2}}^2\right),
    \end{align*}
    by the parallelogram identity, then
    \begin{align*}
        F_{\alpha}(z,x) + F_{\alpha}(z,y) &= \frac14\left(\norm{z +\alpha x}^2 - \norm{z - \alpha x}^2 + \norm{z + \alpha y}^2 - \norm{z - \alpha y}^2\right)\\
                                            &=  \frac12\left(\norm*{z + \frac{\alpha x + \alpha y}{2}}^2 + \norm*{\frac{\alpha x - \alpha y}{2}}^2\right) - \frac12 \left(\norm*{z - \frac{\alpha x + \alpha y}{2}}^2 + \norm*{\frac{\alpha y - \alpha x}{2}}^2\right)\\
                                            &= \frac12 \left(\norm*{z + \alpha \frac{x+y}{2}}^2 - \norm*{z - \alpha \frac{x + y}{2}}^2\right)\\
                                            &= 2 F_{\alpha}\left(z, \frac{x + y}{2}\right).
    \end{align*}
    Since \(F_{\alpha}(0, \noarg) = F_{\alpha}(\noarg, 0) = 0\), setting \(y = 0\) yields \(\frac12F_{\alpha}(u, v) = F_{\alpha}(u, \frac12 v)\) for all \(\alpha \in \mathbb{C}\) and \(u, v \in V\). By the previous result, we have
    \begin{equation*}
        F_{\alpha}(z,x) + F_{\alpha}(z,y) = F_{\alpha}(z,x + y),
    \end{equation*}
    for all \(x,y,z \in V\), that is, \(F_{\alpha}\) is additive in the second argument for all \(\alpha \in \mathbb{C}\). In particular, \(F_1\) and \(F_{-i}\) are additive in the second argument, hence \(\inner{\noarg}{\noarg}\) is additive in the second argument.

    Let us show that \(F_{\alpha}\) is real homogeneous in the second argument, that is, \(F_{\alpha}(u,\lambda v) = \lambda F_{\alpha}(u,v)\) for all \(u,v \in V\) and \(\lambda \in \mathbb{R}\). First, by the definition of the map \(F_{\alpha}\), we have for all \(\alpha, \beta \in \mathbb{C}\) and all \(u, v \in V\) that
    \begin{equation*}
        F_{\alpha}(u, \beta v) = F_{\alpha \beta}(u, v),
    \end{equation*}
    then we want to show \(F_{\alpha \lambda}(u, v) = \lambda F_{\alpha}(u,v)\) for all \(\lambda \in \mathbb{R}\). By the definition, we know this holds for \(\lambda = 0\), and when showing additivity we've found it holds for \(\lambda = \frac12\). Hence, with an inductive argument, it must hold for all \(\lambda \in \mathbb{N}\) by additivity. From the definition it is easy to see that \(F_{\alpha}(u,v) + F_{-\alpha}(u,v) = 0\), then homogeneity holds for \(\lambda = \mathbb{Z}\):
    \begin{equation*}
        F_{\alpha p}(u, v) = p F_{\alpha}(u, v),
    \end{equation*}
    for all \(p \in \mathbb{Z}\). If we replace \(p \mapsto q\) and \(v \mapsto \frac1q v\), with \(q \in \mathbb{N}\), we obtain
    \begin{equation*}
        F_{\alpha q}\left(u, \frac{1}{q}v\right) = F_{\alpha}(u,v)\quad\text{and}\quad qF_{\alpha}\left(u, \frac{1}{q}v\right) = q F_{\frac{\alpha}{q}}(u,v) \implies F_{\frac{\alpha}{q}}(u,v) = \frac{1}{q}F_{\alpha}(u,v),
    \end{equation*}
    then from the additivity and homogeneity thus far shown it follows that homogeneity holds for all \(\lambda \in \mathbb{Q}\). Recall the rationals are dense in the real numbers with respect to the usual metric, then for all \(\lambda \in \mathbb{R}\) there is a sequence \(\family{r_n}{n\in \mathbb{N}}\subset \mathbb{Q}\) such that \(r_n \to \lambda\), then by the continuity of the norm,
    \begin{align*}
        F_{\alpha \lambda}(u,v) &= F_{\alpha}\left(u, \lim_{n\to\infty} r_n v\right)\\
                                &= \frac14 \left[\norm*{u + \alpha \left(\lim_{n\to\infty} r_n\right) v}^2 - \norm*{u - \alpha \left(\lim_{n\to\infty} r_n\right) v}^2\right]\\
                                &= \lim_{n\to\infty}\frac{\norm*{u + \alpha r_n v}^2 + \norm*{u - \alpha r_n v}^2}{4}\\
                                &= \lim_{n\to\infty} F_{\alpha r_n}(u,v)\\
                                &= \left(\lim_{n\to \infty} r_n\right)F_{\alpha}(u,v)\\
                                &= \lambda F_{\alpha}(u,v),
    \end{align*}
    hence \(F_{\alpha}\) is real homogeneous as claimed.

    To complete the proof, we show complex homogeneity in the second argument of \(\inner{\noarg}{\noarg}\). Let \(\lambda \in \mathbb{C}\) with \(\lambda = a + ib\), \(a,b \in \mathbb{R}\), and let \(u, v \in V\), then as \(\inner{\noarg}{\noarg}\) was shown to be additive in the second argument,
    \begin{equation*}
        \inner{u}{\lambda v} = \inner{u}{av} + \inner{u}{ibv} = \left[F_a(u,v) + iF_{-ia}(u,v)\right] + \left[F_{ib}(u,v) + iF_{b}(u,v)\right].
    \end{equation*}
    The real homogeneity of \(F_{\alpha}\) yields
    \begin{align*}
        \inner{u}{\lambda v} &= a\left[F_1(u,v) + iF_{-i}(u,v)\right] + ib\left[F_{1}(u,v)-iF_{i}(u,v)\right]\\
                             &= a\inner{u}{v} + ib\left[F_{1}(u,v) + F_{-i}(u,v)\right]\\
                             &= a\inner{u}{v} + ib \inner{u}{v}\\
                             &= \lambda \inner{u}{v},
    \end{align*}
    hence \(\inner{\noarg}{\noarg}\) is complex homogeneous in the second argument. We've thus shown \(\inner{\noarg}{\noarg}\) is linear in the second argument, hence \(\inner{\noarg}{\noarg}\) is the inner product that induces the norm in this pre-Hilbert space.
\end{proof}

We may sum up the previous results in the following statement.
\begin{corollary}
    Let \(V\) be a linear space. A norm \(\norm{\noarg} : V \to \mathbb{R}\) on \(V\) is induced by an inner product if and only if \((V, \norm{\noarg})\) is a pre-Hilbert space. Furthermore, the inner product satisfies the \emph{polarization identity}.
\end{corollary}

Therefore, a Banach space \((V, \norm{\noarg})\) is a Hilbert space if and only if \((V, \norm{\noarg})\) is a pre-Hilbert space.
\begin{example}{The Banach space \((\mathcal{C}([0,1];\mathbb{C}), \norm{\noarg}_\infty)\) is not a Hilbert space}{continuous_sup_not_hilbert}
    The Banach space \((\mathcal{C}([0,1];\mathbb{C}), \norm{\noarg}_\infty)\) of continuous complex functions defined in the interval \([0,1]\) is not a Hilbert space.
\end{example}
\begin{proof}
    We consider the continuous functions \(f,g \in \mathcal{C}([0,1;\mathbb{C}])\) defined by \(f(x) = x\) and \(g(x) = 1\), then \(\norm{f}_\infty = \norm{g}_\infty = 1\), \(\norm{f + g}_\infty = 2\), and \(\norm{f - g}_\infty = 1\). The parallelogram identity does not hold for such vectors,
    \begin{equation*}
        \norm{f+g}_\infty^2 + \norm{f-g}_\infty^2 = 5\quad\text{and}\quad2\left(\norm{f}_\infty^2 + \norm{g}_\infty^2\right) = 4,
    \end{equation*}
    therefore this Banach space is not a pre-Hilbert space.
\end{proof}

\begin{lemma}{Necessary condition for a pre-Hilbert space}{parallelogram_inequality}
    Let \((V, \norm{\noarg})\) be a normed linear space. If for all \(x,y \in V\)
    \begin{equation*}
        2\left(\norm{x}^2 + \norm{y}^2\right) \leq \norm{x+y}^2 + \norm{x - y}^2,
    \end{equation*}
    then \((V, \norm{\noarg})\) is a pre-Hilbert space.
\end{lemma}
\begin{proof}
    Let \(u, v \in V\). Setting \(x = \frac12(u + v)\) and \(y = \frac12(u - v)\) in the inequality yields
    \begin{equation*}
        2\left(\norm{u}^2 + \norm{v}^2\right) \geq \norm{u+v}^2 + \norm{u - v}^2,
    \end{equation*}
    from which we conclude the parallelogram identity.
\end{proof}
\begin{remark}
    It is obvious we could start from the inequality in the proof and conclude the same result.
\end{remark}
\begin{theorem}{The topological dual of a Hilbert space is a Hilbert space}{dual_hilbert_maybe}
    Let \(\hilbert\) be a Hilbert space. Then, \(\hilbert^\dag\) is a Hilbert space.
\end{theorem}
\begin{proof}
From \cref{thm:bounded_operators_Banach}, we know \(\hilbert^\dag\) is a Banach space with respect to the operator norm. Let \(f, g \in \hilbert^\dag\), then
    \begin{align*}
        \norm{f + g}^2 + \norm{f - g}^2 &= \left(\sup_{\psi \in \hilbert}\frac{\norm{(f+g)\psi}}{\norm{\psi}}\right)^2 + \left(\sup_{\psi \in \hilbert}\frac{\norm{(f-g)\psi}}{\norm{\psi}}\right)^2
    \end{align*}
    \todo[I don't think I can show it now. Plus, I must use the inner product on \(\hilbert\) somehow, otherwise this would be true for topological dual of any Banach space.]
\end{proof}

We may use the parallelogram identity to show the best approximation theorem.
\begin{definition}{Convex sets}{convex_sets}
    Let \(V\) be a linear space. A linear combination of the form \(\lambda u + (1 - \lambda)v\) for \(u, v \in V\) and \(\lambda \in [0,1]\) is said to be a \emph{convex linear combination of \(u\) and \(v\)}. A \emph{convex set} \(A \subset V\) is a set such that contains every convex linear combination of vectors in \(A\).
\end{definition}
\begin{remark}
    It should be obvious linear subspaces are convex spaces. Indeed, a subspace contains any linear combination of its vectors, in particular it contains any convex linear combination of its vectors.
\end{remark}

\begin{proposition}{The closure of a convex set is convex}{closure_convex}
    Let \((V, \norm{\noarg})\) be a normed linear space. If \(U \subset V\) is a convex subset of \(V\), then \(\cl{U}\) is convex.
\end{proposition}
\begin{proof}
    Let \(x, y \in \cl{U}\), then there exists sequences \(\family{x_n}{n\in \mathbb{N}}, \family{y_n}{n\in \mathbb{N}} \subset U\) that converge to \(x\) and \(y\) respectively. Let \(\lambda \in [0,1]\), then the convex linear combination \(\lambda x_n + (1-\lambda)y_n\) belongs to \(U\) for all \(n \in \mathbb{N}\). By continuity of scalar multiplication and vector addition, \(\family{\lambda x_n + (1- \lambda)y_n}{n\in \mathbb{N}}\subset U\) is a sequence of elements in \(U\) that converges to the convex linear combination \(\lambda x + (1 - \lambda)y\). We have thus shown \(\lambda x + (1 - \lambda)y \in \cl{U}\), that is, \(\cl{U}\) is convex.
\end{proof}

\begin{theorem}{Best approximation}{best_approximation}
    Let \(A\) be a closed convex subset of a Hilbert space \(\hilbert\). Then, for every \(x \in \hilbert\), there exists a unique \emph{best approximation \(\eta \in A\) of \(x\) in \(A\)} satisfying
    \begin{equation*}
        \norm{x - \eta} = \inf_{y \in A}\norm{x - y},
    \end{equation*}
    that is, \(\eta\) minimizes the distance of \(x\) to \(A\).
\end{theorem}
\begin{proof}
    We define \(R = \inf_{y \in A}\norm{x - y}\) and show there exists a unique element in \(A\) that satisfies \(\norm{x - y} = R\). Consider a sequence \(\family{y_n}{n \in \mathbb{N}} \subset A\) satisfying the property
    \begin{equation*}
        \norm{x - y_n}^2 < R^2 + \frac1{n+1},
    \end{equation*}
    for all \(n \in \mathbb{N}\), which is guaranteed to exist since \(R\) is the greatest lower bound for the distance \(\norm{x - y}\) with \(y \in A\). Indeed, suppose there exists \(M \in \mathbb{N}\) such that there is no \(y \in A\) such that \(\norm{x - y} < R^2 + \frac{1}{M}\). Then every \(y \in A\) satisfies \(\norm{x - y} \geq R^2 + \frac{1}{M} > R\), and \(R\) wouldn't be the greatest lower bound.

    Next we show such a sequence is a Cauchy. For all \(m,n \in \mathbb{N}\), we have
    \begin{equation*}
        \norm{(y_n - x) + (y_m - x)}^2 + \norm{(y_n - x) - (y_m - x)}^2 = 2\norm{y_n - x}^2 + 2\norm{y_m - x}^2
    \end{equation*}
    by the parallelogram identity. Then
    \begin{equation*}
        \norm{y_n - y_m}^2 \leq 4R^2 + \frac2{n} + \frac{2}{m} - 4\norm*{\frac12y_n + \frac12y_m - x}^2,
    \end{equation*}
    by the sequence definition.
    Since \(\frac12 y_n + \frac12 y_m\) is a convex linear combination of elements in \(A\), it is an element of \(A\), therefore \(\norm*{\frac12 y_n + \frac12 y_m - x} \geq R\), then
    \begin{equation*}
        \norm{y_n - y_m}^2 \leq \frac{2}{n} + \frac{2}{m} \implies \norm{y_n - y_m} \leq \sqrt{\frac{2}{n} + \frac{2}{m}}
    \end{equation*}
    for all \(m,n \in \mathbb{N}\). Let \(\varepsilon > 0\), then \(N > \frac{4}{\varepsilon^2}\) yields \(\norm{y_n - y_m} < \varepsilon\) for all \(n,m \geq N\). Hence, the sequence is Cauchy.

    Since \(A\) is a closed set in a Hilbert space, it follows from \cref{thm:complete_closed} that this sequence converges to some element in \(A\), \(\eta\) say. We show this is the vector that minimizes the distance. By the triangle inequality,
    \begin{equation*}
        \norm{\eta - x} \leq \norm{\eta - y_n} + \norm{y_n - x} < \norm{\eta - y_n} + \sqrt{R^2 + \frac1{n}},
    \end{equation*}
    for all \(n \in \mathbb{N}\). Taking the limit as \(n \to \infty\), we get \(\norm{\eta - x} \leq R\). Since \(\eta \in A,\) we have \(\norm{\eta - x} \geq R\), then we conclude \(\norm{\eta - x} = R\).

    Suppose there is another such vector in \(A\), \(\tilde{\eta}\) say, that minimizes the distance to \(x\). By the parallelogram identity, we have
    \begin{equation*}
        \norm{\eta - \tilde{\eta}}^2 = 2\norm{\eta - x}^2 + 2\norm{\tilde{\eta} - x}^2 - 4\norm*{\frac{\eta + \tilde{\eta}}{2} - x}^2 = 4\left(R^2 - \norm*{\frac12\eta + \frac12\tilde{\eta} - x}^2\right).
\end{equation*}
    By the same argument as before, we have \(\norm*{\frac12\eta + \frac12 \tilde{\eta} - x}^2 \geq R^2\), which yields
    \begin{equation*}
        \norm{\eta - \tilde{\eta}}^2 \leq 0.
    \end{equation*}
    That is, \(\tilde{\eta} = \eta\), thus showing uniqueness.
\end{proof}
\begin{remark}
    In the particular case \(x = 0\), the theorem states there exists a vector \(\eta \in A\) that minimizes the norm in \(A\), \(\norm{\eta} = \inf_{y \in A} \norm{y}\).
\end{remark}

% vim: spl=en_us
\section{Orthogonality}
The inner product in pre-Hilbert spaces allows for a notion of \emph{orthogonality} of two vectors.
\begin{definition}{Orthogonality}{orthogonality}
    Let \((X, \inner{\noarg}{\noarg})\) be a pre-Hilbert space. A vector \(x \in X\) is \emph{orthogonal} to a vector \(y \in X\) if \(\inner{x}{y} = 0\). The \emph{orthogonal complement} of a subset \(Y \subset X\) is the set
    \begin{equation*}
        Y^\perp = \setc{x \in X}{\forall y \in Y : \inner{y}{x} = 0},
    \end{equation*}
    that is, every vector in \(Y^\perp\) is orthogonal to all vectors in \(Y\).
\end{definition}
\begin{remark}
    It should be clear from the definition of an inner product that if \(x\) is orthogonal to \(y\), then \(y\) is orthogonal to \(x\). Moreover, we have \(x\) orthogonal to \(x\) if and only if \(x = 0\), that is, \(Y \cap Y^\perp = \set{0}\).
\end{remark}

\begin{proposition}{Orthogonal complement is a closed subspace}{orthogonal_closed}
    Let \((X, \inner{\noarg}{\noarg})\) be a pre-Hilbert space. If \(Y \subset X\) is a subset, then its orthogonal complement \(Y^\perp\) is a closed linear subspace.
\end{proposition}
\begin{proof}
    Clearly, the zero vector belongs to the orthogonal complement. Let \(u, v \in Y^\perp\) and let \(\alpha \in \mathbb{C}\), then for all \(y \in Y\), we have
    \begin{equation*}
        \inner{y}{u + \alpha v} = \inner{y}{u} + \alpha\inner{y}{v}
    \end{equation*}
    by linearity. Hence, \(\inner{y}{u + \alpha v} = 0\) for all \(y \in Y\), that is, \(Y^\perp\) is a linear subspace.

    Let \(\family{x_n}{n\in \mathbb{N}} \subset Y^\perp\) be a convergent sequence to some \(x \in X\), then by the continuity of the inner product we have
    \begin{equation*}
        \inner{y}{x} = \inner*{y}{\lim_{n\to\infty} x_n} = \lim_{n\to\infty} \inner{y}{x_n} = 0
    \end{equation*}
    for all \(y \in Y\). That is, \(x \in Y^\perp\), so \(Y^\perp\) must be closed.
\end{proof}

We now use the best approximation theorem to show every vector in a Hilbert space has a unique decomposition in a closed subspace and its orthogonal complement.
\begin{theorem}{Orthogonal decomposition}{orthogonal_decomposition}
    Let \(M\) be a closed linear subspace of a Hilbert space \(\hilbert\). Then, every \(x \in \hilbert\) admits a unique decomposition \(x = x_\parallel + x_\perp,\) where \(x_\parallel \in M\) and \(x_\perp \in M^\perp\). Moreover, \(x_\parallel\) is the best approximation of \(x\) in \(M\).
\end{theorem}
\begin{proof}
    Since every closed subspace of a Hilbert space is in particular a closed convex subset, it admits a unique best approximation for every vector in the Hilbert space, by \cref{thm:best_approximation}. For every \(x \in \hilbert\) there exists a unique \(x_\parallel \in M\) such that
    \begin{equation*}
        \norm{x_\parallel - x} = \inf_{y \in M}{\norm{y - x}}.
    \end{equation*}

    Notice for every \(y \in M\) and \(\lambda \in \mathbb{C}\) we have \(x_\parallel + \lambda y \in M\), then
    \begin{equation*}
        \norm{x - x_\parallel}^2 \leq \norm{x - x_\parallel - \lambda y}^2,
    \end{equation*}
    with equality being equivalent to the case \(y = 0\). Let \(x_\perp = x - x_\parallel\), then
    \begin{equation*}
        \norm{x_\perp}^2 \leq \norm{x_\perp}^2 + \abs{\lambda}^2\norm{y}^2 - 2 \Re(\inner{x_\perp}{\lambda y}),
    \end{equation*}
    which yields
    \begin{equation*}
        2 \Re(\inner{x_\perp}{\lambda y}) \leq  \abs{\lambda}^2 \norm{y}^2,
    \end{equation*}
    for all \(y \in M\) and all \(\lambda \in \mathbb{C}\). Let \(\theta \in [0, 2\pi)\) be such that \(\inner{x_\perp}{y} = \abs*{\inner{x_\perp}{y}}e^{i\theta}\) and consider \(\lambda = r e^{-i\theta}\) with \(r > 0\), then \(\Re(\inner{x_\perp}{\lambda y}) = r\abs*{\inner{x_\perp}{y}}\) for all \(y \in M\). This yields
    \begin{equation*}
        \abs*{\inner{x_\perp}{y}} \leq \frac12 r \norm{y}^2
    \end{equation*}
    for all \(r > 0\) and all \(y \in M\), which can only hold if \(\inner{x_\perp}{y} = 0\). We have thus shown \(x_\perp \in M^\perp\).

    Suppose there exists \(\xi_\parallel \in M\) and \(\xi_\perp \in M^\perp\) such that \(x = \xi_\parallel + \xi_\perp\). Then, it must hold that
    \begin{equation*}
        \xi_\parallel - x_\parallel = x_\perp - \xi_\perp.
    \end{equation*}
    Notice \(\xi_\parallel - x_\parallel \in M\) and \(x_\perp - \xi_\perp \in M^\perp\), since \(M\) and \(M^\perp\) are linear subspaces, hence \(\xi_\parallel - x_\parallel \in M \cap M^\perp\). That is, \(\xi_\parallel - x_\parallel = 0\), so \(\xi_\parallel = x_\parallel\) and \(\xi_\perp = x_\perp\), showing uniqueness.
\end{proof}

We now show the relation between the closure of a linear subspace with orthogonal complements. First we prove two lemmas.
\begin{lemma}{Orthogonal complement and inclusion partial order}{complement_subsets}
    Let \(X\) be a pre-Hilbert space. Then if \(A \subset B \subset X\), we have \(B^\perp \subset A^\perp\).
\end{lemma}
\begin{proof}
    Let \(x \in B^\perp\), then \(\inner{y}{x} = 0\) for all \(y \in B\). In particular, \(\inner{y}{x}\) for all \(y \in A \subset B\), then \(x \in A^\perp\).
\end{proof}

\begin{lemma}{Closure of a linear subspace is a linear subspace}{closure_subspace}
    Let \(M\) be a linear subspace of a pre-Hilbert space. Then \(\cl{M}\) is a linear subspace.
\end{lemma}
\begin{proof}
    Recall \(\xi \in \cl{M}\) if and only if there exists a convergent sequence of elements in \(M\) that converges to \(\xi\), by \cref{lem:convergent_closure}.

    Since \(M\) is a linear subspace, \(0 \in M\), then with the constant sequence \(\family{0}{n\in \mathbb{N}} \subset M\) we have \(0 \in \cl{M}\). Let \(x, y \in \cl{M}\), then there exist sequences \(\family{x_n}{n\in \mathbb{N}}, \family{y_n}{n\in \mathbb{N}} \subset M\) that converge to \(x\) and \(y\). Since \(M\) is a linear subspace, for all \(\alpha \in \mathbb{C}\) we have \(x_n + \alpha y_n \in M\) for all \(n \in \mathbb{N}\). By the continuity of vector addition and scalar multiplication, it follows that \(x_n + \alpha y_n \to x + \alpha y\), that is, there exists a convergent sequence \(\family{x_n + \alpha y_n}{n \in \mathbb{N}}\subset M\) that converges to \(x + \alpha y\). We have thus shown that \(x + \alpha y\) is a point of closure of \(M\), that is, \(x + \alpha y \in \cl{M}\) for all \(x, y \in \cl{M}\) and all \(\alpha \in \mathbb{C}\).
\end{proof}

\begin{theorem}{Closure and orthogonal complement}{closure_perp}
    Let \(M\) be a linear subspace of a Hilbert space \(\hilbert\). Then \(\cl{M} = (M^\perp)^\perp\).
\end{theorem}
\begin{proof}
    Let \(x \in M\) and let \(y \in M^\perp\). Then \(x\) is perpendicular to \(y\), that is, \(x \in (M^\perp)^\perp\). This shows \(M \subset (M^\perp)^\perp\), hence \(\cl M \subset (M^\perp)^\perp\) since the orthogonal complement is closed and the closure \(\cl{M}\) is the smallest closed set containing \(M\).

    Let \(x \in (M^\perp)^\perp\). By \cref{lem:closure_subspace}, \(\cl{M}\) is a closed linear subspace, then for all \(y \in \hilbert\) there exist unique \(y_\parallel \in \cl{M}\) and \(y_\perp \in (\cl{M})^\perp\) such that \(y = y_\parallel + y_\perp\) by \cref{thm:orthogonal_decomposition}. In particular, we consider the orthogonal decomposition \(x = x_\parallel + x_\perp\) and we claim \(x_\perp = 0\). For all \(y \in \hilbert\), we have
    \begin{equation*}
        \inner{x_\perp}{y} = \inner{x_\perp}{y_\perp} = \inner{x - x_\parallel}{y_\perp} = \inner{x}{y_\perp},
    \end{equation*}
    since \(\inner{x_\parallel}{y_\perp} = \inner{x_\perp}{y_\parallel} = 0\). By \cref{lem:complement_subsets}, \(M \subset \cl{M}\) yields \((\cl{M})^\perp \subset M^\perp\), that is, \(y_\perp \in M^\perp\) for all \(y \in \hilbert\), then \(\inner{x}{y_\perp} = 0\). We have thus shown \(\inner{x_\perp}{y} = 0\) for all \(y \in \hilbert\), and it follows from non-degeneracy of the inner product that \(x_\perp = 0\) as claimed. That is, \(x = x_\parallel \in \cl{M}\), hence \((M^\perp)^\perp \subset \cl{M}\).
\end{proof}

Before illustrating the previous result with a concrete example, we relate linear independence with orthogonality and, for the sake of completeness, recall the definition of a Hamel basis for the sake of completeness.

\begin{lemma}{Linear independence and orthogonality}{linear_independent_orthogonal}
    Let \((V, \inner{\noarg}{\noarg})\) be a pre-Hilbert space. Let \(F\) be a non-empty orthogonal subset of \(V\), that is, \(0 \notin F\) and for all \(u,v \in F\) we have \(\inner{u}{v} = 0\) if \(u \neq v\). Then, \(F\) is linearly independent.
\end{lemma}
\begin{proof}
    Let \(S\) be a non-empty finite subset of \(F\) with \(n\) elements. Then, we may enumerate \(S\) as the finite family \(\ffamily{v_i}{i=1}{n} = S\). We consider a finite family \(\ffamily{\alpha_i}{i=1}{n}\) of \(n\) complex numbers such that
    \begin{equation*}
        \sum_{i =1}^n \alpha_i v_i = 0.
    \end{equation*}
    For every \(j \in \set{1,2,\dots, n}\), we may take the inner product of the above expression with the element \(v_j\), yielding \(\alpha_j \norm{v_j}^2 = 0\) for all \(j\), since \(\inner{v_j}{\alpha_i v_i} = \alpha_i \norm{v_i}^2 \delta_{ij}\). Since \(0 \notin F\), we have \(\alpha_i = 0\), for all \(i \in \set{1,2,\dots, n}\). Hence, \(F\) is linearly independent.
\end{proof}

\begin{definition}{Linear span}{linear_span}
    Let \(V\) be a linear space and let \(F \subset V\) be a non-empty subset. The \emph{linear span \(\lspan{F}\) of \(F\)} is the set
    \begin{equation*}
        \lspan{F} = \setc*{\sum_{i = 1}^n \lambda_i v_i}{n \in \mathbb{N}, \ffamily{\lambda_i}{i=1}{n} \subset \mathbb{C}, \ffamily{v_i}{i=1}{n}\subset F}
    \end{equation*}
    consisting of all finite linear combinations of elements in \(F\). The set \(F\) is called a \emph{generating set} for \(\lspan{F}\).
\end{definition}
\begin{remark}
    We check \(\lspan F\) is a linear subspace of \(V\). Clearly, \(0 \in \lspan F\). Let \(x, y \in \lspan F\), then \(x = \sum_{i = 1}^{n} \lambda_i x_i\) and \(y = \sum_{j = 1}^{m} \mu_j y_j\) where \(\ffamily{\lambda_i}{i=1}{n}, \ffamily{\mu_j}{j = 1}{m} \subset \mathbb{C}\) and \(\ffamily{x_i}{i=1}{n}, \ffamily{y_j}{j=1}{m} \subset F\) for natural numbers \(n,m \geq 1\). Let \(\alpha \in \mathbb{C}\), then setting \(\eta_k = \lambda_k\), \(v_k = x_k\) for \(k \in \set{1, 2, \dots, n}\) and \(\eta_{k} = \alpha \mu_{k-n}\), \(v_{k} = y_{k-n}\) for \(k \in \set{n+1, n+2, \dots, n+m}\), yields \(x + \alpha y = \sum_{k = 1}^{m+n} \eta_k v_k\), that is \(x + \alpha y \in \lspan F\).
\end{remark}

\begin{proposition}{Linear span of a subset is the smallest linear subspace that contains it}{linear_span_smallest}
    Let \(F \subset V\) be a non-empty subset of a linear space \(V\). If \(E\) is a subset of \(V\) that contains \(F\), then \(\lspan{F} \subset \lspan{E}\). Moreover, if \(E\) is a linear subspace of \(F\), we have \(\lspan{F} \subset E\).
\end{proposition}
\begin{proof}
    Let \(v \in \lspan{F}\). Then \(v\) is a linear combination of elements in \(F\). Since \(F \subset E\), every element in \(F\) is a member of \(E\), that is, \(v\) is a linear combination of vectors in \(E\). Hence, \(v \in \lspan{E}\), then \(\lspan{F} \subset \lspan{E}\). If, in addition, \(E\) is a linear subspace, then \(\lspan{E} = E\), since linear subspaces are closed under linear combinations.
\end{proof}

\begin{definition}{Hamel basis and dimension}{hamel_basis}
    Let \(V\) be a linear space. A \emph{Hamel basis} is a non-empty subset \(F\subset V\) that is linearly independent and a generating set for \(V\). If \(F\) is finite, we say \(V\) is finite dimensional with \emph{dimension} equal to the number of elements in \(F\), otherwise we say \(V\) is infinite dimensional.
\end{definition}
\begin{remark}
    Using Zorn's lemma, it can be shown that every non-trivial vector space (in fact, every non-trivial module over a division ring) admits a Hamel basis. This result is assumed without proof.
\end{remark}
\begin{remark}
    Rigorously it would be necessary to show well-definition of dimension, which we will simply assume for brevity.
\end{remark}

\begin{example}{Dense set in \(\ell_2\)}{dense_l2}
    Let \(\ell_2\) be the Hilbert space of square-summable sequences and let \(\mathfrak{d} \subset \ell_2\) be the set of all sequences with a finite number of non-zero components. Then, \(\mathfrak{d}\) is infinite-dimensional, \((\mathfrak{d}^\perp)^\perp = \ell_2\) and \(\mathfrak{d}\) is dense in \(\ell_2\).
\end{example}
\begin{proof}
    Recall the inner product in \(\ell_2\) is defined as \(\inner{a}{b} = \sum_{n = 1}^\infty \conj{a_n} b_n\), for two sequences \(a = \family{a_n}{n\in \mathbb{N}}\) and \(b = \family{b_n}{n \in \mathbb{N}}\). We show the set \(E = \family{e_i}{n\in \mathbb{N}}\), where \(e_i\) is the sequence defined by
    \begin{align*}
        e_i : \mathbb{N} &\to \mathbb{C}\\
                       j &\mapsto \delta_{ij}
    \end{align*}
    is a Hamel basis for \(\mathfrak{d}\). Clearly, each sequence \(e_i\) is square-summable and each sequence has a finite number of non-zero components, then \(e_i \in \mathfrak{d} \subset \ell_2\), for all \(i \in \mathbb{N}\). Notice \(E\) is an orthogonal set, since for all \(i,j \in \mathbb{N}\) we have
    \begin{equation*}
        \inner{e_i}{e_j} = \sum_{n = 1}^\infty \delta_{in} \delta_{jn} = \delta_{ij},
    \end{equation*}
    then \(E\) is linearly independent by \cref{lem:linear_independent_orthogonal}. Since \(E \subset \mathfrak{d}\) and \(\mathfrak{d}\) is a linear subspace of \(\ell_2\), then \(\lspan{E} \subset \mathfrak{d}\). Let \(s \in \mathfrak{d}\), then there is a maximal finite set \(J \subset \mathbb{N}\) such that \(s(\mathbb{N} \setminus J) = \set{0}\), and as a result, we have
    \begin{equation*}
        s = \sum_{j \in J} s(j) e_j,
    \end{equation*}
    hence \(\mathfrak{d} \subset \lspan{E}\). We have thus shown \(E\) is a Hamel basis for \(\mathfrak{d}\).

    Let \(x = \family{x_n}{n\in \mathbb{N}} \in \mathfrak{d}^\perp\). Notice for all \(i \in \mathbb{N}\), we have \(\inner{e_i}{x} = x_i\). Then, \(x_i = 0\) for all \(i \in \mathbb{N}\), that is, \(x\) is the zero vector. We have shown \(\mathfrak{d}^\perp = \set{0}\). Since every vector is orthogonal to the zero vector by non-degeneracy, we have \((\mathfrak{d}^\perp)^\perp = \ell_2\). Finally, it follows from \cref{thm:closure_perp} that \(\cl\mathfrak{d} = \ell_2\), that is, \(\mathfrak{d}\) is dense in \(\ell_2\).
\end{proof}



% vim: spl=en_us
\section{Orthonormal sets}
In the previous example we constructed a countable orthogonal Hamel basis for a linear subspace dense in a Hilbert space. This means that basis is linearly independent and the closure of its linear span is the entire Hilbert space. We'll develop this idea by studying further properties of orthonormal sets in order to construct a different type of basis that uses the completeness of a Hilbert space.
\begin{definition}{Orthonormal set}{orthonormal_set}
    Let \((V, \inner{\noarg}{\noarg})\) be a pre-Hilbert space. A subset \(F\subset V\) is \emph{orthogonal} if for all \(u, v \in F\), we have \(\inner{u}{u} > 0\) and \(\inner{u}{v} = 0\) for \(v \neq u\). If for all \(u \in V\) we have \(\inner{u}{u} = 1\), then \(F\) is said to be \emph{orthonormal}.
\end{definition}
\begin{remark}
    We have already shown that such a set is linearly independent in \cref{lem:linear_independent_orthogonal}.
\end{remark}

\begin{lemma}{Pythagorean theorem}{pythagoras_finite}
    Let \((V, \inner{\noarg}{\noarg})\) be a pre-Hilbert space. Let \(\ffamily{v_i}{i = 1}{n} \subset V\) be a finite orthonormal set and let \(\ffamily{\lambda_i}{i=1}{n} \subset \mathbb{C}\) be a finite family of complex numbers, then
    \begin{equation*}
        \norm*{\sum_{i = 1}^n \lambda_i v_i}^2 = \sum_{i=1}^n \abs*{\lambda_i}^2
    \end{equation*}
\end{lemma}
\begin{proof}
    By sesquilinearity of the inner product we have
    \begin{equation*}
        \norm*{\sum_{i = 1}^n \lambda_i v_i}^2 = \inner*{\sum_{i = 1}^n \lambda_i v_i}{\sum_{j = 1}^n \lambda_j v_j} = \sum_{i = 1}^n \sum_{j = 1}^n \conj{\lambda_i} \lambda_j \inner{e_i}{e_j}.
    \end{equation*}
    Since \(\ffamily{v_i}{i=1}{n}\) is orthonormal, we have \(\inner{e_i}{e_j} = \delta_{ij}\), then
    \begin{equation*}
        \norm*{\sum_{i=1}^n\lambda_i v_i}^2 = \sum_{i = 1}^n \sum_{j=1}^n \conj{\lambda_i}\lambda_j \delta_{ij} = \sum_{i = 1}^n \abs{\lambda_i}^2,
    \end{equation*}
    as desired.
\end{proof}

We now give a necessary and sufficient condition for a \emph{convergent series} in a Hilbert space. For any sequence \family{s_n}{n\in \mathbb{N}}, a \emph{series} is defined as the sequence \family{\sum_{i=1}^n s_n}{n \in \mathbb{N}} and if it converges (with respect to the appropriate metric space) we denote its limit by \(\sum_{i=1}^\infty s_n\).
\begin{proposition}{Convergent series and countable orthonormal sets}{convergent_series}
    Let \(\hilbert\) be a Hilbert space and let \(\family{v_n}{n\in \mathbb{N}}\subset \hilbert\) be a countable orthonormal set. The sequence \(\family{\lambda_n}{n\in \mathbb{N}} \subset \mathbb{C}\) is square-summable if and only if the sequence \(\family{s_n}{n\in \mathbb{N}} \subset \hilbert\), where \(s_n = \sum_{i=1}^n \lambda_i v_i\), converges in \(\hilbert\).
\end{proposition}
\begin{proof}
    Consider the sequence \(\family{\ell_n}{n\in \mathbb{N}} \subset \mathbb{R}\) where \(\ell_n = \sum_{i = 1}^n \abs{\lambda_i}^2\). For all \(n,m \in \mathbb{N}\) we have
    \begin{equation*}
        \norm{s_n - s_m}^2 = \norm*{\sum_{i = \min\set{m,n} + 1}^{\max\set{m,n}} \lambda_i v_i}^2 = \sum_{i=\min\set{m,n}}^{\max{\set{m,n}}} \abs{\lambda_i}^2 = \abs*{\sum_{i=1}^n \abs{\lambda_i}^2 - \sum_{i=1}^m \abs{\lambda_i}^2} = \abs{\ell_n - \ell_m}
    \end{equation*}
    by \cref{lem:pythagoras_finite}.

    Suppose \(s_n \to s \in \hilbert\). Let \(\varepsilon > 0\), then there exists \(N \in \mathbb{N}\) such that \(\norm{s_n - s_m} < \sqrt{\varepsilon}\) for all \(n, m \geq N\), then
    \begin{equation*}
        n, m \geq N \implies \abs{\ell_n - \ell_m} = \norm{s_n - s_m}^2 < \varepsilon.
    \end{equation*}
    Since \family{\ell_n}{n\in \mathbb{N}} is a Cauchy sequence of real numbers, it converges in \((\mathbb{R}, \abs{\noarg}\), therefore \(\family{\lambda_n}{n \in \mathbb{N}}\) is square-summable.

    Suppose \(\family{\lambda_n}{n \in \mathbb{N}}\) is square-summable. Then, there exists \(N \in \mathbb{N}\) such that \(\abs{\ell_n - \ell_m} < \varepsilon^2\) for all \(m,n \geq N\), then
    \begin{align*}
        n, m \geq N &\implies \abs{\ell_n - \ell_m} = \norm{s_n - s_m}^2 < \varepsilon^2\\
                    &\implies \norm{s_n - s_m} < \varepsilon.
    \end{align*}
    Since \family{s_n}{n\in \mathbb{N}} is a Cauchy sequence of in a Hilbert space, it converges.
\end{proof}

We recall the Gram-Schmidt orthonormalization process, in order to show that every finite-dimensional linear subspace in a Hilbert space is closed.
\begin{lemma}{Gram-Schmidt orthonormalization process}{gram_schmidt}
    Let \((V, \inner{\noarg}{\noarg})\) be a pre-Hilbert space. If \(E \subset V\) is a non-trivial finite-dimensional linear subspace, then it admits an orthonormal Hamel basis.
\end{lemma}
\begin{proof}
    By the existence of a Hamel basis, we let \(\ffamily{v_i}{i=1}{n}\) be a linearly independent generating set for \(E\), where \(n\) is the dimension of \(E\). We claim for all \(m \in \set{1, 2, \dots, n}\) the set \(F_m = \ffamily{u_i}{i=1}{m}\) is orthogonal, where \(u_1 = v_1\) and
    \begin{equation*}
        u_k = v_k - \sum_{i = 1}^{k-1} \frac{\inner{u_i}{v_k}}{\inner{u_i}{u_i}} u_i
    \end{equation*}
    for \(k \in \set{2, 3, \dots, m}\). Clearly, this holds for \(m = 1\). Suppose it it true for some \(\ell < n\), and we check if \(u_{\ell+1} \in F_\ell^\perp \setminus \set{0}\). Since \(F_\ell\) is orthogonal, we have \(\inner{u_k}{u_j} = \inner{u_j}{u_j} \delta_{jk}\) for all \(j,k \in \set{1,\dots, \ell}\), then
    \begin{align*}
        \inner{u_k}{u_{\ell+1}} &= \inner*{u_k}{v_{\ell+1} - \sum_{j = 1}^{\ell} \frac{\inner{u_j}{v_{\ell+1}}}{\inner{u_j}{u_j}} u_j}\\
                                &= \inner{u_k}{v_{\ell+1}} - \sum_{j=1}^{\ell} \frac{\inner{u_j}{v_{\ell+1}}}{\inner{u_j}{u_j}} \inner{u_k}{u_j}\\
                                &= \inner{u_k}{v_{\ell+1}} - \sum_{j=1}^{\ell} \inner{u_j}{v_{\ell+1}}\delta_{kj}\\
                                &= \inner{u_k}{v_{\ell+1}} - \inner{u_k}{v_{\ell+1}} = 0
    \end{align*}
    for \(k \in \set{1,2, \dots, \ell}\), hence \(u_{\ell+1} \in F_\ell^\perp\). Notice \(u_{\ell+1} \neq 0\), since \(\ffamily{v_i}{i=1}{\ell+1}\) is linearly independent, and \(u_{\ell+1} \in \lspan\ffamily{v_i}{i=1}{\ell+1}\). We have thus shown \(F_{\ell+1}\) is orthogonal. By the principle of finite induction, \(F_m\) is orthogonal for all \(m \in \set{1,2,\dots, n}\).

    Since \(F_n\) is linearly independent, by \cref{lem:linear_independent_orthogonal}, it remains to show \(\lspan{F} = E\). Since \(F \subset E\), we have \(\lspan{F} \subset E\). Let \(v \in E\), then there exist \(\ffamily{\lambda_i}{i=1}{n}\) such that \(v = \sum_{i=1}^n \lambda_i v_i\). By the definition of \(u_k\), we have
    \begin{equation*}
        v = \lambda_1 u_1 + \sum_{i=2}^n \lambda_i\left(u_i + \sum_{j = 1}^{i-1} \frac{\inner{u_j}{v_i}}{\inner{u_j}{u_j}}u_j\right),
    \end{equation*}
    that is, \(v\) is a linear combination of elements in \(F\), hence \(v \in \lspan{F}\). That is, \(\lspan{F} = E\), therefore \(F\) is an orthogonal Hamel basis for \(E\). In particular the set \(\ffamily{\frac1{\norm{u_i}}u_i}{i=1}{n}\) is an orthonormal Hamel basis for \(E\).
\end{proof}
\begin{remark}
    If the Hamel basis were countable, the argument and result would follow analogously.
\end{remark}

\begin{proposition}{Finite dimensional subspaces of a pre-Hilbert space are closed}{finite_closed}
    Let \((V, \inner{\noarg}{\noarg})\) be a pre-Hilbert space. If \(E \subset V\) is a finite-dimensional linear subspace, then it is closed.
\end{proposition}
\begin{proof}
    By \cref{lem:gram_schmidt}, there exists an orthonormal Hamel basis \(\ffamily{e_i}{i=1}{d} \subset E\), where \(d\) is the dimension of \(E\). Let \(v : \mathbb{N} \to E\) be a sequence that converges to \(\tilde{v} \in \hilbert\), then for all \(n \in \mathbb{N}\), we have
    \begin{equation*}
        v(n) = \sum_{i = 1}^{d} \lambda_i(n) e_i,
    \end{equation*}
    where \(\lambda_i : \mathbb{N} \to \mathbb{C}\) is a sequence of complex numbers, for \(i \in\set{1,2,\dots, d}\). Notice
    \begin{equation*}
        \lambda_j(n) = \inner{e_j}{v(n)}
    \end{equation*}
    for all \(j \in \set{1,2,\dots,d}\) and all \(n \in \mathbb{N}\). Setting \(\tilde{\lambda}_j = \inner{e_j}{\tilde{v}}\) yields
    \begin{equation*}
        \abs{\tilde{\lambda}_j - \lambda_j(n)} = \abs{\inner{e_j}{\tilde{v} - v(n)}} \leq \norm{\tilde{v} - v(n)}
    \end{equation*}
    by the Cauchy-Schwarz inequality. Since \(v(n) \to \tilde{v}\), we have \(\lambda_j(n) \to \tilde{\lambda}_j\).

    We claim \(\tilde{v} = \sum_{i=1}^d \tilde{\lambda}_i e_i \in E\), showing that \(E\) is closed by \cref{thm:convergent_closed}. Indeed, by subadditivity
    \begin{align*}
        \norm*{\tilde{v} - \sum_{i=1}^d \tilde{\lambda}_i e_i} &\leq \norm*{\tilde{v} - v(n)} + \norm*{v(n) - \sum_{i=1}^d \tilde{\lambda}_i e_i} = \norm*{\tilde{v} - v(n)} + \norm*{\sum_{i=1}^d \left[\tilde{\lambda}_i - \lambda_i(n)\right]e_i}
    \end{align*}
    for all \(n \in \mathbb{N}\). \cref{lem:pythagoras_finite} yields
    \begin{equation*}
        \norm*{\tilde{v} - \sum_{i=1}^d \tilde{\lambda}_i e_i} \leq \norm*{\tilde{v} - v(n)} + \sqrt{\sum_{i=1}^d \abs*{\tilde{\lambda}_i - \lambda_i(n)}^2}.
    \end{equation*}
    As \(v(n) \to \tilde{v}\) and \(\lambda_i(n) \to \tilde{\lambda}_i\), the right hand size can be made arbitrarily small, hence
    \begin{equation*}
        \norm*{\tilde{v} - \sum_{i=1}^d \tilde{\lambda}_ie_i} = 0.
    \end{equation*}
    That is, \(\tilde{v} \in E\).
\end{proof}
\begin{corollary}
    The linear span of a finite orthonormal set is closed.
\end{corollary}
\begin{remark}
    From what we've seen in \cref{exam:dense_l2}, we cannot extend this result for linear subspaces generated by countable orthonormal sets.
\end{remark}

Since finite-dimensional linear subspaces of a Hilbert space are closed, we may use the best approximation theorem and, as a direct consequence, the orthogonal decomposition theorem.
\begin{proposition}{Best approximation on a finite-dimensional linear subspace}{best_approximation_finite}
    Let \(\hilbert\) be a Hilbert space and let \(F = \ffamily{e_i}{i=1}{d} \subset \hilbert\) be a orthonormal set with \(d \in \mathbb{N}\). For all \(x \in \hilbert\) the best approximation of \(x\) in \(\lspan{F}\) is given by
    \begin{equation*}
        \tilde{x} = \sum_{i=1}^{d} \inner{e_i}{x}e_i.
    \end{equation*}
    Moreover,
    \begin{equation*}
        \norm{x - \tilde{x}} = \sqrt{\norm{x}^2 - \sum_{i = 1}^d \abs*{\inner{e_i}{x}}^2}
    \end{equation*}
    is the infimum of the distance between \(x\) and \(\lspan{F}\).
\end{proposition}
\begin{proof}
    Let \(y \in \lspan F\), then there exist \(\ffamily{\lambda_i}{i=1}{d} \subset \mathbb{C}\) such that \(y = \sum_{i=1}^d \lambda_i e_i\). By \cref{lem:pythagoras_finite}, we have
    \begin{equation*}
        \norm{y}^2 = \sum_{i = 1}^d \abs{\lambda_i}^2,
    \end{equation*}
    then for all \(x \in \hilbert\),
    \begin{align*}
        \norm*{x - y}^2 &= \norm{x}^2 + \norm{y}^2 - \inner{x}{y} - \inner{y}{x}\\
                        &= \norm{x}^2 + \sum_{i=1}^d \left(\abs{\lambda_i}^2 - \inner{x}{\lambda_i e_i} - \inner{\lambda_i e_i}{x}\right)\\
                        &= \norm{x}^2 + \sum_{i=1}^d \left(\abs{\lambda_i}^2 - \lambda_i\conj{\inner{e_i}{x}} - \conj{\lambda_i}\inner{e_i}{x} + \abs*{\inner{e_i}{x}}^2\right) - \sum_{i=1}^d \abs*{\inner{e_i}{x}}^2\\
                        &= \norm{x}^2 + \sum_{i = 1}^d \left[\left(\lambda_i - \inner{e_i}{x}\right)\conj{\left(\lambda_i - \inner{e_i}{x}\right)}\right] - \sum_{i=1}^d \abs*{\inner{e_i}{x}}^2\\
                        &= \norm{x}^2 + \sum_{i=1}^d \abs*{\lambda_i - \inner{e_i}{x}}^2 - \sum_{i=1}^d \abs*{\inner{e_i}{x}}^2.
    \end{align*}
    Since \(\norm{x}^2\) and \(\sum_{i=1}^d\abs*{\inner{e_i}{x}}^2\) are fixed for each \(x\) and the other term is a sum of non-negative real numbers, we have
    \begin{equation*}
        \norm{x - y} = \sqrt{\norm{x}^2 + \sum_{i=1}^d \abs*{\lambda_i - \inner{e_i}{x}}^2 - \sum_{i=1}^d \abs*{\inner{e_i}{x}}^2} \geq \sqrt{\norm{x}^2 - \sum_{i=1}^d \abs*{\inner{e_i}{x}}^2},
    \end{equation*}
    that is,
    \begin{equation*}
        \inf_{y \in \lspan{F}} \norm{x - y} = \sqrt{\norm{x}^2 - \sum_{i=1}^d \abs*{\inner{e_i}{x}}^2}.
    \end{equation*}
    The vector in \(\lspan{F}\) that realizes the minimal distance between \(x\) and \(\lspan{F}\) is \(\tilde{x} = \sum_{i=1}^d \inner{e_i}{x} e_i\). By \cref{thm:best_approximation}, this is the only such element.
\end{proof}
\begin{remark}
    By \cref{thm:orthogonal_decomposition}, we identify \(x_\parallel = \sum_{i=1}^d \inner{e_i}{x}e_i \in \lspan{F}\) and \(x_\perp = x - x_\parallel \in (\lspan{F})^\perp\) with \(\norm{x_\parallel} = \sum_{i=1}^d \abs*{\inner{e_i}{x}}^2\) and \(\norm{x_\perp}^2 = \norm{x}^2 - \sum_{i=1}^d \abs*{\inner{e_i}{x}}^2\).
\end{remark}

We may use the previous result to show Bessel's inequalities.
\begin{lemma}{Bessel inequality}{Bessel_inequality}
    Let \(\hilbert\) be a Hilbert space. For all \(d \in \mathbb{N}\), if \(\ffamily{e_i}{i=1}{d}\) is a finite orthonormal set, then
    \begin{equation*}
        \sum_{i=1}^d \abs*{\inner{e_i}{x}}^2 \leq \norm{x}^2
    \end{equation*}
    for all \(x \in \hilbert\). If \(\family{e_i}{i\in \mathbb{N}}\) is a countable orthonormal set, then
    \begin{equation*}
        \sum_{i=1}^\infty \abs*{\inner{e_i}{x}}^2 \leq \norm{x}^2
    \end{equation*}
    for all \(x \in \hilbert\).
\end{lemma}
\begin{proof}
    Let \(F_d = \ffamily{e_i}{i=1}{d}\subset \hilbert\) be a finite orthonormal set, then by \cref{prop:best_approximation_finite} the best approximation \(\tilde{x}\) for \(x \in \hilbert\) in \(\lspan{F_d}\) satisfies
    \begin{equation*}
        \norm{x - \tilde{x}}^2 = \norm{x}^2 - \sum_{i=1}^d \abs*{\inner{e_i}{x}}^2.
    \end{equation*}
    Since \(\norm{x - \tilde{x}} \geq 0\), we have
    \begin{equation*}
        \sum_{i=1}^d \abs*{\inner{e_i}{x}}^2 \leq \norm{x}^2,
    \end{equation*}
    as claimed.

    If \(F_d\) is a finite subset of a countable orthonormal set \(\family{e_n}{n\in \mathbb{N}}\), then we have shown the sequence \(\family{s_n}{n\in \mathbb{N}} \subset \mathbb{R}\), where \(s_n = \sum_{i=1}^n \abs*{\inner{e_i}{x}}^2\), is bounded above by \(\norm{x}^2\). As a sum of non-negative real numbers, it is an increasing sequence bounded above, therefore it converges to its supremum, satisfying \(\sup_{n \in \mathbb{N}} s_n \leq \norm{x}^2\).

    Indeed, for all \(n \in \mathbb{N}\), \(s_n \leq \norm{x}^2\), therefore \(s = \sup_{n\in \mathbb{N}} \leq \norm{x}^2\). Let \(\varepsilon > 0\), then there exists \(N \in \mathbb{N}\) such that \(s - \varepsilon < s_N \leq s\), otherwise \(s\) would not be the least upper bound. Since the sequence does not decrease, we have \(s_n \geq s_N\) for all \(n \geq N\), that is
    \begin{equation*}
        n \geq N \implies \abs{s - s_n} < \varepsilon,
    \end{equation*}
    therefore the sequence is convergent.
\end{proof}



% vim: spl=en_us
\section{Complete orthonormal basis}
We now define a different basis and prove its existence on any Hilbert space.
\begin{definition}{Complete orthonormal basis}{complete_basis}
    A \emph{complete orthonormal basis} \(F\) is an orthonormal subset of a Hilbert space \(\hilbert\) such that the only vector of \(\hilbert\) that is orthogonal to every vector of \(F\) is the zero vector.
\end{definition}
\begin{remark}
    Recall \cref{exam:dense_l2}. The (countable) Hamel basis \(\family{e_n}{n \in \mathbb{N}} \subset \mathcal{d}\) there defined is in fact a complete orthonormal basis for the space of square-summable sequences \(\ell_2\).
\end{remark}

The existence of a complete orthonormal basis is shown with Zorn's lemma.
\begin{theorem}{Existence of a complete orthonormal basis}{complete_orthonormal_basis_exists}
    Every non-trivial Hilbert space has at least one complete orthonormal basis.
\end{theorem}
\begin{proof}
    Let \((\mathcal{O}, \subset)\) be the collection of orthonormal sets in some non-trivial Hilbert space \(\hilbert\) partially ordered by inclusion. Notice \(\mathcal{O}\) is not empty since every finite-dimensional linear subspace of \(\hilbert\) admits an orthonormal Hamel basis, in particular we may construct orthonormal sets from any collection of vectors in \(\hilbert\).

    Let \(\mathcal{E} \subset \mathcal{O}\) be a non-empty linearly ordered subset of \(\mathcal{O}\). We claim the union
    \begin{equation*}
        F = \bigcup \mathcal{E} = \setc{v \in \hilbert}{\exists A \in \mathcal{E} : v \in A}
    \end{equation*}
    is an upper bound for \(\mathcal{E}\) in \(\mathcal{O}\). Let \(u \in F\), then there exists an element \(A \in \mathcal{E}\) such that \(u \in A\). Since \(A\) is orthonormal, \(\norm{u} = 1\) and \(\inner{a}{u} = 0\) for all \(a \in A \setminus{u}\). Let \(v \in F\) with \(v \neq u\), then, analogously, it belongs to some element \(B \in \mathcal{E}\), with \(\norm{v} = 1\) and \(\inner{b}{v} = 0\) for all \(b \in B\). Since \(\mathcal{E}\) is a linearly ordered subset, we either have \(B \subset A\) or \(A \subset B\). For the former we have \(v \in A\) and for the latter, \(u \in B\); then either case yields the same conclusion, \(\inner{u}{v} = 0\). That is, for all \(u \in F\), \(\norm{u} = 1\) and \(v \in F \setminus{u} \implies \inner{v}{u} = 0\), hence \(F\) is an orthonormal set. We have thus shown every linearly ordered subset of \(\mathcal{O}\) has an upper bound in \(\mathcal{O}\).

    By \nameref{thm:zorn}, \(\mathcal{O}\) has at least one maximal element, \(\mathcal{B}\) say. That is, \(\mathcal{B}\) is an orthonormal set in \(\hilbert\) that is not properly contained in any other orthonormal set. Suppose, by contradiction, \(\mathcal{B}\) is not a complete orthonormal basis, then there exists \(u \in \hilbert\setminus\set{0}\) that is orthogonal to every vector in \(\mathcal{B}\). Notice \(u \notin \mathcal{B}\), otherwise \(\inner{u}{u} = 0\), contrary to the hypothesis \(u \neq 0\). In this case, \(\mathcal{B} \cup \set{\frac{1}{\norm{u}}u} \in \mathcal{O}\) is an orthonormal set that contains \(\mathcal{B}\). Then, \(\mathcal{B}\) is not a maximal element of \((\mathcal{O}, \subset)\). This contradiction shows \(\mathcal{B}\) is a complete orthonormal basis.
\end{proof}

To motivate the \enquote{basis} name in \cref{def:complete_basis}, we will show a complete orthonormal basis is a \emph{topological basis} on a Hilbert space.
\begin{definition}{Topological basis}{topological_basis}
    Let \(V\) be a topological vector space. A \emph{topological basis} is a linearly independent subset \(F\subset V\) such that \(\cl(\lspan F) = V\).
\end{definition}
In contrast to a Hamel basis (or algebraic basis), elements in the topological vector space can be expressed as a limit of a sequence of linear combinations of elements of a topological basis.

Before showing a complete orthonormal basis on a Hilbert space is indeed a topological basis, we show the existence of a useful countable subset of an orthonormal set.
\begin{lemma}{Countable subset of an orthonormal set}{countable_orthonormal}
    Let \(F\) be a non-empty orthonormal subset of a Hilbert space \(\hilbert\). For all \(x \in \hilbert\), the subset \(F^x = \setc{v \in F}{\inner{x}{v} \neq 0}\) is countable.
\end{lemma}
\begin{proof}
    If \(x = 0\), the subset \(F^x\) is empty, therefore the statement holds vacuously. In what follows, we assume \(x \neq 0\).

    Notice \(v \in F^x\) if and only if \(\abs*{\inner{x}{v}}^2 \neq 0\). Moreover, by \cref{lem:Bessel_inequality}, \(0 < \abs*{\inner{x}{v}}^2 \leq \norm{x}^2\) if and only if \(v \in F^x\), since \(\set{v}\) is a finite orthonormal set. Then, \(F^x = \setc{v \in F}{0 < \abs*{\inner{x}{v}}^2 \leq \norm{x}^2}\).

    For all \(n \in \mathbb{N}\), we claim the subset
    \begin{equation*}
        F^x_n = \setc*{v \in F^x}{\abs*{\inner{x}{v}}^2 \in \left(\frac{\norm{x}^2}{n+1}, \frac{\norm{x}^2}{n}\right]}
    \end{equation*}
    has at most \(n\) elements. Suppose by contradiction it has \(m > n\) elements, then
    \begin{equation*}
        \forall v \in F^x_n, \abs*{\inner{v}{x}}^2 > \frac{\norm{x}^2}{n+1} \implies \sum_{v \in F^x_n} \abs*{\inner{v}{x}}^2 > \frac{m}{n+1} \norm{x}^2 \geq \norm{x}^2.
    \end{equation*}
    This contradicts the Bessel inequality, proving our claim.

    Let \(n,m \in \mathbb{N}\), and consider the intersection \(F^x_n \cap F^x_m\). Suppose this intersection is not the empty set, then let \(v \in F^x_n \cap F^x_m\). Since \(v \in F^x\), there exists \(\lambda \geq 1\) such that \(\abs*{\inner{x}{v}}^2 = \frac{\norm{x}^2}{\lambda}\). Since \(v \in F^x_n\) and \(v \in F^x_m\), it follows that \(n \leq \lambda < n+1\) and \(m \leq \lambda < m+1\). Then \(\lambda \in [n, n+1) \cap [m,m+1)\), from which we conclude \(n = m\), as there are no natural numbers between a natural number and its successor and this intersection would be empty otherwise.

    The contrapositive yields \(n \neq m \implies F^x_n \cap F^x_m = \emptyset,\) then \(F^x\) is the disjoint union
    \begin{equation*}
        F^x = \bigcupdot_{n=1}^\infty F^x_n.
    \end{equation*}
    Indeed, let \(v \in F^x\), then there exists \(\lambda \geq 1\) such that \(\abs*{\inner{x}{v}}^2 = \frac{\norm{x}^2}{\lambda}\). Since the natural numbers are not bounded above, there exists \(n \in \mathbb{N}\) such that \(n \leq \lambda < n+1\), that is, \(v \in F^x_n \subset \bigcupdot_{n=1}^\infty F^x_n\). This yields \(F^x \subset \bigcupdot_{n=1}^\infty F^x_n\), which proves our claim, as \(F^x_n \subset F^x\) by construction. We have thus shown \(F^x\) is countable, as it is a countable disjoint union of finite sets.
\end{proof}

We finally show the linear span of a complete orthonormal basis is dense in the Hilbert space, concluding it is a topological basis.
\begin{theorem}{Complete orthonormal basis is a topological basis}{closure_basis}
    Let \(F\) be a complete orthonormal basis on a Hilbert space \(\hilbert\). For each \(x \in \hilbert\), there exists a sequence \(\family{e_n}{n\in \mathbb{N}} \subset F\) such that \(x \in \cl(\lspan \family{e_n}{n\in \mathbb{N}})\), with
    \begin{equation*}
        x = \sum_{n=1}^\infty \inner{e_n}{x} e_n\quad\text{and}\quad
        \norm{x}^2 = \sum_{n=1}^\infty \abs*{\inner{e_n}{x}}^2.
    \end{equation*}
\end{theorem}
\begin{proof}
    Let \(x \in \hilbert\), then by \cref{lem:countable_orthonormal} the subset
    \begin{equation*}
        F^x = \setc{v \in F}{\inner{x}{v} \neq 0}
    \end{equation*}
    is a countable subset of \(F\). Let \(\family{e_n}{n\in \mathbb{N}} \subset F^x\) be an enumeration of \(F^x\), then we consider the sequence \(\family{\tilde{x}_n}{n\in \mathbb{N}} \subset \lspan F^x\), where
    \begin{equation*}
        \tilde{x}_n = \sum_{i=1}^n \inner{e_i}{x}e_i.
    \end{equation*}
    By \nameref{lem:Bessel_inequality}, we have \(\sum_{i=1}^\infty \abs*{\inner{e_i}{x}}^2 \leq \norm{x}^2\), then by \cref{prop:convergent_series}, we conclude \(\family{\tilde{x}_n}{n\in \mathbb{N}}\) converges to some \(\tilde{x} \in \hilbert\).

    Let \(v \in F\) and consider \(\tilde{x} - x \in \hilbert\). Continuity of the inner product yields
    \begin{equation*}
        \inner{v}{x - \tilde{x}} = \inner{v}{x} - \lim_{n\to\infty}\inner{v}{\tilde{x}_n} = \inner{v}{x}{ - \lim_{n\to\infty}} \sum_{i=1}^n \inner{e_i}{x}\inner{v}{e_i}.
    \end{equation*}
    Clearly, we have the disjoint union \(F = (F \setminus F^x) \cup F^x\), that is, either \(v \in F \setminus F^x\) or \(v \in F^x\). In the former, \(\inner{x}{v} = 0\) by construction, and \(\inner{v}{e_i} = 0\) for all \(i \in \mathbb{N}\), since \(F\) is orthonormal, hence \(\inner{v}{x - \tilde{x}} = 0\). In the latter, \(v = e_j\) for some \(j \in \mathbb{N}\), which yields \(\inner{v}{e_i} = \delta_{ji}\), then
    \begin{equation*}
        \inner{v}{x - \tilde{x}} = \inner{e_j}{x} - \lim_{n\to\infty} \sum_{i=1}^n \inner{e_i}{x} \delta_{ji} = \inner{e_j}{x} - \inner{e_j}{x} = 0.
    \end{equation*}
    We have shown \(\inner{v}{x - \tilde{x}} = 0\) for all \(v \in F\). That is, \(x - \tilde{x}\) is a vector orthogonal to every vector of \(F\), hence \(x = \tilde{x}\) since \(F\) is a complete orthonormal basis.

    By \cref{prop:best_approximation_finite}, for every \(n \in \mathbb{N}\), \(\tilde{x}_n\) is the best approximation of \(x\) in the finite-dimensional linear subspace generated by the finite orthonormal set \ffamily{e_i}{i=1}{n} and satisfies
    \begin{equation*}
        \norm{x - \tilde{x}_n}^2 = \norm{x}^2 - \sum_{i=1}^n \abs*{\inner{e_i}{x}}^2.
    \end{equation*}
    Let \(\varepsilon > 0\), then there exists \(N \in \mathbb{N}\) such that \(\norm{x - \tilde{x}_n} < \sqrt{\varepsilon}\) for all \(n \geq N\). Consequently,
    \begin{equation*}
        n \geq N \implies \abs*{\norm{x}^2 - \sum_{i=1}^n \abs*{\inner{e_i}{x}}^2} < \varepsilon,
    \end{equation*}
    that is,
    \begin{equation*}
        \sum_{i=1}^\infty \abs*{\inner{e_i}{x}}^2 = \norm{x}^2
    \end{equation*}
    as desired.
\end{proof}

Furthermore, we show the equivalence of orthonormal topological basis and complete orthonormal basis.
    \begin{theorem}{A orthonormal topological basis is a complete orthonormal basis}{complete_topological_basis}
    Let \(F\) be an orthonormal subset of a Hilbert space \(\hilbert\). The following statements are equivalent
    \begin{enumerate}[label=(\alph*)]
        \item \(F\) is a complete orthonormal basis for \(\hilbert\);
        \item \(F\) is a topological basis for \(\hilbert\); and
        \item For all \(x \in \hilbert\), the set \(F^x = \setc{v \in F}{\inner{x}{v} \neq 0}\) is countable and \(\norm{x}^2 = \sum_{v \in F^x} \abs*{\inner{v}{x}}^2\).
    \end{enumerate}
\end{theorem}
\begin{proof}
    Along with \cref{lem:linear_independent_orthogonal,lem:countable_orthonormal}, \cref{thm:closure_basis} shows \(\text{(a)} \implies \text{(b)}\) and \(\text{(a)}\implies \text{(c)}\). It remains to show \(\text{(b)}\implies \text{(a)}\) and \(\text{(c)}\implies \text{(a)}\).

    Suppose (b) holds. Suppose, by contradiction, there exists \(x \in \hilbert\setminus \set{0}\) that is orthogonal to every vector in \(F\). In particular, any vector in \(F\) is orthogonal to \(x,\) that is, \(F \subset \set{x}^\perp\). Since the orthogonal complement is a closed linear subspace, we have \(F \subset \lspan F \subset \subset \cl(\lspan F) \subset \set{x}^\perp\) since the linear span of \(F\) is the smallest linear subspace that contains \(F\) and since the closure of \(\lspan F\) is the smallest subset that contains \(\lspan F\). By hypothesis, \(F\) is a topological basis, thus \(\hilbert \subset \set{x}^\perp\), that is, \(x \neq 0\) is orthogonal to every vector in the Hilbert space. This contradiction shows no such \(x\) exists, therefore \(F\) is a complete orthonormal basis, hence \(\text{(b)}\implies \text{(a)}\).

    Suppose (c) holds. Suppose, by contradiction, there exists \(x \in \hilbert\setminus\set{0}\) that is orthogonal to every vector in \(F\). By construction, \(F^x = \emptyset\) and, by (c), \(\norm{x}^2 = 0\). This contradiction, shows \(F\) is a complete orthonormal basis, hence \(\text{(c)}\implies \text{(a)}\).
\end{proof}

% vim: spl=en_us
\section{Separable Hilbert spaces}
In \cref{exam:dense_l2} we've constructed a countable complete orthonormal basis for the Hilbert space of square-summable sequences \(\ell_2\). Then, it was a countable topological basis, therefore \(\ell_2\) is a \emph{separable topological space}, that is, it has a dense subset that is countable. In fact, we'll see \(\ell_2\) is the only such infinite-dimensional \emph{separable Hilbert space} up to unitary equivalence, which will be later defined.
\begin{definition}{Separable Hilbert space}{separable_Hilbert_space}
    A Hilbert space \(\hilbert\) is \emph{separable} if it is a separable topological space, with respect to the metric topology, that is, if there exists a countable dense subset \(X \subset \hilbert\).
\end{definition}

Our aim is to relate the separability of a Hilbert space with the collection of its complete orthonormal basis. Before we are ready to show this, we construct a dense subset of a closure of a linear span. For any subset \(F\) of a Hilbert space \(\hilbert\), we denote \(\lspan_\mathbb{Q}F\) as the linear span of \(F\) by rationals, that is, the subset
\begin{equation*}
    \lspan_\mathbb{Q}F = \setc*{\sum_{i=1}^n r_i v_i}{n \in \mathbb{N}, \ffamily{r_i}{i=1}{n} \subset \mathbb{Q}\mathbb{C}, \ffamily{v_i}{i=1}{n}},
\end{equation*}
where \(\mathbb{Q}\mathbb{C} = \setc{x + iy}{x,y \in \mathbb{Q}} \subset \mathbb{C}\) is dense in \(\mathbb{C}\), since \(\mathbb{Q}\) is dense in \(\mathbb{R}\).
\begin{lemma}{The closure of rational linear span is the closure of the linear span}{rational_span}
    Let \(F\) be a subset of a Hilbert space \(\hilbert\). Then \(\cl(\lspan_\mathbb{Q} F) = \cl(\lspan F)\).
\end{lemma}
\begin{proof}
    Notice \(\lspan_\mathbb{Q} F \subset \lspan F\), since every linear combination by rationals is a linear combination. Recall the closure of a subset is the smallest closed set containing the subset, therefore \(\cl(\lspan_\mathbb{Q} F) \subset \cl(\lspan F)\) and \(\lspan F \subset \cl(\lspan F)\).

    Let \(v \in \lspan F\), then for some \(n \in \mathbb{N}\), there exist finite sets \(\ffamily{\lambda_i}{i=1}{n}\subset \mathbb{C}\) and \(\ffamily{v_i}{i=1}{n} \subset F\) such that \(v = \sum_{i=1}^n \lambda_i v_i\). Since \(\mathbb{Q}\mathbb{C}\) is dense in \(\mathbb{C}\), for each \(i \in \set{1, 2,\dots, n}\), there exists a sequence \(r_i : \mathbb{N} \to \mathbb{Q}\mathbb{C}\) that converges to \(\lambda_i\). Consider the sequence
    \begin{align*}
        \tilde{v} : \mathbb{N} &\to \lspan_\mathbb{Q} F\\
                             m &\mapsto \sum_{i = 1}^n r_i(m) v_i,
    \end{align*}
    then for all \(m \in \mathbb{N}\), we have
    \begin{equation*}
        \norm{v - \tilde{v}_m} = \norm*{\sum_{i=1}^n \left[\lambda_i - r_i(m)\right]v_i} \leq \sum_{i=1}^n\abs*{\lambda_i - r_i(m)}\cdot\norm{v_i}.
    \end{equation*}
    The right hand side can be made arbitrarily small, then we conclude \(\tilde{v}_m \to v\), thus showing \(\lspan F \subset \cl(\lspan F) \subset \cl(\lspan_\mathbb{Q}F)\).
\end{proof}

We are now in a position to show that if a Hilbert space has a countable complete orthonormal basis is a necessary and sufficient condition for its separability.
\begin{theorem}{A complete orthonormal basis of a separable Hilbert space is countable}{countable_separable}
    A Hilbert space is separable if and only if it has a countable complete orthonormal basis.
\end{theorem}
\begin{proof}
    Suppose a Hilbert space \(\hilbert\) has a countable complete orthonormal basis \(F\), then \(\cl(\lspan F) = \hilbert\). By \cref{lem:rational_span}, we also have \(\cl(\lspan_\mathbb{Q} F) = \hilbert\), then \(\lspan_\mathbb{Q} F\) is dense in \(\hilbert\). Notice
    \begin{equation*}
        \lspan_\mathbb{Q} F = \bigcup_{n\in \mathbb{N}} \left[\bigcup_{\ffamily{v_i}{i=1}{n} \subset F} \left(\bigcup_{\ffamily{r_i}{i=1}{n} \subset \mathbb{Q}\mathbb{C}} \set*{\sum_{i=1}^n r_i v_i}\right)\right],
    \end{equation*}
    then, since \(F\) is countable, \todo[\(\lspan_{\mathbb{Q}}F\) is a countable union of countable sets, therefore countable]. We have thus constructed a countable dense subset in \(\hilbert\), therefore \(\hilbert\) is a separable Hilbert space.

    Suppose a Hilbert space \(\hilbert\) is separable, then there exists a countable dense subset \(D\) in \(\hilbert\). Let \(F\) be a complete orthonormal basis for \(\hilbert\) and let
    \begin{equation*}
        F_D = \bigcup_{x \in D} F^x = \bigcup_{x \in D} \setc{v \in F}{\inner{x}{v} \neq 0}.
    \end{equation*}
    By \cref{thm:complete_topological_basis}, \(F_D\) is countable, as a countable union of countable sets. From \cref{thm:closure_basis}, we know for all \(x \in D\) we have \(x \in \cl(\lspan B^x)\), hence \(D \subset \cl(\lspan F_D)\). Since \(D\) is dense in \(\hilbert\), we must have \(\cl(\lspan F_D) = \hilbert\), that is, \(\lspan F_D\) is dense in \(\hilbert\). Since \(F_D\) is a non-empty subset of the orthonormal set \(F\), it is an orthonormal set, therefore it is a linearly independent set. We have thus shown \(F_D\) is a countable orthonormal topological basis for \(\hilbert\), therefore it is a countable complete orthonormal basis.
\end{proof}
\begin{corollary}
    A Hilbert space \(\hilbert\) is separable if and only if every complete orthonormal basis of \(\hilbert\) is countable.
\end{corollary}
\begin{proof}
    Suppose every complete orthonormal basis of \(\hilbert\) is countable. In particular, it has a countable orthonormal basis, then it is separable.

    Suppose \(\hilbert\) is separable. Let \(F\) and \(F_D\) as before, we aim to show \(F_D = F\). Suppose, by contradiction, \(F_D\) is a proper subset of \(F\), then there exists \(v \in F \setminus F_D\) and \(v \neq 0\). Since \(F\) is orthonormal, for all \(u \in F_D \subsetneq F\), we have \(\inner{u}{v} = 0\), as \(u \neq v\). That is, \(v\neq 0\) is a vector orthogonal to every vector of the complete orthonormal basis \(F_D\). This contradiction shows \(F_D = F\), hence \(F\) is a countable complete orthonormal basis. Since the choice of \(F\) was arbitrary, it follows that every complete orthonormal basis of \(\hilbert\) is countable.
\end{proof}
\begin{corollary}
    A Hilbert space is not separable if it has an uncountable orthonormal set.
\end{corollary}
\begin{proof}
    Let \(A\) be an uncountable orthonormal set in a non-trivial Hilbert space \(\hilbert\) and consider the collection \(\mathfrak{O}_A = \setc{S \in \mathfrak{O}}{A \subset S}\) partially ordered by inclusion, where \(\mathfrak{O}\) is the collection of orthonormal sets in \(\hilbert\). Notice \(\mathfrak{O}_A\) is not empty, since \(A \in \mathfrak{O}_A\).

    Let \(\mathcal{E} \subset \mathfrak{O}_A\) be a non-empty linearly ordered subset of \(\mathfrak{O}_A\). As was shown in \cref{thm:complete_orthonormal_basis_exists}, \(\mathcal{E}\) has an upper bound in \(\mathfrak{O}\), defined by \(\bigcup \mathcal{E}\). Since it is the union of sets that contain \(A\), it follows that it contains \(A\), hence it is an upper bound for \(\mathcal{E}\) in \(\mathfrak{O}_A\).

    By \nameref{thm:zorn}, \(\mathfrak{O}_A\) has at least one maximal element, \(F\) say. We claim \(F\) is a complete orthonormal basis that contains \(A\), therefore uncountable. Suppose, by contradiction, \(F\) is not a complete orthonormal basis, then there exists \(v \in \hilbert \setminus \set{0}\) such that \(v \in F^\perp\). In such a case, \(F \cup \set{\frac{1}{\norm{v}}v}\) is an orthonormal set that contains \(F\), and thereby \(A\), contradicting the fact that \(F\) is a maximal element of \(\mathfrak{O}_A\). This shows not every complete orthonormal basis of \(\hilbert\) is countable, therefore \(\hilbert\) is not separable.
\end{proof}

We may restate \cref{thm:closure_basis} for separable Hilbert spaces.
\begin{theorem}{Complete orthonormal basis in a separable Hilbert space}{basis_separable}
    Let \(F\) be a complete orthonormal basis on a separable Hilbert space \(\hilbert\). For any enumeration \(\family{e_n}{n\in \mathbb{N}} \subset F\) of this basis, we have for all \(x \in \hilbert\) that
    \begin{equation*}
        x = \sum_{n=1}^\infty \inner{e_n}{x}e_n\quad\text{and}\quad
        \norm{x}^2 = \sum_{n=1}^\infty \abs*{\inner{e_n}{x}}^2.
    \end{equation*}
\end{theorem}

We now illustrate the fact a countable orthonormal set in a separable Hilbert space is not necessarily a complete orthonormal basis.
\begin{example}{Countable orthonormal set in a separable Hilbert space}{L2_characteristic}
    For \(A \subset \mathbb{R}\), we define the characteristic function \(\chi_A\) of \(A\) by the map defined by \(\chi_A(x) = 1\) if \(x \in A\) and \(\chi_A(x) = 0\) if \(x \notin A\). Let
    \begin{align*}
        \psi : \mathbb{Z} &\to L^2(\mathbb{R}, \dl{x})\\
                        m &\mapsto \chi_{[m, m+1)}
    \end{align*}
    be the sequence of characteristic functions on the intervals \([m, m+1)\). Then \(\family{\psi_n}{n\in \mathbb{Z}}\) is a countable orthonormal set in the separable Hilbert space \(L^2(\mathbb{R}, \dl{x})\), but it is not a complete orthonormal basis.
\end{example}
\begin{proof}
    Notice \(\conj{\psi_n} \psi_m = \delta_{nm} \psi_n\) for all \(n,m \in \mathbb{Z}\). Then \(\family{\psi_n}{n\in \mathbb{Z}}\) is a countable orthonormal set in \(L^2(\mathbb{R}, \dl{x})\) since
    \begin{equation*}
        \inner{\psi_n}{\psi_m} = \int_{\mathbb{R}} \dli{x} \conj{\psi_n(x)}\psi_m(x) = \delta_{nm} \int_{\mathbb{R}} \dli{x}\chi_{[n, n+1)}(x) = \delta_{nm}
    \end{equation*}
    for all \(n,m \in \mathbb{Z}\).

    Consider the function \(\phi : \mathbb{R} \to \mathbb{C}\) defined by \(x \mapsto \sin(2\pi x)\psi_1(x)\). Notice \(\phi \in L^2(\mathbb{R}, \dl{x})\) since
    \begin{equation*}
        \int_{\mathbb{R}} \dli{x} \abs*{\sin(2\pi x) \psi_0(x)}^2 = \int_{\mathbb{R}} \dli{x} \sin^2(2\pi x) \chi_{[0,1)} = \int_{[0,1)} \dli{x} \sin^2(2\pi x) = \frac1{2},
    \end{equation*}
    and we have \(\norm{\phi} = \frac1{\sqrt{2}}.\) Notice \(\phi \in \family{\psi_n}{n\in \mathbb{Z}}^\perp\). Indeed,
    \begin{equation*}
        \inner{\psi_n}{\phi} = \int_{\mathbb{R}} \dli{x} \chi_{[n,n+1)}(x) \chi_{[0,1)}(x) \sin(2\pi x) = \delta_{n0} \int_{[0,1)} \dli{x} \sin(2\pi x) = 0
    \end{equation*}
    for all \(n \in \mathbb{Z}\). Hence, \(\family{\psi_n}{n\in \mathbb{Z}}\) is not a complete orthonormal basis.
\end{proof}

% vim: spl=en_us
\section{Riesz's representation theorem}
The orthogonal decomposition theorem has a very useful consequence in the form of Riesz's representation theorem, that states that a continuous linear functional on a Hilbert space can be uniquely represented by a vector.
\begin{theorem}{Riesz's representation theorem}{riesz_representation}
    Let \(\hilbert\) be a Hilbert space. If \(\ell \in \hilbert^\dag\), then there exists a unique \(\psi \in \hilbert\), called the \emph{Riesz representation of \(\ell\)}, such that
    \begin{equation*}
        \ell(x) = \inner{\psi}{x}
    \end{equation*}
    for all \(x \in \hilbert\).
\end{theorem}
\begin{proof}
    We consider the kernel \(\ker \ell\) of the continuous linear functional \(\ell\). Notice \(\ker\ell\) is closed as it is the preimage of the closed set \(\set{0} \subset \mathbb{C}\) under a continuous map. That is, \(\ker \ell \subset \hilbert\) is a closed subspace of \(\hilbert\). If \(\ker \ell = \hilbert\), then \(\psi = 0\) satisfies the claim, since it is the unique vector that is orthogonal to every vector in \(\hilbert\) by non-degeneracy of the inner product. Henceforth we may assume \(\ker \ell \subsetneq \hilbert\), then the closed subspace \(\ker\ell^\perp\) is non-trivial. Indeed, by \cref{thm:orthogonal_decomposition}, for all \(x \in \hilbert\) we have the unique decomposition \(x = x_\parallel + x_\perp\) where \(x_\parallel \in \ker \ell\) and \(x_\perp \in \ker \ell^\perp\). If \(\ker\ell^\perp\) were trivial, then \(x_\perp = 0\) for all \(x,\) that is, \(\ker \ell = \hilbert\), which contradicts our hypothesis, proving our claim.

    Let \(z \in \ker\ell^\perp\setminus\set{0}\), then \(\ell(z) \neq 0\). For all \(x \in \hilbert\) we have \(\ell(z)x - \ell(x)z \in \ker\ell\). Indeed,
    \begin{equation*}
        \ell\left(\ell(z)x - \ell(x)z\right) = \ell(z)\ell(x) - \ell(x)\ell(z) = 0.
    \end{equation*}
    It follows immediately that \(z\) and \(\ell(z)x - \ell(x)z\) are orthogonal, then
    \begin{equation*}
        \inner{z}{\ell(z)x - \ell(x)z} = \ell(z) \inner{z}{x} - \ell(x) \norm{z}^2 = 0 \implies \ell(x) = \inner*{\frac{\conj{\ell(z)}}{\norm{z}^2}z}{x},
    \end{equation*}
    for all \(x \in \hilbert\). That is, defining
    \begin{equation*}
        \psi = \frac{\conj{\ell(z)}}{\norm{z}^2}z
    \end{equation*}
    yields \(\ell(x) = \inner{\psi}{x}\) for all \(x \in \hilbert\) as desired.

    Since \(z\) is arbitrary, we have to show well definition. Let \(\tilde{z} \in \ker\ell^\perp\setminus\set{0}\). First, notice \(\lspan\set{z}\) is a closed subset, then \(\tilde{z} = \tilde{z}_\parallel + \tilde{z}_\perp\), with \(\tilde{z}_\parallel \in \lspan\set{z}\) and \(\tilde{z}_\perp \in \lspan\set{z}^\perp\). We must have \(\tilde{z}_\perp \in \ker\ell^\perp\), otherwise \(\tilde{z} \notin \ker\ell^\perp\), since \(\ker\ell^\perp\) is a subspace. Notice \(\ell(\tilde{z}_\perp) = 0\),
    \begin{equation*}
        \ell(\tilde{z}_\perp) = \inner*{\frac{\conj{\ell(z)}}{\norm{z}^2}z}{\tilde{z}_\perp} = \frac{\ell(z)}{\norm{z}^2} \inner{z}{\tilde{z}_\perp} = 0.
    \end{equation*}
    This implies \(\tilde{z}_\perp = 0\), since \(\tilde{z}_\perp \in \ker\ell \cap \ker\ell^\perp\). We have thus shown \(\ker\ell^\perp = \lspan{z}\), that is, there exists \(\alpha \in \mathbb{C}\setminus \set{0}\) such that \(\tilde{\tilde{z}} = \alpha z\), then
    \begin{equation*}
        \tilde{\psi} = \frac{\conj{\ell(\tilde{z})}}{\norm{\tilde{z}^2}}\tilde{z} = \frac{\conj{\alpha \ell(z)}}{\abs{\alpha}^2\norm{z}^2}\alpha z = \frac{\conj{\ell(z)}}{\norm{z}^2}z = \psi,
    \end{equation*}
    which shows well definition of \(\psi\).

    Finally, we show uniqueness. Suppose there exists \(\psi'\in \hilbert\) such that \(\ell(x) = \inner{\psi'}{x}\) for all \(x \in \hilbert\). Then, \(\inner{\psi}{x} = \inner{\psi'}{x}\), which yields \(\inner{\psi - \psi'}{x} = 0\) for all \(x \in \hilbert\). By non-degeneracy we have \(\psi - \psi' = 0\), thus concluding the proof.
\end{proof}
\begin{corollary}
    Let \(\ell \in \hilbert^\dag\setminus\set{0}\) be a non zero continuous linear functional, then \(\ker \ell^\perp\) is the one-dimensional linear subspace spanned by \(\psi\).
\end{corollary}

Notice \nameref{thm:riesz_representation} defines a map
\begin{align*}
    \riesz : \hilbert^\dag &\to \hilbert\\
                           \ell &\mapsto \psi_\ell,
\end{align*}
where \(\psi_\ell\) is the Riesz representation of \(\ell\). Moreover, the theorem shows the map is injective. In fact, it is bijective. Indeed, let \(\phi \in \hilbert,\) then
\begin{align*}
    \ell_\phi : \hilbert &\to \mathbb{C}\\
                       x &\mapsto \inner{\phi}{x}
\end{align*}
defines a continuous linear functional with \(\riesz(\ell_\phi) = \phi\), thus the range of \(\riesz\) is \(\hilbert\). Moreover, the association \(\phi\mapsto\ell_\phi\) defines the inverse map \(\riesz^{-1} : \hilbert \to \hilbert^\dag\). Notice, however, \(\riesz\) and \(\riesz^{-1}\) are \emph{antilinear} maps. Indeed, let \(f, g \in \hilbert^\dag\) and let \(\alpha, \beta \in \mathbb{C}\), then for all \(x \in \hilbert\) we have
\begin{equation*}
    (\alpha f + \beta g)(x) = \alpha \inner{\riesz(f)}{x} + \beta \inner{\riesz(g)}{x} = \inner{\conj{\alpha}\riesz(f) + \conj{\beta}\riesz(g)}{x},
\end{equation*}
that is, \(\riesz(\alpha f + \beta g) = \conj{\alpha}\riesz(f) + \conj{\beta}\riesz(g)\).  Let \(u, v \in \hilbert\), then for all \(x \in \hilbert\)
\begin{equation*}
    \riesz^{-1}(\alpha u + \beta v)(x) = \inner{\alpha u + \beta v}{x} = \conj{\alpha} \inner{u}{x} + \conj{\beta}\inner{v}{x} = \left(\conj{\alpha}\riesz^{-1}(u)+ \conj{\beta}\riesz^{-1}(v)\right)(x),
\end{equation*}
then \(\riesz^{-1}(\alpha u + \beta v) = \conj{\alpha}\riesz^{-1}(u) + \conj{\beta}\riesz^{-1}(v)\).

We may use the Riesz representation map to define an inner product on \(\hilbert^\dag\).
\begin{proposition}{Inner product on topological dual}{inner_product_dual}
    Let \((\hilbert, \inner{\noarg}{\noarg})\) be a Hilbert space. The map
    \begin{align*}
        \inner{\noarg}{\noarg}_{\hilbert^\dag} : \hilbert^\dag \times \hilbert^\dag &\to \mathbb{C}\\
        (f,g) &\mapsto \inner{\riesz(g)}{\riesz(f)}
    \end{align*}
    is an inner product on the topological dual \(\hilbert^\dag\).
\end{proposition}
\begin{proof}
    Let \(f, g, h \in \hilbert^\dag\) and let \(\alpha, \beta \in \mathbb{C}\), then by the antilinearity of the Riesz representation map, we have
    \begin{equation*}
        \inner{h}{\alpha f + \beta g}_{\hilbert^\dag} = \inner{\riesz(\alpha f + \beta g)}{\riesz(h)} = \inner{\conj{\alpha}\riesz(f) + \conj{\beta}\riesz(g)}{\riesz(h)}.
    \end{equation*}
    By linearity in the second argument of the inner product in \(\hilbert\), we conclude
    \begin{equation*}
        \inner{h}{\alpha f + \beta g}_{\hilbert^\dag} = \alpha \inner{\riesz(f)}{\riesz(h)} + \beta \inner{\riesz(g)}{\riesz(h)} = \alpha \inner{h}{f}_{\hilbert^\dag} + \beta \inner{h}{g}_{\hilbert^\dag},
    \end{equation*}
    hence \(\inner{\noarg}{\noarg}_{\hilbert^\dag}\) is linear in the second argument. The conjugate symmetry of the inner product in \(\hilbert\) yields
    \begin{equation*}
        \inner{g}{f}_{\hilbert^\dag} = \inner{\riesz(f)}{\riesz(g)} = \conj{\inner{\riesz(g)}{\riesz(f)}} = \conj{\inner{f}{g}_{\hilbert^\dag}},
    \end{equation*}
    hence, \(\inner{\noarg}{\noarg}_{\hilbert^\dag}\) is conjugate symmetric. Finally, positive-definiteness of the inner product in \(\hilbert\) induces positive-definiteness on \(\inner{\noarg}{\noarg}_{\hilbert^\dag}\). Indeed,
    \begin{equation*}
        \inner{f}{f}_{\hilbert^\dag} = \inner{\riesz(f)}{\riesz(f)} \geq 0
    \end{equation*}
    with equality being equivalent to \(\riesz(f) = 0\). Since \(\riesz\) is an antilinear isomorphism, we have shown this map is a inner product on \(\hilbert^\dag\).
\end{proof}
\begin{remark}
    Notice \(\inner{f}{g}_{\hilbert^\dag} = g(\riesz(f))\) for all \(f, g \in \hilbert^\dag\).
\end{remark}

We have shown that \((\hilbert^\dag, \inner{\noarg}{\noarg}_{\hilbert^\dag})\) is a pre-Hilbert space and we already know \((\hilbert^\dag, \norm{\noarg})\) is a Banach space with respect to the operator norm \(\norm{\noarg}\). In fact, \((\hilbert^\dag, \inner{\noarg}{\noarg}_{\hilbert^\dag})\) is a Hilbert space, since the operator norm is actually induced by this inner product.
\begin{theorem}{Topological dual is a Hilbert space}{dual_hilbert}
    The norm \(\norm{\noarg}_{\hilbert^\dag}\) induced by the inner product \(\inner{\noarg}{\noarg}_{\hilbert^\dag}\) is equal to the operator norm on the topological dual \(\hilbert^\dag\) of a Hilbert space \(\hilbert\).
\end{theorem}
\begin{proof}
    Trivially, \(\norm{0}_{\hilbert^\dag} = \norm{0}\). Let \(\ell \in \hilbert^\dag\setminus\set{0}\), then
    \begin{equation*}
        \norm{\ell} = \sup_{x \in \hilbert} \frac{\abs{\ell(x)}}{\norm{x}} = \sup_{x \in \hilbert} \frac{\abs{\inner{\riesz(\ell)}{x}}}{\norm{x}}.
    \end{equation*}
    By the Cauchy-Schwarz inequality, we have
    \begin{equation*}
        \norm{\ell} \leq \sup_{x\in \hilbert} \norm{\riesz(\ell)} = \norm{\ell}_{\hilbert^\dag}.
    \end{equation*}
    At the same time, the supremum yields
    \begin{equation*}
        \norm{\ell} \geq \frac{\abs{\inner{\riesz(\ell)}{\riesz(\ell)}}}{\norm{\riesz(\ell)}} = \norm{\ell}_{\hilbert^\dag}.
    \end{equation*}
    Hence, the norms are equal.
\end{proof}

As an immediate consequence, we show every Hilbert space is reflexive.
\begin{theorem}{Hilbert spaces are reflexive}{hilbert_reflexive}
    If \((\hilbert, \inner{\noarg}{\noarg})\) is a Hilbert space, then \(\eval(\hilbert) = (\hilbert^\dag)^\dag\), that is, \(\hilbert\) is reflexive.
\end{theorem}
\begin{proof}
    Since \((\hilbert, \inner{\noarg}{\noarg})\) and \((\hilbert, \inner{\noarg}{\noarg}_{\hilbert^\dag})\) are Hilbert spaces, \nameref{thm:riesz_representation} yields the antilinear bijective representation maps \(\riesz : \hilbert^\dag \to \hilbert\) and \(\mathscr{S} : (\hilbert^\dag)^\dag \to \hilbert^\dag\), with antilinear inverse maps.

    We consider the composition \(\mathscr{S}^{-1} \circ \riesz^{-1} : \hilbert \to (\hilbert^\dag)^\dag\). Let \(u, v \in \hilbert\) and \(\alpha, \beta \in \mathbb{C}\), then
    \begin{equation*}
        \mathscr{S}^{-1} \circ \riesz^{-1} (\alpha u + \beta v) = \mathscr{S}^{-1}\left(\conj{\alpha} \riesz^{-1}(u) + \conj{\beta}\riesz^{-1}(v)\right) = \alpha \mathscr{S}^{-1} \circ \riesz^{-1}(u) + \beta \mathscr{S}^{-1} \circ \riesz^{-1}(v),
    \end{equation*}
    that is, the composition is linear. As a composition of bijections, it is a bijection, hence \(\mathscr{S}^{-1} \circ \riesz^{-1}\) is a linear isomorphism.

    This composition is, in fact, the evaluation map \(\eval : \hilbert \to (\hilbert^\dag)^\dag\). Indeed, let \(u \in \hilbert\), then for all \(\ell \in \hilbert^\dag\)
    \begin{align*}
        \left(\mathscr{S}^{-1}\circ \riesz^{-1}(u)\right)(\ell) &= \inner{\riesz^{-1}(u)}{\ell}_{\hilbert^\dag}\\
                                                                &= \inner{\riesz(\ell)}{u}\\
                                                                &= \ell(u),
    \end{align*}
    that is \(\mathscr{S}^{-1} \circ \riesz^{-1}(u) = \eval(u)\), as claimed. Thus, the evaluation map is a linear isomorphism.
\end{proof}

% vim: spl=en_us
\section{Bounded operators in Hilbert spaces}
We recall \crefname{thm:orthogonal_decomposition}, that for every vector \(v \in \hilbert\), there exist unique \(v_\parallel \in V\) and \(v_\perp \in V^\perp\) such that \(v = v_\parallel + v_\perp\), and \(v_\parallel\) is the unique best approximation of \(v\) on \(V\). This motivates the definition of a map that realizes the best approximation.
\begin{definition}{Orthogonal projector}{orthogonal_projector}
    Let \(\hilbert\) be a Hilbert space. If \(V \subset \hilbert\) is a closed linear subspace of \(\hilbert\), the \emph{orthogonal projector onto \(V\)} is the map
    \begin{align*}
        P_V : \hilbert &\to \hilbert\\
                     v &\mapsto v_\parallel
    \end{align*}
    where \(v_\parallel \in V\) is the unique best approximation of \(v\) on \(V\). If \(x \in \hilbert\), we may write \(P_x\) as shorthand for the orthogonal projector \(P_{\lspan{\set{x}}} : \hilbert \to \hilbert\).
\end{definition}

\begin{proposition}{Orthogonal projector is a bounded operator}{orthogonal_projector_bounded}
    Let \(\hilbert\) be a Hilbert space. If \(V\) is a closed linear subspace of \(\hilbert\), then \(P_V \in \bounded(\hilbert)\) with \(\norm{P_V} \leq 1\).
\end{proposition}
\begin{proof}
    Let \(x, y \in \hilbert\) and \(\alpha \in \mathbb{C}\), then \cref{thm:orthogonal_decomposition} guarantees the existence and uniqueness of \(x_\parallel, y_\parallel \in V\) and \(x_\perp, y_\perp \in V^\perp\) such that \(x = x_\parallel + x_\perp\) and \(y = y_\parallel + y_\perp\), then
    \begin{equation*}
        x + \alpha y = (x_\parallel + \alpha y_\parallel) + (x_\perp + \alpha y_\perp).
    \end{equation*}
    That is, \(x_\parallel + \alpha y_\parallel\) is the best approximation of \(x + \alpha y\) in \(V\), hence \(P_V(x + \alpha y) = P_V(x) + \alpha P_V(y)\). It is clear \(P_V\) is bounded since
    \begin{equation*}
        \sup_{v \in \hilbert} \frac{\norm{P_Vv}}{\norm{v}} = \sup_{v \in \hilbert} \frac{\norm{v_\parallel}}{\sqrt{\norm{v_\parallel}^2 + \norm{v_\perp}^2}} \leq 1,
    \end{equation*}
    that is, \(\norm{P_V} \leq 1\).
\end{proof}

An important consequence of the orthogonal decomposition theorem is that every bounded linear operator defined on a subset of a Hilbert space can be extended to the entire linear space.
\begin{theorem}{Isometric extension of a bounded operator defined on a Hilbert space}{isometric_extension_bounded_hilbert}
    Let \(T : \domain{T} \subset \hilbert \to X\) be a bounded linear operator, where \(\hilbert\) is a Hilbert space and \(X\) is a Banach space. Then \(T\) can be uniquely extended to \(\hat{T} : \hilbert \to X\), where the extension satisfies \(\hat{T}((\cl_\hilbert\domain{T})^\perp) = \set{0}\) and \(\norm{\hat{T}} = \norm{T}\).
\end{theorem}
\begin{proof}
    Notice \(\domain{T}\) is dense in the Banach space \(V = \cl_\hilbert{\domain{T}}.\) Using \cref{thm:blt} if necessary, there exists a unique bounded linear extension \(\tilde{T} : V \to X\) to \(T\) such that \(\norm{\tilde{T}} = \norm{T}\).

    We define the map
    \begin{align*}
        \hat{T} : \hilbert &\to X\\
                         v &\mapsto \tilde{T} \circ P_V (v),
    \end{align*}
    which is manifestly a bounded linear operator, as a composition of bounded linear operators. Notice \(P_Vu = u\) for all \(u \in \domain{T}\subset V\), since \(u\) is its best approximation in \(V\). Then, since \(\tilde{T}\) extends \(T\), \(\hat{T}\) extends \(T\). As \(\ker{P_V} = V^\perp\), it follows that \(\hat{T}(V^\perp) = \set{0}\). Moreover, we have
    \begin{equation*}
        \norm{\hat{T}} = \sup_{v \in \hilbert}{\frac{\norm{\hat{T}v}}{\norm{v}}}= \sup_{v \in \hilbert}{\frac{\norm{\tilde{T}v_\parallel}}{\sqrt{\norm{v_\parallel}^2 + \norm{v_\perp}^2}}} \leq \sup_{v\in V}{\frac{\norm{\tilde{T}v}}{\norm{v}}} = \norm{\tilde{T}} = \norm{T}
    \end{equation*}
    and
    \begin{equation*}
        \norm{\hat{T}} = \sup_{v \in \hilbert}{\frac{\norm{\hat{T}v}}{\norm{v}}} \geq \sup_{v\in\domain{T}}{\frac{\norm{Tv}}{\norm{v}}} = \norm{T},
    \end{equation*}
    and we conclude \(\norm{\hat{T}} = \norm{T}\).

    Suppose there exists \(S \in \bounded(\hilbert, X)\) that extends \(T\) such that \(\norm{S} = \norm{T}\) and \(S(V^\perp) = \set{0}\). Then, \(\restrict{S}{V}\) is a bounded extension to \(T\) with \(\norm{\restrict{S}{V}} = \norm{T}\), hence the uniqueness of \(\tilde{T}\) yields \(\restrict{S}{V} = \tilde{T}\). For every \(v \in V\) we have
    \begin{equation*}
        S(v) = S\circ P_V(v) + S(v - P_V(v)) = \restrict{S}{V} \circ P_V(v) = \tilde{T}\circ P_V (v) = \hat{T}(v),
    \end{equation*}
    then \(S = \hat{T}\), showing uniqueness.
\end{proof}
Due to \cref{thm:isometric_extension_bounded_hilbert}, we can always assume a \emph{bounded} linear operator is defined on an entire Hilbert space, rather than just a linear subspace.

\subsection{Adjoint of a bounded operator}
We consider bounded operators \(A : \hilbert_1 \to \hilbert_2\), where \(\hilbert_1\) and \(\hilbert_2\) are Hilbert spaces. As a consequence of \nameref{thm:riesz_representation}, we'll show any bounded operator \(A \in \bounded(\hilbert_1, \hilbert_2)\) defines a unique operator \(B \in \bounded(\hilbert_2, \hilbert_1)\) such that \(\inner{y}{Ax}_{\hilbert_2} = \inner{By}{x}_{\hilbert_1}\), for all \(x \in \hilbert_1\) and \(y \in \hilbert_2\).
\begin{lemma}{Bicontinuous sesquilinear form defines a unique operator}{sesquilinear_adjoint}
    Let \(\mathscr{A} : \hilbert_2 \times \hilbert_1 \to \mathbb{C}\) be a sesquilinear form defined on Hilbert spaces \(\hilbert_1\) and \(\hilbert_2\). Suppose \(\mathscr{A}\) is \emph{bicontinuous}, that is, there exists \(M > 0\) such that \(\abs{\mathscr{A}(u,v)} \leq M\norm{u}_{\hilbert_2} \norm{v}_{\hilbert_1}\) for all \(u \in \hilbert_2\) and all \(v \in \hilbert_1\). Then, there exists a unique linear operator \(B \in \bounded(\hilbert_2, \hilbert_1)\) such that
    \begin{equation*}
        \mathscr{A}(u,v) = \inner{Bu}{v}_{\hilbert_1},
    \end{equation*}
    for all \(u \in \hilbert_2\) and all \(v \in \hilbert_1\).
\end{lemma}
\begin{proof}
    As \(\mathscr{A}\) is bicontinuous, it follows that for each \(u \in \hilbert_2\), the map
    \begin{align*}
        \ell_u : \hilbert_1 &\to \mathbb{C}\\
                          v &\mapsto \mathscr{A}(u,v)
    \end{align*}
    is a bounded linear functional on \(\hilbert_1\). That is, there exists a map \(\ell : \hilbert_2 \to \hilbert_1^\dag\) defined by \(\ell(u) = \ell_u\). Let \(u_1, u_2 \in \hilbert_2\) and \(\alpha \in \mathbb{C}\), then
    \begin{align*}
        \ell(u_1 + \alpha u_2)(v) = \ell_{u_1 + \alpha u_2}(v) &= \mathscr{A}(u_1 + \alpha u_2, v)\\
                                                               &= \mathscr{A}(u_1, v) + \conj{\alpha} \mathscr{A}(u_2, v)\\
                                                               &= \ell_{u_1}(v) + \conj{\alpha} \ell_{u_2}(v)\\
                                                               &= \left[\ell(u_1) + \conj{\alpha} \ell(u_2)\right](v)
    \end{align*}
    for all \(v \in \hilbert_1\). We infer \(\ell(u_1 + \alpha u_2) = \ell(u_1) + \conj{\alpha}\ell(u_2)\), thus showing \(\ell\) is an antilinear map.

    We claim the map \(B = \riesz \circ \ell\) is a bounded linear operator such that \(\mathscr{A}(u,v) = \inner{B(u)}{v}_{\hilbert_1}\), where \(\riesz : \hilbert_1^\dag \to \hilbert_1\) is the Riesz representation map. \nameref{thm:riesz_representation} ensures that \(B(u)\) is the unique vector in \(\hilbert_1\) such that \(\ell_u(v) = \inner{B(u)}{v}_{\hilbert_1}\) for all \(v \in \hilbert_1\). As a result, for all \(u \in \hilbert_2\) and \(v \in \hilbert_1\), we have \(\inner{B(u)}{v}_{\hilbert_1} = \mathscr{A}(u,v)\). It is clear \(B\) is a linear map since it is the composition of two antilinear maps. Moreover, since \(\mathscr{A}\) is bicontinuous, we have
    \begin{equation*}
        \norm{Bu}_{\hilbert_1}^2 = \abs*{\inner{Bu}{Bu}_{\hilbert_1}} = \abs*{\mathscr{A}(u, Bu)} \leq M \norm{u}_{\hilbert_2} \norm{Bu}_{\hilbert_1},
    \end{equation*}
    that is, \(M\) is an upper bound for the set \(\setc*{\frac{\norm{Bu}_{\hilbert_1}}{\norm{u}_{\hilbert_2}}}{u \in \hilbert_2 \setminus \set{0}}\), hence \(B \in \bounded(\hilbert_2, \hilbert_1)\), proving our claim.

    Let \(\tilde{B} \in \bounded(\hilbert_2, \hilbert_1)\) such that \(\mathscr{A}(u,v) = \inner{\tilde{B}u}{v}_{\hilbert_1}\) for all \(u \in \hilbert_2\) and \(v \in \hilbert_1\). Then, \(\inner{Bu}{v}_{\hilbert_1} = \inner{\tilde{B}u}{v}_{\hilbert_1}\), which implies \(\inner{(B - \tilde{B})u}{v}_{\hilbert_1} = 0\) for all \(u \in \hilbert_2\) and all \(v \in \hilbert_1\). Since the inner product is non-degenerate this yields \((B - \tilde{B})u = 0\) for all \(u \in \hilbert_2\), hence \(B = \tilde{B}\), which shows uniqueness.
\end{proof}

\begin{theorem}{Existence of the adjoint of a bounded operator}{adjoint_bounded_hilbert}
    Let \(\hilbert_1, \hilbert_2\) be Hilbert spaces. If \(A \in \bounded(\hilbert_1, \hilbert_2)\) is a bounded linear operator, then there exists a unique bounded linear operator \(B \in \bounded(\hilbert_2, \bounded_1)\) such that
    \begin{equation*}
        \inner{y}{Ax}_{\hilbert_2} = \inner{By}{x}_{\hilbert_1}
    \end{equation*}
    for all \(x \in \hilbert_1\) and \(y \in \hilbert_2\).
\end{theorem}
\begin{proof}
    We claim the map
    \begin{align*}
        \mathscr{A} : \hilbert_2 \times \hilbert_1 &\to \mathbb{C}\\
                                             (y,x) &\mapsto \inner{y}{Ax}_{\hilbert_2}
    \end{align*}
    is a bicontinuous sesquilinear form. Sesquilinearity of \(\mathscr{A}\) follows from linearity of the operator \(A\) and from sesquilinearity of the inner product defined on \(\hilbert_2\). \nameref{thm:cauchy_schwarz} yields
    \begin{equation*}
        \abs{\mathscr{A}(y,x)} = \abs*{\inner{y}{Ax}_{\hilbert_2}} \leq \norm{y}_{\hilbert_2} \norm{Ax}_{\hilbert_2} \leq \norm{A}_{\bounded(\hilbert_1, \hilbert_2)} \norm{x}_{\hilbert_1} \norm{y}_{\hilbert_2},
    \end{equation*}
    for all \(x \in \hilbert_1\) and \(y \in \hilbert_2\), hence \(\mathscr{A}\) is bicontinuous. By \cref{lem:sesquilinear_adjoint}, there exists a unique bounded linear operator \(B \in \bounded(\hilbert_2, \hilbert_1)\) such that \(\mathscr{A}(y,x) = \inner{By}{x}_{\hilbert_1}\), that is,
    \begin{equation*}
        \inner{y}{Ax}_{\hilbert_2} = \inner{By}{x}_{\hilbert_1},
    \end{equation*}
    for all \(x \in \hilbert_1\) and \(y \in \hilbert_2\).
\end{proof}

Notice \cref{thm:adjoint_bounded_hilbert} establishes a map from \(\bounded(\hilbert_1, \hilbert_2)\) to \(\bounded(\hilbert_2, \hilbert_1)\).
\begin{definition}{Adjoint of a bounded operator}{adjoint_bounded_hilbert}
    Let \(\hilbert_1, \hilbert_2\) be Hilbert spaces. The map
    \begin{align*}
        ^* : \bounded(\hilbert_1, \hilbert_2) &\to \bounded(\hilbert_2, \hilbert_1)\\
                                           A &\mapsto A^*,
    \end{align*}
    where \(A^*\) is the unique bounded operator such that
    \begin{equation*}
        \inner{y}{Ax}_{\hilbert_2} = \inner{A^*y}{x}_{\hilbert_1}
    \end{equation*}
    for all \(x \in \hilbert_1\) and \(y \in \hilbert_2\), is called an \emph{adjoint operation}, later \emph{involution}, on \(\bounded(\hilbert_1, \hilbert_2)\). The operator \(A^*\) is called the \emph{adjoint of \(A\)}.
\end{definition}
The Banach space adjoint and the Hilbert space adjoint unfortunately share the same name, despite not being the same operation. They are, however, related by the Riesz representation map.

\begin{proposition}{Banach space adjoint and Hilbert space adjoint}{adjoint_relation}
    Let us denote the Banach space adjoint operation by \(' : \bounded(\hilbert_1, \hilbert_2) \to \bounded(\hilbert_2^\dag, \hilbert_1^\dag)\) and the Hilbert space adjoint operation by \(^* : \bounded(\hilbert_1, \hilbert_2) \to \bounded(\hilbert_2, \hilbert_1)\), where \(\hilbert_1, \hilbert_2\) are Hilbert spaces. If \(T \in \bounded(\hilbert_1, \hilbert_2)\) is a bounded operator, then \(T^* = \riesz_{\hilbert_1} \circ T' \circ \riesz^{-1}_{\hilbert_2}\).
\end{proposition}
Let \(T \in \bounded(\hilbert_1, \hilbert_2)\), with \(T' \in \bounded(\hilbert_2^\dag, \hilbert_1^\dag)\) being the Banach space adjoint and \(T^* \in \bounded(\hilbert_2, \hilbert_1)\) the Hilbert space adjoint, then for all \(x \in \hilbert_1\) and \(y \in \hilbert_2\) we have
    \begin{align*}
        \inner{\riesz_{\hilbert_1} \circ T' \circ \riesz_{\hilbert_2}^{-1}(y)}{x}_{\hilbert_1}
        &= \inner{\riesz_{\hilbert_1}(\riesz_{\hilbert_2}^{-1}(y) \circ T)}{x}_{\hilbert_1}\\
        &= \riesz_{\hilbert_2}^{-1}(y) \circ T(x)\\
        &= \inner{y}{Tx}_{\hilbert_2},
    \end{align*}
    hence \(T^* = \riesz_{\hilbert_1} \circ T' \circ \riesz^{-1}_{\hilbert_2}\) by uniqueness of the Hilbert space adjoint.

\begin{theorem}{Adjoint operation properties}{involution_properties}
    We'll denote Hilbert spaces by \(\hilbert_n\) and the adjoint operations on \(\bounded(\hilbert_n, \hilbert_m)\) simply as \(^*\). Then, the adjoint operation
    \begin{enumerate}[label=(\alph*)]
        \item is an involution: \((A^*)^* = A\), for all \(A \in \bounded(\hilbert_1, \hilbert_2)\);
        \item is an isometry: \(\norm{A^*} = \norm{A},\) for all \(A \in \bounded(\hilbert_1, \hilbert_2)\);
        \item satisfies the \(C^*\) property: \(\norm{A^* \circ A} = \norm{A}^2\), for all \(A \in \bounded(\hilbert_1, \hilbert_2)\);
        \item is antilinear: \((\alpha A + \beta B)^* = \conj{\alpha} A^* + \conj{\beta} B^*\) for all \(\alpha, \beta \in \mathbb{C}\) and \(A, B \in \bounded(\hilbert_1, \hilbert_2)\);
        \item The adjoint operation is antidistributive: \((B \circ A)^* = A^*\circ B^*\) for all \(A \in \bounded(\hilbert_1, \hilbert_2)\) and \(B \in \bounded(\hilbert_2, \hilbert_3)\);
        \item preserves the identity: if \(\unity \in \bounded(\hilbert_1)\) is the identity operator, then \(\unity^* = \unity\);
        \item maps the inverse to the inverse of the adjoint: if \(A \in \bounded(\hilbert_1, \hilbert_2)\) admits an inverse map \(A^{-1} \in \bounded(\hilbert_2, \hilbert_1)\), then \((A^{-1})^* = (A^*)^{-1}\).
    \end{enumerate}
\end{theorem}
\begin{proof}[Proof of (a)]
    Let \(A \in \bounded(\hilbert_1, \hilbert_2)\), then
    \begin{equation*}
        \inner{(A^*)^*x}{y}_{\hilbert_2} = \inner{x}{A^*y}_{\hilbert_1} = \conj{\inner{A^*y}{x}_{\hilbert_1}} = \conj{\inner{y}{Ax}_{\hilbert_2}} = \inner{Ax}{y}_{\hilbert_2}
    \end{equation*}
    for all \(x \in \hilbert_1\) and \(y \in \hilbert_2.\) By non-degeneracy of the inner product, we have \(A = (A^*)^*\), showing involutivity.
\end{proof}
\begin{proof}[Proof of (b)]
    We consider the Banach space adjoint operation
    \begin{align*}
        ' : \bounded(\hilbert_1, \hilbert_2) &\to \bounded(\hilbert_2^\dag, \hilbert_1^\dag)\\
                                           A &\mapsto A'
    \end{align*}
    which we have shown in \cref{prop:adjoint_Banach} to be an isometry. Since the maps \(\riesz_{\hilbert_1}\) and \(\riesz_{\hilbert_2}^{-1}\) are bijective isometries, it follows that
    \begin{align*}
        \norm{A^*}_{\bounded(\hilbert_2, \hilbert_1)} = \sup_{x \in \hilbert_2}{\frac{\norm{A^*x}_{\hilbert_1}}{\norm{x}_{\hilbert_2}}}
        &= \sup_{x \in \hilbert_2}{\frac{\norm{\riesz_{\hilbert_1}\circ A' \circ \riesz^{-1}_{\hilbert_2}(x)}_{\hilbert_1}}{\norm{x}_{\hilbert_2}}}\\
        &= \sup_{x \in \hilbert_2}{\frac{\norm{A' \circ \riesz^{-1}_{\hilbert_2}(x)}_{\hilbert_1^\dag}}{\norm{\riesz_{\hilbert_2}^{-1}(x)}_{\hilbert_2^\dag}}}\\
        &= \sup_{\ell \in \hilbert_2^\dag}{\frac{\norm{A'(\ell)}_{\hilbert_1^\dag}}{\norm{\ell}_{\hilbert_2^\dag}}}\\
        &= \norm{A'}_{\bounded(\hilbert_2^\dag, \hilbert_1^\dag)} = \norm{A}_{\bounded(\hilbert_1,\hilbert_2)}
    \end{align*}
    for all \(A \in \bounded(\hilbert_1, \hilbert_2)\).
\end{proof}
\begin{proof}[Proof of (c)]
    Let \(A \in \bounded(\hilbert_1, \hilbert_2)\), then \(A^* \circ A \in \bounded(\hilbert_1)\) is a bounded operator as a composition of continuous maps. It should be clear that \(\norm{A^*\circ A}_{\bounded(\hilbert_1)} \leq \norm{A}_{\bounded(\hilbert_1, \hilbert_2)}^2\). Indeed,
    \begin{equation*}
        \norm{A^* \circ A}_{\bounded(\hilbert_1)} = \opnorm{A^*\circ A}{x}{\hilbert_1}{\hilbert_1} \leq \norm{A^*}_{\bounded(\hilbert_2, \hilbert_1)}\opnorm{A}{x}{\hilbert_1}{\hilbert_2} = \norm{A}_{\bounded(\hilbert_1, \hilbert_2)}^2.
    \end{equation*}
    For all \(x \in \hilbert_1\), \(\inner{A^*\circ Ax}{x}_{\hilbert_1}\) is a non-negative real number since
    \begin{equation*}
        \inner{A^*\circ Ax}{x}_{\hilbert_1} = \inner{Ax}{Ax}_{\hilbert_2} = \norm{Ax}_{\hilbert_2}^2 \geq 0.
    \end{equation*}
    The \nameref{thm:cauchy_schwarz} yields
    \begin{equation*}
        \norm{Ax}^2_{\hilbert_2} = \inner{A^*\circ A x}{x}_{\hilbert_1} \leq \norm{A^* \circ Ax}_{\hilbert_1} \norm{x}_{\hilbert_1} \leq \norm{A^* \circ A}_{\bounded(\hilbert_1)} \norm{x}_{\hilbert_1}^2
    \end{equation*}
    for all \(x \in \hilbert_1\). That is, \(\norm{A^* \circ A}_{\bounded(\hilbert_1)} \geq \norm{A}_{\bounded(\hilbert_1, \hilbert_2)}^2\).
\end{proof}
\begin{proof}[Proof of (d)]
    Let \(\alpha, \beta \in \mathbb{C}\) and \(A, B \in \bounded(\hilbert_1, \hilbert_2)\), then for all \(x \in \hilbert_1\) and \(y \in \hilbert_2\) we have
    \begin{align*}
        \inner{(\alpha A + \beta B)^*y}{x}_{\hilbert_1} &= \inner{y}{(\alpha A + \beta B)x}_{\hilbert_2}\\
                                                        &= \alpha\inner{y}{Ax}_{\hilbert_2} + \beta\inner{y}{Bx}_{\hilbert_2}\\
                                                        &= \alpha \inner{A^*y}{x}_{\hilbert_1} + \beta \inner{B^*y}{x}_{\hilbert_1}\\
                                                        &= \inner{\conj{\alpha}A^*y + \conj{\beta}B^*y}{x}_{\hilbert_1}\\
                                                        &= \inner{(\conj{\alpha}A^* + \conj{\beta}B^*)y}{x}_{\hilbert_1},
    \end{align*}
    which yields
    \begin{equation*}
        \inner*{\left[\left(\alpha A + \beta B\right)^* - \left(\conj{\alpha}A^* + \conj{\beta}B^*\right)\right]y}{x}_{\hilbert_1} = 0.
    \end{equation*}
    As the inner product is non-degenerate, we have \(\left[\left(\alpha A + \beta B\right)^* - \left(\conj{\alpha}A^* + \conj{\beta}B^*\right)\right]y = 0\) for all \(y \in \hilbert_2\), hence \(\left(\alpha A + \beta B\right)^* = \conj{\alpha}A^* + \conj{\beta}B^*\).
\end{proof}
\begin{proof}[Proof of (e)]
    Let \(A \in \bounded(\hilbert_1, \hilbert_2)\) and let \(B \in \bounded(\hilbert_2, \hilbert_3)\), then for all \(x \in \hilbert_1\) and \(y \in \hilbert_3\), we have
    \begin{align*}
        \inner{(B\circ A)^*y}{x}_{\hilbert_1} &= \inner{y}{B \circ Ax}_{\hilbert_3}\\
                                              &= \inner{B^*y}{Ax}_{\hilbert_2}\\
                                              &= \inner{A^* \circ B^*y}{x}_{\hilbert_1}.
    \end{align*}
    The non-degeneracy of the inner product yields \((B \circ A)^* = A^* \circ B^*\).
\end{proof}
\begin{proof}[Proof of (f)]
    Let \(\unity \in \bounded(\hilbert_1)\) be the identity map, then for all \(x, y \in \hilbert_1\),
    \begin{equation*}
        \inner{\unity^*y}{x}_{\hilbert_1} = \inner{y}{\unity x}_{\hilbert_1} = \inner{y}{x},
    \end{equation*}
    hence \(\unity^*y = y\), thus showing \(\unity^* = \unity\).
\end{proof}
\begin{proof}[Proof of (g)]
    Let \(A \in \bounded(\hilbert_1, \hilbert_2)\) be a bijective bounded operator and \(A^{-1} \in \bounded(\hilbert_2, \hilbert_1)\) its inverse map. By (e) and (f), we have
    \begin{equation*}
        A^* \circ (A^{-1})^* = (A^{-1} \circ A)^* = \unity_{\hilbert_1}^* = \unity_{\hilbert_1}
    \end{equation*}
    and
    \begin{equation*}
        (A^{-1})^* \circ A^* = (A \circ A^{-1})^* = \unity_{\hilbert_2}^* = \unity_{\hilbert_2}
    \end{equation*}
    and we conclude \((A^{-1})^* = (A^*)^{-1}\).
\end{proof}
\todo

\subsection{Self-adjoint bounded operators}
\todo
\begin{theorem}{Norm of a self-adjoint bounded operator}{norm_self_adjoint}
    Let \(T \in \bounded(\hilbert)\) be a self-adjoint bounded operator defined on a Hilbert space \(\hilbert\). Then its norm satisfies
    \begin{equation*}
        \norm{T} = \sup_{x \in \hilbert}\frac{\abs*{\inner{x}{Tx}}}{\norm{x}^2}.
    \end{equation*}
\end{theorem}
\begin{proof}
    \todo
\end{proof}

\subsection{Normal operators}
\begin{lemma}{Application of the polarization identity}{normal}
    Let \(\hilbert\) be a Hilbert space. The bounded operators \(A, B \in \bounded(\hilbert)\) satisfy \(A^* \circ A = B^*\circ B\) if and only if \(\norm{Ax} = \norm{Bx}\) for all \(x \in \hilbert\).
\end{lemma}
\begin{proof}
    Suppose \(A^* \circ A = B^* \circ B\), then for all \(x \in \hilbert\), we have
    \begin{equation*}
        \norm{Ax}^2 = \inner{Ax}{Ax} = \inner{A^*\circ Ax}{x} = \inner{B^*\circ Bx}{x} = \inner{Bx}{Bx} = \norm{Bx}^2,
    \end{equation*}
    hence \(\norm{Ax} = \norm{Bx}\).

    Suppose that for all \(x \in \hilbert\) we have \(\norm{Ax} = \norm{Bx}\). Then,
    \begin{equation*}
        \inner{x}{A^* \circ Ax} = \inner{(A^*)^*x}{Ax} = \inner{Ax}{Ax} = \inner{Bx}{Bx} = \inner{(B^*)^*x}{Bx} = \inner{x}{B^* \circ Bx}
    \end{equation*}
    for all \(x \in \hilbert\). Let \(u, v \in \hilbert\), then
    \begin{align*}
        \inner{u}{A^* \circ A v} &= \frac14\sum_{n=0}^3 i^n\inner{u + i^{-n}v}{A^*\circ A(u + i^{-n}v)}\\
                                 &= \frac14 \sum_{n=0}^3 i^n\inner{u + i^{-n}v}{B^* \circ B(u + i^{-n}v)}\\
                                 &= \inner{u}{B^* \circ B v}
    \end{align*}
    follows from \cref{prop:polarization_identity}. That is, for all \(u, v \in \hilbert\), \(\inner{u}{(A^*\circ A - B^*\circ B)v} = 0\), hence the non-degeneracy of the inner product yields \(A^*\circ A = B^*\circ B\).
\end{proof}

\begin{proposition}{Necessary and sufficient condition for a normal operator}{normal}
    A bounded operator \(A \in \bounded(\hilbert)\) in a Hilbert space \(\hilbert\) is normal if and only if \(\norm{Ax} = \norm{A^* x}\) for all \(x \in \hilbert\).
\end{proposition}
\begin{proof}
    Setting \(B = A^*\) in \cref{lem:normal} yields
    \begin{align*}
        \forall x \in \hilbert, \norm{Ax} = \norm{A^*x} &\iff A^*\circ A = (A^*)^*\circ A^*\\
                                                        &\iff A^* \circ A = A \circ A^*\\
                                                        &\iff A\text{ is normal,}
    \end{align*}
    as desired.
\end{proof}

\subsection{Unitary operators}
\todo

% vim: spl=en_us

% c*-algebras
% vim: spl=en_us
\chapter{C*-algebras}
Bounded operators on a Hilbert space give rise to an abstraction that proves fruitful. Previously we have shown that bounded operators on a Hilbert space can be made into a unital C*-algebra with the operations of map composition and adjoint operation.
\begin{definition}{Algebra}{algebra}
    An \emph{algebra} \(\algebra{A}\) is a linear space \(\algebra{A}\) over \(\mathbb{C}\) where there is a product \(\cdot : \algebra{A} \times \algebra{A} \to \algebra{A}\) satisfying
\begin{enumerate}[label=(\alph*)]
    \item Distributivity with respect to the vector addition:
        \begin{equation*}
            a\cdot(b + c) = a\cdot b + a\cdot c
            \quad\text{and}\quad
            (b + c) \cdot a = b \cdot a + c \dot a
        \end{equation*}
        for all \(a, b, c \in \algebra{A}\); and
    \item Compatibility with scalar multiplication:
        \begin{equation*}
            \alpha(a \cdot b) = (\alpha a)\cdot b = a \cdot (\alpha b),
        \end{equation*}
        for all \(a, b \in \algebra{A}\) and \(\alpha \in \mathbb{C}\).
\end{enumerate}
If the product is associative, that is,
\begin{equation*}
    a \cdot (b \cdot c) = (a \cdot b) \cdot c
\end{equation*}
for all \(a, b, c \in \algebra{A}\), we say \(\algebra{A}\) is an associative algebra and we denote the product simply by juxtaposition with no ambiguity.

If the product is commutative, that is,
\begin{equation*}
    a \cdot b = b \cdot a
\end{equation*}
for all \(a, b \in \algebra{A}\), we say \(\algebra{A}\) is an abelian algebra.

If there exists \(\unity \in \algebra{A}\) satisfying
\begin{equation*}
    a \cdot \unity = \unity \cdot a = a
\end{equation*}
for all \(a \in \algebra{A}\), then we say \(\algebra{A}\) is a unital algebra and call \(\unity\) an identity element.
\end{definition}

It is easy to see a unital algebra has a unique identity element.
\begin{proposition}{Unital algebra has a unique identity element}{unital_algebra_unique_identity}
    Let \(\algebra{A}\) be a unital algebra and let \(\unity \in \algebra{A}\) be an identity element. Then, \(\unity\) is the only identity element of \(\algebra{A}\).
\end{proposition}
\begin{proof}
    Let \(e \in \algebra{A}\) be an identity element of \(\algebra{A}\). Since \(\unity \in \algebra{A}\), we have \(e \cdot \unity = \unity \cdot e = \unity\). Since \(\unity\) is an identity element, we also have \(e \cdot \unity = \unity \cdot e = e\), hence \(e = \unity\).
\end{proof}

We'll henceforth consider associative algebras. For any \(n \in \mathbb{N}\), we denote \(a^n = a^{n-1}a = aa^{n-1}\), with \(a^1 = a\), for all vectors \(a\) in the algebra.
\begin{proposition}{Identities for the difference of powers}{difference_powers}
    Let \(\algebra{A}\) be an associative algebra. Then
    \begin{equation*}
        x^{n+1} - y^{n+1} = \frac12 \left[(x^n + y^n)(x - y) + (x^n - y^n)(x+y)\right]
    \end{equation*}
    and
    \begin{equation*}
        x^{n+1} - y^{n+1} = \frac12(x^n + y^n)(x-y) + \sum_{k=1}^{n-1}\frac{1}{2^{k+1}}(x^{n-k}+y^{n-k})(x-y)(x+y)^k + \frac1{2^n}(x - y)(x+y)^n
    \end{equation*}
    hold for all \(x,y \in \algebra{A}\) and all \(n \in \mathbb{N}\).
\end{proposition}
\begin{proof}
    Let \(u,v \in \algebra{A}\), then
    \begin{align*}
        \frac12\left[(u^n + v^n)(u - v) + (u^n - v^n)(u+v)\right] &= \frac12 \left[u^{n+1} - u^n v + v^n u - v^{n+1} + u^{n+1} + u^nv - v^nu -v^{n+1}\right]\\&= u^{n+1} - v^{n+1}
    \end{align*}
    for all \(n \in \mathbb{N}\).

    For ease of notation, let \(S_n(x,y)\) denote the sum in the second identity for \(x,y \in \algebra{A}\) and \(n \in \mathbb{N}\).
    Let \(M \subset \mathbb{N}\) be the set of natural numbers for which the second identity holds for all \(x,y \in \algebra{A}\). For \(n = 1\), \(S_1(x,y)\) yields the zero vector, then
    \begin{equation*}
        \frac12 (x + y)(x - y) + \frac12(x-y)(x+y) = x^2 - y^2,
    \end{equation*}
    that is, \(1 \in M\). In particular, \(M\) is non-empty. Let \(m \in M\), then from the first identity we have
    \begin{equation*}
        x^{m+2} - y^{m+2} = \frac{1}{2}\left[(x^{m+1} + y^{m+1})(x - y) + (x^{m+1} - y^{m+1})(x+y)\right]
    \end{equation*}
    for all \(x,y \in \algebra{A}\), hence
    \begin{equation*}
        \begin{split}
            x^{m+2} - y^{m+2} &= \frac12 (x^{m+1} + y^{m+1})(x - y) + \\
                              &+\frac12\left[\frac12(x^{m} + y^{m})(x-y) + S_{m}(x,y) + \frac1{2^{m}}(x - y)(x+y)^{m} \right] (x+y)\\
                              % &= \frac12 (x^m + y^m)(x - y) + \frac{1}{2^m}(x-y)(x+y)^m+\\
                              % &+ \left[\frac14 (x^{m-1} + y^{m-1})(x-y) + \frac12 S_{m-1}(x,y)\right](x+y)\\
                                &=\frac12 (x^{m+1} + y^{m+1})(x - y) + \frac{1}{2^{m+1}}(x-y)(x+y)^{m+1}+\\
                              &+\frac14 (x^{m} + y^{m})(x-y)(x+k) +  \sum_{k=1}^{m-1}\frac{1}{2^{k+2}}(x^{m-k} + y^{m-k})(x-y)(x+y)^{k+1}\\
                              &=\frac12 (x^{m+1} + y^{m+1})(x - y) + \frac{1}{2^{m+1}}(x-y)(x+y)^{m+1}+\\
                              &+\frac14 (x^{m} + y^{m})(x-y)(x+k) +  \sum_{k=2}^{m}\frac{1}{2^{k+1}}(x^{m+1-k} + y^{m+1-k})(x-y)(x+y)^{k}\\
                              &=\frac12 (x^{m+1} + y^{m+1})(x - y) + S_{m+1}(x,y) + \frac{1}{2^{m+1}}(x-y)(x+y)^{m+1},
        \end{split}
    \end{equation*}
    that is, \(m+1 \in M\). By the principle of finite induction, we have shown that \(M = \mathbb{N}\).
\end{proof}

If the underlying linear space of an algebra is equipped with a norm, we may study the algebraic structure with respect to the metric topology.
\begin{definition}{Normed algebra}{normed_algebra}
    A \emph{normed algebra} \(\algebra{A}\) is an associative algebra equipped with a norm \(\norm{\noarg} : \algebra{A} \to \mathbb{R}\) satisfying
    \begin{enumerate}[label=(\alph*)]
        \item \(\norm{ab} \leq \norm{a}\cdot\norm{b}\) for all \(a,b \in \algebra{A}\); and
        \item if \(\algebra{A}\) is unital, \(\norm{\unity} = 1\).
    \end{enumerate}
\end{definition}
With the above definition, we may use the norm for estimating certain quantities.
\begin{lemma}{Estimates for the difference of powers}{estimate_difference_power}
    Let \(\algebra{A}\) be a normed algebra. Then,
    \begin{equation*}
        \norm{x^{n+1} - y^{n+1}}\leq \left(\norm{x} + \norm{y}\right)^n \norm{x-y}
    \end{equation*}
    for all \(n \in \mathbb{N}_0\) and all \(x,y \in \algebra{A}\). Moreover, for all \(x, y \in \algebra{A}\) with \(\norm{x} \leq 1\) and \(\norm{y} \leq 1\),
    \begin{equation*}
        \norm{x^{n+1} - y^{n+1}} \leq (n+1)\norm{x+y}
    \end{equation*}
    for all \(n \in \mathbb{N}_0\).
\end{lemma}
\begin{proof}
    Let \(S\subset \mathbb{N}_0\) be the set of integers for which the first inequality holds for all \(x, y \in \algebra{A}\) and let \(R \subset \mathbb{N}_0\) be the set analogously defined for the second inequality. Quite trivially, \(0 \in S \cap R\), so neither set is empty. Let \(k \in S\), then for all \(a \in \algebra{A}\) we have \(\norm{a^{k+1}} \leq \norm{a}^k\). Then, by \cref{prop:difference_powers} we have
    \begin{align*}
        \norm{x^{k+2} - y^{k+2}} &\leq \frac12 \norm{(x^{k+1} + y^{k+1})(x - y)} + \frac12\norm{(x^{k+1} - y^{k+1})(x+y)}\\
                                 &\leq \frac12 \norm{x^{k+1} + y^{k+1}} \norm{x - y}\\
                                 &\leq \frac12\left(\norm{x^{k+1}} + \norm{y^{k+1}}\right)\norm{x - y}\\
                                 &\leq \frac12\left(\norm{x}^{k+1} + \norm{y}^{k+1}\right)\norm{x - y},
    \end{align*}
    that is, \(k + 1 \in S\). By the principle of finite induction, we have \(S = \mathbb{N}_0\). Let \(x, y \in \setc{a \in \algebra{A}}{\norm{a} \leq 1}\). Then for all \(n \in \mathbb{N} \subset S\), \(\norm{x^n + y^n} \leq \norm{x^n} + \norm{y^n} \leq \norm{x}^n + \norm{y}^n\leq 2\) and \(\norm{x + y}^n \leq (\norm{x} + \norm{y})^n \leq 2^n\). Since \(\algebra{A}\) is a normed algebra, \cref{prop:difference_powers} yields
    \begin{equation*}
        \norm{x^{n+1} - y^{n+1}} \leq \norm{x - y} + \sum_{k=1}^{n-1} \norm{x-y} + \norm{x-y} = (n+1)\norm{x-y},
    \end{equation*}
    hence \(R = \mathbb{N}_0\).
\end{proof}

We now abstract the adjoint operation on \(\bounded(\hilbert)\) to associative algebras.
\begin{definition}{Involution and *-algebra}{involutive_algebra}
    Let \(\algebra{A}\) be an associative algebra. An \emph{involution} is a map
    \begin{align*}
        * : \algebra{A} &\to \algebra{A}\\
                      a &\mapsto a^*
    \end{align*}
    satisfying
    \begin{enumerate}[label=(\alph*)]
        \item involutivity: \((a^*)^* = a\), for all \(a \in \algebra{A}\);
        \item antidistributivity: \((ab)^* = b^* a^*\), for all \(a, b \in \algebra{A}\);
        \item antilinearity: \((\alpha a + \beta b)^* = \conj{\alpha}a^* + \conj{\beta}b^*\), for all \(a,b \in \algebra{A}\) and all \(\alpha, \beta \in \mathbb{C}\); and
        \item if \(\algebra{A}\) is unital, then \(\unity^* = \unity\).
    \end{enumerate}
    If \(\algebra{A}\) has an involution, we say \(\algebra{A}\) is a \emph{*-algebra} or \emph{involutive algebra}.
\end{definition}


A usual, we define the maps that preserve the structure of a given space. In the case of involutive algebras, we have *-morphisms.
\begin{definition}{*-morphism}{star_morphism}
    Let \(\algebra{A}, \algebra{B}\) be *-algebras. A \emph{*-morphism} is a linear map \(\pi : \algebra{A} \to \algebra{B}\) satisfying
    \begin{enumerate}[label=(\alph*)]
        \item \(\pi(ab) = \pi(a) \pi(b)\) for all \(a, b \in \algebra{A}\); and
        \item \(\pi(a^*) = \pi(a)^*\) for all \(a \in \algebra{A}\).
    \end{enumerate}
    If there exists a bijective *-morphism \(\pi: A \to B\), we say \(A\) and \(B\) are *-isomorphic and that \(\pi\) is a *-isomorphism.
\end{definition}

The following example uses the adjoint operation to define another involution on \(\bounded(\hilbert)\).
\begin{example}{Involution on bounded operators of a Hilbert space}{involution_adjoint}
    Let \(\hilbert\) be a Hilbert space and let \(d \in \bounded(\hilbert)\) such that \(d\) is self-adjoint and unitary. The map
    \begin{align*}
        \dag : \bounded(\hilbert) &\to \bounded(\hilbert)\\
                                a &\mapsto a^\dag,
    \end{align*}
    where \(a^\dag = d^* a^* d\) defines an involution on \(\bounded(\hilbert)\), where \(a^*\) is the adjoint operator of \(a\).
\end{example}
\begin{proof}
    Let \(a \in \bounded(\hilbert)\), then
    \begin{equation*}
        (a^\dag)^\dag = d^* (a^\dag)^* d = d^* (d^* a^* d)^* d = d^* d^* a d d = a,
    \end{equation*}
    since \(d\) is unitary and self-adjoint, \(d^* d = d d^* = \unity\) and \(d^* = d\). Let \(b \in \bounded(\hilbert)\), then
    \begin{equation*}
        (ab)^\dag = d^*(ab)^* d = d^* b^* a^* d = d^* b^* d d^* a^* d = b^\dag a^\dag.
    \end{equation*}
    Let \(\alpha, \beta \in \mathbb{C}\), then
    \begin{equation*}
        (\alpha a + \beta b)^\dag = d^* (\alpha a + \beta b)^* d = \conj{\alpha} d^* a^* d + \conj{\beta} d^* b^* d = \conj{\alpha} a^\dag + \conj{\beta} b^\dag.
    \end{equation*}
    Finally, we have
    \begin{equation*}
        \unity^\dag = d^* \unity d = d^* d = \unity,
    \end{equation*}
    hence \(\dag\) is an involution on \(\bounded(\hilbert)\).
\end{proof}

Our main object of study will be on involutive algebras, in particular the ones which are complete with respect to the norm.
\begin{definition}{Banach algebras and C*-algebras}{c_algebra}
    A Banach algebra \(\algebra{B}\) is a normed algebra that is a complete metric space with respect to its norm. If \(\algebra{B}\) has an involution that satisfies \(\norm{a} = \norm{a^*}\) for all \(a \in \algebra{B}\), we say \(\algebra{B}\) is a Banach *-algebra. If, in addition, the norm and the involution satisfy \(\norm{a^*a} = \norm{a}^2\) for all \(a \in \algebra{B}\), then \(\algebra{B}\) is a C*-algebra.
\end{definition}

% vim: spl=en_us
\section{Ideals and quotient spaces}
Let \(\algebra{A}\) be an algebra. A \emph{subalgebra} \(\algebra{B}\) is a linear subspace \(\algebra{B} \subset \algebra{A}\) satisfying \(ab \in \algebra{B}\) for all \(a, b \in \algebra{B}\). If \(\algebra{A}\) has more structure we name the subalgebra accordingly if it is closed with respect to the additional properties. For example, if \(\algebra{A}\) is involutive, \(\algebra{B}\) is a *-subalgebra if \(a \in \algebra{B} \implies a^* \in \algebra{B}\).
\begin{definition}{Self-adjoint subset of an involutive algebra}{self_adjointness}
    Let \(\algebra{A}\) be an involutive algebra. A \emph{self-adjoint} element \(a \in \algebra{A}\) satisfies \(a^* = a\). If \(B \subset \algebra{A}\) satisfies \(B^* = B\), then it is called a \emph{self-adjoint} subset of \(\algebra{A}\).
\end{definition}
\begin{remark}
    It is clear that a *-subalgebra is a self-adjoint subset of an involutive algebra.
\end{remark}
\begin{proposition}{Necessary and sufficient conditions for self-adjoint subset}{self_adjoint_subset}
    Let \(\algebra{A}\) be a *-algebra and \(B \subset \algebra{A}\) a non-empty subset. The following statements are equivalent:
    \begin{enumerate}[label=(\alph*)]
        \item \(B^* \subset B\);
        \item \(B \subset B^*\);
        \item \(B\) is self adjoint.
    \end{enumerate}\
\end{proposition}
\begin{proof}
    It is evident that (c) implies (a). Suppose \(B^* \subset B\) and let \(b \in B\). Since \(b^* \in B^* \subset B\), \(b\) is the adjoint to some element in \(B\), namely \(b^*\), hence \(b \in B^*\).  That is, (a) implies (b).

    Suppose \(B \subset B^*\) and let \(a \in B^*\). There exists \(b \in B\) such that \(a = b^*\), but \(b\) is the adjoint to some \(c \in B\), by hypothesis. That is, \(a = (c^*)^* = c \in B\), and we have shown \(B^* \subset B\), hence \(B\) is self-adjoint.
\end{proof}

Let \(A\) and \(B\) be non-empty subsets of an associative algebra \(\algebra{A}\). The image of the product restricted to \(A \times B\) is denoted by \(AB\), that is,
\begin{equation*}
    AB = \setc{x \in \algebra{A}}{\exists a \in A, \exists b \in B : x = ab}.
\end{equation*}
It should be clear \(AB\) is not generally equal to \(BA\).
\begin{definition}{Ideals}{ideal}
    Let \(\algebra{A}\) be an associative algebra. A linear subspace \(B \subset \algebra{A}\) is a
    \begin{enumerate}[label=(\alph*)]
        \item \emph{left ideal of \(\algebra{A}\)} if \(\algebra{A}B \subset B;\)
        \item \emph{right ideal of \(\algebra{A}\)} if \(B\algebra{A} \subset B.\)
    \end{enumerate}
    If \(B\) is a left ideal and a right ideal of \(\algebra{A}\), then we say \(B\) is a \emph{two-sided ideal of \(\algebra{A}\)}.
\end{definition}
\begin{remark}
    It is easy to conclude every ideal is a subalgebra. Indeed, suppose, for definiteness, \(B \subset \algebra{A}\) is a left ideal of \(\algebra{A}\), then \(BB \subset \algebra{A}B \subset B\), that is, \(B\) is an algebra.
\end{remark}

\begin{proposition}{Self-adjoint ideal is two-sided ideal}{self_adjoint_ideal}
    Let \(\algebra{A}\) be a *-algebra. If \(\algebra{B} \subset \algebra{A}\) is a left (or right) ideal of \(\algebra{A}\) and self-adjoint, then \(\algebra{B}\) is a two-sided ideal.
\end{proposition}
\begin{proof}
    Let \(b \in \algebra{B}\), then for all \(a \in \algebra{A}\), we have \(ab \in \algebra{B}\). Self-adjointness yields \(a^*b^* \in \algebra{B}\), hence \(ba \in \algebra{B}\) for all \(a \in \algebra{A}\). Since \(b\) is arbitrary, \(\algebra{B}\) is a right ideal.
\end{proof}
\begin{remark}
    If an ideal is self-adjoint we'll say it is a *-ideal.
\end{remark}

Recall a linear subspace \(S\) on a linear space \(V\) defines an equivalence relation \(\sim_S\) by \(u \sim_S v\) if \(u - v \in S\). We write the quotient space \(V/S\) to denote the quotient set \(V/\sim_S\) and we call the map \(V \ni v \mapsto [v] \in V/S\) the \emph{quotient map}. Moreover, with the operations \(\alpha[v] = [\alpha v]\) and \([u]+[v] = [u+v]\), \(V/S\) is made into a linear space.

Let \(\algebra{A}\) be a Banach *-algebra and let \(\algebra{I}\subset \algebra{A}\) be a closed *-ideal of \(\algebra{A}\). We'll show that with the maps
\begin{align*}
    \cdot : \algebra{A}/\algebra{I} \times \algebra{A}/\algebra{I} &\to \algebra{A}/\algebra{I}&
    * : \algebra{A}/\algebra{I} &\to \algebra{A}/\algebra{I}&
    \norm{\noarg} : \algebra{A}/\algebra{I} &\to \mathbb{R}\\
    ([x],[y]) &\mapsto [xy]&
    [x] &\mapsto [x^*]&
    [x] &\mapsto \inf_{j \in \algebra{I}}{\norm{x + j}},
\end{align*}
the quotient space \(\algebra{A}/\algebra{I}\) is a Banach *-algebra. We'll split the proof in several lemmas showing intermediate results.

\begin{lemma}{Quotient space by a two-sided ideal is an associative algebra}{quotient_two_sided_associative_algebra}
    Let \(\algebra{A}\) be an associative algebra. If \(\algebra{I}\) is a two-sided ideal, then \(\algebra{A}/\algebra{I}\) is an associative algebra. If, furthermore, \(\algebra{A}\) is unital, then \([\unity]\) is the identity of the quotient space \(\algebra{A}/\algebra{I}\).
\end{lemma}
\begin{proof}
    We begin by showing the algebraic product is well-defined. Let \(x_1, x_2, y_1, y_2 \in \algebra{A}\) with \([x_1] = [x_2]\) and \([y_1] = [y_2]\), then there exist \(j_x = x_2 - x_1\) and \(j_y = y_2 - y_1\) with \(j_x, j_y \in \algebra{I}\). We have
    \begin{equation*}
        x_1y_1 - x_2 y_2 = x_1 y_1 - (x_1 + j_x) y_2 = x_1 (y_1 - y_2) - j_x y_2 = x_1 j_y - j_x y_2,
    \end{equation*}
    hence \([x_1 y_1] = [x_2 y_2]\) because \(\algebra{I}\) is a two-sided ideal. Therefore, the product is well-defined.

    Let \([a], [b], [c] \in \algebra{A}/\algebra{I}\) and \(\alpha \in \mathbb{C}\), then
    \begin{equation*}
        [a] \cdot ([b]+[c]) = [a]\cdot[b+c] = [ab + ac] = [ab] + [ac]
    \end{equation*}
    and
    \begin{equation*}
        ([b]+[c])\cdot[a] = [b+c]\cdot[a] = [ba + ca] = [ba] + [ca],
    \end{equation*}
    that is, the product is distributive with respect to vector addition. We also have
    \begin{align*}
        \alpha([a]\cdot[b]) = \alpha[ab] = [\alpha ab] &= ([\alpha a])\cdot[b] = (\alpha[a])\cdot[b]\\
                                                       &= [a]\cdot([\alpha b]) = [a]\cdot (\alpha[b]),
    \end{align*}
    then it follows that the product is compatible with scalar multiplication. Moreover, the product is associative since
    \begin{equation*}
        [a]\cdot([b]\cdot[c]) = [a]\cdot[bc]=[abc] = [ab] \cdot [c] = ([a]\cdot[b])\cdot[c],
    \end{equation*}
    hence \(\algebra{A}/\algebra{I}\) is an associative algebra. If, in addition, \(\algebra{A}\) is unital we have \([\unity]\cdot[x] = [\unity x] = [x]\) and \([x]\cdot[\unity] = [x \unity] = [x]\) for all \([x] \in \algebra{A}/\algebra{I}\), hence \([\unity]\) is the identity of \(\algebra{A}/\algebra{I}\).
\end{proof}
\begin{lemma}{Quotient space by a self-adjoint subspace is a *-algebra}{quotient_self_adjoint_star}
    Let \(\algebra{A}\) be a *-algebra. If \(\algebra{I}\) is a self-adjoint linear subspace, then \(\algebra{A}/\algebra{I}\) is a *-algebra.
\end{lemma}
\begin{proof}
    The well-definition of the map \([x] \mapsto [x^*]\) follows from the self-adjointness of the ideal. Indeed, let \(z_1, z_2 \in \algebra{A}\) with \([z_1] = [z_2]\), then there exists \(j \in \algebra{I}\) such that \(j = z_2 - z_1 \in \algebra{I}\), hence \(j^* = z_2^* - z_1^* \in \algebra{I}\), which shows \([z_1^*] = [z_2^*]\). Let \([a], [b] \in \algebra{A}/\algebra{I}\) and \(\lambda\in \mathbb{C}\), then * satisfies
    \begin{enumerate}[label=(\alph*)]
        \item involutivity: \(\left([a]^*\right)^* = [a^*]^* = [(a^*)^*] = [a]\);
        \item antidistributivity: \(([a]\cdot[b])^* = [ab]^* = [(ab)^*] = [b^*a^*] = [b^*]\cdot[a^*] = [b]^* \cdot [a]^*\); and
        \item antilinearity: \(([a] + \lambda[b])^* = [a + \lambda b]^* = [(a + \lambda b)^*] = [a^* + \conj{\lambda}b^*] = [a^*] + [\conj{\lambda} b^*] = [a]^* + \conj{\lambda}[b]^*\).
    \end{enumerate}
    If \(\algebra{A}\) is unital, we have \([\unity]^* = [\unity^*] = [\unity]\), hence \(*\) is an involution on \(\algebra{A}/\algebra{I}\).
\end{proof}

\begin{lemma}{Quotient space by a closed subspace is a normed linear space}{quotient_closed_normed}
    Let \(\algebra{A}\) be a normed linear space. If \(\algebra{I}\) is a closed subspace, then \(\algebra{A}/\algebra{I}\) is a normed linear space.
\end{lemma}
\begin{proof}
    Let \(w_1, w_2 \in \algebra{A}\) with \([w_1] = [w_2]\), then the sets \(\setc{w_1 + j}{j \in \algebra{I}}\) and \(\setc{w_2 + j}{j \in \algebra{I}}\) are equal, hence the map \(\norm{\noarg}\) is well-defined.

    Let \(\lambda \in \mathbb{C}\setminus\set{0}\), then
    \begin{equation*}
        \norm{\lambda [y]} = \norm{[\lambda y]} = \inf_{j \in \algebra{I}}{\norm{\lambda y + \lambda j}}= \abs{\lambda} \inf_{j \in \algebra{I}}{\norm{y} + j} = \abs{\lambda}\cdot \norm{[y]}
    \end{equation*}
    for all \([y] \in \algebra{A}/\algebra{I}\). We also have \(\norm{0 [y]} = \norm{[0]} = \inf_{j \in \algebra{I}} \norm{j} = 0,\) for all \([y] \in \algebra{A}/\algebra{I}\), hence \(\norm{\noarg}\) is absolute homogeneous.

    Let \([a], [b] \in \algebra{A}/\algebra{I}\), then
    \begin{equation*}
        \norm{[a] + [b]} = \norm{[a+b]} = \inf_{n,m \in \algebra{I}}{\norm{a+b+n+m}} \leq \inf_{n,m \in \algebra{I}}{\left(\norm{a + n} + \norm{b+m}\right)} = \norm{[a]} + \norm{[b]},
    \end{equation*}
    hence \(\norm{\noarg}\) is subadditive.

    From the nonnegativity of the norm defined on \(\algebra{A}\), it follows that this map is non-negative. Let \([x] \in \algebra{A}/\algebra{I}\) be such that \(\norm{[x]} = 0\), then there exists no \(\epsilon > 0\) such that \(\norm{j - x} \geq \epsilon > 0\) for all \(j \in \algebra{I}\). As a result, every open ball centered at \(x\) has non-empty intersection with \(\algebra{I}\), then \(x \in \cl_{\algebra{A}}\algebra{I}\). Since \(\algebra{I}\) is closed, we have \(x \in \algebra{I}\). The positive-definiteness of the norm and the previous result yield
    \begin{equation*}
        \norm{[x]} = 0 \iff x \in \algebra{I} \iff [x] = [0],
    \end{equation*}
    hence \(\norm{\noarg}\) is positive-definite. That is, \(\norm{\noarg}\) defines a norm on \(\algebra{A}/\algebra{I}\).
\end{proof}
\begin{lemma}{Quotient space by a closed two-sided ideal is a normed algebra}{quotient_closed_ideal_normed}
    Let \(\algebra{A}\) be a normed algebra. If \(\algebra{I}\) is a closed two-sided ideal, then \(\algebra{A}/\algebra{I}\) is a normed algebra.
\end{lemma}
\begin{proof}
    Let \([u],[v] \in \algebra{A}/\algebra{I},\) then
    \begin{align*}
        \norm{[u]\cdot[v]} = \norm{[uv]} = \inf_{j\in \algebra{I}}{\norm{uv + j}}
        &\leq \inf_{k,\ell \in \algebra{I}}{\norm{uv + \ell k}}\\
        &\leq \inf_{k,\ell \in \algebra{I}}{\norm{uv + k\ell + u \ell + kv}}
    \end{align*}
    since \(\algebra{I}\) is a two-sided ideal and, in particular, a subalgebra. Notice \(uv + k\ell + u \ell + kv = (u + k)(v + \ell)\), then
    \begin{equation*}
        \norm{[u]\cdot[v]} \leq \inf_{k,\ell \in \algebra{I}}{\norm{(u+k)(v + \ell)}}
                           \leq \inf_{k\in\algebra{I}}{\norm{u+k}}\inf_{\ell \in \algebra{I}}{\norm{v+\ell}}
                           = \norm{[u]}\cdot\norm{[v]}.
    \end{equation*}
    If \(\algebra{A}\) is unital we have
    \begin{equation*}
        \norm{[x]} = \norm{[\unity]\cdot[x]} \leq \norm{[\unity]}\cdot\norm{[x]},
    \end{equation*}
    for all \([x] \in \algebra{A}/\algebra{I}\). In particular, we have \(\norm{[\unity]}\left(\norm{[\unity]} - 1\right) \geq 0\), hence either \(\norm{[\unity]} = 0\) or \(\norm{[\unity]}\geq 1\). However we have
    \begin{equation*}
        \norm{[\unity]} = \inf_{j\in \algebra{I}}{\norm{\unity + j}} \leq \norm{\unity}+\inf_{j\in\algebra{I}}{\norm{j}} = 1,
    \end{equation*}
    hence \(\norm{[\unity]} = 1\) or \(\norm{[\unity]} = 0\). In the second case we have \(\norm{[x]} = 0\) for all \([x] \in \algebra{A}/\algebra{I}\), that is, the quotient space is a trivial linear space, which is trivially a normed algebra. We have thus shown that \(\algebra{A}/\algebra{I}\) is a normed algebra.
\end{proof}

\begin{lemma}{Quotient space by a closed subspace is a Banach space}{quotient_closed_banach}
    Let \(\algebra{A}\) be a Banach space. If \(\algebra{I}\) is a closed subspace, then \(\algebra{A}/\algebra{I}\) is a Banach space.
\end{lemma}
\begin{proof}
    Let \(\family{[x_n]}{n \in \mathbb{N}} \subset \algebra{A}/\algebra{I}\) be a sequence in \(\algebra{A}/\algebra{I}\). For all \(n,m,p \in \mathbb{N}\) and \(\eta > 0\) there exists \(\ell(n,m,p, \eta) \in \algebra{I}\) satisfying
    \begin{equation*}
        \norm{x_m - x_n + \ell(m,n,p,\eta)} \leq \norm{[x_m] - [x_n]} + \frac{\eta}{2^{p+1}},
    \end{equation*}
    otherwise for some \(\tilde{m}, \tilde{n}, \tilde{p} \in \mathbb{N}\) and \(\tilde{\eta} > 0\) we would have
    \begin{equation*}
        \forall \ell \in \algebra{I}: \norm{[x_{\tilde{m}}] - [x_{\tilde{n}}]} + \frac{\eta}{2^{\tilde{p} + 1}} < \norm{x_{\tilde{m}} - x_{\tilde{n}} + \ell},
    \end{equation*}
    contradicting the fact that \(\norm{[x_{\tilde{m}}] - [x_{\tilde{n}}]}\) is the greatest lower bound for \(\setc{\norm{x_{\tilde{m}} - x_{\tilde{n}} + \ell}}{\ell \in \algebra{I}}\).
    We now suppose the sequence is Cauchy, then for all \(\varepsilon> 0\) there exists \(N_{\varepsilon} > 0\) such that for all \(m, n > N_{\varepsilon}\) we have \(\norm{[x_m] - [x_n]} < \varepsilon\). Then, we may choose a subsequence \(\family{[x_{i_j}]}{j \in \mathbb{N}}\) satisfying
    \begin{equation*}
        \norm{[x_{i_j}] - [x_{i_k}]} \leq \frac{1}{2^{k+1}}\varepsilon
    \end{equation*}
    for all \(j > k\). As a result we have for all \(j > k\) and \(p \in \mathbb{N}\) that
    \begin{equation*}
        \norm{x_{i_j} - x_{i_k} + \ell(i_j, i_k, p)} \leq \left(\frac{1}{2^{k+1}} + \frac{1}{2^{p+1}}\right)\varepsilon \leq \frac{1}{2^{\max\set{k,p}}}\varepsilon.
    \end{equation*}

    For all \(m \in \mathbb{N}\) set
    \begin{equation*}
        y_m = x_{i_m} + \sum_{p = 1}^m \ell(i_p, i_{p-1}, p),
    \end{equation*}
    then \(y_m \in [x_{n_m}]\). For \(m > n\), we have
    \begin{equation*}
        y_m - y_n = \sum_{k = n+1}^m (y_k - y_{k-1}) = \sum_{k=n+1}^m \left[x_{i_k} - x_{i_{k-1}} + \ell(i_k, i_{k-1},k)\right],
    \end{equation*}
    which yields
    \begin{equation*}
        \norm{y_m - y_n} \leq \sum_{k=n+1}^m \norm{x_{i_k} - x_{i_{k-1}} + \ell(i_k, i_{k-1},k)} \leq \left(\sum_{k=n+1}^m \frac{1}{2^k}\right) \varepsilon< \left(\sum_{k=n+1}^\infty \frac{1}{2^k}\right) \varepsilon = \frac{1}{2^n}\varepsilon.
    \end{equation*}
    Hence \(\family{y_n}{n \in \mathbb{N}}\) is a Cauchy sequence in the Banach space \(\algebra{A}\), and it must converge against some \(\tilde{y} \in \algebra{A}\). It is also the case that \([x_n] \to [\tilde{y}]\), since
    \begin{equation*}
        \norm{[x_n] - [\tilde{y}]} = \norm{[y_n - \tilde{y}]} = \inf_{a\in \algebra{I}}{\norm{y_n - \tilde{y} + a}} \leq \norm{y_n - \tilde{y}},
    \end{equation*}
    which goes to zero as we take \(n\) sufficiently large.
\end{proof}

\begin{theorem}{Quotient space of a Banach *-algebra by a closed *-ideal}{closed_ideal_Bstar}
    Let \(\algebra{A}\) be a Banach *-algebra. If \(\algebra{I}\) is a closed *-ideal of \(\algebra{A}\), then \(\algebra{A}/\algebra{I}\) is a Banach *-algebra.
\end{theorem}
\begin{proof}
    \cref{lem:quotient_two_sided_associative_algebra,lem:quotient_self_adjoint_star,lem:quotient_closed_ideal_normed,lem:quotient_closed_normed,lem:quotient_closed_banach} show \(\algebra{A}/\algebra{I}\) is a Banach algebra, so it remains to show the Banach *-algebra property. Let \([x] \in \algebra{A}/\algebra{I}\), then
    \begin{equation*}
        \norm{[x]^*} = \inf_{j\in\algebra{I}}{\norm{x^* + j}}=\inf_{j\in\algebra{I}}{\norm{(x + j)^*}} = \inf_{j\in\algebra{I}}{\norm{x+j}} = \norm{[x]}
    \end{equation*}
    follows from self adjointness of \(\algebra{I}\). That is, \(\algebra{A}/\algebra{I}\) is a Banach *-algebra.
\end{proof}

% vim: spl=en_us
\section{Adjoining an identity}
A usual, we define the maps that preserve the structure of a given space. In the case of involutive algebras, we have *-homomorphisms.
\begin{definition}{*-homomorphism}{star_morphism}
    Let \(\algebra{A}, \algebra{B}\) be *-algebras. A \emph{*-homomorphism} is a linear map \(\pi : \algebra{A} \to \algebra{B}\) satisfying
    \begin{enumerate}[label=(\alph*)]
        \item \(\pi(ab) = \pi(a) \pi(b)\) for all \(a, b \in \algebra{A}\); and
        \item \(\pi\circ \adjoint_{\algebra{A}} = \adjoint_{\algebra{B}}\circ \pi\).
    \end{enumerate}
    If there exists a bijective *-homomorphism \(\pi: \algebra{A} \to \algebra{B}\), we say \(\algebra{A}\) is *-isomorphic to \(\algebra{B}\) and that \(\pi\) is a *-isomorphism.
\end{definition}

\begin{lemma}{Composition of *-homomorphisms is a *-homomorphism}{composition_star_homomorphism}
    Let \(\algebra{A}, \algebra{B}, \algebra{C}\) be *-algebras. If \(\pi_1 : \algebra{A} \to \algebra{B}\) and \(\pi_2 : \algebra{B} \to \algebra{C}\) are *-homomorphisms, then \(\pi_2 \circ \pi_1\) is a *-homomorphism.
\end{lemma}
\begin{proof}
    Let \(a, b \in \algebra{A}\), then
    \begin{equation*}
        \pi_2\circ \pi_1(ab) = \pi_2(\pi_1(a) \pi_1(b)) = \pi_2\circ \pi_1(a) \pi_2\circ\pi_1(b)
    \end{equation*}
    and
    \begin{equation*}
        \pi_2\circ \pi_1 \circ \adjoint_{\algebra{A}} = \pi_2 \circ \adjoint_{\algebra{B}} \circ \pi_1 = \adjoint_{\algebra{C}} \circ \pi_2 \circ \pi_1,
    \end{equation*}
    as desired.
\end{proof}
\begin{lemma}{*-isomorphism if and only if its inverse is a *-isomorphism}{inverse_star_isomorphism}
    Let \(\algebra{A}, \algebra{B}\) be *-algebras. The bijective map \(\pi : \algebra{A} \to \algebra{B}\) is a *-isomorphism if and only if \(\pi^{-1} : \algebra{B} \to \algebra{A}\) is a *-isomorphism.
\end{lemma}
\begin{proof}
    Recall that \(\pi \circ \pi^{-1} = \id{\algebra{B}}\) and \(\pi^{-1} \circ \pi = \id{\algebra{A}}\). Suppose \(\pi\) is a *-isomorphism, and let \(b_1, b_2 \in \algebra{B}\). Then,
    \begin{equation*}
        \pi^{-1} (b_1b_2) = \pi^{-1} \left(\pi\circ \pi^{-1}(b_1)\pi\circ \pi^{-1}(b_2)\right) = \pi^{-1} \circ \pi \left(\pi^{-1}(b_1)\pi^{-1}(b_2)\right) = \pi^{-1}(b_1) \pi^{-1}(b_2)
    \end{equation*}
    and
    \begin{equation*}
        \pi^{-1} \circ \adjoint_{\algebra{B}} = \pi^{-1} \circ \adjoint_{\algebra{B}} \circ \pi \circ \pi^{-1} = \pi^{-1} \circ \pi \circ \adjoint_{\algebra{A}} \circ \pi^{-1} = \adjoint_{\algebra{A}}\circ \pi^{-1}
    \end{equation*}
    hence \(\pi^{-1}\) is a *-isomorphism. The converse is shown mutatis mutandis.
\end{proof}
\begin{proposition}{Equivalence relation of *-algebras}{star_isomorphism}
    The relation
    \begin{equation*}
        \algebra{A} \sim_* \algebra{B} \iff \algebra{A}\text{ is *-isomorphic to }\algebra{B}
    \end{equation*}
    is an equivalence relation on *-algebras.
\end{proposition}
\begin{proof}
    \cref{lem:inverse_star_isomorphism,lem:composition_star_homomorphism} show \(\sim_*\) is transitive and symmetric, respectively. For any *-algebra \(\algebra{A}\), the identity map \(\id{\algebra{A}}\) is a *-isomorphism since
    \begin{equation*}
        \id{\algebra{A}}(ab) = \id{\algebra{A}}(a) \id{\algebra{A}}(b)
        \quad\text{and}\quad
        \id{\algebra{A}}(a^*) = \id{\algebra{A}}(a)^*
    \end{equation*}
    hold trivially for any \(a, b \in \algebra{A}\), hence \(\sim_*\) is reflexive.
\end{proof}
\begin{remark}
    If \(\algebra{A}\) is *-isomorphic to \(\algebra{B}\), we say \(\algebra{A}\) and \(\algebra{B}\) are *-isomorphic.
\end{remark}

We show unital *-algebras cannot be *-isomorphic to *-algebras without identity.
\begin{proposition}{*-isomorphism maps identity to identity}{star_isomorphism_identity}
    Let \(\pi : \algebra{A} \to \algebra{B}\) be *-isomorphism of *-algebras. If \(\algebra{A}\) is unital with identity element \(\unity_{\algebra{A}}\), then \(\algebra{B}\) is unital with identity element \(\pi(\unity_{\algebra{A}})\).
\end{proposition}
\begin{proof}
    Suppose \(\algebra{A}\) is unital and let \(\pi : \algebra{A} \to \algebra{B}\) be a *-isomorphism. Let \(b \in \algebra{B}\), then
    \begin{align*}
        b = \pi \circ \pi^{-1}(b) &= \pi(\unity_{\algebra{A}}\pi^{-1}(b)) = \pi(\unity_{\algebra{A}})\pi\circ \pi^{-1}(b) = \pi(\unity_{\algebra{A}})b\\
                                  &= \pi(\pi^{-1}(b) \unity_{\algebra{A}}) = \pi\circ \pi^{-1}(b) \pi(\unity_{\algebra{A}}) = b\pi(\unity_\algebra{A}),
    \end{align*}
    hence \(\pi(\unity_{\algebra{A}})\) is the identity element of \(\algebra{B}\).
\end{proof}
\begin{corollary}
    Let \(\algebra{A}\) and \(\algebra{B}\) be *-isomorphic *-algebras. \(\algebra{A}\) is unital if and only if \(\algebra{B}\) is unital.
\end{corollary}



Even though a *-algebra need not be unital, we may \emph{always} *-isomorphically identify it as a *-subalgebra of a *-algebra with identity. We first show a *-algebra defines a unital *-algebra.
\begin{proposition}{Construction of a unital *-algebra from a *-algebra}{c_plus_star_algebra}
    Let \(\algebra{A}\) be a *-algebra. The operations
    \begin{align*}
        \cdot : \left(\mathbb{C}\times\algebra{A}\right) \times \left(\mathbb{C} \times \algebra{A}\right) &\to \mathbb{C} \times \algebra{A}\\
        \left((\alpha, x),(\beta,y)\right)&\mapsto (\alpha\beta, \alpha y + \beta x + xy)
    \end{align*}
    and
    \begin{align*}
        * : \mathbb{C} \times \algebra{A} &\to \mathbb{C}\times\algebra{A}\\
        (\alpha, x)&\mapsto (\conj{\alpha}, x^*)
    \end{align*}
    define a unital *-algebra on the linear space \(\mathbb{C} \times \algebra{A}\), which we will denote by \(\mathbb{C} \ltimes \algebra{A}\), where \((1, 0) \in \mathbb{C} \ltimes \algebra{A}\) is the identity element.
\end{proposition}
\begin{proof}
    Let \((\alpha, x), (\beta, y), (\gamma, z) \in \mathbb{C} \ltimes \algebra{A}\) and \(\lambda \in \mathbb{C}\), then
    \begin{align*}
        (\alpha, x) \cdot \left((\beta, y) + (\gamma,z)\right)
        &= (\alpha, x) \cdot (\beta + \gamma, y + z)\\
        &= (\alpha \beta + \alpha \gamma, \alpha y + \beta x + xy + \alpha z + \gamma x + xz)\\
        &= (\alpha \beta, \alpha y + \beta x + xy) + (\alpha \gamma, \alpha z + \gamma x + xz)\\
        &= (\alpha, x)\cdot (\beta, y) + (\alpha, x)\cdot(\gamma, z),
    \end{align*}
    \begin{align*}
        \left((\beta, y) + (\gamma,z)\right)\cdot (\alpha, x)
        &= (\beta + \gamma, y + z)\cdot (\alpha, x) \\
        &= (\beta \alpha + \gamma \alpha,  \beta x +\alpha y + yx +  \gamma x + \alpha z +zx)\\
        &= (\beta \alpha,  \beta x + \alpha y +yx) + (\gamma \alpha,  \gamma x +\alpha z + zx)\\
        &= (\beta, y)\cdot (\alpha, x) + (\gamma, z)\cdot(\alpha, x),
    \end{align*}
    \begin{align*}
        (\alpha, x) \cdot \left(\lambda (\beta, y)\right) &= (\alpha, x) \cdot (\lambda \beta, \lambda y)\\
                                                          &= (\alpha \lambda \beta, \alpha \lambda y + \lambda \beta x + \lambda x y) = \lambda \cdot \left((\alpha, x) \cdot (\beta, y)\right)\\
                                                          &= (\alpha \lambda, \lambda x)\cdot (\beta, y) = \left(\lambda (\alpha, x)\right)\cdot (\beta, y),
    \end{align*}
    \begin{align*}
        \left((\alpha,x)\cdot(\beta, y)\right)\cdot (\gamma, z) &= (\alpha \beta, \alpha y + \beta x + xy) \cdot (\gamma, z)\\
                                                                &= (\alpha \beta \gamma, \alpha \beta z + \gamma \alpha y + \gamma \beta x + \gamma xy + \alpha y z + \beta x z + xy z),
    \end{align*}
    \begin{align*}
        (\alpha,x)\cdot\left((\beta, y)\cdot(\gamma, z)\right) &= (\alpha, x) \cdot (\beta \gamma, \beta z + \gamma y + yz)\\
                                                               &= (\alpha \beta \gamma, \alpha \beta z + \alpha \gamma y + \alpha yz + \beta \gamma x + \beta x z + \gamma xy + xyz),
    \end{align*}
    and
    \begin{align*}
        (1, 0) \cdot (\alpha,x) = (\alpha, x) = (\alpha, x) \cdot (1,0)
    \end{align*}
    that is, \(\mathbb{C} \ltimes \algebra{A}\) is an associative algebra with identity. Let \(\kappa \in \mathbb{C}\), then
    \begin{equation*}
        \left((\alpha, x)^*\right)^* = (\conj{\alpha}, x^*)^* = (\alpha, x),
    \end{equation*}
    \begin{equation*}
        \left((\alpha, x)\cdot(\beta, y)\right)^* = (\alpha \beta, \alpha y + \beta x + xy)^* = (\conj{\alpha}\conj{\beta}, \conj{\alpha}y^* + \conj{\beta}x^* + y^* x^*) = (\conj{\beta}, y^*)\cdot(\conj{\alpha}, x^*) = (\beta, y)^*\cdot(\alpha, x)^*,
    \end{equation*}
    \begin{equation*}
        \left(\lambda (\alpha, x) + \kappa (\beta, y)\right)^* = (\lambda \alpha + \kappa \beta, \lambda x + \kappa y)^* = (\conj{\lambda} \conj{\alpha} + \conj{\kappa}\conj{\beta}, \conj{\lambda}x^* + \conj{\kappa}y^*) = \conj{\lambda}(\alpha, x)^* + \conj{\kappa}(\beta, y)^*,
    \end{equation*}
    and \((1,0)^* = (1,0)\), hence \(\mathbb{C} \ltimes \algebra{A}\) is a *-algebra.
\end{proof}

We now show every *-algebra may be *-isomorphically identified with a *-ideal of a unital *-algebra.
\begin{lemma}{*-isomorphism between \(\algebra{A}\) and a *-ideal of \(\mathbb{C} \ltimes \algebra{A}\)}{subalgebra_unit_isomorphism}
    Let \(\algebra{A}\) be a *-algebra without identity. The set \(\algebra{A}_0 = \setc{(\alpha, x) \in \mathbb{C} \ltimes \algebra{A}}{\alpha = 0}\) is a *-ideal of \(\mathbb{C} \ltimes \algebra{A}\). Moreover,
    \begin{align*}
        \pi : \algebra{A} &\to \mathbb{C} \ltimes \algebra{A}\\
                        a &\mapsto (0, a)
    \end{align*}
    is a *-homomorphism between \(\algebra{A}\) and \(\mathbb{C} \ltimes \algebra{A}\) with \(\pi(\algebra{A}) = \algebra{A}_0\), that is, \(\algebra{A}\) is *-isomorphic to \(\algebra{A}_0\).
\end{lemma}
\begin{proof}
    It is clear \(\pi\) is injective and has range equal to \(\algebra{A}_0\). Let \(a, b \in \algebra{A}\), then
    \begin{equation*}
        \pi(ab) = (0,ab) = (0, a) \cdot (0, b) = \pi(a) \pi(b)
        \quad\text{and}\quad
        \pi(a^*) = (0, a^*) = (0, a)^* = \pi(a)^*,
    \end{equation*}
    hence \(\algebra{A}_0\) is self-adjoint and \(\pi\) is a *-homomorphism between \(\algebra{A}\) and \(\mathbb{C} \ltimes \algebra{A}\). Let \((\alpha, x) \in \mathbb{C}\ltimes \algebra{A}\) and let \(a \in \algebra{A}\), then
    \begin{equation*}
        (\alpha, x)\cdot \pi(a) = (0, \alpha a + xa) \in \algebra{A}_0,
    \end{equation*}
    hence \(\algebra{A}_0\) is a *-ideal, by \cref{prop:self_adjoint_ideal}.
\end{proof}

If \(\algebra{A}\) is a C*-algebra we may in fact define a norm on \(\mathbb{C} \ltimes \algebra{A}\) such that it becomes a C*-algebra. This embedding of a C*-algebra without identity in one with identity alleviates some of the difficulties when studying \(\algebra{A}\) due to the lack of an identity.

\begin{theorem}{Isometric *-isomorphism between a \(\algebra{A}\) and a C*-subalgebra of \(\mathbb{C}\ltimes \algebra{A}\)}{adjoin_unity}
    Let \(\algebra{A}\) be a C*-algebra without identity. Then, \(\mathbb{C} \ltimes \algebra{A}\) is a unital C*-algebra with respect to the norm
    \begin{equation*}
        \norm{(\alpha, x)} = \sup_{\substack{d \in \algebra{A}\\\norm{d}=1}}{\norm{\alpha d + xd}}.
    \end{equation*}
    Moreover, the map
    \begin{align*}
        \pi : \algebra{A} &\to \mathbb{C} \ltimes \algebra{A}\\
                        a &\mapsto (0, a)
    \end{align*}
    is a isometric *-homomorphism between \(\algebra{A}\) and \(\mathbb{C} \ltimes \algebra{A}\) , that is, \(\algebra{A}\) is isometrically *-isomorphic to the C*-subalgebra \(\pi(\algebra{A})\).
\end{theorem}
\begin{proof}
    Let us show that \(\norm{\noarg}_{\mathbb{C}\ltimes\algebra{A}}\) is a norm on \(\mathbb{C} \ltimes \algebra{A}\).  Let \((\alpha, x), (\beta, y) \in \mathbb{C} \ltimes \algebra{A},\) and \(\lambda \in \mathbb{C}\), then
    \begin{equation*}
        \norm{\lambda (\alpha, x)} = \sup_{\substack{d \in \algebra{A}\\\norm{d} = 1}}{\norm{\lambda\alpha d + \lambda xd}} = \abs{\lambda}\sup_{\substack{d \in \algebra{A}\\\norm{d} = 1}}{\norm{\alpha d + xd}} = \abs{\lambda}\cdot\norm{(\alpha, x)},
    \end{equation*}
    and
    \begin{align*}
        \norm{(\alpha, x) + (\beta, y)} &= \sup_{\substack{d \in \algebra{A}\\\norm{d} = 1}}{\norm{(\alpha + \beta)d + (x + y)d}}\\
                                        &= \sup_{\substack{d \in \algebra{A}\\\norm{d} = 1}}{\norm{(\alpha + \beta)d + (x + y)d}}\\
                                        &\leq \sup_{\substack{d \in \algebra{A}\\\norm{d} = 1}}{\norm{\alpha d + x d}} + \sup_{\substack{d \in \algebra{A}\\\norm{d} = 1}}{\norm{\beta d + yd}}\\
                                        &= \norm{(\alpha, x)} + \norm{(\beta, y)},
    \end{align*}
    hence \(\norm{\noarg}_{\mathbb{C} \ltimes \algebra{A}}\) is absolute homogeneous and subadditive.

    It is clear it is non-negative since \(\norm{\noarg}_{\algebra{A}}\) is a norm, and it is clear that \(\norm{(0,0)} = 0\). Let \((\gamma, z) \in \mathbb{C} \ltimes \algebra{A}\) such that \(\norm{(\gamma, z)} = 0\).  Suppose \(\gamma \neq 0\), then absolute homogeneity yields
    \begin{equation*}
        \norm{(\gamma, z)} = \abs{\gamma}\cdot\norm{(1, \gamma^{-1}z)},
    \end{equation*}
    therefore \(\norm{(1, -e)} = 0\), where \(e = -\gamma^{-1}z\). That is, for every \(d \in \algebra{A}\) with \(\norm{d} = 1\), we have \(L_{e}d = d\). Since \(L_v\) is linear, we have \(L_{e} = \id{\algebra{A}},\) which shows \(e\) is a left identity for all \(a \in \algebra{A}\). Since the involution is bijective, we also have \(R_{e^*} = \id{\algebra{A}}\), showing \(e^*\) is a right identity for all \(a \in \algebra{A}\). In particular, \(L_e \circ R_{e^*} = R_{e^*}\circ L_e = \id{\algebra{A}}\), that is, \(e\) is self-adjoint. We have just shown \(e\) is an identity for \(\algebra{A},\) contradicting the hypothesis of the lack of an identity for \(\algebra{A}\). This ensures that \(\norm{(\gamma, z)} = 0\) implies we have \(\gamma = 0\). Now we have
    \begin{equation*}
        \norm{(0, z)} = \sup_{\substack{d \in \algebra{A}\\\norm{d} = 1}}{zd} = \norm{z},
    \end{equation*}
    hence the positive-definiteness of the norm \(\norm{\noarg}_{\algebra{A}}\) ensures \(z = 0\). We have thus shown
    \begin{equation*}
        \norm{(\gamma, z)} = 0 \iff (\gamma,z) = (0,0),
    \end{equation*}
    that is, \(\norm{\noarg}_{\mathbb{C} \ltimes \algebra{A}}\) is positive-definite and, therefore, a norm.

    We now show the norm is submultiplicative. Trivially, \(\norm{(\alpha, x)\cdot(\beta,y)} \leq \norm{(\alpha, x)}\norm{(\beta, y)}\) is satisfied for \((\beta,y) = (0,0)\). Suppose \(\norm{(\beta,y)} > 0\), then the sets
    \begin{equation*}
        M = \setc{d \in \algebra{A}}{\norm{d} = 1 \land \norm{\beta d + yd} > 0}
        \quad\text{and}\quad
        N = \setc{d \in \algebra{A}}{\norm{d} = 1 \land \beta d = - yd}
    \end{equation*}
    are disjoint and cover \(\setc{d \in \algebra{A}}{\norm{d} = 1}\). Let \(d \in M\), then
    \begin{equation*}
        \alpha \beta d + (\alpha y + \beta x + xy)d = \alpha \beta d + \alpha (-\beta d) + \beta x d + x(-\beta d) = 0,
    \end{equation*}
    hence
    \begin{equation*}
        \sup_{d \in M}{\norm{\alpha \beta d + (\alpha y + \beta x + xy)d}} \leq \sup_{\substack{d \in \algebra{A}\\\norm{d} = 1}}{\norm{\alpha \beta d + (\alpha y + \beta x + xy)d}} = \sup_{d \in N}{\norm{\alpha \beta d + (\alpha y + \beta x + xy)d}}.
    \end{equation*}
    Let \(\tilde{d} \in N\) and set \(c = \frac{1}{\norm{\beta \tilde{d} + y \tilde{d}}}(\beta \tilde{d} + y \tilde{d})\), then
    \begin{equation*}
        \alpha \beta \tilde{d} + (\alpha y + \beta x + xy)\tilde{d} = \alpha(\beta \tilde{d} + y \tilde{d}) + x(\beta \tilde{d} + y \tilde{d}) =  \norm{\beta \tilde{d} + y \tilde{d}}(\alpha c + xc),
    \end{equation*}
    which yields
    \begin{align*}
        \norm{\alpha \beta \tilde{d} + (\alpha y + \beta x + xy)\tilde{d}}
        &= \norm{\norm{\beta \tilde{d} + y \tilde{d}}(\alpha c + xc)}\\
        &\leq \norm{\beta \tilde{d} + y \tilde{d}} \sup_{\substack{d \in \algebra{A}\\\norm{d}=1}}{\norm{\alpha d + xd}}\\
        &= \norm{(\alpha,x)}\norm{\beta \tilde{d} + y \tilde{d}}.
    \end{align*}
    We then get the desired result since
    \begin{align*}
        \norm{(\alpha, x)\cdot(\beta, y)} &= \sup_{\substack{d \in \algebra{A}\\\norm{d} = 1}}{\norm{\alpha \beta d + (\alpha y + \beta x + xy)d}}\\
                                          &= \sup_{\tilde{d} \in N}{\norm{\alpha \beta \tilde{d} + (\alpha y + \beta x + xy)\tilde{d}}}\\
                                          &\leq \norm{(\alpha,x)} \sup_{\tilde{d} \in N}{\norm{\beta \tilde{d} + y \tilde{d}}}\\
                                          &\leq \norm{(\alpha, x)}\norm{(\beta,y)}.
    \end{align*}
    We also have \(\norm{(1,0)} = \sup{\setc{\norm{d}}{d \in \algebra{A} : \norm{d} = 1}} = 1\), from which we conclude \(\mathbb{C} \ltimes \algebra{A}\) is a normed algebra.

    The C*-property on \(\algebra{A}\) yields
    \begin{align*}
        \norm{\lambda b + ab}^2 &= \norm{(\lambda b + ab)^*(\lambda b + ab)}\\
                                &= \norm{b^* (\conj{\lambda} \lambda b + \conj{\lambda}ab + \lambda a^*b + a^*a b)}\\
                                &\leq \norm{b}\norm{\conj{\lambda} \lambda b + \conj{\lambda}ab + \lambda a^*b + a^*a b}
    \end{align*}
    for all \(a,b \in \algebra{A}\), and \(\lambda \in \mathbb{C}\). In particular, we let \((\alpha, x) \in \mathbb{C} \ltimes \algebra{A}\), then
    \begin{align*}
        \norm{(\alpha, x)}^2 &= \sup_{\substack{d \in \algebra{A}\\\norm{d} = 1}}{\norm{\alpha d + xd}^2}\\
                             &\leq \sup_{\substack{d \in \algebra{A}\\\norm{d} = 1}}{\norm{\alpha \conj{\alpha} d  + \conj{\alpha}xd + \alpha x^*d + x^*x d}}\\
                             &\leq \norm{(\alpha \conj{\alpha}, \conj{\alpha}x + \alpha x^* + x^*x)} = \norm{(\alpha, x)^*(\alpha, x)}\\
                             &\leq\norm{(\alpha, x)^*}\norm{(\alpha, x)},
    \end{align*}
    and we may conclude \(\norm{(\alpha, x)} \leq \norm{(\alpha, x)^*}\). Replacing \((\alpha, x)\) with \((\alpha, x)^*\) yields \(\norm{(\alpha, x)^*} \leq \norm{(\alpha, x)}\), hence showing the B*-property. Moreover, we have shown \(\norm{(\alpha, x)}^2 \leq \norm{(\alpha, x)^*(\alpha, x)}\), then the B*-property and submultiplicativity show
    \begin{equation*}
        \norm{(\alpha, x)}^2 \leq \norm{(\alpha, x)^*(\alpha, x)} \leq \norm{(\alpha, x)^*}\norm{\alpha, x)} \leq \norm{(\alpha, x)}^2,
    \end{equation*}
    hence \(\mathbb{C} \ltimes \algebra{A}\) has the C*-property.

    \cref{lem:subalgebra_unit_isomorphism} shows us \(\pi\) *-isomorphically identifies \(\algebra{A}\) with \(\pi(\algebra{A})\), a *-ideal of \(\mathbb{C}\ltimes \algebra{A}\). This map is also a isometry since
    \begin{equation*}
        \norm{\pi(a)} = \norm{(0,a)} =\sup_{\substack{d \in \algebra{A}\\\norm{d}=1}}{\norm{ad}} = \norm{a}.
    \end{equation*}
    An immediate consequence of this isometry is that a sequence \(\family{a_n}{n\in \mathbb{N}}\subset \algebra{A}\) is Cauchy if and only if \(\family{\pi(a_n)}{n\in \mathbb{N}}\subset \mathbb{C} \ltimes \algebra{A}\) is Cauchy. Furthermore, the sequence in \(\algebra{A}\) converges to \(\tilde{a}\) if and only if the sequence in \(\pi(\algebra{A})\) converges to \(\pi(\tilde{a})\). In particular, this ensures \(\pi(\algebra{A})\) is a closed *-ideal of \(\mathbb{C} \ltimes \algebra{A}\).

    Let \(\family{(\alpha_n, x_n)}{n \in \mathbb{N}}\subset \mathbb{C} \ltimes \algebra{A}\) be a Cauchy sequence, then there exists \(M > 0\) such that \(\norm{(\alpha_n, x_n)} < M\) for all \(n \in \mathbb{N}\). First, we claim \(\family{\alpha_n}{n \in \mathbb{N}}\subset \mathbb{C}\) is bounded. Suppose it is not, then there exists a subsequence \(\family{\alpha_{n_j}}{j \in \mathbb{N}} \subset \mathbb{C}\) that diverges, that is, \(\norm{\alpha_{n_j}} \to \infty\) as \(j \to \infty\). We may remove, if necessary, elements from the sequence \(\family{n_j}{j \in \mathbb{N}}\subset \mathbb{N}\) until \(\alpha_{n_j} \neq 0\) for all \(j \in \mathbb{N}\), without changing the fact that the resulting subsequence diverges. Then, for all \(j \in \mathbb{N}\) we have
    \begin{equation*}
        \norm{(1, \alpha_{n_j}^{-1}x_{n_j})} = \frac1{\norm{\alpha_{n_j}}} \norm{(\alpha_{n_j}, x_{n_j})} = \frac{M}{\norm{\alpha_{n_j}}},
    \end{equation*}
    hence \((1, \alpha_{n_j}^{-1}x_{n_j}) \to (0,0)\). Continuity yields \(\pi(-\alpha_{n_j}^{-1} x_{n_j}) = (0, -\alpha_{n_j}^{-1} x_{n_j}) \to (1, 0) \notin \pi(\algebra{A})\), a contradiction. It must be the case, then, that \(\family{\alpha_n}{n \in \mathbb{N}}\) is bounded.

    Let \(\varepsilon > 0\), then there exists \(N > 0\) such that \(n, m \geq N \implies \norm{(\alpha_m - \alpha_n, x_m - x_n)} < \frac12 \varepsilon\). As a bounded sequence of complex numbers, we may take a subsequence \(\family{\alpha_{n_k}}{k\in \mathbb{N}} \subset \mathbb{C}\) that converges to some \(\tilde{\alpha} \in \mathbb{C}\), per Bolzano-Weierstrass. Then, there exists \(K > 0\) such that for all \(k, \ell \geq K\) we have \(\abs{\alpha_{n_k} - \alpha_{n_\ell}} < \frac12 \varepsilon\). As a result, for all \(k, \ell \geq \max{\set{N, K}}\), we have
    \begin{align*}
        \norm{(0, x_{n_k}) - (0, x_{n_\ell})} &= \norm*{\left((\alpha_{n_k}, x_{n_k}) - (\alpha_{n_k}, 0)\right) - \left((\alpha_{n_\ell}, x_{n_\ell}) - (\alpha_{n_\ell}, 0)\right)}\\
                                              &\leq \norm{(\alpha_{n_k} - \alpha_{n_\ell}, x_{n_k} - x_{n_\ell})} + \norm{(\alpha_{n_k} - \alpha_{n_\ell}, 0)}\\
                                              &< \varepsilon,
    \end{align*}
    where we have used that the map \(\mathbb{C} \ltimes \algebra{A} \ni (\lambda,0) \mapsto \lambda \in \mathbb{C}\) is an isometry. The previous result states \(\family{\pi(x_{n_k})}{k \in \mathbb{N}}\) is a Cauchy sequence, hence \(\family{x_{n_k}}{k \in \mathbb{N}}\) is a Cauchy sequence, which, by completeness, converges to some \(\tilde{x} \in \algebra{A}\) and, in turn, \(\pi(x_{n_k}) \to \pi(\tilde{x})\). Finally, we have
    \begin{align*}
        \norm{(\alpha_n, x_n) - (\tilde{\alpha}, \tilde{x})} &= \sup_{\substack{d\in\algebra{A}\\\norm{d}=1}}{\norm{(\alpha_n - \tilde{\alpha})d + (x_n - \tilde{x})d}}\\
                                                             &\leq \sup_{\substack{d\in\algebra{A}\\\norm{d}=1}}{\left[\norm{(\alpha_n - \tilde{\alpha})d} + \norm{(x_n - \tilde{x})d}\right]}\\
                                                             &\leq \sup_{\substack{d\in\algebra{A}\\\norm{d}=1}}{\left(\abs{\alpha_n - \tilde{\alpha}} + \norm{x_n - \tilde{x}}\right)\norm{d}}\\
                                                             &=\abs{\alpha_n - \tilde{\alpha}} + \norm{x_n - \tilde{x}}
    \end{align*}
    for all \(n \in \mathbb{N}\), hence \((\alpha_n, x_n) \to (\tilde{\alpha}, \tilde{x})\), establishing \(\mathbb{C} \ltimes \algebra{A}\) as a C*-algebra with unity.
\end{proof}


% vim: spl=en_us
\section{Inverse of bounded operators}
Prior to defining the spectrum of an operator of a Banach algebra, we study the inverse of linear maps in Banach spaces and the generalization for unital Banach algebras.
\begin{theorem}{Inverse of a bounded operator}{inverse_bounded}
    Let \(T \in \bounded(X)\) be a bounded operator acting on a Banach space \(X\). Only one of the following statements must hold:
    \begin{enumerate}[label=(\alph*)]
        \item If \(T\) is bijective, then \(T^{-1} \in \bounded(X)\).
        \item If \(T\) is injective, not surjective, and \(\cl_X(\range{T}) = X\), then \(T^{-1} : \range{T} \to X\) exists but it is not bounded.
        \item If \(T\) is injective, not surjective, and \(\cl_X(\range{T}) \subsetneq X\), then \(T^{-1} : \range{T} \to X\) exists.
        \item If \(T\) is not injective, then we may not define an inverse on \(\range{T}\).
    \end{enumerate}
\end{theorem}
\begin{proof}
    It is clear the situations are mutually exclusive. The case (a) is merely restating \cref{thm:bounded_inverse_theorem}. Case (d) is evident, as a map may not be a one-to-many relation. We move on to cases (b) and (c).

    If \(T\) is injective and not surjective, then we may define a map \(T^{-1} : \range{T} \to X\) such that \(T^{-1} \circ T = \id{X}\) and \(T \circ T^{-1} = \id{\range{T}}\). If, in addition, \(\range{T}\) is dense in \(X\), then we show \(T^{-1} \notin \bounded(\range{T})\). Suppose, by contradiction, \(T^{-1}\) is bounded, then there exists a unique \(W \in \bounded(X)\) that extends \(T^{-1}\) by the BLT theorem. Since it is a extension, we have
    \begin{equation*}
        W \circ T = \restrict{W}{\range{T}} \circ T = T^{-1} \circ T = \id{X}.
    \end{equation*}
    Let \(x \in X\), then for every sequence \(\family{x_n}{n \in \mathbb{N}}\subset \range{T}\) that converges against \(x\), we have
    \begin{equation*}
        T\circ W(x) = T\left(\lim_{n\to \infty} T^{-1}x_n\right) = \lim_{n\to \infty} T\circ T^{-1} x_n = \lim_{n\to\infty} x_n = x,
    \end{equation*}
    that is, \(T \circ W = \id{X}\). This shows \(T\) is bijective, contradicting the hypothesis that \(T\) is not surjective, hence we conclude (b) and (c).
\end{proof}
\begin{corollary}
    Let \(T\in \bounded(X)\) be a bounded operator in a Banach space \(X\). If \(T^{-1}\) exists and is bounded, then \(\range{T}\) is closed.
\end{corollary}
\begin{proof}
    Let \(\family{y_n}{n\in \mathbb{N}} \subset \range{T}\) be a sequence that converges against some \(\tilde{y} \in X\), then there exists \(\family{x_n}{n \in \mathbb{N}} \subset X\) such that \(x_n = T^{-1} y_n\). Since \(T^{-1}\) is uniformly continuous, \(x_n\) is a Cauchy sequence, hence converges against some \(\tilde{x} \in X\). For all \(n \in \mathbb{N}\), we have
    \begin{equation*}
        \norm{T\tilde{x} - y_n} = \norm{T(\tilde{x} - x_n)} \leq \norm{T}\norm{\tilde{x} - x_n},
    \end{equation*}
    hence \(y_n \to T\tilde{x}\). That, is \(\tilde{y} = T\tilde{x} \in \range{T}\).
\end{proof}

By abstracting the notion of inverses to Banach algebras, we may obtain results in this context and particularize them for bounded operators in Banach or Hilbert spaces.
\begin{definition}{Inverse of an operator}{inverse_operator}
    Let \(\algebra{A}\) be an associative algebra with identity. An operator \(a \in \algebra{A}\) is said to be \emph{invertible} if there exists \(b \in \algebra{A}\) such that \(ab = ba = \unity\), which is then said to be the \emph{inverse element}, or simply \emph{inverse}, of \(a\). We denote the set of invertible elements of \(\algebra{A}\) by \(\invertible{\algebra{A}}\).
\end{definition}
\begin{remark}
    It should be clear that an inverse element is unique from associativity.
\end{remark}

\begin{proposition}{Invertibility compatible with scalar multiplication}{invertible_scalar}
    Let \(\algebra{A}\) be an associative algebra with identity. An operator \(w \in \algebra{A}\) is invertible if and only if \(\lambda w \in \invertible{\algebra{A}}\) for all \(\lambda \in \mathbb{C} \setminus \set{0}.\)
\end{proposition}
\begin{proof}
    Suppose \(w \in \invertible{\algebra{A}}\) and let \(\lambda \in \mathbb{C} \setminus \set{0
    }\). Then
    \begin{equation*}
        (\lambda^{-1} w^{-1})(\lambda w) = w^{-1} w = \unity
        \quad\text{and}\quad
        (\lambda w)(\lambda^{-1} w^{-1}) = w w^{-1} = \unity
    \end{equation*}
    that is, \(\lambda w \in \invertible{\algebra{A}}\). The converse is obviously true.
\end{proof}

Unlike finite dimensional spaces, it is necessary to require both \(ab = \unity\) and \(ba = \unity\) in infinite dimensional spaces, as the following example illustrates.
\begin{example}{Shift operators in \(\ell_2\)}{shift_example}
    Let \(a \in \ell_2\), the \emph{shift operator} \(S : \ell_2 \to \ell_2\) is defined by the sequence \(Sa\), where \(Sa(1) = 0\) and \(Sa(n+1) = a(n)\) for all \(n \in \mathbb{N}\). Then its adjoint map \(S^*\) is defined by \(Sa(n) = a(n+1)\) for all \(n \in \mathbb{N}\) and satisfies \(S^*\circ S = \unity\), but does not satisfy \(S \circ S^* = \unity\).
\end{example}
\begin{proof}
    Let \(x,y \in \ell_2\), then
    \begin{align*}
        \inner{x}{Sy} = \sum_{k=1}^\infty \conj{x(k)} Sy(k)
                      = \sum_{k=2}^\infty \conj{x(k)} y(k-1)
                      = \sum_{k=1}^\infty \conj{x(k+1)}y(k)
                      = \sum_{k=1}^\infty \conj{S^*x(k)}y(k)
                      = \inner{S^*x}{y},
    \end{align*}
    as claimed. We also have
    \begin{equation*}
        S^* \circ Sx(n) = S^*x(n+1) = x(n),
    \end{equation*}
    for all \(n \in \mathbb{N}\), that is \(S^* \circ S = \unity\). Notice, however the sequence \(S \circ S^* x\) has \(S\circ S^* x(1) = 0 \neq x(1),\) hence \(S \circ S^* \neq \unity\).
\end{proof}

We recall the definition of a group.
\begin{definition}{Group}{group}
    A \emph{group} \((S, \bullet)\) is a non-empty set \(S\) equipped with an associative product \(\bullet : S \times S \to S\) satisfying the following properties:
    \begin{enumerate}[label=(\alph*)]
        \item there exists a unique element \(e \in S\), called the \emph{identity element}, such that \(e\bullet s = s \bullet e = s\) for all \(s \in S\); and
        \item for each \(s \in S\) there exists a unique element \(s^{-1} \in S\), called the \emph{inverse element of \(s\)}, such that \(s \bullet s^{-1} = s^{-1} \bullet s = e\).
    \end{enumerate}
    If, in addition, \topology{S} is a topological space, we say \(S\) is a \emph{topological group}, or \emph{continuous group}, if the maps \(\bullet : S \times S \to S\) and \(^{-1} : S \to S\) are continuous.
\end{definition}

We'll follow the group notation and denote the inverse element of \(a \in \algebra{A}\) by \(a^{-1}\).
\begin{proposition}{Group of invertible operators}{group_invertible}
    Let \(\algebra{A}\) be an associative algebra with identity. Then \((\invertible{\algebra{A}}, \cdot)\) is a group.
\end{proposition}
\begin{proof}
    Let \(a, b \in \invertible{\algebra{A}}\), then
    \begin{equation*}
        (b^{-1} a^{-1}) (ab) = b^{-1} (a^{-1}a)b = b^{-1}b = \unity\quad\text{e}\quad
        (ab) (b^{-1} a^{-1}) = a(bb^{-1})a^{-1} = aa^{-1} = \unity,
    \end{equation*}
    that is, the product is closed in \(\invertible{\algebra{A}}\). It is clear that the identity element is \(\unity\) and the existence of the inverse elements follows by construction.
\end{proof}
\begin{proposition}{Unitary operators form a subgroup of the invertible operators}{unitary_subgroup}
    Let \(\algebra{A}\) be a unital *-algebra. Then the set
    \begin{equation*}
        U = \setc{a \in \algebra{A}}{a^*a = \unity \land aa^* = \unity}
    \end{equation*}
     is a self-adjoint subgroup of \(\invertible{\algebra{A}}\).
\end{proposition}
\begin{proof}
    Noting that \(\unity \in U\), we have in particular that \(U\) is non-empty. Let \(a \in U\), then \(a^*\) satisfies \(a^*(a^*)^* = a^*(a^*)^*= \unity\), hence \(a^* \in U\), that is, \(U\) is self-adjoint.

    Notice that for any \(a \in U\), we have \(a^{-1} = a^*,\) and since \(U\) is self-adjoint, we have \(a^{-1} \in U\). Let \(u,v \in U\), then
    \begin{equation*}
        (uv)^*(uv) = v^*u^*uv = v^* v = \unity = uu^* = uvv^*u^* = (uv)(uv)^*,
    \end{equation*}
    that is, \(uv \in U\). Since \(\unity\) is the identity element of \(\invertible{\algebra{A}},\) we have \(U\) as a subgroup of the group of invertible operators.
\end{proof}

If \(\algebra{A}\) is an involutive algebra, the adjoint operation is compatible with the inverse map.
\begin{proposition}{Involution is compatible with the inverse map}{involution_inverse}
    Let \(\algebra{A}\) be a unital *-algebra. Then, \(\invertible{\algebra{A}}\) is self-adjoint and
    \begin{equation*}
        (w^{-1})^* = (w^*)^{-1}
    \end{equation*}
    for all \(w \in \invertible{\algebra{A}}\).
\end{proposition}
\begin{proof}
    Let \(w \in \invertible{\algebra{A}}\), then
    \begin{equation*}
        (w^{-1}w)^* = \unity^* \implies w^* (w^{-1})^* = \unity
    \end{equation*}
    and
    \begin{equation*}
        (ww^{-1})^* = \unity^* \implies (w^{-1})^* w^* = \unity,
    \end{equation*}
    that is, \(w^* \in \invertible{\algebra{A}}\), with \((w^*)^{-1} = (w^{-1})^*\).
\end{proof}

It is often important to know sufficient conditions for the existence of the inverse of an operator.
\begin{proposition}{Sufficient condition for the invertibility of an operator}{prop411}
    Let \(\algebra{A}\) be a unital associative algebra. Let \(u, v \in \algebra{A}\), then \(\unity - uv \in \invertible{\algebra{A}}\) if and only if \(\unity - vu \in \invertible{\algebra{A}}\).
\end{proposition}
\begin{proof}
    Suppose \(\unity - vu \in \invertible{\algebra{A}}\) and set \(w = (\unity - vu)^{-1}\). Then,
    \begin{align*}
        (\unity - uv)(\unity + uwv) &= \unity + uwv - uv - uvuwv&
        (\unity + uwv)(\unity - uv) &= \unity + uwv - uv - uwvuv\\
                                    &= \unity - uv + u(\unity - vu)wv&
                                    &= \unity - uv + uw(\unity - vu)v\\
                                    &= \unity - uv + uv&
                                    &= \unity - uv + uv\\
                                    &= \unity&
                                    &= \unity,
    \end{align*}
    that is, \(\unity - uv \in \invertible{\algebra{A}}\). The converse is shown by simply replacing \(u\) with \(v\).
\end{proof}

Recall \cref{thm:neumann-series} where we provided a strong limit for the inverse of a bounded operator defined on a Banach space. We now show that series is actually uniformly convergent.
\begin{theorem}{C. Neumann series}{neumann_series_algebra}
    Let \(\algebra{A}\) be a Banach algebra with identity. If \(w \in \algebra{A}\) is such that \(\norm{w} < 1\), then \(\unity - w \in \invertible{\algebra{A}}\) with
    \begin{equation*}
        (\unity - w)^{-1} = \unity + \sum_{k=1}^\infty w^k,
    \end{equation*}
    with convergence given with respect to the norm of \(\algebra{A}\).
\end{theorem}
\begin{proof}
    Let \(s_n = \unity + \sum_{k=1}^n w^k\) for all \(n \in \mathbb{N}\).  Then, for all \(n, m \in \mathbb{N}\) with \(m > n\) we have
    \begin{equation*}
        \norm{s_n - s_m} \leq \sum_{k=n+1}^m \norm*{w^k}
        \leq \sum_{k = n+1}^m \norm{w}^k
        \leq \norm{w}^{n+1} \sum_{k=0}^{m-n-1} \norm{w}^k
        < \norm{w}^{n+1} \sum_{k=0}^{\infty} \norm{w}^k = \frac{\norm{w}^{n+1}}{1 - \norm{w}}.
    \end{equation*}
    We may take \(n\) arbitrarily large as to make \(\norm{s_n - s_m}\) arbitrarily small, hence \(\family{s_n}{n\in \mathbb{N}}\) is a Cauchy sequence in \(\algebra{A}\). By completeness, we've shown \(\family{s_n}{n\in \mathbb{N}}\) converges against some  \(v \in \algebra{A}\).

    Note also that \(w^k \to 0\) since \(\norm{w^k} < \norm{w}^k \to 0\). Then, the continuity of the product yields
    \begin{align*}
        wv &= w + w\left(\lim_{n\to \infty} \sum_{k=1}^n w^k\right)&
        vw &= w + \left(\lim_{n\to \infty} \sum_{k=1}^n w^k\right)w\\
           &= w + \lim_{n\to\infty} \sum_{k=1}^n w^{k+1}&
           &= w + \lim_{n\to\infty} \sum_{k=1}^n w^{k+1}\\
           &= w + \lim_{n \to \infty} \left(w^{n+1} - w + \sum_{k = 1}^n w^k\right)&
           &= w + \lim_{n \to \infty} \left(w^{n+1} - w + \sum_{k = 1}^n w^k\right)\\
           &= \lim_{n \to \infty} w^{n+1} + \lim_{n\to \infty} \sum_{k=1}^\infty w^k&
           &= \lim_{n \to \infty} w^{n+1} + \lim_{n\to \infty} \sum_{k=1}^\infty w^k\\
           &= v - \unity&
           &= v - \unity,
    \end{align*}
    that is, \( (\unity-w)v = v(\unity- w) =\unity\).
\end{proof}

The following results will show the group of invertible elements in a unital Banach algebra is a topological group with respect to the uniform topology.
\begin{proposition}{Set of invertible elements is open in the uniform topology}{invertible_open}
    Let \(\algebra{A}\) be a unital Banach algebra. If \(v \in \algebra{A}\) satisfies \(\norm{\unity - v w^{-1}} < 1\) for some some \(w \in \invertible{\algebra{A}}\), then \(v \in \invertible{\algebra{A}}\) with
    \begin{equation*}
        v^{-1} = w^{-1} \left[\unity + \sum_{k=1}^\infty\left(\unity - vw^{-1}\right)^k\right],
    \end{equation*}
    converging in the uniform norm.
\end{proposition}
\begin{proof}
    By \cref{thm:neumann_series_algebra}, we have \(vw^{-1} \in \invertible{\algebra{A}}\), hence \(v = vw^{-1}w\) is invertible. We also have
    \begin{equation*}
        wv^{-1} = (vw^{-1})^{-1} = \left[\unity - (\unity - vw^{-1})\right]^{-1} = \unity + \sum_{k=1}^\infty (\unity - vw^{-1})^{k},
    \end{equation*}
    and the result follows.
\end{proof}
\begin{corollary}
    Let \(\algebra{A}\) be a unital Banach algebra. Then the set of invertible elements \(\invertible{\algebra{A}}\) is open in the uniform topology.
\end{corollary}
\begin{proof}
    Let \(w \in S\), then consider the open ball \(S\) of radius \(\norm{w^{-1}}^{-1}\) centered at \(w\). Let \(v \in S\), then
    \begin{equation*}
        \norm{\unity - vw^{-1}} \leq \norm{v - w}\norm{w^{-1}} < \norm{w^{-1}}^{-1}\norm{w^{-1}} = 1,
    \end{equation*}
    and it follows that \(v \in \invertible{\algebra{A}}\). That is, \(\invertible{\algebra{A}} \subset \inte_{\algebra{A}}\invertible{\algebra{A}}\).
\end{proof}

\begin{proposition}{The group of invertible operators is a continuous group}{invertible_continuous_group}
    Let \(\algebra{A}\) be a unital Banach algebra. Then the map
    \begin{align*}
        \noarg^{-1} : \invertible{\algebra{A}} &\to \invertible{\algebra{A}}\\
                                             w &\mapsto w^{-1}
    \end{align*}
    is continuous with respect to the topology on \(\invertible{\algebra{A}}\) induced by the uniform topology.
\end{proposition}
\begin{proof}
    Let \(\tilde{v} \in \invertible{\algebra{A}}\), and let \(S\) be the open ball around \(\tilde{v}\) with radius \(\norm{\tilde{v}^{-1}}^{-1}\), which is contained in \(\invertible{\algebra{A}}\) by the previous result. For all \(u \in S\) we have
    \begin{equation*}
        u = u - \tilde{v} + \tilde{v} = \tilde{v}\tilde{v}^{-1}(u - \tilde{v}) + \tilde{v} = \tilde{v}\left[\unity + \tilde{v}^{-1}(u -\tilde{v})\right] \implies u^{-1} = \left[\unity + \tilde{v}^{-1}(u -\tilde{v})\right]^{-1} \tilde{v}^{-1},
    \end{equation*}
    hence
    \begin{equation*}
        u^{-1} - \tilde{v}^{-1} = \left\{\left[\unity + \tilde{v}^{-1}(u -\tilde{v})\right]^{-1} - \unity\right\}\tilde{v}^{-1}.
    \end{equation*}
    Now \(\norm{\tilde{v}^{-1}(u - \tilde{v})} \leq \norm{\tilde{v}^{-1}}\norm{u-\tilde{v}} < \norm{\tilde{v}^{-1}}\norm{\tilde{v}^{-1}}^{-1} = 1\), then
    \begin{equation*}
        u^{-1} - \tilde{v}^{-1} = \left\{\sum_{k=1}^\infty (-1)^k \left[\tilde{v}^{-1}(u-\tilde{v})\right]^k \right\}\tilde{v}^{-1},
    \end{equation*}
    for all \(u \in S\).

    Let \(\varepsilon > 0\) and set
    \begin{equation*}
        \delta = \frac{\varepsilon\norm{\tilde{v}^{-1}}^{-1}}{\norm{\tilde{v}^{-1}} + \varepsilon} < \norm{\tilde{v}^{-1}}^{-1}
    \end{equation*}
    then for all \(v \in \invertible{\algebra{A}}\) such that \(\norm{\tilde{v} - v} < \delta\),  we have \(v \in S\) and, as a consequence,
    \begin{equation*}
        \norm{v^{-1} - \tilde{v}^{-1}} \leq \norm{\tilde{v}^{-1}} \sum_{k=1}^\infty \norm{\tilde{v}^{-1}(v - \tilde{v})}^k = \norm{v^{-1}} \frac{\norm{\tilde{v}^{-1}(v - \tilde{v})}}{1 - \norm{\tilde{v}^{-1}(v - \tilde{v})}} \leq \frac{\norm{\tilde{v}^{-1}}^2\norm{v - \tilde{v}}}{1 - \norm{\tilde{v}^{-1}}\norm{v - \tilde{v}}},
    \end{equation*}
    since \(\norm{\tilde{v}^{-1}(v-\tilde{v})} < 1\). Notice
    \begin{equation*}
        1 - \delta \norm{\tilde{v}^{-1}} < 1 - \norm{\tilde{v}^{-1}}\norm{\tilde{v} - v} < 1,
    \end{equation*}
    from which follows
    \begin{align*}
        \norm{v^{-1} - \tilde{v}^{-1}} \leq \frac{\norm{\tilde{v}^{-1}}^2\norm{v - \tilde{v}}}{1 - \norm{\tilde{v}^{-1}} \norm{\tilde{v} - v}} < \frac{\norm{\tilde{v}^{-1}}^2\norm{v - \tilde{v}}}{1 - \delta \norm{\tilde{v}^{-1}}} = \norm{\tilde{v}^{-1}}^2\norm{v - \tilde{v}}\left(\norm{\tilde{v}^{-1}}^{-1}\varepsilon + 1\right) < \varepsilon
    \end{align*}
    whenever \(\norm{\tilde{v} - v} < \delta\).
\end{proof}
\begin{corollary}
    Let \(\algebra{A}\) be a unital Banach algebra. Then, \(\invertible{\algebra{A}}\) is a topological group with respect to the topology induced on \(\invertible{\algebra{A}}\) by the uniform topology.
\end{corollary}

% vim: spl=en_us
\section{Spectra of operators in Banach algebras}
As we have introduced the set of invertible operators of a unital associative algebra, we may study the notion of spectra and resolvent sets. In Banach algebras, the spectrum of an operator is much simpler than in general algebras, as we will show. The results here derived will be particularized to bounded operators in Banach or Hilbert spaces at a later point; for now we focus solely on Banach algebras.
\begin{definition}{Resolvent and spectrum}{resolvent_spectrum}
    Let \(\algebra{A}\) be a unital associative algebra and let \(a \in \algebra{A}\). The set
    \begin{equation*}
        \rho(a) = \setc*{\lambda \in \mathbb{C}}{\lambda \unity - a \in \invertible{\algebra{A}}}
    \end{equation*}
    is the \emph{resolvent set of \(a\)}.  The set
    \begin{equation*}
        \sigma(a) = \setc*{\lambda \in \mathbb{C}}{\lambda \unity - a \notin \invertible{\algebra{A}}}
    \end{equation*}
    is the \emph{spectrum \(\sigma(a)\) of \(a\)}.
\end{definition}
\begin{remark}
    It is clear the spectrum is the complement of the resolvent set, that is, \(\sigma(a) = \mathbb{C} \setminus \rho(a)\).
\end{remark}

\cref{prop:prop411} can be stated in terms of the spectra and resolvent sets of \(uv\) and \(vu\).
\begin{corollary}
    Let \(\algebra{A}\) be a unital associative algebra. Let \(u, v \in \algebra{A}\), then \(\sigma(uv) \setminus \set{0} = \sigma(vu) \setminus\set{0}\) and \(\rho(uv)\setminus\set{0} = \rho(vu)\setminus\set{0}\).
\end{corollary}
\begin{proof}
    Let \(\lambda \in \rho(vu)\setminus\set{0}\), then \(\lambda \unity - vu \in \invertible{\algebra{A}}\). By \cref{prop:invertible_scalar}, we have \(\unity - \lambda^{-1} vu \in \invertible{\algebra{A}}\), hence \( \unity - \lambda^{-1} uv \in \invertible{\algebra{A}}\) by \cref{prop:prop411}, that is, \(\rho(vu) \setminus \set{0} \subset \rho(uv)\setminus \set{0}\). The converse is shown by replacing \(u\) and \(v\).

    Let \(\lambda \in \sigma(uv) \setminus \set{0},\) then \(\lambda \unity - uv \notin \invertible{\algebra{A}}\). \cref{prop:invertible_scalar} shows that \(\unity - \lambda^{-1} uv\) must not be invertible. \cref{prop:prop411} lets us conclude \(\unity - \lambda^{-1} vu \notin \invertible{\algebra{A}}\), hence \(\lambda \in \sigma(vu) \setminus\set{0}\). The converse is shown \emph{mutatis mutandis}.
\end{proof}

That is, the spectra of \(uv\) and \(vu\) may only be different at zero, which we illustrate with the shift operators in \(\ell_2\).
\begin{example}{Spectra of the shift operators}{spectra_shift}
    Consider the shift operators \(S, S^* \in \bounded(\ell_2)\) defined in \cref{exam:shift_example}. Then \(\sigma(S\circ S^*) = \set{0,1}\) and \(\sigma(S^* \circ S) = \set{1}\).
\end{example}
\begin{proof}
    It is clear \(\rho(S^*\circ S) = \mathbb{C} \setminus \set{1}\), since \(\lambda\unity - S^* \circ S = (\lambda - 1)\unity\), which is invertible for all \(\lambda \neq 1\). To see \(\set{0,1}\subset\sigma(S \circ S^*)  \), we consider the sequences \(u,v \in \ell_2\) defined by
    \begin{equation*}
        u(n) = \begin{cases}
            1,&\text{if }n = 1\\
            0,&\text{otherwise}.
        \end{cases}
        \quad\text{and}\quad
        v(n) = \begin{cases}
            0,&\text{if }n = 1\\
            2^{-n},&\text{otherwise}
        \end{cases}
    \end{equation*}
    Then, \(S \circ S^*u = 0\) and \((\unity - S \circ S^*)v = 0\), hence \(S \circ S^*\) and \((\unity - S\circ S^*)\) are not injective. The previous corollary guarantees the spectrum of \(S\circ S^*\) is \(\set{0,1}\).
\end{proof}

An immediate consequence of the previous corollary is the similarity invariance of the spectrum.
\begin{proposition}{Similarity invariance}{similarity_invariance}
    Let \(\algebra{A}\) be a unital associative algebra. If \(u \in \invertible{\algebra{A}}\), then \(\sigma(uvu^{-1}) = \sigma(v)\) for all \(v \in \algebra{A}\).
\end{proposition}
\begin{proof}
    It is clear that \(\sigma(uvu^{-1})\setminus\set{0} = \sigma(v)\setminus\set{0}\). Since
    \begin{equation*}
        0 \in \rho(v) \iff v \in \invertible{\algebra{A}} \iff uvu^{-1} \in \invertible{\algebra{A}} \iff 0 \in \rho(uvu^{-1}),
    \end{equation*}
    we have \(\sigma(uvu^{-1}) = \sigma(v)\).
\end{proof}
\begin{corollary}
    If \(u,v \in \invertible{\algebra{A}}\), then \(\sigma(uv) = \sigma(vu)\).
\end{corollary}
\begin{proof}
    We have \(\sigma(u(vu)u^{-1}) = \sigma(vu)\) by the previous proposition.
\end{proof}

The spectra of the adjoint and of the inverse operators to an invertible operator can be easily related to its spectrum.
\begin{proposition}{Spectrum of the inverse operator}{spectrum_inverse}
    Let \(\algebra{A}\) be a unital associative algebra. If \(u \in \invertible{\algebra{A}}\), then \(\sigma(u^{-1}) = \setc{\lambda \in \mathbb{C}}{\lambda^{-1} \in \sigma(u)}\).
\end{proposition}
\begin{proof}
    Since \(u \in \invertible{\algebra{A}}\), \(0 \notin \sigma(u)\). For \(\lambda \in \mathbb{C} \setminus \set{0}\), we have
    \begin{equation*}
        \lambda \unity - u^{-1} \notin \invertible{\algebra{A}} \iff -\lambda^{-1} u \left(\lambda \unity - u^{-1}\right) \notin \invertible{\algebra{A}} \iff \lambda^{-1}\unity - u \notin \invertible{\algebra{A}},
    \end{equation*}
    hence \(\lambda \in \sigma(u^{-1})\) if and only if \(\lambda^{-1} \in \sigma(u)\).
\end{proof}
\begin{proposition}{Spectrum of the adjoint of an operator}{spectrum_adjoint}
    Let \(\algebra{A}\) be a unital *-algebra. If \(u \in \invertible{\algebra{A}}\), then \(\sigma(u^*) = \setc{\lambda \in \mathbb{C}}{\conj{\lambda} \in \sigma(u)}\).
\end{proposition}
\begin{proof}
    Since \(\invertible{\algebra{A}}\) is self-adjoint, we have
    \begin{equation*}
        \lambda \unity - u^* \notin \invertible{\algebra{A}} \iff (\lambda \unity - u^*)^* \notin \invertible{\algebra{A}} \iff \conj{\lambda}\unity - u \notin \invertible{\algebra{A}},
    \end{equation*}
    hence \(\lambda \in \sigma(u^*)\) if and only if \(\conj{\lambda} \in \sigma(u)\).
\end{proof}
For an invertible operator \(u \in \invertible{\algebra{A}}\) in a unital *-algebra \(\algebra{A}\), we'll denote
\begin{equation*}
    \sigma(u)^{-1} = \setc{\lambda \in \mathbb{C}}{\lambda^{-1} \in \sigma(u)} \quad\text{and}\quad \conj{\sigma(u)} = \setc{\lambda \in \mathbb{C}}{\conj{\lambda} \in \sigma(u)}
\end{equation*}
for the spectra of \(u^{-1}\) and \(u^*\), respectively.

\subsection{Topological properties of the spectrum}
We may associate an operator with each complex number in the resolvent set of an operator.
\begin{definition}{Resolvent operator}{resolvent_operator}
    Let \(\algebra{A}\) be a unital associative algebra. Let \(a \in \algebra{A}\) and \(\lambda \in \rho(a)\), the operator
    \begin{equation*}
        R_\lambda(a) = (\lambda \unity - a)^{-1}
    \end{equation*}
    is the \emph{resolvent of \(a\) at \(\lambda\)}.
\end{definition}

By studying the resolvent operator, we may derive many conclusions about the spectra of operators. We begin with simple properties of the resolvent.
\begin{proposition}{Resolvent and operator commute}{resolvent_commute}
    Let \(\algebra{A}\) be a unital associative algebra. If \(R_{\lambda}(a) \in \algebra{A}\) is the resolvent of \(a \in \algebra{A}\) at \(\lambda \in \rho(a)\), then \(R_{\lambda}(a)a = aR_{\lambda}(a)\).
\end{proposition}
\begin{proof}
    Since \(R_{\lambda}(a) \in \invertible{\algebra{A}}\), we have
    \begin{equation*}
        R_{\lambda}(a) R_{\lambda}(a)^{-1} = \unity \implies a R_{\lambda}(a) R_{\lambda}(a)^{-1} = a \implies a R_{\lambda}(a) = a R_{\lambda}(a),
    \end{equation*}
    as desired.
\end{proof}

\begin{proposition}{First resolvent identity}{first_resolvent_identity}
    Let \(\algebra{A}\) be a unital associative algebra and let \(a \in \algebra{A}\). Then
    \begin{equation*}
        R_{\lambda}(a) - R_{\mu}(a) = (\mu - \lambda)R_{\lambda}(a)R_{\mu}(a)
    \end{equation*}
    for all \(\lambda, \mu \in \rho(a)\).
\end{proposition}
\begin{proof}
    We have
    \begin{equation*}
        R_{\lambda}(a) = R_{\lambda}(a)(\mu \unity - a)R_{\mu}(a) = R_{\lambda}(a) \left[(\mu - \lambda)\unity + (\lambda \unity - u)\right]R_{\mu}(a) = (\mu - \lambda)R_{\lambda}(a) R_{\mu}(a) - R_{\mu}(a)
    \end{equation*}
    for all \(\lambda, \mu \in \rho(a)\).
\end{proof}
\begin{corollary}
    Let \(\algebra{A}\) be a unital associative algebra and let \(a \in \algebra{A}\). Then, \(R_{\lambda}(a)R_{\mu}(a) = R_{\mu}(a)R_{\lambda}(a)\) for all \(\lambda, \mu \in \rho(a)\).
\end{corollary}
\begin{proof}
    The first resolvent identity yields
    \begin{equation*}
        (\mu - \lambda)R_{\lambda}(a)R_{\mu}(a) = R_{\lambda}(a) - R_{\mu}(a) = -(\lambda - \mu)R_{\mu}(a)R_{\lambda}(a),
    \end{equation*}
    then
    \begin{equation*}
        (\mu - \lambda)\left[R_{\lambda}(a) R_{\mu}(a) - R_{\mu}(a) R_{\lambda}(a)\right] = 0
    \end{equation*}
    for all \(\mu,\lambda \in \rho(a)\) and the result follows.
\end{proof}
\begin{proposition}{Second resolvent identity}{second_resolvent_identity}
    Let \(\algebra{A}\) be a unital associative algebra and let \(u, v \in \algebra{A}\). Then,
    \begin{equation*}
        R_{\lambda}(u) - R_{\lambda}(v) = R_{\lambda}(u) (u - v) R_{\mu}(v)
    \end{equation*}
    for all \(\lambda \in \rho(u) \cap \rho(v)\).
\end{proposition}
\begin{proof}
    We have
    \begin{equation*}
        R_{\lambda}(u) (u-v) R_{\lambda}(v) = R_{\lambda}(u) \left[(\lambda \unity - v) - (\lambda \unity - u)\right] R_{\lambda}(v) = R_{\lambda}(u) - R_{\lambda}(v)
    \end{equation*}
    for all \(\lambda \in \rho(u) \cap \rho(v)\).
\end{proof}
\begin{corollary}
    Let \(\algebra{A}\) be a unital associative algebra and let \(u, v \in \algebra{A}\). Then,
    \begin{equation*}
        R_{\lambda}(u) - R_{\lambda}(v) = R_{\lambda}(v) (u - v) R_{\mu}(u)
    \end{equation*}
    for all \(\lambda \in \rho(u) \cap \rho(v)\).
\end{corollary}
\begin{proof}
    Exchanging \(u\) and \(v\) in the second identity yields the result.
\end{proof}
\begin{proposition}{Third resolvent identity}{third_resolvent_identity}
    Let \(\algebra{A}\) be a unital associative algebra and let \(u, v \in \algebra{A}\). Then,
    \begin{equation*}
        R_{\lambda}(uv) = \lambda^{-1}\left[\unity + uR_{\lambda}(vu)v\right]
    \end{equation*}
    for all \(\lambda \in \rho(vu)\setminus\set{0}\).
\end{proposition}
\begin{proof}
    Let \(\lambda \in \rho(vu)\setminus\set{0},\) then \(R_{\lambda}(vu) = \lambda^{-1}(\unity - \lambda^{-1}vu)^{-1}\). We repeat the proof of \cref{prop:prop411}:
    \begin{align*}
        \left[\lambda^{-1}\unity + \lambda^{-1}uR_{\lambda}(vu)v\right](\lambda\unity - uv)
        &= \unity + uR_{\lambda}(vu)v - \lambda^{-1}uv - \lambda^{-1}uR_{\lambda}(vu)vuv\\
        &= \unity - \lambda^{-1}uv + uR_{\lambda}(vu)(\unity - \lambda^{-1}vu)v\\
        &= \unity - \lambda^{-1}uv + \lambda^{-1}uv\\
        &= \unity,
    \end{align*}
    and
    \begin{align*}
        (\lambda\unity - uv)\left[\lambda^{-1}\unity + \lambda^{-1}uR_{\lambda}(vu)v\right]
        &= \unity - \lambda^{-1}uv + uR_{\lambda}(vu)v - \lambda^{-1}uvuR_{\lambda}(vu)v\\
        &= \unity - \lambda^{-1}uv + u(\unity - \lambda^{-1}vu)R_{\lambda}(vu)v\\
        &= \unity - \lambda^{-1}uv + \lambda^{-1}uv\\
        &= \unity,
    \end{align*}
    proving our claim.
\end{proof}

\begin{lemma}{Resolvent set is open}{resolvent_set_open}
    Let \(\algebra{A}\) be a unital Banach algebra and let \(a \in \algebra{A}\). If \(\mu \in \rho(a)\), then the open ball \(B_{\norm{R_\mu(a)}^{-1}}(\mu)\) is contained in \(\rho(a)\) and
    \begin{equation*}
        R_{\lambda}(a) = R_\mu(a)\left\{\unity + \sum_{n = 1}^{\infty} \left[(\mu - \lambda)R_\mu(a)\right]^n\right\} = \left\{\unity + \sum_{n = 1}^{\infty}\left[(\mu - \lambda)R_\mu(a)\right]^n\right\}R_{\mu}(a)
    \end{equation*}
    for all \(\lambda \in B_{\norm{R_\mu(a)}^{-1}}(\mu)\).
\end{lemma}
\begin{proof}
    Notice
    \begin{equation*}
        (\kappa \unity - a)R_\mu(a) = \left[(\kappa - \mu)\unity + (\mu \unity - a)\right]R_{\mu}(a) = \unity + (\kappa - \mu)R_{\mu}(a),
    \end{equation*}
    for all \(\kappa \in \mathbb{C}\).

    Let \(\lambda \in B_{\norm{R_\mu(a)}^{-1}}(\mu)\), then \(\norm{(\mu - \lambda)R_{\mu}(a)} < 1\), hence \(\unity - (\mu - \lambda)R_{\mu}(a) \in \invertible{\algebra{A}}\) and
    \begin{equation*}
        \unity + \sum_{n = 1}^{\infty} \left[(\mu - \lambda)R_\mu(a)\right]^n = \left[\unity - (\mu - \lambda)R_{\mu}(a)\right]^{-1} = \left[(\lambda \unity - a)R_\mu(a)\right]^{-1}
    \end{equation*}
    by \cref{thm:neumann_series_algebra}. Since \(\invertible{\algebra{A}}\) is closed under the product and \(R_\mu(a) \in \invertible{\algebra{A}}\), we have \(\lambda \in \rho(a)\) and
    \begin{equation*}
        \unity + \sum_{n = 1}^{\infty} \left[(\mu - \lambda)R_\mu(a)\right]^n = (\mu\unity - a)R_{\lambda}(a),
    \end{equation*}
    concluding our proof.
\end{proof}
\begin{corollary}
    Let \(\algebra{A}\) be a unital Banach algebra. The spectrum of an operator is a closed subset of \(\mathbb{C}\).
\end{corollary}
\begin{proof}
    Let \(a \in \algebra{A}.\) If \(\rho(a)\) is empty, then \(\sigma(a) = \mathbb{C}\) is closed, so we may assume \(\rho(a)\) is non-empty. Since every point of \(\rho(a)\) is an interior point by \cref{lem:resolvent_set_open}, we have \(\mathbb{C} \setminus \rho(a)\) open.
\end{proof}

\begin{lemma}{Linear functionals define holomorphic maps}{holomorphic}
    Let \(\algebra{A}\) be a unital Banach algebra and \(a \in \algebra{A}\). Let \(\ell \in \algebra{A}^\dag\) be a continuous linear functional, then the map
    \begin{align*}
        f_{\ell} : \rho(a) &\to \mathbb{C}\\
                   \lambda &\mapsto \ell(R_\lambda(a))
    \end{align*}
    is holomorphic in each connected component of \(\rho(a)\).
\end{lemma}
\begin{proof}
    Let \(\mu \in \rho(a)\) and consider \(\lambda \in B_{\norm{R_\mu(a)}^{-1}}(\mu)\setminus \set{\mu},\) then
    \begin{equation}
        f_\ell(\lambda) = \ell(R_{\lambda}(a)) = \ell\left(R_{\mu}(a) + \sum_{n = 1}^{\infty} (\mu - \lambda)^n R_{\mu}(a)^{n+1}\right)
    \end{equation}
    by \cref{lem:resolvent_set_open}. Since the series converges to \((\mu \unity - a)R_{\lambda}(a) - \unity\) uniformly and \(\ell\) is uniformly continuous, we have
    \begin{equation*}
        f_\ell(\lambda) = \ell(R_{\mu}(a)) + \sum_{n = 1}^\infty (-1)^n\ell\left(R_\mu(a)^{n+1}\right)(\lambda - \mu)^n,
    \end{equation*}
    which converges absolutely in \(B_{\norm{R_\mu(a)}^{-1}}\). That is, \(f_\ell\) is holomorphic in \(B_{\norm{R_\mu(a)}^{-1}}\), hence it can be extended to the connected component of \(\rho(a)\) containing \(\mu\).
\end{proof}

\begin{theorem}{Spectrum of an operator is compact}{spectrum_compact}
    Let \(\algebra{A}\) be a unital Banach algebra and \(a \in \algebra{A}\). Then \(\sigma(a)\) is non-empty and contained in \(\setc{\lambda \in \mathbb{C}}{\abs{\lambda} < \norm{a}}\).
\end{theorem}
\begin{proof}
    Suppose, by contradiction, \(\rho(a) = \mathbb{C}\), then by \cref{lem:holomorphic} the map \(f_\ell\) is entire for every \(\ell \in \algebra{A}^\dag\). Let \(\lambda \in \setc{\lambda \in \mathbb{C}}{\abs{\lambda} > \norm{a}}\), then
    \begin{equation*}
        R_{\lambda}(a) = \lambda^{-1}(\unity - \lambda^{-1}a)^{-1} = \lambda^{-1}\left(\unity + \sum_{n = 1}^{\infty} \lambda^{-n}u^n\right)
    \end{equation*}
    by \cref{thm:neumann_series_algebra}. We have
    \begin{equation*}
        \norm{R_{\lambda}(a)} \leq \abs{\lambda}^{-1}\left[1 + \sum_{n = 1}^\infty \left(\frac{\norm{u}}{\abs{\lambda}}\right)^n\right] = \abs{\lambda}^{-1} \left[1 + \frac{\frac{\norm{u}}{\abs{\lambda}}}{1 - \frac{\norm{u}}{\abs{\lambda}}}\right] = \frac{1}{\abs{\lambda} - \norm{u}}
    \end{equation*}
    which shows \(\norm{R_{\lambda}(a)} \to 0\) as we take \(\abs{\lambda} \to \infty\). Since \(\ell\) is continuous, we have \(\abs{f_\ell(\lambda)} \leq \norm{\ell}\cdot \norm{R_{\lambda}(a)}\), from which follows \(\abs{f_\ell(\lambda)} \to 0\) as \(\abs{\lambda} \to \infty\). Liouville's theorem ensures \(f_\ell\) is identically zero in the entire complex plane. This implies \(R_\lambda(a) \in \bigcap_{\ell \in \algebra{A}^\dag}\ker{\ell},\) hence \(R_{\lambda}(a) = 0\), which is not invertible. This contradiction shows \(\rho(a) \neq \mathbb{C}\), and we conclude \(\sigma(a)\) is non-empty. Moreover, we have shown \(\setc{\lambda \in \mathbb{C}}{\abs{\lambda} > \norm{a}} \subset \rho(a),\) hence \(\sigma(a) \subset \setc{\lambda \in \mathbb{C}}{\abs{\lambda} < \norm{a}}\).
\end{proof}
\begin{corollary}
    Let \(\algebra{A}\) be a unital Banach algebra and let \(a \in \algebra{A}\). Then \(\sigma(a)\) is compact.
\end{corollary}
\begin{proof}
    Since the complex plane has the Heine-Borel property, this result follows from the fact \(\norm{a}\) is a bound for the closed set \(\sigma(a)\).
\end{proof}

\subsection{Spectrum of projectors}
Abstracting the definition of projectors in Banach spaces and orthogonal projectors in Hilbert spaces we define projectors in general associative algebras.
\begin{definition}{Projectors and orthogonal projectors}{projector_algebra}
    Let \(\algebra{A}\) be an associative algebra. A \emph{projector} is an idempotent operator \(p \in \algebra{A}\), with \(p^2 = p\). If \(\algebra{A}\) is involutive, an \emph{orthogonal projector} is a self-adjoint projector.
\end{definition}

As expected from orthogonal projectors in Hilbert spaces, the norm of a orthogonal projector in C*-algebras is either zero or one.
\begin{proposition}{Norm of projectors}{norm_projector}
    Let \(\algebra{A}\) be a normed algebra. If \(p\in \algebra{A}\) is a projector, then either \(p = 0\) or \(\norm{p} \geq 1\). In addition, if \(\algebra{A}\) is a C*-algebra and \(p\) is an orthogonal projector, then \(\norm{p} \in \set{0,1}.\)
\end{proposition}
\begin{proof}
    In a normed algebra, we have \( \norm{p} = \norm{p^2} \leq \norm{p}^2\), which yields \(\norm{p} \in \set{0}\cup(1,\infty)\). If follows from the C* property and self-adjointness of the orthogonal projector that
    \begin{equation*}
        \norm{p} = \norm{p^2} = \norm{p^*p} = \norm{p}^2,
    \end{equation*}
    hence \(\norm{p} \in \set{0,1}\).
\end{proof}

\begin{proposition}{Necessary and sufficient condition for a projector}{projector_sufficient}
    Let \(\algebra{A}\) be a unital associative algebra. An operator \(p \in \algebra{A}\) is a projector if and only if \(\unity - p\) is a projector.
\end{proposition}
\begin{proof}
    If \(p\) is a projector, then
    \begin{equation*}
        (\unity - p)^2 = \unity - 2p + p^2  = \unity - p,
    \end{equation*}
    that is, \(\unity - p\) is a projector.

    Suppose \(\unity - p\) is a projector, then
    \begin{equation*}
        \unity - p=(\unity - p)^2 = \unity - 2p + p^2 \implies p^2 = p
    \end{equation*}
    that is, \(p\) is a projector.
\end{proof}
\begin{proposition}{Identity is the only invertible projector}{projector_invertible}
    Let \(\algebra{A}\) be a unital associative algebra. A projector \(p \in \algebra{A}\) is invertible if and only if \(p = \unity\).
\end{proposition}
\begin{proof}
    It is clear that \(\unity\) is a projector, since \(\unity^2 = \unity\). Suppose \(p \in \invertible{\algebra{A}}\) is a projector, then
    \begin{equation*}
        p^{-1}p^2 = p^{-1}p \implies p = \unity,
    \end{equation*}
    as desired.
\end{proof}

Let \(\algebra{A}\) be a unital Banach algebra, \(p \in \algebra{A}\) a projector, and \(\lambda \in \setc{\alpha \in \mathbb{C}}{\abs{\alpha} > 1}.\) Then \(\norm{\lambda^{-1}p} < 1\) and
\begin{equation*}
    \left(\unity - \lambda^{-1}p\right)^{-1} = \unity + \sum_{k = 1}^{\infty} \lambda^{-n}p^n  = \unity + \left(\sum_{k=1}^\infty \lambda^{-n}\right)p =\unity + \frac{1}{\lambda - 1}p
\end{equation*}
by \cref{thm:neumann_series_algebra}. \todo[Meromorphic map extension]
\begin{lemma}{Resolvent of a projector}{resolvent_projector}
    Let \(\algebra{A}\) be a unital associative algebra. If \(p\in\algebra{A}\) is a projector, then
    \begin{equation*}
        R_{\lambda}(p) = \frac{1}{\lambda}\unity + \frac{1}{\lambda(\lambda - 1)}p
    \end{equation*}
    is the resolvent of \(p\) at \(\lambda \in \mathbb{C} \setminus \set{0,1}.\)
\end{lemma}
\begin{proof}
    Let \(\lambda \in \mathbb{C}\setminus\set{0,1}\), then
    \begin{align*}
        \left[\frac1\lambda\unity + \frac1{\lambda(\lambda - 1)}p\right](\lambda \unity - p)
        &= \unity + \left(\frac{1}{\lambda-1} - \frac{1}{\lambda}\right)p - \frac{1}{\lambda(\lambda - 1)}p^2\\&= \unity + \left[\left(\frac{1}{\lambda-1} - \frac{1}{\lambda}\right) - \frac{1}{\lambda(\lambda - 1)}\right]p = \unity
    \end{align*}
    and
    \begin{align*}
        (\lambda \unity - p)\left[\frac1\lambda\unity + \frac1{\lambda(\lambda - 1)}p\right]
        &= \unity + \left(\frac{1}{\lambda-1} - \frac{1}{\lambda}\right)p - \frac{1}{\lambda(\lambda - 1)}p^2 \\&= \unity + \left[\left(\frac{1}{\lambda-1} - \frac{1}{\lambda}\right) - \frac{1}{\lambda(\lambda - 1)}\right]p = \unity,
    \end{align*}
    hence \(\lambda \in \rho(p)\) and the result follows.
\end{proof}

\begin{proposition}{Spectra of projectors}{spectra_projector}
    Let \(\algebra{A}\) be a unital associative algebra. The spectrum of a projector \(p \in \algebra{A}\) is contained in the set \(\set{0,1}\), with \(\sigma(p) = \set{0}\) if and only if \(p = 0\) and \(\sigma(p) = \set{1}\) if and only if \(p = \unity\).
\end{proposition}
\begin{proof}
    \cref{lem:resolvent_projector} guarantees \(\mathbb{C} \setminus \set{0,1}\subset \rho(p)\), hence \(\sigma(p) \subset \set{0,1}\) for any projector.

    In particular, \(0\) is not invertible, but \(\lambda \unity\) is invertible for all \(\lambda \in \mathbb{C} \setminus \set{0},\) thus showing that \(\sigma(0) = \set{0},\) and \(\unity\) is invertible with \((\lambda - 1)\unity\) invertible for all \(\lambda \in \mathbb{C}\setminus\set{1},\) that is, \(\sigma(\unity) = \set{1}\).

    Suppose \(\sigma(p) = \set{0},\) then \(\unity - p \in \invertible{\algebra{A}}\), hence \(p = 0\) by \cref{prop:projector_invertible}. Suppose \(\sigma(p) = \set{1}\), then \(-p \in \invertible{\algebra{A}}\), hence \(p = \unity\).
\end{proof}
\subsection{Spectral mapping theorem}

% vim: spl=en_us
\section{Gelfand representation}
Let \(\algebra{A}\) be a unital C*-algebra and let \(a \in \algebra{A}\) be a self-adjoint operator. We'll denote by \(\mathcal{C}(\sigma(a), \mathbb{C})\) the set of continuous, with respect to the subspace topology on \(\sigma(a) \subset \mathbb{R}\) and the usual topology on \(\mathbb{C}\), complex-valued maps from defined on the spectrum of \(a\) and by \(\norm{\noarg}_{\infty}\) the supremum norm, that is, \(\left(\mathcal{C}(\sigma(a), \mathbb{C}), \norm{\noarg}_{\infty}\right)\) is a Banach *-algebra. Since the set of polynomials defined on \(\sigma(a)\) is a subalgebra of \(\mathcal{C}(\sigma(a), \mathbb{C})\), we may improve on the spectral mapping theorem and show the supremum norm of a such defined polynomial \(\varphi\) is equal to the norm \(\norm{\varphi(a)}\) in the C*-algebra.
\begin{proposition}{Norm of a polynomial defined on the spectrum of an operator}{norm_polynomial}
    Let \(\algebra{A}\) be a unital C*-algebra and let \(a \in \algebra{A}\) be a self-adjoint operator. Then \(\norm{\varphi}_\infty = \norm{\varphi(a)}\) for all polynomials \(\varphi : \sigma(a) \to \mathbb{C}\).
\end{proposition}
\begin{proof}
    Let us write \(\varphi(z) = \sum_{k = 0}^{n} c_k z^k\) for all \(z \in \sigma(a)\). It follows from self-adjointness of \(a\) that
    \begin{equation*}
        \varphi(a)^*\varphi(a) = \sum_{k=0}^n \conj{c_k} a^k \sum_{\ell = 0}^n c_{\ell} a^\ell = \sum_{k = 0}^n \sum_{\ell = 0}^n \conj{c_k} c_k a^{\ell + k} = (\conj{\varphi}\varphi)(a).
    \end{equation*}
    \cref{thm:spectral_radius_cstar,thm:spectral_mapping} yield
    \begin{equation*}
        \norm{\varphi(a)^*\varphi(a)} = \norm{(\conj{\varphi}\varphi)(a)} = \sup_{\lambda \in \sigma\left[(\conj{\varphi}\varphi)(a)\right]}{\abs{\lambda}} = \sup_{\lambda \in \sigma(a)}{\abs*{(\conj{\varphi}\varphi)(\lambda)}} = \sup_{\lambda\in \sigma(a)}{\abs{\varphi(\lambda)}^2} = \norm{\varphi}_{\infty}^2,
    \end{equation*}
    then the proposition follows from the C* property, \(\norm{\varphi(a)}^2 = \norm{\varphi(a)^*\varphi(a)} = \norm{\varphi}_{\infty}^2\).
\end{proof}

\begin{proposition}{Norm of a power of a normal operator}{norm_normal}
    Let \(\algebra{A}\) be a unital C*-algebra. If \(a \in \algebra{A}\) is a normal operator, then \(\norm{a}^n = \norm{a^n}\) for all \(n \in \mathbb{N}\).
\end{proposition}
\begin{proof}
    Let \(n \in \mathbb{N}\), then for a normal operator we have \(\norm{a^n}^2 = \norm{(a^*)^n a^n} = \norm{(a^*a)^n}\). Since \(a^*a\) is self-adjoint, we have by \cref{prop:norm_polynomial} that \(\norm{(a^*a)^n} = \norm{a^*a}^n = \norm{a}^{2n}\). This yields \(\norm{a^n} = \norm{a^n}\) as desired.
\end{proof}

We recall the set of polynomials is dense in the set of continuous functions defined on a compact set with respect to the topology induced by the supremum norm. Indeed, we show the standard \nameref{thm:weierstrass_polynomial} for maps defined in the compact interval \([0,1]\), which may be generalized to continuous maps defined in any compact subset of \(\mathbb{R}\).
\begin{theorem}{Weierstrass polynomial approximation theorem}{weierstrass_polynomial}
    Let \(f \in \mathcal{C}\left([0,1], \mathbb{C}\right)\) be a continuous function. Then the sequence \(\family{\varphi_n}{n \in \mathbb{N}}\subset\mathcal{C}\left([0,1], \mathbb{C}\right)\) of polynomials defined by
    \begin{equation*}
        \varphi_n(x) = \sum_{k = 0}^n \binom{n}{k} f\left(\frac{k}{n}\right) x^k (1 - x)^{n - k}
    \end{equation*}
    converges uniformly to \(f\), that is, for all \(\varepsilon > 0\) there exists \(N \in \mathbb{N}\) such that \(\norm{f - \varphi_n}_\infty < \epsilon\) for all \(n \geq N\).
\end{theorem}
\begin{proof}
    Let \(x, y \in [0,1]\) and \(n \in \mathbb{N}_0\), then
    \begin{equation*}
        nx(x + y)^{n-1} = x\diffp*{(x + y)^n}{x} = x\diffp*{\sum_{k = 0}^n\binom{n}{k} x^k y^{n-k}}{x} = \sum_{k=0}^n \binom{n}{k} k x^{k}y^{n - k}
    \end{equation*}
    and
    \begin{equation*}
        n(n-1)x^2(x + y)^{n-2} = x^2\diffp*[2]{(x + y)^n}{x} = x^2\diffp*{ \sum_{k=0}^n \binom{n}{k} k x^{k-1}y^{n - k}}{x} =  \sum_{k = 0}^n \binom{n}{k} k (k-1) x^{k} y^{n - k}.
    \end{equation*}
    Writing \(b^n_k(x) = \binom{n}{k} x^k (1 - x)^{n - k}\), we get
    \begin{equation*}
        \sum_{k = 0}^n b^n_k(x) = 1,
        \quad
        \sum_{k=0}^n k b^n_k(x) = n x,
        \quad\text{and}\quad
        \sum_{k=0}^n k(k-1) b^n_k(x) = n(n-1)x^2
    \end{equation*}
    for all \(x \in [0,1]\). This yields
    \begin{align*}
        \sum_{k = 0}^n (k - nx)^2 b^n_k(x) &= \sum_{k = 0}^n k^2 b^n_k(x) - 2nx\sum_{k=0}^n kb^n_k(x) + n^2 x^2 \sum_{k = 0}^n b^n_k(x)\\
                                           &= \sum_{k = 0}^n k(k-1) b^n_k(x) + \sum_{k=0}^n k b^n_k(x) - n^2 x^2 \\
                                           &= n(n-1)x^2 + nx - n^2x^2\\
                                           &= nx(1 - x).
    \end{align*}

    \todo[Since \(f\) is defined on a compact set and continuous, it is uniformly continuous.] Let \(\varepsilon > 0\), then there exists \(\delta > 0\) such that \(\abs{f(x) - f(y)} < \frac12\epsilon\) whenever \(\abs{x - y} < \delta\). Let \(x \in [0,1]\), and we consider \(S_n = \setc{m \in \mathbb{N}_0}{m \leq n \land \abs{m - xn} < \delta n}\) and \(R_n = \setc{m \in \mathbb{N}_0}{m \leq n \land \abs{m - xn} \geq \delta n}\), then
    \begin{align*}
        \abs*{f(x) - \varphi_n(x)} &= \abs*{\sum_{k = 0}^n \left[f(x) - f\left(\frac{k}{n}\right)\right]b^n_k(x)}\\
                                  &\leq \sum_{k = 0}^n \abs*{f(x) - f\left(\frac{k}{n}\right)}b^n_k(x)\\
                                  &= \sum_{k \in S_n} \abs*{f(x) - f\left(\frac{k}{n}\right)}b^n_k(x) + \sum_{k \in R_n} \abs*{f(x) - f\left(\frac{k}{n}\right)}b^n_k(x)\\
                                  &< \frac12\varepsilon \sum_{k \in S_n} b^n_k(x) + 2 \norm{f}_\infty \sum_{k \in R_n} b^n_k(x)\\
                                  &\leq \frac12 \varepsilon \sum_{k = 0}^n b^n_k(x) + \frac{2\norm{f}_\infty}{n^2 \delta^2}\sum_{k=0}^n (k - nx)^2 b^n_k(x)\\
                                  &= \frac12 \varepsilon + \frac{2 \norm{f}_{\infty}x(1 - x)}{n \delta^2}\\
                                  &\leq \frac12 \varepsilon + \frac{\norm{f}_\infty}{2 \delta^2 n}.
    \end{align*}
    If \(\norm{f}_\infty\), we are done, so we may assume \(\norm{f}_\infty > 0\). We set \(N = \ceil*{\frac{\norm{f}_\infty}{\delta^2 \varepsilon}}\), then
    \begin{equation*}
        n \geq N \implies \forall x \in [0,1]: \abs{f(x) - \varphi_n(x)} < \varepsilon,
    \end{equation*}
    that is, \(\norm{f - \varphi_n}_\infty < \varepsilon\) for all \(n \geq N\).
\end{proof}

Let us denote by \(\mathcal{P}(\sigma(a)) \subset \mathcal{C}(\sigma(a), \mathbb{C})\) the subalgebra of polynomials defined on the spectrum of the self-adjoint operator \(a \in \algebra{A}\). Then by the generalized Weierstrass approximation theorem, we have \(\cl{\mathcal{P}(\sigma(a))} = \mathcal{C}(\sigma(a))\). \cref{prop:norm_polynomial} shows us the linear map
\begin{align*}
    \tilde{\Phi}_a : \mathcal{P}(\sigma(a)) &\to \algebra{A}\\
                                    \varphi &\mapsto \varphi(a)
\end{align*}
is isometric, thus bounded. \cref{thm:blt} guarantees us the uniqueness and existence of a linear map
\begin{align*}
    \Phi_a : \mathcal{C}\left(\sigma(a), \mathbb{C}\right) &\to \algebra{A}\\
                                                         f &\mapsto f(a)
\end{align*}
that isometrically extends \(\tilde{\Phi}_a\) to \(\mathcal{C}\left(\sigma(a), \mathbb{C}\right)\) by defining
\begin{equation*}
    f(a) = \lim_{n\to\infty}{\varphi_n(a)},
\end{equation*}
where \(\family{\varphi_n}{n\in \mathbb{N}} \subset \mathcal{P}(\sigma(a))\) is \emph{any} sequence of polynomials that uniformly converges against \(f\).

\begin{theorem}{Gelfand's homomorphism in C*-algebras}{gelfand_homomorphism}
    Let \(\algebra{A}\) be unital C*-algebra, let \(a \in \algebra{A}\) be a self-adjoint operator and consider the above defined map \(\Phi_a : \mathcal{C}\left(\sigma(a), \mathbb{C}\right) \to \algebra{A}\). Then \(\Phi_a\) is a *-homomorphism and for all \(f \in \mathcal{C}\left(\sigma(a), \mathbb{C}\right)\) we have
    \begin{enumerate}[label=(\alph*)]
        \item if \(f(\lambda) \geq 0\) for all \(\lambda \in \sigma(a)\), then \(\sigma(\Phi_a(f)) \subset [0, \infty)\);
        \item if a sequence \(\family{f_n}{n\in \mathbb{N}}\subset \mathcal{C}\left(\sigma(a), \mathbb{C}\right)\) converges uniformly against \(f\) then \(\family{\Phi_a(f_n)}{n\in \mathbb{N}} \subset \algebra{A}\) converges against \(\Phi_a(f)\);
        \item \(f(\sigma(a)) = \sigma(\Phi_a(f))\).
    \end{enumerate}
\end{theorem}
\begin{proof}
    Let \(f, g \in \mathcal{C}\left(\sigma(a), \mathbb{C}\right)\) and let \(\family{\varphi_n}{n \in \mathbb{N}}, \family{\psi_n}{n \in \mathbb{N}} \subset \mathcal{P}(\sigma(a))\) be sequences of polynomials that uniformly converge to \(f\) and \(g\), respectively. Then for all \(\alpha, \beta \in \mathbb{C}\) we have
    \begin{equation*}
        \Phi_a( \alpha f + \beta g) = \lim_{n \to \infty}{(\alpha \varphi_n + \beta \psi_n)(a)}= \alpha \lim_{n\to\infty}{\varphi_n(a)} + \beta\lim_{n\to\infty}{\psi_n(a)} = \alpha \Phi_a(f) + \beta\Phi_a(g),
    \end{equation*}
    that is, \(\Phi_a\) is linear. The map preserves the product,
    \begin{equation*}
        \Phi_a(fg) = \lim_{n\to\infty}{(\varphi_n\psi_n)(a)} = \lim_{n\to\infty}{\varphi_n(a)\psi_n(a)}=\left[\lim_{n\to\infty}{\varphi_n(a)}\right]\left[\lim_{n\to\infty}{\psi_n(a)}\right] = \Phi_a(f)\Phi_a(g),
    \end{equation*}
    commutes with the involution,
    \begin{equation*}
        \Phi_a(\conj{f}) = \lim_{n\to\infty}{\conj{\varphi_n}(a)} = \lim_{n\to\infty}{\varphi_n(a)^*} = \left[\lim_{n\to\infty}{\varphi_n(a)}\right]^* = \Phi_a(f)^*,
    \end{equation*}
    and maps the identity element of \(\mathcal{C}(\sigma(a), \mathbb{C})\), the constant polynomial \(1\), to the identity element of the C*-algebra, \(\unity\), hence it is shown that it is a *-homomorphism.

    Suppose \(f\) satisfies \(f(\lambda) \geq 0\) for all \(\lambda \in \sigma(a)\). Then there exists \(h \in \mathcal{C}(\sigma(a), \mathbb{R})\) such that \(f = h^2\). Since \(\Phi_a\) is a *-homomorphism, we know \(\Phi_a(h)\) is self-adjoint, hence its spectrum lies in \([-\norm{h}_\infty, \norm{h}_\infty]\). With the spectral mapping theorem, we know \(\sigma(\Phi_a(h)^2) \subset [0, \norm{h}_\infty^2]\) and we conclude (a).

    Let \(\family{f_n}{n \in \mathbb{N}} \subset \mathcal{C}\left(\sigma(a), \mathbb{C}\right)\) be a sequence that converges uniformly against \(f\), then
    \begin{equation*}
        \norm{\Phi_a(f) - \Phi_a(f_n)}= \norm{\Phi_a(f - f_n)} = \norm{f - f_n}_{\infty},
    \end{equation*}
    for all \(n \in \mathbb{N}\), since \(\Phi_a\) is a isometry. From this we conclude \(\family{\Phi_a(f_n)}{n\in \mathbb{N}}\) converges against \(\Phi_a(f)\), hence (b) follows.

    Let \(\mu \in f(\sigma(a))\), then there exists \(\lambda \in \sigma(a)\) such that \(\mu = f(\lambda)\). Suppose, by  contradiction, \(\mu \in \rho(\Phi_a(f))\), then \(f(\lambda)\unity - \Phi_a(f)\) is invertible. Let \(\varepsilon > 0\), then there exists \(\varphi \in \mathcal{P}(\sigma(a))\) such that \(\norm{f - \varphi}_{\infty} < \frac12 \varepsilon\). Then
    \begin{equation*}
        \norm*{[f(\lambda)\unity - \Phi_a(f)] - [\varphi(\lambda)\unity - \Phi_a(\varphi)]} \leq \abs{f(\lambda) - \varphi(\lambda)} + \norm*{\Phi_a(f) - \Phi_a(\varphi)} \leq 2 \norm{f - \varphi}_\infty < \varepsilon,
    \end{equation*}
    that is, \(\varphi(\lambda)\unity - \Phi_a(\varphi)\) lies in the open ball centered in \(\mu \unity - \Phi_a(f)\) with radius \(\varepsilon\). If we set \(\varepsilon\) sufficiently small, we conclude by \cref{prop:invertible_continuous_group} that \(\varphi(\lambda) \unity - \Phi_a(\varphi)\) is invertible, hence \(\varphi(\lambda) \in \rho(\Phi_a(\varphi))\). However, we know from the spectral mapping theorem that \(\varphi(\lambda) \in \sigma(\Phi_a(\varphi)).\) This contradiction shows \(\mu \in \sigma(\Phi_a(f))\), that is, \(f(\sigma(a)) \subset \sigma(\Phi_a(f))\).

    Let \(\kappa \in \mathbb{C} \setminus f(\sigma(a))\), then the map
    \begin{align*}
        r_{\kappa} : \sigma(a) &\to \mathbb{C}\\
                           \xi &\mapsto \frac{1}{\kappa - f(\xi)}
    \end{align*}
    is continuous, that is, \(r_{\kappa} \in \mathcal{C}\left(\sigma(a), \mathbb{C}\right)\). Since \(\Phi_a\) is a *-homomorphism, we have
    \begin{equation*}
        1 = r_{\kappa} (\kappa - f) \implies \Phi_a(r_{\kappa}) \Phi_a(\kappa - f) = \unity \implies \Phi_a(r_{\kappa}) \left[\kappa \unity - \Phi_a(f)\right] = \unity
    \end{equation*}
    that is, \(\Phi_a(r_{\kappa}) = \left[\kappa\unity - \Phi_a(f)\right]^{-1}\), as \(\mathcal{C}\left(\sigma(a), \mathbb{C}\right)\) is an abelian algebra. This shows \(\kappa \in \rho(\Phi_a(f))\), that is, \(\kappa \notin \sigma(\Phi_a(f))\). Therefore, \(\sigma(\Phi_a(f)) \subset f(\sigma(a))\), and we conclude (c).
\end{proof}

\begin{corollary}
    The set \(\algebra{F} = \setc{\Phi_a(f)}{f \in \mathcal{C}\left(\sigma(a), \mathbb{C}\right)}\) is a unital abelian C*-subalgebra of \(\algebra{A}\).
\end{corollary}
\begin{proof}
    Let \(\family{u_n}{n \in \mathbb{N}} \subset \algebra{F}\) be a convergent sequence of operators in \(\algebra{F}\), then there exists a sequence \(\family{f_n}{n \in \mathbb{N}} \subset \mathcal{C}\left(\sigma(a), \mathbb{C}\right)\) such that \(\Phi_a(f_n) = u_n\) for all \(n \in \mathbb{N}\). As \(\Phi_a\) is an isometry and \family{u_n}{n \in \mathbb{N}} is a Cauchy sequence, it follows that \family{f_n}{n \in \mathbb{N}} is a Cauchy sequence. By completeness there exists a continuous function \(f : \sigma(a) \to \mathbb{C}\) against which the sequence \family{f_n}{n\in \mathbb{N}} converges uniformly. By \cref{thm:gelfand_homomorphism}, we know \family{u_n}{n\in \mathbb{N}} converges to \(\Phi_a(f) \in \algebra{F}\), that is, \(\algebra{F}\) is a closed subset of \(\algebra{A}\). The *-homomorphism properties there shown also ensure \(\algebra{F}\) is an abelian self-adjoint subalgebra of \(\algebra{A}\), therefore it is a C*-subalgebra of \(\algebra{A}\).
\end{proof}

\begin{definition}{C*-subalgebra generated by a set of operators}{cstar_generated}
    Let \(\hilbert\) be a Hilbert space. If \(S\subset \bounded(\hilbert)\) is a non-empty subset of bounded operators, then the \emph{C*-subalgebra generated by \(S\)}, denoted by \(C^*[S]\), is the intersection of all C*-subalgebras of \(\bounded(\hilbert)\) that contain \(S\).
\end{definition}
\begin{remark}
    The intersection of C*-subalgebras of a C*-algebra is a C*-subalgebra. Indeed, let \(\algebra{A}\) be a C*-algebra, let \(\family{A_{j}}{j \in J} \subset \algebra{A}\) be a family of C*-subalgebras of \(\algebra{A}\) with \(J\) an arbitrary indexing set, and denote \(A = \bigcap_{j \in J}{A_j}\). Since each \(A_j\) is a closed subspace of \(\algebra{A}\), it is clear that \(A\) is a closed subspace of \(\algebra{A}\). Moreover, since each \(A_j\) is a C*-algebra, it follows that \(A\) has the C*-property and is closed under the product, hence \(A\) is a C*-algebra.
\end{remark}

\begin{proposition}{C*-subalgebra generated by an operator}{cstar_generated}
    Let \(\hilbert\) be a Hilbert space and let \(a \in \bounded(\hilbert)\) be a self-adjoint operator. We denote the C*-subalgebra of \(\bounded(\hilbert)\) generated by \(\set{\unity, a}\) by \(C^*[a] = C^*[\set{\unity, a}]\). If \(\Phi_a : \mathcal{C}(\sigma(a), \mathbb{C}) \to \bounded(\hilbert)\) denotes Gelfand's homomorphism, then \(\Phi_a(\mathcal{C}(\sigma(a), \mathbb{C})) = C^*[a]\).
\end{proposition}
\begin{proof}
    We have already shown that \(\Phi_a(\mathcal{C}(\sigma(a), \mathbb{C}))\) is a C*-subalgebra of \(\bounded(\hilbert)\) that contains \(\set{\unity, a}\), then it contains \(C^*[a]\) by definition. Let \(f \in \mathcal{C}(\sigma(a), \mathbb{C})\), then there exists a sequence of polynomials \(\family{\varphi_n}{n \in \mathbb{N}} \subset \mathcal{P}(\sigma(a))\) that converges uniformly against \(f\). If \(\algebra{A} \subset \bounded(\hilbert)\) is a C*-subalgebra that contains \(\set{\unity,a}\), then for all \(n \in \mathbb{N}\) the operator \(\Phi_a(\varphi_n)\) is obtained by linear combination of \(\unity\) and powers of \(a\), that is, \(\Phi_a(\varphi_n) \in \algebra{A}\). Since Gelfand's homomorphism is a isometry, this means \(\family{\Phi_a(\varphi_n)}{n \in \mathbb{N}} \subset \algebra{A}\) is a Cauchy sequence in a C*-algebra, then it converges to \(\Phi_a(f) \in \algebra{A}\) by \cref{thm:gelfand_homomorphism}. This shows \(\Phi_a(f)\) is an element of every C*-subalgebra that contains \(\set{\unity, a}\), that is, \(\Phi_a(f) \in C^*[a]\).
\end{proof}

% vim: spl=en_us
\section{Square root of operators in Banach algebras}
Consider the function \(f : \mathbb{C} \to \mathbb{C}\) defined by \(f(z) = \sqrt{1 - z}\), which is analytic in the unit disk with
\begin{equation*}
    f(z) = 1 - \frac12z - \sum_{n = 2}^\infty \frac{(2n - 3)!!}{(2n)!!} z^n,
\end{equation*}
for all \(\abs{z} < 1\).
\begin{lemma}{Property of Taylor coefficients for \(z \mapsto \sqrt{1 - z}\) in the unit disk}{coefficients_square_root}
    Denote the Taylor coefficients of the map \(z \mapsto \sqrt{1-z}\) with \(\abs{z} < 1\) by
    \begin{equation*}
        c_0 = 1,\quad c_1 = -\frac12,\quad\text{and}\quad c_{n+1} = -\frac{(2n-1)!!}{(2n+2)!!}
    \end{equation*}
    for all \(n \in \mathbb{N}\). Then
    \begin{equation*}
        \sum_{k = 0}^\infty \abs{c_k} \leq 2
    \end{equation*}
    and the Taylor series for \(z \mapsto \sqrt{1-z}\) is absolutely convergent for all \(\abs{z} \leq 1\). Moreover,
    \begin{equation*}
        \sum_{\substack{k + \ell = m\\k,\ell \in \mathbb{N}_0}}{c_k c_\ell} = 0
    \end{equation*}
    for all \(m \geq 2\).
\end{lemma}
\begin{proof}
    Notice \(\abs{c_k} = - c_k\) for all \(k \in \mathbb{N}\), then
    \begin{equation*}
        \sum_{k = 0}^n (\abs{c_k} + c_k) = 2c_0 = 2 \implies \sum_{k = 0}^n \abs{c_k} = 2 - \sum_{k = 0}^n c_k
    \end{equation*}
    for all \(n \in \mathbb{N}_0\). Let \(t \in (0,1)\), then
    \begin{equation*}
        \sum_{k=0}^n c_k t^k = \sqrt{1 - t} - \sum_{k = {n+1}}^\infty c_k t^k = \sqrt{1 - t} + \sum_{k = n+1} \abs{c_k}t^k \geq \sqrt{1 - t}
    \end{equation*}
    for all \(n \in \mathbb{N}_0\). We then have
    \begin{equation*}
        \sum_{k = 0}^n \abs{c_k} = 2 - \sum_{k = 0}^n c_k = 2 - \lim_{t \to 1^{-}}\sum_{k = 0}^{n} c_k t^k \leq 2 - \lim_{t \to 1^{-}} \sqrt{1 - t} = 2
    \end{equation*}
    for all \(n \in \mathbb{N}_0\), which guarantees
    \begin{equation*}
        \sum_{k = 0}^\infty \abs{c_k} \leq 2.
    \end{equation*}
    Then, for all \(z \in \mathbb{C}\) with \(\abs{z} \leq 1\), we have
    \begin{equation*}
        \sum_{k = 0}^n \abs{c_k}\abs{z}^k \leq \sum_{k = 0}^n \abs{c_k} \leq 2
    \end{equation*}
    for all \(n \in \mathbb{N}_0\), that is, \(\sum_{k = 0}^\infty c_k z^k\) is absolutely convergent for all \(\abs{z} \leq 1\).

    Let \(z \in \mathbb{C}\) with \(\abs{z} < 1\), then
    \begin{equation*}
        1 - z = \left(\sum_{k = 0}^\infty c_k z^k\right)^2 = \sum_{k = 0}^\infty \sum_{\ell = 0}^\infty c_k c_\ell z^{k + \ell} = \sum_{m = 0}^\infty \sum_{\substack{k + \ell = m\\k,\ell \in \mathbb{N}_0}} c_kc_\ell z^m = 1 - z + \sum_{m = 2}^\infty z^m\sum_{\substack{k + \ell = m\\k,\ell \in \mathbb{N}_0}}c_k c_\ell.
    \end{equation*}
    From the uniqueness of the Taylor series we must have \(\sum_{\substack{k+\ell=m\\k,\ell \in \mathbb{N}_0}} c_k c_\ell = 0\) for all \(m \geq 2\).
\end{proof}

\begin{theorem}{Square root of an operator}{square_root}
    Let \(\algebra{A}\) be a unital Banach algebra, let \(u \in \algebra{A}\) with \(\norm{u} \leq 1\), and let \(c_n\) be the coefficients defined in \cref{lem:coefficients_square_root} with \(n \in \mathbb{N}_0\). There exists \(v \in \algebra{A}\) such that \(v^2 = \unity - u\). The operator defined by the convergent series
    \begin{equation*}
        v = \unity + \sum_{n = 1}^\infty c_n u^n
    \end{equation*}
    satisfies \(v^2 = \unity - u\).
\end{theorem}
\begin{proof}
    Let \(\family{v_n}{n \in \mathbb{N}}\subset \algebra{A}\) and \(\family{s_n}{n \in \mathbb{N}}\) be the sequences defined by the partial sums
    \begin{equation*}
        v_n = \unity + \sum_{k=1}^n c_k u^k\quad\text{and}\quad s_n = \sum_{k = 0}^n \abs{c_k}
    \end{equation*}
    for all \(n \in \mathbb{N}\). \cref{lem:coefficients_square_root} guarantees the sequence \(s_n\) is convergent, as it is a bounded and monotonic sequence of real numbers, hence it is a Cauchy sequence. Then, for all \(n,m \in \mathbb{N}\) with \(n \geq m\) we have
    \begin{equation*}
        \norm{v_n - v_m} = \norm*{\sum_{k=m+1}^n c_k u^k} \leq \sum_{k = m+1}^n \abs{c_k} \norm{u^k} \leq \sum_{k = m+1}^n \abs{c_k} = \abs{s_n - s_m},
    \end{equation*}
    that is, \(v_n\) is a Cauchy sequence. By completeness, there exists \(v \in \algebra{A}\) such that \(v_n \to v\).

    Let us write \(N_p = \setc{j \in \mathbb{N}_0}{j \leq p}\) for any \(p \in \mathbb{N}_0\) and use the convention \(u^0 = \unity\). Then
    \begin{equation*}
        v_n^2 = \sum_{k = 0}^n \sum_{\ell = 0}^n c_k c_\ell u^{k + \ell} = \sum_{m = 0}^{2n}u^m\sum_{\substack{k + \ell = m\\k, \ell \in N_n}} c_k c_\ell
    \end{equation*}
    for all \(n \in \mathbb{N}\) and, in particular, for \(n \geq 2\) we have
    \begin{align*}
        v_n^2 &= \unity - u + \sum_{m = 2}^{2n} u^m \sum_{\substack{k + \ell = m\\k, \ell \in N_n}}c_k c_\ell + \sum_{m = n + 1}^{2n} u^m \sum_{\substack{k + \ell = m\\k, \ell \in N_n}} c_k c_\ell\\
              &= \unity - u + \sum_{m = 2}^n u^m \sum_{\substack{k + \ell = m\\k,\ell \in \mathbb{N}_0}} c_k c_\ell + \sum_{m = n+1}^{2n} u^m \sum_{\substack{k + \ell = m\\k,\ell \in N_n}} c_k c_\ell\\
              &= \unity - u + \sum_{m = n+1}^{2n} u^m \sum_{\substack{k + \ell = m\\k,\ell \in N_n}} c_k c_\ell
    \end{align*}
    by \cref{lem:coefficients_square_root}. Notice we have for \(n + 1 \leq m \leq 2n\) and \(n \geq 2\) that
    \begin{equation*}
        \sum_{\substack{k + \ell = m\\k,\ell \in N_n}} c_k c_\ell = \sum_{k = m - n}^n c_k c_{m - k} = \sum_{k = 0}^n c_k c_{m - n} - \sum_{k = 0}^{m - n - 1} c_k c_{m - k} = \sum_{\substack{k + \ell = m\\k, \ell \in \mathbb{N}_0}} c_k c_\ell - \sum_{k = 0}^{m - n - 1} c_k c_{m - k} = - \sum_{k = 0}^{m - n - 1} c_k c_{m - k},
    \end{equation*}
    then
    \begin{align*}
        \norm{v_n^2 - (\unity - u)} &\leq \sum_{m = n + 1}^{2n} \norm{u}^m \abs*{\sum_{k = 0}^{m - n - 1} c_k c_{m-k}}\\
                                    &\leq \sum_{m = 0}^{n-1} \sum_{k = 0}^{m} \abs{c_k}\abs{c_{m+n+1-k}}\\
                                    &= \sum_{k = 0}^{n-1} \abs{c_k} \sum_{m = k}^{n-1} \abs{c_{m+n+1-k}}\\
                                    &= \sum_{k = 0}^{n-1} \abs{c_k} \sum_{m = n+1}^{2n - k} \abs{c_m}\\
                                    &\leq \sum_{k = 0}^{n-1} \abs{c_k} \sum_{m = n+1}^{2n} \abs{c_m}\\
                                    &\leq 2 \sum_{m = n+1}^{2n} \abs{c_m}\\
                                    &= 2 \abs{s_{2n} - s_{n+1}}.
    \end{align*}
    Since \(s_n\) is Cauchy, we may take \(n\) sufficiently big such that the right hand side becomes arbitrarily small, that is, \(v_n^2\) converges against \(\unity - u\). From the continuity of the product with respect to the uniform topology, this yields \(v^2 = \unity - u\).
\end{proof}
\begin{corollary}
    Let \(u \in \algebra{A}\) with \(\norm{\unity - u} \leq 1\), then there exists \(v \in \algebra{A}\) such that \(v^2 = u\). The operator defined by
    \begin{equation*}
        v = \unity + \sum_{n=1}^\infty c_n (\unity - u)^n
    \end{equation*}
    satisfies \(v^2 = u\).
\end{corollary}
\begin{proof}
    The operator \(\unity - u\) satisfies the hypothesis in \cref{thm:square_root}, then there exists
    \begin{equation*}
        v = \unity + \sum_{n = 1}^\infty c_n (\unity - u)^n
    \end{equation*}
    which satisfies \(v^2 = \unity - (\unity - u) = u\).
\end{proof}
\begin{corollary}
    If \(\algebra{A}\) is a unital Banach *-algebra and \(u \in \algebra{A}\) is self-adjoint with \(\norm{\unity - u} \leq 1\), then
    \begin{equation*}
        v = \unity + \sum_{n = 1}^{\infty}c_n(\unity - u)^n
    \end{equation*}
    is self adjoint.
\end{corollary}
\begin{proof}
    Let \(\family{v_n}{n \in \mathbb{N}} \subset \algebra{A}\) be defined by
    \begin{equation*}
        v_n = \unity + \sum_{k = 1}^n c_k (\unity - u)^k
    \end{equation*}
    for all \(n \in \mathbb{N}\). Since \(c_k \in \mathbb{R}\) for all \(k \in \mathbb{N}\), we have
    \begin{equation*}
        v_n^* = \unity^* + \sum_{k=1}^n c_k\left[(\unity - u)^k\right]^* = \unity + \sum_{k=1}^n c_k (\unity - u)^k = v_n,
    \end{equation*}
    for all \(n \in \mathbb{N}\), since \(\unity - u\) is self-adjoint. From the continuity of the adjoint operation, it follows that \(\displaystyle v = \lim_{n\to\infty} v_n\) is self-adjoint.
\end{proof}
\begin{corollary}
    If \(\algebra{A}\) is a unital Banach *-algebra and if \(u \in \algebra{A}\) with \(\norm{u} \leq 1\), then there exists \(v \in \algebra{A}\) self-adjoint such that \(v^2 = \unity - u^*u\).
\end{corollary}
\begin{proof}
    We have \(\norm{u^*u} \leq \norm{u^*}\norm{u} = \norm{u}^2 \leq 1\), then the result follows by \cref{thm:square_root} and by the previous corollary.
\end{proof}
\begin{corollary}
    If \(u \in \algebra{A} \setminus\set{0}\) satisfies \(\norm{\unity - \norm{u}^{-1} u} \leq 1\), then there exists \(v \in \algebra{A}\) such that \(v^2 = u\). The operator defined by
    \begin{equation*}
        v = \norm{u}^{\frac12} \left[\unity + \sum_{n = 1}^\infty c_n\left(\unity - \norm{u}^{-1}u\right)^n\right]
    \end{equation*}
    satisfies \(v^2 = u\). If, in addition, \(\algebra{A}\) is a Banach *-algebra and \(u\) is self-adjoint, then \(v\) is self-adjoint.
\end{corollary}
\begin{proof}
    Using \cref{thm:square_root} to the operator \(\norm{u}^{-1}u\) yields \(v_0 \in \algebra{A}\) defined by
    \begin{equation*}
        v_0  = \unity + \sum_{n = 1}^\infty c_n\left(\unity - \norm{u}^{-1}u\right)^n
    \end{equation*}
    satisfying \(v_0^2 = \norm{u}^{-1} u\). Then \(v = \norm{u}^{\frac12}v_0\) satisfies \(v^2 = \norm{u} v_0^2 = u\). If \(u\) is self-adjoint, then \(v_0\) is self-adjoint, and the result follows.
\end{proof}

Let \(D = \setc{a \in \algebra{A}}{\norm{a} \leq 1}\) be the unit closed disk around the zero operator. We now show the map
\begin{align*}
    \psi:D&\to \algebra{A}\\
        u &\mapsto \unity + \sum_{k = 1}^\infty c_k u^k,
\end{align*}
satisfying \(\psi(u)^2 = \unity - u\) for all \(u \in D\), is continuous with respect to the uniform topology.
\begin{proposition}{Uniform continuity of the square root of an operator}{square_root_continuous}
    Let \(\algebra{A}\) be a unital Banach algebra and let \(\psi : D \to \algebra{A}\) be as above. Then \(\psi\) is a continuous map with respect to the uniform topology.
\end{proposition}
\begin{proof}
    For convenience, we write
    \begin{equation*}
        \psi_n(w) = \unity + \sum_{k = 1}^n c_k w^k
    \end{equation*}
    for all \(n \in \mathbb{N}\) and \(w \in D\). Let \(\varepsilon > 0\), then \cref{lem:coefficients_square_root} shows us there exists \(N \in \mathbb{N}\) such that
    \begin{equation*}
        \norm*{\psi(w) - \psi_n(w)} \leq \sum_{k = n + 1}^\infty \abs{c_k} \norm{w}^k \leq \sum_{k = n + 1}^\infty \abs{c_k} < \frac13\varepsilon
    \end{equation*}
    for all \(n \geq N\) and all \(w \in D\).

    Let \(\family{u_n}{n\in \mathbb{N}} \subset D\) be a sequence that converges against \(u \in D\), then it follows from \cref{lem:estimate_difference_power} that
    \begin{equation*}
        \norm*{\psi_N(u_m) - \psi_N(u)} \leq \sum_{k = 1}^N \abs{c_k} \norm{u_m^k-u^k} \leq \sum_{k = 1}^N k\abs{c_k}\norm{u_m - u}
    \end{equation*}
    for all \(m \in \mathbb{N}\). Since \(u_n \to u\), there exists \(M \in \mathbb{N}\) such that \(\norm{u_m - u} < \frac{\varepsilon}{6N}\) for all \(m \geq M\), then
    \begin{equation*}
        \norm*{\psi_N(u_m) - \psi_N(u)} < \sum_{k = 1}^N \frac{k \varepsilon}{6N} \abs{c_k} \leq \frac16 \varepsilon\sum_{k = 1}^N \abs{c_k} \leq \frac13\varepsilon
    \end{equation*}
    for all \(m \geq M\). Then, for all \(m \geq \max\set{N, M}\) we have by the triangle inequality that
    \begin{equation*}
        \norm{\psi(u) - \psi(u_m)} \leq \norm*{\psi(u) - \psi_N(u)} + \norm*{\psi_N(u) - \psi_N(u_m)} + \norm*{\psi_N(u_m) - \psi(u_m)} < \varepsilon,
    \end{equation*}
    that is, \(\psi\) is continuous with respect to the uniform topology.
\end{proof}

% vim: spl=en_us
\section{Positive elements of a C*-algebra}
We now show every non-zero self-adjoint operator in a C*-algebra with a positive spectrum admits a unique square root. Let us denote the real half-lines by \(\mathbb{R}_+ = [0,\infty)\) and \(\mathbb{R}_- = (-\infty, 0]\).
\begin{definition}{Positive element of an involutive algebra}{positive}
    Let \(\algebra{A}\) be a *-algebra. A \emph{positive element} \(a \in \algebra{A}\) is self-adjoint and its spectrum lies in the positive half-line, \(\sigma(a) \subset \mathbb{R}_+\). The set of all positive elements is denoted by \(\algebra{A}_+\).
\end{definition}

\begin{proposition}{Norm of a positive operator is in its spectrum}{norm_positive}
    Let \(\algebra{A}\) be a unital C*-algebra. If \(a \in \algebra{A}_+\), then \(\norm{a} \in \sigma(a)\).
\end{proposition}
\begin{proof}
    By \cref{thm:spectra_self_adjoint} we have \(\sigma(a) \subset [0, \norm{a}]\). Suppose, by contradiction, there exists \(M \in [0, \norm{a})\) such that \(\sigma(a) \subset [0, M]\). Then by \cref{thm:spectral_radius_cstar}, we have
    \begin{equation*}
        \norm{a} = r(a) = \sup_{\lambda \in \sigma(a)}{\abs{\lambda}} \leq \sup_{\lambda \in [0, M]} \abs{\lambda} = M < \norm{a},
    \end{equation*}
    which shows there exists no such \(M\).
\end{proof}

\begin{lemma}{Positive elements of a C*-algebra has no pair of distinct opposite operators}{positive_salient}
    Let \(\algebra{A}\) be a unital C*-algebra. Then \(\algebra{A}_+ \cap (-\algebra{A}_+) = \set{0}\).
\end{lemma}
\begin{proof}
    Notice \(0 \in \algebra{A}\) is a positive element, as it is self-adjoint with \(\sigma(0) = \set{0} \subset \mathbb{R}_+\). As \(-0 = 0\), we have \(0 \in \algebra{A}_+ \cap (-\algebra{A}_+)\). Let \(a \in \algebra{A}_+ \cap (-\algebra{A}_+)\), then \(\sigma(a) \subset \mathbb{R}_+\) and there exists \(b \in \algebra{A}_+\) such that \(a = -b\). \cref{thm:spectral_mapping}, yields \(\sigma(a) = -\sigma(b) \subset \mathbb{R}_-\). As \(\sigma(a)\in \algebra{A}_+\), we have \(\sigma(a) \subset \mathbb{R}_+\), then it follows that \(\sigma(a) = \set{0}\). That is, \(r(a) = 0\), and we conclude from \cref{thm:spectral_radius_cstar} that \(a = 0\).
\end{proof}
\begin{corollary}
    Let \(\algebra{A}\) be a unital C*-algebra. If \(a, b\in \algebra{A}_+\) such that \(a + b = 0\), then \(a = b = 0\).
\end{corollary}
\begin{proof}
    If \(a + b = 0\), then \(\algebra{A}_+ \ni a = -b \in (-\algebra{A}_+)\), that is, \(a \in \algebra{A}_+ \cap (-\algebra{A}_+)\). We conclude \(a = 0\) from \cref{lem:positive_salient}, hence \(b = 0\).
\end{proof}

\begin{lemma}{If two positive operators commute, then their product is positive}{positive_commute}
    Let \(\algebra{A}\) be a unital C*-algebra. If \(a,b \in \algebra{A}_+\) such that \(ab = ba\), then \(ab \in \algebra{A}_+\).
\end{lemma}
\begin{proof}
    Gelfand homomorphism yields \(c,d \in \algebra{A}\) such that \(c^2 = a\) and \(d^2 = d\), where \(c\) and \(d\) are obtained by the limit of sequences of polynomials of \(a\) and \(b\), hence \(cd = dc\) as \(a\) and \(b\) commute. Then \(ab = c^2 d^2 = (cd)^2\), and we conclude \(\sigma(ab) \subset [0, \norm{cd}^2] \subset \mathbb{R}_+\) from \cref{thm:spectral_mapping}.
\end{proof}

\begin{theorem}{Square root in C*-algebra}{square_root_cstar}
    Let \(\algebra{A}\) be a unital C*-algebra. If \(u \in \algebra{A}\setminus\set{0}\) is self-adjoint, then the following statements are equivalent:
    \begin{enumerate}[label=(\alph*)]
        \item \(u \in \algebra{A}_+\);
        \item \(\norm*{\unity - \norm{u}^{-1}u} \leq 1\); and
        \item there exists \(v \in \algebra{A}\) self-adjoint such that \(v^2 = u\).
    \end{enumerate}
    In addition, if \(u\) is positive, then there exists a unique \(w \in \algebra{A}_+\) such that \(w^2 = u\), and we say \(w = \sqrt{u}\) is \emph{the positive square root of \(u\)}.
\end{theorem}
\begin{proof}
    Suppose (a) and consider the polynomial \(\varphi(z) = 1 - \frac{z}{\norm{u}}\). By \cref{thm:spectral_mapping}, we have \(\sigma(\varphi(u)) = \varphi(\sigma(u)) \subset \varphi([0, \norm{u}]) = [0,1]\). Then, (b) follows from \cref{thm:spectral_radius_cstar}. Supposing (b), (c) follows from \cref{thm:square_root}. Supposing (c), we have \(\sigma(u) = \sigma(v^2) = \sigma(v)^2 \subset [0, \norm{v}^2]\) by \cref{thm:spectral_mapping,thm:spectral_radius_cstar}, hence \(u \in \algebra{A}_+\), and we conclude (a).

    If \(u \in \algebra{A}_+\setminus\set{0}\), then \(\sigma(u) \subset [0, \norm{u}]\). The map \(\id{\sigma(u)} : \sigma(u) \to \sigma(u)\) is continuous and so is the map \(\sqrt{\noarg} : \mathbb{R} \to \mathbb{R}\), then the map \(\psi = \restrict{\sqrt{\noarg}}{\sigma(u)} \circ \id{\sigma(u)} : \sigma(u) \to \mathbb{R}\) is continuous by \cref{prop:restriction_map} and satisfies \(\psi(\lambda) \geq 0\) for all \(\lambda \in \sigma(u)\). By \cref{thm:gelfand_homomorphism}, we know \(w = \Phi_u(\psi)\) is self-adjoint and satisfies both \(\sigma(w) \subset \mathbb{R}_+\) and
    \begin{equation*}
        w^2 = \Phi_u(\psi)^2 = \Phi_u(\id{\sigma(u)}) = u.
    \end{equation*}
    It remains to show \(w \in \algebra{A}_+\) is the only such positive element of \(\algebra{A}\).

    Let \(v \in \algebra{A}_+\) such that \(v^2 = u\), then \(v\) commutes with \(u\), as \(uv = v^3 = vu\). As \(w\) is the limit of polynomials of \(u\), it commutes with \(u\), and therefore it commutes with any operator that commutes with \(u\) therefore, in particular, we have \(wv = vw\). This yields
    \begin{equation*}
        0 = (u - u)(v - w) = (v^2 - w^2)(v - w) = (v + w)(v - w)^2 = v(v - w)^2 + w(v - w)^2.
    \end{equation*}
    Notice \((v - w)^2 \in \algebra{A}_+\), then \cref{lem:positive_commute} guarantees \(v(v - w)^2\) and \(w(v - w)^2\) are positive. We have then written \(0\) as the sum of two positive operators, and we conclude by \cref{lem:positive_salient} that both must be equal to the zero operator. Then,
    \begin{equation*}
        0 = v(v - w)^2 - w(v - w)^2 = (v - w)^3 \implies (v - w)^4 = 0,
    \end{equation*}
    hence
    \begin{equation*}
        \norm{(v - w)}^4 = \norm{(v - w)^2}^2 = \norm{(v - w)^4} = 0
    \end{equation*}
    follows from the self-adjointness of \(v - w\) and the C*-property. That is, \(v - w = 0\), which shows the uniqueness of the positive square root.
\end{proof}
\begin{corollary}
    Let \(\algebra{A}\) be a unital C*-algebra. If \(u \in \algebra{A}\) is self-adjoint and \(\norm{u} \leq 1\), then there exists a unique \(v \in \algebra{A}_+\) such that \(v^2 = \unity - u\), and we write \(v = \sqrt{\unity - u}\).
\end{corollary}
\begin{proof}
    We consider \(w = \unity - u\), then \(\norm{\unity - w} = \norm{u} \leq 1\), then it follows from \cref{thm:square_root} that there exists \(v \in \algebra{A}\) such that \(v^2 = w\). By \cref{thm:square_root_cstar}, we know \(w \in \algebra{A}_+\) and we may take \(v\) as the unique element of \(\algebra{A}_+\) such that \(v^2 = w\).
\end{proof}


\begin{theorem}{Set of positive elements of a C*-algebra is a closed convex salient cone}{positive_cone}
    Let \(\algebra{A}\) be a unital C*-algebra. Then
    \begin{enumerate}[label=(\alph*)]
        \item \(\algebra{A}_+\) is a salient cone, that is, if \(u \in \algebra{A}_+\) and \(\lambda \in \mathbb{R}_+\), then \(\lambda u \in \algebra{A}_+\) and it has the property \(\algebra{A}_+ \cap (-\algebra{A}_+) = \set{0}\);
        \item \(\algebra{A}_+\) is convex, that is, if \(u, v \in \algebra{A}_+\) and \(\lambda \in [0,1]\), then \(\lambda u + (1 - \lambda)v \in \algebra{A}_+\); and
        \item \(\algebra{A}_+\) is closed in the uniform topology;
    \end{enumerate}
\end{theorem}
\begin{proof}
    Let \(u \in \algebra{A}_+\) and \(\lambda \in \mathbb{R}_+\), then \(\sigma(\lambda u) = \lambda \sigma(u) \subset [0, \lambda \norm{u}] \subset \mathbb{R}_+\) and \(\lambda u\) is self-adjoint, hence \(\lambda u \in \algebra{A}_+\), that is, \(\algebra{A}_+\) is a cone. We have shown that it is salient in \cref{lem:positive_salient}, thus (a) follows.

    Let us consider \(a \in \algebra{A}_+\) and \(\mu \geq \norm{a}\) with \(\mu \neq 0\). Then by \cref{thm:spectral_mapping} we have
    \begin{equation*}
        \sigma(\unity - \mu^{-1} a) = \setc*{1 - \frac{\lambda}{\mu}}{\lambda \in \sigma(a)} \subset \left[1 - \frac{\norm{a}}{\mu}, 1\right] \subset [0, 1],
    \end{equation*}
    and it follows from \cref{thm:spectral_radius_cstar} that \(\norm{\unity - \mu^{-1} a} \leq 1\). Let \(u,v \in \algebra{A}_+\) and let \(\lambda \in [0,1]\), then for any \(\kappa \geq \max\set{\norm{u}, \norm{v}}\) with \(\kappa \neq 0\), we have
    \begin{align*}
        \norm*{\unity - \kappa^{-1}\left[\lambda u + (1 - \lambda)v\right]}
        &= \norm*{\lambda\left(\unity - \kappa^{-1} u\right) + (1- \lambda) (\unity - \kappa^{-1}v)}\\
        &\leq \lambda \norm{\unity - \kappa^{-1}u} + (1 - \lambda) \norm{\unity - \kappa^{-1} v}\\
        &\leq \lambda + 1 - \lambda = 1,
    \end{align*}
    thus showing \(\sigma\left\{\unity - \kappa^{-1}\left[\lambda u + (1 - \lambda)v\right]\right\} \subset [-1,1]\) and as a result, \(\sigma\left[\lambda u + (1 - \lambda)v\right] \subset [0, 2 \kappa]\). That is, \(\lambda u + (1 - \lambda)v \in \algebra{A}_+\) and we conclude (b).

    Let \(\family{u_n}{n \in \mathbb{N}} \subset \algebra{A}_+\) be a sequence of positive operators that converge against \(u \in \algebra{A}\). We may assume without loss of generality that \(u_n \neq 0\) for all \(n \in \mathbb{N}\), for if the sequence were to converge against \(0 \in \algebra{A}_+\) there would be nothing to show and if the sequence converges against \(u \in \algebra{A}\setminus{0}\), we may take a convergent subsequence. For \(n \in \mathbb{N}\), we have \(u_n \in \algebra{A}_+\setminus\set{0}\), then \cref{thm:square_root_cstar} yields \(\norm*{\unity - \norm{u_n}^{-1}u_n}\leq 1\) and we have
    \begin{equation*}
        \norm*{\unity - \norm{u}^{-1}u} = \lim_{n \to \infty} \norm*{\unity - \norm{u_n}^{-1}u_n} \leq \lim_{n \to \infty} 1 = 1,
    \end{equation*}
    that is, \(u \in \algebra{A}_+\).
\end{proof}
\begin{corollary}
    Let \(\algebra{A}\) be a unital C*-algebra. If \(u,v\in \algebra{A}_+\), then \(u + v \in \algebra{A}_+\).
\end{corollary}
\begin{proof}
    Since \(\frac12 u + \frac12 v\) is a convex linear combination of positive operators, it is a positive operator. As \(\algebra{A}_+\) is a cone, \(2 \left(\frac12 u + \frac12 v\right) = u + v \in \algebra{A}_+\).
\end{proof}
\begin{corollary}
    Let \(\algebra{A}\) be a unital C*-algebra. If \(u \in \algebra{A}\) is such that \(-u^*u \in \algebra{A}_+\), then \(u = 0\).
\end{corollary}
\begin{proof}
    Since \(\sigma(u^*u) \setminus\set{0} = \sigma(uu^*)\setminus\set{0}\), we know that \(-u^*u\) is positive if and only if \(-uu^*\) is positive, then by \cref{thm:positive_cone}, we know \(\frac12 uu^* + \frac12 u^*u \in -\algebra{A}_+\). We define the self adjoint operators \(x = \frac12 (u + u^*)\) and \(y = \frac1{2i}(u - u^*)\), with
    \begin{equation*}
        x^2 + y^2 = \frac12(u^*u + uu^*) \in -\algebra{A}_+.
    \end{equation*}
    Notice \(x^2, y^2 \in \algebra{A}_+\), then by the previous corollary \(x^2 + y^2 \in \algebra{A}_+\). Since \(\algebra{A}_+\) is a salient cone, we have \(x^2 = y^2 = 0\). The C* property then yields \(x = y = 0\), thus showing \(u = x + iy = 0\).
\end{proof}

We will now show every positive element of a C*-algebra is of the form \(x^*x\). First we show the following decomposition result.
\begin{lemma}{Orthogonal decomposition of an operator}{orthogonal_decomposition_cstar}
    Let \(\algebra{A}\) be a unital C*-algebra. If \(u \in \algebra{A}\) is self-adjoint, then there exists unique positive operators \(u_+, u_- \in \algebra{A}_+\) such that \(u = u_+ - u_-\) and \(u_+u_- = u_-u_+ = 0\).
\end{lemma}
\begin{proof}
    We consider the continuous maps
    \begin{align*}
        f_+ : \sigma(u) &\to \mathbb{R}&
        f_- : \sigma(u) &\to \mathbb{R}\\
                \lambda &\mapsto \frac12\left(\abs{\lambda} + \lambda\right)&
                \lambda &\mapsto \frac12\left(\abs{\lambda} - \lambda\right)
    \end{align*}
    which verify
    \begin{equation*}
        (f_+\cdot f_-)(\lambda) = f_+(\lambda)f_-(\lambda) = \frac14 \left(\abs{\lambda}^2 - \lambda^2\right) = 0,
    \end{equation*}
    \begin{equation*}
        (f_+ - f_-)(\lambda) = f_+(\lambda) - f_-(\lambda) = \lambda,
    \end{equation*}
    and
    \begin{equation*}
        (f_+ + f_-)(\lambda) = f_+(\lambda) + f_-(\lambda) = \abs{\lambda} = \sqrt{\lambda^2}
    \end{equation*}
    for all \(\lambda \in \sigma(u)\), that is, \(f_+ f_- = \id{\sigma(u)}\) and \(f_+ - f_- = 0\). We define \(u_+ = \Phi_u(f_+)\) and \(\Phi_u(f_-)\), then \(u_+ - u_- = u\) and \(u_+ u_- = 0\). As the image of Gelfand homomorphism is a commutative C*-subalgebra, we also have \(u_- u_+ = 0\).

    Let \(\tilde{u}_+, \tilde{u}_- \in \algebra{A}_+\) such that \(\tilde{u}_+ - \tilde{u}_- = u\) and \(\tilde{u}_+\tilde{u}_- = \tilde{u}_-\tilde{u}_+ = 0\). Then
    \begin{equation*}
        u^2 = \left(\tilde{u}_+ - \tilde{u}_-\right)^2 = \tilde{u}_+^2 + \tilde{u}_-^2 = \left(\tilde{u}_+ + \tilde{u}_-\right)^2 \implies u_+ + u_- = \sqrt{u^2} = \tilde{u}_+ + \tilde{u}_-,
    \end{equation*}
    which yields \(\tilde{u}_+ = u_+\) and \(\tilde{u}_- = u_-\).
\end{proof}

\begin{theorem}{Characterization of positive elements of a C*-algebra}{positive_cstar}
    Let \(\algebra{A}\) be a unital C*-algebra. Then \(\algebra{A}_+ = \setc{u^*u}{u \in \algebra{A}}\).
\end{theorem}
\begin{proof}
    Let \(u \in \algebra{A}_+\), then \(\sqrt{u} \in\algebra{A}_+\) satisfies \(\sqrt{u}^* \sqrt{u} = \sqrt{u}^2 = u\).

    Let \(a \in \algebra{A}\) and let \(b = a^*a\), with orthogonal decomposition \(b_+, b_- \in \algebra{A}_+\). Consider \(c = ab_-\), then \(- c^*c = - b_- a^* a b_- = b_-bb_- = (b_-)^3 \in \algebra{A}_+\). By \cref{thm:positive_cone}, we know \(c = 0\), which yields \(0 = a^*c = b b_- = -(b_-)^2\), and we conclude \(b_- = 0\) by the C* property. That is, \(a^*a = b_+ \in \algebra{A}_+\).
\end{proof}

\begin{corollary}
    Let \(\algebra{A}\) be a unital C*-algebra. If \(a \in \algebra{A}\), then \(\norm{a}^2 \in \sigma(a^*a)\).
\end{corollary}
\begin{proof}
    Since \(a^*a \in \algebra{A}_+\), then \(\norm{a}^2 = \norm{a^*a} \in \sigma(a^*a)\).
\end{proof}

Recall \cref{prop:polarization_star_algebra}, which shows we may write any operator \(a \in \algebra{A}\) of a unital *-algebra as
\begin{equation*}
    a = \frac14 \sum_{k = 0}^3 i^k(a + i^k \unity)^*(a + i^k \unity).
\end{equation*}
In a C*-algebra, this shows every operator can be written as a linear combination of four positive operators.
\begin{proposition}{Unitary decomposition}{unitary_decomposition}
    Let \(\algebra{A}\) be a unital C*-algebra. If \(a \in \algebra{A}\) is self-adjoint, then there exist \(u_+, u_- \in \algebra{A}\) unitary such that \(a = \frac{\norm{a}}{2}(u_+ + u_-)\). If \(b \in \algebra{A}\), then for \(k \in \set{1,2,3,4}\) there exist \(u_k \in \algebra{A}\) unitary and \(\beta_k \in \setc{\lambda \in \mathbb{C}}{\abs{\lambda} \leq \frac12 \abs{b}}\) such that \(b = \sum_{k = 1}^4 \beta_k u_k\).
\end{proposition}
\begin{proof}
    Let \(a \in \algebra{A}\) be a self-adjoint operator, then we may assume \(a \neq 0\), since \(0 = \frac{0}{2}(\unity - \unity)\). Notice \(\norm{\norm{a}^{-2}a^2} = 1\) by the C*-property, then there exists \(\sqrt{\unity - \norm{a}^{-2}a^2} \in \algebra{A}_+\). We define
    \begin{equation*}
        u_\pm = \norm{a}^{-1} a \pm i \sqrt{\unity - \norm{a}^{-2} a^2},
    \end{equation*}
    then \(a = \frac{\norm{a}}{2}(u_+ + u_-)\). By self-adjointness of \(a\) and the square root, we have \(u_\pm^* = u_\mp\), then
    \begin{equation*}
        u_\pm^*u_\pm = \left(\norm{a}^{-1} a \mp i \sqrt{\unity - \norm{a}^{-2}a^2}\right)\left(\norm{a}^{-1} a \pm i \sqrt{\unity - \norm{a}^{-2}a^2}\right) = \norm{a}^{-2}a^2 + \sqrt{\unity - \norm{a}^{-2}a^2}^2 = \unity
    \end{equation*}
    and
    \begin{equation*}
        u_\pm u_\pm^* = \left(\norm{a}^{-1} a \pm i \sqrt{\unity - \norm{a}^{-2}a^2}\right)\left(\norm{a}^{-1} a \mp i \sqrt{\unity - \norm{a}^{-2}a^2}\right) = \norm{a}^{-2}a^2 + \sqrt{\unity - \norm{a}^{-2}a^2}^2 = \unity,
    \end{equation*}
    that is, \(u_\pm\) is unitary.

    Let \(b \in \algebra{A}\) be an operator, then \(b = a_1 + ia_2\), where \(a_1 = \frac{1}{2}(b + b^*)\) and \(a_2 = \frac{1}{2i}(b - b^*)\) are self-adjoint. By the previous result, we may decompose \(a_1\) and \(a_2\) with two unitary operators, \(a_j = \frac12\norm{a_j}(u_j^+ + u_i^-)\) for \(j \in \set{1,2}\), then
    \begin{equation*}
        b = \frac14\norm{b + b^*} (u_1^+ + u_1^-) + \frac14\norm{b - b^*} (u_2^+ + u_2^-),
    \end{equation*}
    with \(\frac14\norm{b \pm b^*} \leq \frac14\norm{b} + \frac14\norm{b^*} = \frac12 \norm{b}\).
\end{proof}

% In \cref{lem:orthogonal_decomposition_cstar}, we have used the Gelfand homomorphism to define the modulus of a self-adjoint operator. In fact, we may generalize such a map for any operator in a C*-algebra.
% \begin{proposition}{Polar decomposition}{polar_decomposition_cstar}
%     Let \(\algebra{A}\) be a unital C*-algebra. The modulus of an operator is defined by the map
%     \begin{align*}
%         \abs{\noarg} : \algebra{A} &\to \algebra{A}_+\\
%                                  a &\mapsto \sqrt{a^*a}.
%     \end{align*}
%     Then for every operator \(a \in \algebra{A}\), there exists an unitary operator \(u \in \algebra{A}\) in the C*-subalgebra generated by \(a\) and \(a^*\) such that \(a = u\abs{a}\).
% \end{proposition}
% \begin{proof}
%     The map is well-defined as \(a^*a \in \algebra{A}_+\) by \cref{thm:positive_cstar} and there exists a unique positive square root for every positive operator. Notice
% \end{proof}

\subsection{Partial order induced by positive elements of a C*-algebra}
The set of positive elements \(\algebra{A}_+\) of a C*-algebra \(\algebra{A}\) induces naturally a partial order on \(\algebra{A}\) defined as \(a \preceq b\) if \(b - a \in \algebra{A}_+\). Reflexivity follows from \(0 \in \algebra{A}_+\), anti-symmetry follows from \cref{lem:positive_salient}, and transitivity follows from \(\algebra{A}_+\) being a convex cone. In fact, the set of self-adjoint operators is linearly ordered by this partial order, and we write \(a \leq b\) if \(a\) and \(b\) are self-adjoint with \(a \preceq b\). If \(b \leq a\), we may also write \(a \geq b\), and in addition to \(b \neq a\), we write \(a > b\) and, analogously, \(b < a\).

\begin{proposition}{Congruence transformations are order preserving}{congruence_order}
    Let \(\algebra{A}\) be a unital C*-algebra. If \(a,b \in \algebra{A}\) are self-adjoint with \(a \geq b\), then \(c^*ac \geq c^* b c\) for all \(c \in \algebra{A}\).
\end{proposition}
\begin{proof}
    If \(u\in \algebra{A}\) is self-adjoint, then \((c^*uc)^* = c^* u^* c = c^* u c\) for any \(c \in \algebra{A}\). Then, \(c^*ac - c^*bc = c^*(a - b)c = c^*\sqrt{a - b} \sqrt{a - b}c = (\sqrt{a - b} c)^* (\sqrt{a - b}c) \in \algebra{A}_+\).
\end{proof}

\begin{proposition}{Properties of the order induced by positive elements}{properties_order}
    Let \(\algebra{A}\) be a unital C*-algebra.
    \begin{enumerate}[label=(\alph*)]
        \item If \(u\in\algebra{A}_+\), then \(\norm{u} \unity \geq u \geq 0\).
        \item If \(u \in \algebra{A}_+\), then \(\norm{u} u \geq u^2 \geq 0\).
        \item If \(u,v \in \algebra{A}_+\) with \(u \geq v\), then \(\norm{u}\geq \norm{v}\).
    \end{enumerate}
\end{proposition}
\begin{proof}
    If \(u \in \algebra{A}_+\), then \(u - 0 \in \algebra{A}_+\), that is, \(u \geq 0\). \cref{thm:spectral_mapping} yields
    \begin{equation*}
        \sigma(\norm{u} \unity - u) = \setc{\norm{u} - \lambda}{\lambda \in \sigma(u)} \subset [0, \norm{u}],
    \end{equation*}
    and
    \begin{equation*}
        \sigma(\norm{u} u - u^2) = \setc{\norm{u}\lambda - \lambda^2}{\lambda \in \sigma(u)} \subset \setc{\norm{u}\lambda - \lambda^2}{\lambda \in [0, \norm{u}]} \subset \left[0, \frac{\norm{u}^2}{4}\right],
    \end{equation*}
    that is, \(\norm{u} \unity - u \in \algebra{A}_+\) and \(\norm{u} u - u^2 \in \algebra{A}_+\), and we conclude (a) and (b).

    If \(0 \leq v \leq u\), then by transitivity we have \(v \leq \norm{u} \unity\), then
    \begin{equation*}
        \sigma(\norm{u}\unity - v) = \setc{\norm{u} - \lambda}{\lambda \in \sigma(v)} \subset \mathbb{R}_+,
    \end{equation*}
    which implies \(\norm{u} - \lambda \geq 0\) for all \(\lambda \in \sigma(v)\). In particular, by \cref{prop:norm_positive}, we have thus shown \(\norm{u} - \norm{v} \geq 0\), and we conclude (c).
\end{proof}

\begin{proposition}{Positive resolvent and order relations}{positive_resolvent}
    Let \(\algebra{A}\) a unital C*-algebra. If \(a \in \algebra{A}_+\) and \(\lambda \in \mathbb{R}_+\), then \(\unity + \lambda a \in \invertible{\algebra{A}}\) and \((\unity + \lambda a)^{-1} \in \algebra{A}_+\). Moreover, if \(b \in \algebra{A}\) with \(b \leq a\), then \((\unity +\lambda b)^{-1} \geq (\unity + \lambda a)^{-1}\).
\end{proposition}
\begin{proof}
    We may assume \(\lambda > 0\), since \(\unity = \unity^{-1} \in \algebra{A}_+\) and \(\unity \geq \unity\). Then \(-\lambda^{-1} \in \mathbb{R}_-\) and we conclude from positivity that \(-\lambda^{-1} \notin \sigma(a)\). That is, \(-\lambda^{-1} \unity - a \in \invertible{\algebra{A}}\) which yields \(\unity + \lambda a \in \invertible{\algebra{A}}\). \cref{prop:spectrum_inverse,thm:spectral_mapping} yields
    \begin{equation*}
        \sigma\left((\unity + \lambda a)^{-1}\right) = \setc{\mu^{-1}}{\mu \in \sigma(\unity + \lambda a)} = \setc*{\frac1{1 + \mu \lambda}}{\mu \in \sigma(a)} \subset \mathbb{R}_+,
    \end{equation*}
    that is, \((\unity + \lambda a)^{-1} \in \algebra{A}_+\).

    Notice \(\unity + \lambda b \in \algebra{A}_+\), then there exists a unique positive square root \(\sqrt{\unity + \lambda b} \in \algebra{A}_+\) and by the previous result, we have \(\sqrt{\unity + \lambda b}^{-1} = \sqrt{(\unity + \lambda b)^{-1}} \in \algebra{A}_+\). Since the positive square root \(\sqrt{(\unity + \lambda b)^{-1}}\) is in the C*-subalgebra generated by \(\unity\) and \((\unity + \lambda b)^{-1}\), it commutes with \((\unity + \lambda b)^{-1}\) and therefore with \(\unity + \lambda b\) by \cref{prop:resolvent_commute}. Then \cref{prop:congruence_order} yields
    \begin{equation*}
        a \geq b \implies \unity + \lambda a \geq \unity + \lambda b \implies  u = \sqrt{(\unity + \lambda b)^{-1}} (\unity + \lambda a) \sqrt{(\unity + \lambda b)^{-1}} \geq \unity.
    \end{equation*}
    Notice \(u \in \invertible{\algebra{A}}\) with \(u^{-1} = \sqrt{\unity + \lambda b} (\unity + \lambda a)^{-1} \sqrt{\unity + \lambda b}\), then \(\sigma(u) \subset [1, \infty)\), which yields \(\sigma(u^{-1}) \subset (0, 1]\). That is, \(u^{-1} \leq \unity\) and we have
    \begin{equation*}
        \sqrt{\unity + \lambda b} (\unity + \lambda a)^{-1} \sqrt{\unity + \lambda b} \leq \unity \implies (\unity + \lambda a)^{-1} \leq (\unity + \lambda b)
    \end{equation*}
    by \cref{prop:congruence_order}.
\end{proof}

% vim: spl=en_us
\section{Approximate identity in C*-algebras}
Even though it is always possible to adjoin an identity to a C*-algebra, it is sometimes more worthwhile to intrinsically work with the original algebra, using \emph{approximate identities.}

%TODO: maybe move this definition to the topology part?
\begin{definition}{Directed sets and convergence}{directed_set}
    A \emph{directed set} \((\Lambda, \preceq)\) is a set \(\Lambda\) with a partial order \(\preceq\) with the property that for all \(\lambda, \lambda' \in \Lambda\), there exists \(\mu \in \Lambda\) such that \(\lambda \preceq \mu\) and \(\lambda' \preceq \mu\).

    Let \(X\) be a non-empty set. If \(x : \Lambda \to X\) is a map, we say \(x\) is a \emph{net} on \(X\) based on the directed set \(\Lambda\). As with sequences, we may denote a net by \(\family{x_{\lambda}}{\lambda \in \Lambda}\), if the partial order is understood by context. If \(\tau\) is a topology on \(X\), we say \(\tilde{x} \in X\) is a limit point of \(x\) with respect to \(\tau\) if for any open neighborhood \(U \in \tau\) of \(\tilde{x}\) there exists \(\kappa\in \Lambda\) such that \(x_{\lambda} \in U\) for all \(\lambda \succeq \kappa\). If \(\tilde{x}\) is a limit point of \(x\) with respect to \(\tau\) we write \(\lim_{\lambda} x_{\lambda} = \tilde{x}\) if the topology and the directed set are understood by context.
\end{definition}

\begin{definition}{Approximate identity}{approximate_identity}
    Let \(\algebra{A}\) be a C*-algebra and let \(\algebra{I}\) be a right ideal of \(\algebra{A}\). An \emph{approximate identity on \(\algebra{A}\) by elements of the right ideal \(\algebra{I}\)} is a net \(e : \Lambda \to \algebra{I}\) satisfying
    \begin{enumerate}[label=(\alph*)]
        \item \(e_{\lambda} \in \algebra{A}_+\) for all \(\lambda \in \Lambda\);
        \item \(\norm{e_{\lambda}}\leq 1\) for all \(\lambda \in \Lambda\);
        \item if \(\lambda, \lambda' \in \Lambda\) with \(\lambda \succeq \lambda'\), then \(e_{\lambda} \geq e_{\lambda'}\); and
        \item \(\lim_{\lambda}{\norm{a - e_{\lambda}a}}= 0\) for all \(a \in \algebra{I}\).
    \end{enumerate}
    If \(\algebra{I} = \algebra{A}\), then we say \(e : \Lambda \to \algebra{A}\) is an \emph{approximate identity on \(\algebra{A}\)}.
\end{definition}

\begin{theorem}{Existence of approximate identity on a ideal of a unital C*-algebra}{approximate_identity}
    Let \(\algebra{A}\) be a unital C*-algebra. If \(\algebra{I} \subset \algebra{A}\) is a right ideal, then there exists an approximate identity on \(\algebra{A}\) by elements of \(\algebra{I}\).
\end{theorem}
\begin{proof}
    Consider the set
    \begin{equation*}
        \Lambda = \setc{\lambda \in \mathbb{P}(\algebra{I})}{\lambda\text{ is finite}}
    \end{equation*}
    partially ordered by inclusion, that is, \(\lambda' \preceq \lambda\) if \(\lambda' \subset \lambda \subset \algebra{I}\). It is clear that \((\Lambda, \preceq)\) is a directed set, since for all \(\lambda, \lambda' \in \Lambda\) we have \(\lambda \preceq \lambda \cup \lambda'\) and \(\lambda' \preceq \lambda \cup \lambda'\), with \(\lambda \cup \lambda'\) finite.

    We consider the net \(f\) on \(\algebra{I}\) based on \(\Lambda\) by
    \begin{equation*}
        f_{\lambda} = \sum_{u \in \lambda} uu^*,
    \end{equation*}
    which clearly lies in \(\algebra{I}\) since it is a right ideal. By \cref{thm:positive_cstar}, for all \(a \in \algebra{A}\) the operator \(aa^*\) is positive,
    \begin{equation*}
        aa^* = (a^*)^*(a^*) \in \algebra{A}_+,
    \end{equation*}
    then \(f_{\lambda}\) is a finite sum of positive elements of \(\algebra{A}\), hence positive. Thus we conclude \(\unity + \nu f_{\lambda} \in \invertible{\algebra{A}}\) and \((\unity + \nu f_{\lambda})^{-1} \in \algebra{A}_+\) for all \(\nu \in \mathbb{R}_+\) by \cref{prop:positive_resolvent}. In particular, we consider \(\abs{\lambda} \in \mathbb{R}^+\), where \(\abs{\lambda} \in \mathbb{N}_0\) denotes the number of elements of \(\lambda \in \Lambda\), and define the net
    \begin{align*}
        e : \Lambda &\to \algebra{I}\\
            \lambda &\mapsto \abs{\lambda} f_{\lambda} (\unity + \abs{\lambda} f_{\lambda})^{-1}
    \end{align*}
    with \(e_{\lambda} \in \algebra{A}_+\) for all \(\lambda \in \Lambda\) by \cref{lem:positive_commute,prop:resolvent_commute}. Notice
    \begin{equation*}
        e_{\lambda} = (\unity + \abs{\lambda} f_{\lambda} - \unity)(\unity + \abs{\lambda}f_{\lambda})^{-1} = \unity - (\unity + \abs{\lambda} f_{\lambda})^{-1},
    \end{equation*}
    then \(\unity - e_{\lambda} \in \algebra{A}_+\), that is, \(e_{\lambda} \leq \unity\), which yields \(\norm{e}_{\lambda} \leq 1\) by \cref{prop:properties_order}. If \(\lambda \succeq \lambda'\), then \(f_{\lambda} \geq f_{\lambda'}\),  since we have
    \begin{equation*}
        f_{\lambda} - f_{\lambda'} = \sum_{u \in \lambda \setminus \lambda'} uu^* \geq 0,
    \end{equation*}
    which yields
    \begin{equation*}
        e_{\lambda} - e_{\lambda'} = \left[\unity - (\unity + \abs{\lambda} f_{\lambda})^{-1}\right] - \left[\unity - (\unity + \abs{\lambda'} f_{\lambda'})^{-1}\right] = (\unity + \abs{\lambda'}f_{\lambda'})^{-1} - (\unity + \abs{\lambda}f_{\lambda})^{-1} \geq 0
    \end{equation*}
    by \cref{prop:positive_resolvent}, that is, \(e_{\lambda} \geq e_{\lambda'}\).

    Let \(a \in \algebra{I}\), then
    \begin{equation*}
    (a - e_{\lambda}a)(a - e_{\lambda}a)^* = (\unity - e_{\lambda})a a^* (\unity - e_{\lambda})^* = (\unity + \abs{\lambda}f_{\lambda})^{-1}aa^* (\unity + \abs{\lambda}f_{\lambda})^{-1}
    \end{equation*}
    for all \(\lambda \in \Lambda\). For all \(\lambda \succeq \set{a} \in \Lambda\), we have \(f_{\lambda} \geq f_{\set{a}} = aa^*\) and \(\abs{\lambda} \in \mathbb{N}\), therefore \cref{prop:congruence_order} yields
    \begin{align*}
        (a - e_{\lambda}a)(a - e_{\lambda}a)^* &\leq (\unity + \abs{\lambda}f_{\lambda})^{-1} f_{\lambda} (\unity + \abs{\lambda}f_{\lambda})^{-1}\\
                                               &= \frac{1}{\abs{\lambda}} (\unity + \abs{\lambda}f_{\lambda})^{-1} (\unity + \abs{\lambda}f_{\lambda} - \unity) (\unity + \abs{\lambda}f_{\lambda})^{-1}\\
                                               &= \frac{1}{\abs{\lambda}} \left[\unity - (\unity + \abs{\lambda}f_{\lambda})^{-1}\right](\unity + \abs{\lambda}f_{\lambda})^{-1} = \frac{1}{\abs{\lambda}} g_{\lambda},
    \end{align*}
    where \(g_{\lambda} = \left[\unity - (\unity + \abs{\lambda}f_{\lambda})^{-1}\right](\unity + \abs{\lambda}f_{\lambda})^{-1}\). \cref{prop:properties_order} ensures
    \begin{equation*}
        \norm{a - e_{\lambda}a}^2 = \norm{(a - e_{\lambda}a)(a - e_{\lambda}a)^*} \leq \frac{1}{\abs{\lambda}} \norm{g_{\lambda}}
    \end{equation*}
    for all \(\lambda \succ \set{a}\). \cref{thm:spectral_mapping} yields
    \begin{equation*}
        \sigma(g_{\lambda}) = \setc*{\left[1 - \frac{1}{1 + \abs{\lambda}z}\right]\frac{1}{1 + \abs{\lambda}z}}{z \in \sigma(f_{\lambda})} \subset \setc*{\frac{\abs{\lambda}z}{(1 + \abs{\lambda}z)^2}}{z \in \mathbb{R}^+} = \setc{\xi(\abs{\lambda}z)}{z \in \mathbb{R}_+},
    \end{equation*}
    where we defined the real function \(\xi(x) = \frac{x}{(1 + x)^2}\) for \(x\in\mathbb{R}_+\), with range in \(\mathbb{R}_+\). Notice \(\xi'(x) = \frac{1 - x}{(1 + x)^3}\), then \(\xi'\) changes sign from positive to negative at \(x = 1\), hence it is a local maximum and \(\xi\) is decreasing in \((1, \infty)\). That is, \(\xi(1) = \frac14\) is a global maximum of \(\xi\) and we conclude \(\norm{g_{\lambda}} \leq \frac14\) for all \(\lambda \succ \set{a}\) by \cref{thm:spectral_radius_cstar}. We have thus shown
    \begin{equation*}
        \norm{a - e_{\lambda}a}^2 \leq \frac{1}{4\abs{\lambda}}
    \end{equation*}
    for all \(\lambda \succ \set{a}\), that is, \(\lim_{\lambda}\norm{a - e_{\lambda}a} = 0\).
\end{proof}
\begin{corollary}
    Let \(\algebra{A}\) be a C*-algebra. Then there exists an approximate identity on \(\algebra{A}\).
\end{corollary}
\begin{proof}
    We may assume \(\algebra{A}\) has no identity, then we adjoin one. \cref{thm:adjoin_unity} guarantees \(\algebra{A}\) is isometrically *-isomorphic to the *-ideal \(\pi(\algebra{A})\) of \(\mathbb{C} \ltimes \algebra{A}\). Let \(e : \Lambda \to \pi(\algebra{A})\) be an approximate identity on \(\mathbb{C} \ltimes \algebra{A}\) by elements of \(\pi(\algebra{A})\), then \(\pi^{-1} \circ e : \Lambda \to \algebra{A}\) is an approximate identity on \(\algebra{A}\).
\end{proof}

\begin{lemma}{Norm of elements of an approximate identity in a unital C*-algebra}{norm_approximant}
    Let \(\algebra{A}\) be a unital C*-algebra and let \(\algebra{I} \subset \algebra{A}\) be a right ideal. If \(e : \Lambda \to \algebra{I}\) is an approximate identity on \(\algebra{A}\) by elements of \(\algebra{I}\), then \(\norm{\unity - e_{\lambda}} \leq 1\) for all \(\lambda \in \Lambda\).
\end{lemma}
\begin{proof}
    For all \(\lambda \in \Lambda\), the spectrum of the positive operator \(e_{\lambda}\) lies in the interval \([0,1]\) by the \cref{thm:spectral_radius_cstar}. Since \(\unity - e_{\lambda}\) is self-adjoint, we have
    \begin{equation*}
        \norm{\unity - e_{\lambda}} = r(\unity - e_{\lambda}) = \sup_{x \in \sigma(e_{\lambda})}{\abs{1 - x}} \leq \sup_{x \in [0,1]}{\abs{1 - x}} = 1,
    \end{equation*}
    by the \nameref{thm:spectral_mapping}.
\end{proof}

We recall the construction of \cref{thm:adjoin_unity} and the isometric *-isomorphism \(\pi : \algebra{A} \to \mathbb{C} \ltimes \algebra{A}\).
\begin{lemma}{Approximate identity in the C*-algebra with an adjoined identity}{approximate_identity_adjoin}
    Let \(\algebra{A}\) be a C*-algebra, \(\algebra{I}\) a closed two-sided ideal of \(\algebra{A}\) and \(e : \Lambda \to \algebra{I}\) an approximate identity on \(\algebra{A}\) by elements in \(\algebra{I}\), where \((\Lambda, \preceq)\) is a directed set. Then \(\pi(\algebra{I})\) is a closed two-sided ideal of \(\mathbb{C} \ltimes \algebra{A}\) and \(\pi \circ e\) is an approximate identity on \(\mathbb{C} \ltimes \algebra{A}\) by elements of \(\pi(\algebra{I})\).
\end{lemma}
\begin{proof}
    It is clear that \(\pi(\algebra{I})\) is a linear subspace of \(\mathbb{C}\ltimes \algebra{A}\) since \(\pi\) is linear. Let \((\alpha, a) \in \mathbb{C} \ltimes \algebra{A}\) and \(b \in \algebra{I}\), then
    \begin{equation*}
        (\alpha, a)\cdot (0, b) = (0, \alpha b + ab) \in \pi(\algebra{I})
        \quad\text{and}\quad
        (0, b)\cdot (\alpha, a) = (0, \alpha b + ba) \in \pi(\algebra{I}),
    \end{equation*}
    hence \(\pi(\algebra{I})\) is a two-sided ideal. Let \(\xi : \mathbb{N} \to \pi(\algebra{I})\) is a converging sequence to some \(\tilde{\xi} \in \mathbb{C} \ltimes \algebra{A}\), then there exists a unique sequence \(x : \mathbb{N} \to \algebra{I}\) such that \(\xi = \pi\circ x\), since \(\pi\) bijectively associates \(\algebra{I}\) with \(\pi(\algebra{I})\). As \(\xi\) is Cauchy and \(\pi\) is an isometry, then \(x\) is Cauchy, hence convergent to some \(\tilde{x} \in \algebra{I}\), as the ideal is a C*-subalgebra. Then \(\tilde{\xi} = \pi(\tilde{x})\) since for all \(n \in \mathbb{N}\) we have
    \begin{equation*}
        \norm{\xi_n - \pi(\tilde{x})} = \norm{\pi(x_n - \tilde{x})} = \norm{x_n - \tilde{x}},
    \end{equation*}
    and we conclude \(\pi(\algebra{I})\) is a closed two-sided ideal.

    Let \(\lambda \in \Lambda\), then \(e_{\lambda} \in \algebra{A}_+\) and there must exist \(a \in \algebra{A}\) such that \(e_{\lambda} = a^*a\), and as \(\pi\) is a *-isomorphism, we have
    \begin{equation*}
        \pi(e_{\lambda}) = \pi(a^*a) = \pi(a^*) \pi(a) = \pi(a)^* \pi(a) \in (\mathbb{C} \ltimes \algebra{A})_+,
    \end{equation*}
    hence \(\pi \circ e\) is a net of positive elements of \(\mathbb{C} \ltimes \algebra{A}\). As \(\pi\) is an isometry, it is clear that \(\norm{\pi(e_{\lambda})} = \norm{e_{\lambda}} \leq 1\) for all \(\lambda \in \Lambda\). Let \(\lambda, \lambda' \in \Lambda\), then
    \begin{align*}
        \lambda \succeq \lambda' &\implies e_{\lambda} \geq e_{\lambda'}\\
                                 &\implies e_{\lambda} - e_{\lambda'} \in \algebra{A}_+\\
                                 &\implies \exists b \in \algebra{A} : b^*b = e_{\lambda} - e_{\lambda'}\\
                                 &\implies \exists b \in \algebra{A} : \pi(e_{\lambda}) - \pi(e_{\lambda'}) = \pi(b)^*\pi(b)\\
                                 &\implies \pi(e_{\lambda}) - \pi(e_{\lambda'}) \in (\mathbb{C}\ltimes \algebra{A})_+\\
                                 &\implies \pi(e_{\lambda}) \geq \pi(e_{\lambda'}).
    \end{align*}
    Let \((0,c) \in \pi(\algebra{I})\), then
    \begin{equation*}
        \lim_{\lambda}{\norm{(0,c) - \pi(e_{\lambda})(0,c)}} = \lim_{\lambda}{\norm{\pi(c) - \pi(e_{\lambda})\pi(c)}} = \lim_{\lambda}{\norm{\pi(c - e_{\lambda}c)}} = \lim_{\lambda}{\norm{c - e_{\lambda}c}} = 0,
    \end{equation*}
    and we conclude \(\pi \circ e\) is an approximate identity.
\end{proof}

\subsection{Cosets by closed two-sided ideals in C*-algebras}
If \(\algebra{A}\) is a C*-algebra and \(\algebra{I}\) is a closed *-ideal, we have already established in \cref{thm:closed_ideal_Bstar} that the coset \(\algebra{A} / \algebra{I}\) is a Banach *-algebra. We now show that if \(\algebra{I}\) is a closed two-sided ideal, then the coset \(\algebra{A}/\algebra{I}\) is a C*-algebra.

\begin{proposition}{Closed two-sided ideal of a C*-algebra is a *-ideal}{closed_bi_ideal_cstar}
    Let \(\algebra{A}\) be a C*-algebra. Let \(\algebra{I} \subset \algebra{A}\) be a two-sided ideal that is closed with respect to the uniform topology in \(\algebra{A}\). Then \(\algebra{I}\) is self-adjoint.
\end{proposition}
\begin{proof}
    By \cref{thm:approximate_identity}, there exists an approximate identity \(e : \Lambda \to \algebra{I}\) on \(\algebra{A}\) by elements of \(\algebra{I}\), where \((\Lambda, \preceq)\) is a directed set. Let \(j \in \algebra{I}\), then for all \(\lambda \in \Lambda\) we have
    \begin{equation*}
        \lim_{\lambda}\norm{j^* - j^*e_{\lambda}} = \lim_{\lambda} \norm{(j - e_{\lambda}j)^*} = \lim_{\lambda} \norm{j - e_{\lambda}j} = 0,
    \end{equation*}
    hence \(j^* = \lim_{\lambda} j^* e_{\lambda}\). Since \(\algebra{I}\) is a two-sided ideal, then for all \(\lambda \in \Lambda\) the operator \(j^* e_{\lambda}\) lies in \(\algebra{I}\), hence the net \(f : \Lambda \to \algebra{I}\) defined by \(\lambda \mapsto j^* e_{\lambda}\) converges to \(j^*\), that is, \(j^* \in \algebra{I}\) since the ideal is closed.
\end{proof}
\begin{remark}
    We may conclude a closed two-sided ideal is a C*-subalgebra of a C*-algebra.
\end{remark}


We need only show the coset \(\algebra{A}/\algebra{I}\) has the C* property, but first we have to show the
\begin{lemma}{Norm of an element of the coset of a C*-algebra and a closed two sided ideal}{norm_coset}
    Let \(\algebra{A}\) be a C*-algebra, \(\algebra{I}\) a closed two-sided ideal of \(\algebra{A}\) and \(e : \Lambda \to \algebra{I}\) an approximate identity of elements in \(\algebra{I}\), with \((\Lambda, \preceq)\) a directed set. Then
    \begin{equation*}
        \norm{[a]} = \lim_{\lambda}{\norm{a - e_{\lambda} a}}= \lim_{\lambda}{\norm{a - a e_{\lambda}}}
    \end{equation*}
    for all \(a \in \algebra{A}\).
\end{lemma}
\begin{proof}
    We may assume without loss of generality that \(\algebra{A}\) is a unital C*-algebra by \cref{lem:approximate_identity_adjoin}. Let \(a \in \algebra{A}\), then for all \(\varepsilon > 0\) there exists \(j_{\varepsilon} \in \algebra{I}\) such that \(\norm{[a]} \leq \norm{a + j_{\varepsilon}} \leq \norm{[a]} + \frac12\varepsilon\), and there also exists \(\lambda_{\varepsilon} \in \Lambda\) such that \(\norm{j_{\lambda_{\varepsilon}} - e_{\lambda_{\varepsilon}}j_{\varepsilon}} \leq \frac12\varepsilon\). Notice
    \begin{equation*}
        a - e_{\lambda_{\varepsilon}}a = j_{\varepsilon} + a - (j_{\varepsilon} - e_{\lambda_{\varepsilon}}j_{\varepsilon}) - (e_{\lambda_{\varepsilon}}j_{\varepsilon} + e_{\lambda_{\varepsilon}}a) = (\unity - e_{\lambda_{\varepsilon}})(j_{\epsilon} + a) - (j_{\varepsilon} - e_{\lambda_{\varepsilon}} j_{\varepsilon}),
    \end{equation*}
    therefore
    \begin{align*}
        \norm{a - e_{\lambda_{\varepsilon}}a} &\leq \norm{(\unity - e_{\lambda_{\varepsilon}})(a + j_{\lambda_{\varepsilon}})} + \norm{j_{\varepsilon} - e_{\lambda_{\varepsilon}}j_{\varepsilon}}\\
                                              &\leq \norm{\unity - e_{\lambda_{\varepsilon}}}\norm{a + j_{\varepsilon}} + \frac12 \varepsilon\\
                                              &\leq \norm{a + j_{\varepsilon}} + \frac12 \varepsilon\\
                                              &\leq \norm{[a]} + \varepsilon
    \end{align*}
    by \cref{lem:norm_approximant}. Since \(-e_{\lambda_{\varepsilon}}a \in \algebra{I}\), we have shown \(\norm{[a]} \leq \norm{a - e_{\lambda_{\varepsilon}}a} \leq \norm{[a]} + \varepsilon\), hence we conclude the limit \(\norm{[a]} = \lim_{\lambda}{\norm{a - e_{\lambda}a}}\). Notice
    \begin{equation*}
        \lim_{\lambda}{\norm{a - a e_{\lambda}}}= \lim_{\lambda}{\norm{a^* - e_{\lambda}a^*}} = \norm{[a^*]} = \norm{[a]^*} = \norm{[a]}
    \end{equation*}
    for all \(a \in \algebra{A}\), concluding the proof.
\end{proof}

\begin{theorem}{Coset of a C*-algebra by a closed two-sided ideal is a C*-algebra}{coset_cstar}
    Let \(\algebra{A}\) be a C*-algebra. If \(\algebra{I}\) is a closed two-sided ideal of \(\algebra{A}\), then the coset \(\algebra{A}/\algebra{I}\) is a C*-algebra.
\end{theorem}
\begin{proof}
    We adjoin an identity to \(\algebra{A}\), if necessary. Let \(a \in \algebra{A}\) and let \(e : \Lambda \to \algebra{I}\) be a approximate identity of \(\algebra{A}\) by elements of the ideal \(\algebra{I}\), then by \cref{lem:norm_approximant} we have
    \begin{equation*}
        \norm{[a]}^2 = \lim_{\lambda}{\norm{a - ae_{\lambda}}^2} = \lim_{\lambda}{\norm{(a - ae_{\lambda})^*(a - ae_{\lambda})}} = \lim_{\lambda}{\norm{(\unity - e_{\lambda})a^*a(\unity - e_{\lambda})}}.
    \end{equation*}
    For all \(j \in \algebra{I}\) we have
    \begin{align*}
        \norm{[a]}^2 &= \lim_{\lambda}{\norm{(\unity - e_{\lambda})(a^*a + j)(\unity - e_{\lambda}) - (\unity - e_{\lambda})j(\unity - e_{\lambda})}}\\
                     &\leq \norm{\unity - e_{\lambda}}^2\lim_{\lambda}{\left(\norm{a^*a + j} + \norm{j}\right)}\\
                     &\leq \lim_{\lambda}{\norm{a^*a + j}},
    \end{align*}
    that is, \(\norm{[a]^2} \leq \inf_{j \in \algebra{I}}{\norm{a^*a + j}} = \norm{[a^*a]}\), and we conclude the coset is a C*-algebra.
\end{proof}

% vim: spl=en_us
\section{An introduction to von Neumann algebras}
In this section, we consider the C*-algebra of bounded operators \(\bounded(\hilbert)\) on a Hilbert space \(\hilbert\).
\begin{definition}{Commutant of a set of operators}{}
    Let \(\hilbert\) be a Hilbert space. The \emph{commutant} \(\algebra{M}'\) of a subset \(\algebra{M} \subset \bounded(\hilbert)\) is the set
    \begin{equation*}
        \algebra{M}' = \setc{b \in \bounded(\hilbert)}{\forall a \in \algebra{M} : ab = ba}.
    \end{equation*}
    The \emph{bicommutant} \(\algebra{M}''\) of \(\algebra{M}\) is the commutant of \(\algebra{M}'\).
\end{definition}
\begin{remark}
    If \(\algebra{M} \subset \algebra{N} \subset \bounded(\hilbert)\) are sets, then
    \begin{equation*}
        b \in \algebra{N}' \implies \forall a \in \algebra{N} : ab = ba \implies \forall a \in \algebra{M} : ab = ba \implies b \in \algebra{M}',
    \end{equation*}
    that is, \(\algebra{N}' \subset \algebra{M}'\).
\end{remark}

\begin{proposition}{Commutant is a Banach subalgebra}{commutant_Banach}
    Let \(\hilbert\) be a Hilbert space and \(\algebra{M} \subset \bounded(\hilbert)\) a subset of bounded operators. The commutant \(\algebra{M}'\) is a unital Banach subalgebra of \(\bounded(\hilbert)\). If \(\algebra{M}\) is self-adjoint, then \(\algebra{M}'\) is a unital C*-subalgebra of \(\bounded(\hilbert)\).
\end{proposition}
\begin{proof}
    The commutant of the subset \(\algebra{M}\) of \(\bounded(\hilbert)\) is a subalgebra since if \(\algebra{M}\) is the empty set, then its commutant is \(\bounded(\hilbert)\), and if it is not empty, then for all \(\lambda \in \mathbb{C}\) and all \(a,b \in \algebra{M}'\), then
    \begin{equation*}
        c(ab) = acb = (ab)c,\quad
        c(a+b) = ca + cb = ac + bc = (a+b)c,\quad\text{and}\quad
        c(\lambda a) = \lambda ca = (\lambda a)c,
    \end{equation*}
    for all \(c \in \algebra{M}\). Moreover, it is trivial to check \(\algebra{M}'\) contains the identity.

    Let \(x : \mathbb{N} \to \algebra{M}'\) be a convergent sequence to some \(\tilde{x} \in \bounded(\hilbert)\). If \(a \in \algebra{M}\), then for all \(n \in \mathbb{N}\) we have
    \begin{align*}
        \norm{\tilde{x}a - a\tilde{x}} &\leq \norm{(\tilde{x} - x_n)a - a(\tilde{x} - x_n) + x_na - a x_n} \\
                                       &\leq \norm{a(\tilde{x} - x_n)} + \norm{(\tilde{x}- x_n)a} \\
                                       &\leq 2\norm{a} \norm{\tilde{x} - x_n},
    \end{align*}
    therefore \(\norm{\tilde{x} a - a\tilde{x}} = 0\), that is, \(\tilde{x} \in \algebra{M}'\). As a closed subalgebra of a complete algebra, we have shown \(\algebra{M}'\) is a unital Banach subalgebra of \(\bounded(\hilbert)\).

    Suppose \(\algebra{M}\) is self-adjoint and let \(b \in \algebra{M}'\), then
    \begin{align*}
        b \in \algebra{M}' \implies \forall a \in \algebra{M} : a^*b = ba^* \implies \forall a \in \algebra{M} : b^*a = ab^* \implies b^* \in \algebra{M}',
    \end{align*}
    that is, the commutant \(\algebra{M}'\) is self-adjoint, therefore a C*-subalgebra of \(\bounded(\hilbert)\).
\end{proof}

\begin{proposition}{Higher order commutant of a set}{motivation_von_neumann}
    Let \(\hilbert\) be a Hilbert space and \(\algebra{M} \subset \bounded(\hilbert)\) a subset of bounded operators. For \(n \in \mathbb{N}\), let us briefly denote the commutant of \(\algebra{M}^{(n)}\) as \(\algebra{M}^{(n+1)}\), where \(\algebra{M}^{(1)} = \algebra{M}'\). Then
    \begin{enumerate}[label=(\alph*)]
        \item \(\algebra{M} \subset \algebra{M}'',\)
        \item \({(\algebra{M}^{(n)})}^{(m)} = \algebra{M}^{(n+m)},\)
        \item \(\algebra{M}^{(2n)} = \algebra{M}^{(2m)},\) and
        \item \(\algebra{M}^{(2n+1)} = \algebra{M}^{(2m+1)}\),
    \end{enumerate}
    for all \(n, m \in \mathbb{N}\).
\end{proposition}
\begin{proof}
    It is clear a set is contained in its bicommutant, as \(a \in \algebra{M}\) commutes with every operator of \(\algebra{M}'\), hence \(a \in \algebra{M}''\) and we conclude (a).

    Let \(n \in \mathbb{N}\), and we show (b) by induction on \(m \in \mathbb{N}\). First \({(\algebra{M}^{(n)})}^{(1)} = {(\algebra{M}^{(n)})}' = \algebra{M}^{(n+1)}\) follows by definition. Suppose it holds for some \(m \in \mathbb{N}\), then
    \begin{equation*}
        {(\algebra{M}^{(n)})}^{(m+1)} = {\left({(\algebra{M}^{(n)})}^{(m)}\right)}' = {(\algebra{M}^{(n+m)})}' = \algebra{M}^{n+m+1},
    \end{equation*}
    hence it also holds for \(m + 1\). By the principle of finite induction, it holds for all \(m \in \mathbb{N}\) and we conclude (b) since \(n\) is arbitrary.

    In order to show (c) and (d) we must only prove \(\algebra{M}' = \algebra{M}^{(3)}\), since by (b) this would imply \(\algebra{M}^{(k)} = \algebra{M}^{(k + 2)}\) for all \(k \in \mathbb{N}\). Taking the commutant of (a) yields \(\algebra{M}^{(3)} \subset \algebra{M}'\), and since \(\algebra{M}^{(3)}\) is the bicommutant of \(\algebra{M}'\), we have from (a) that \(\algebra{M}' \subset \algebra{M}^{(3)}\), concluding the proof.
\end{proof}

\begin{definition}{von Neumann algebras}{von_neumann}
    A \emph{von Neumann algebra \(\algebra{M}\) on a Hilbert space \(\hilbert\)} is a *-subalgebra \(\algebra{M}\) of \(\bounded(\hilbert)\) such that \(\algebra{M}'' = \algebra{M}\). The \emph{center of a von Neumann algebra \(\algebra{M}\)} is the intersection \(\mathfrak{Z}(\algebra{M}) = \algebra{M} \cap \algebra{M}'\). A von Neumann algebra \(\algebra{M}\) is a \emph{factor} if its center is trivial, that is, \(\mathfrak{Z}(\algebra{M}) = \mathbb{C} \unity\), where \(\mathbb{C} \unity\) denotes the algebra generated by the identity.
\end{definition}

\begin{proposition}{Trivial factors}{trivial_factors}
    Let \(\hilbert\) be a Hilbert space. Then \(\bounded(\hilbert)\) and \(\mathbb{C} \unity\) are factors with \(\bounded(\hilbert)' = \mathbb{C} \unity\) and \(\mathbb{C} \unity' = \bounded(\hilbert)\).
\end{proposition}
\begin{proof}
    From \cref{prop:motivation_von_neumann}, we have \(\bounded(\hilbert) \subset \bounded(\hilbert)'' \subset \bounded(\hilbert)\), hence \(\bounded(\hilbert)\) is a von Neumann algebra. As \(\bounded(\hilbert)\) is self-adjoint, then its commutant \(\bounded(\hilbert)'\) is self-adjoint and contains \(\mathbb{C} \unity\) by \cref{prop:commutant_Banach}. For \(\phi \in \hilbert\) with \(\norm{\phi} = 1\), let us denote by \(p_{\phi} \in \bounded(\hilbert)\) the orthogonal projector onto the linear subspace spanned by \(\phi\). If \(a \in \bounded(\hilbert)'\), then for all unitary vectors \(\phi \in \hilbert\) we have \(a p_\phi = p_\phi a\) and \(a^* p_\phi = p_\phi a^*\), hence
    \begin{equation*}
        \inner{\phi}{a \psi}\phi = p_\phi a\psi = ap_\phi \psi  = \inner{\phi}{\psi}a\phi
    \end{equation*}
    for all \(\psi \in \hilbert\). In the particular case \(\psi = \phi\), this yields \(a\phi = \inner{\phi}{a\phi}\phi\) and, analogously, \(a^*\phi = \inner{\phi}{a^*\phi}\phi\).
\end{proof}

% vim: spl=en_us
\section{Representations of C*-algebras}
We'll refer to *-homomorphisms between involutive algebras simply as *-morphisms. We show now a *-(iso)morphism preserves the order relation defined by positivity and that a *-morphism is always continuous with respect to the uniform topology, with a *-isomorphism even being a isometry.
\begin{proposition}{*-morphism preserves positivity}{morphism_positivity}
    Let \(\pi : \algebra{A} \to \algebra{B}\) be a *-morphism between C*-algebras \(\algebra{A}\) and \(\algebra{B}\). Then \(\pi\) preserves positivity, \(\pi(\algebra{A}_+)\subset \algebra{B}_+\) and it is continuous with respect to the uniform topology, enjoying the property \(\norm{\pi(a)} \leq \norm{a}\) for all \(a \in \algebra{A}\).
\end{proposition}
\begin{remark}
    It is clear that \(\pi(\algebra{A}_+) \subset \algebra{B}_+\) implies that if \(a \geq b\) then \(\pi(a) \geq \pi(b)\) as \(\pi\) is linear and \(\pi(a-b) \in \algebra{B}_+\).
\end{remark}
\begin{proof}
    Let \(a \in \algebra{A}_+\), then there exists \(b \in \algebra{A}\) such that \(a = b^*b\), hence \(\pi(a) = \pi(b)^* \pi(b) \in \algebra{B}_+\), that is, \(\pi(\algebra{A}_+)\subset \algebra{B}_+\).

    It is clear that if \(a \in \ker\pi\), then \(\norm{\pi(a)} \leq \norm{a}\). Recall \cref{prop:properties_order,thm:positive_cstar}, then for all \(a \in \algebra{A}\setminus\ker{\pi}\) we have
    \begin{align*}
        \norm{a^*a} a^*a \geq (a^*a)^2 &\implies \norm{a^*a} \pi(a^*a) \geq \pi(a^*a)^2\\
                                       &\implies \norm{a}^2 \pi(a^*a) \geq \pi(a^*a)^2\\
                                       &\implies \norm*{\norm{a}^2 \pi(a^*a)} \geq \norm*{\pi(a^*a)^2}\\
                                       &\implies \norm{a}^2 \norm{\pi(a^*a)} \geq \norm*{\pi(a^*a)^* \pi(a^*a)}\\
                                       &\implies \norm{a}^2 \norm{\pi(a^*a)} \geq \norm*{\pi(a^*a)}^2\\
                                       &\implies \norm{a}^2 \geq \norm{\pi(a^*a)}\\
                                       &\implies \norm{a}^2 \geq \norm{\pi(a)^* \pi(a)}\\
                                       &\implies \norm{a}^2 \geq \norm{\pi(a)}^2\\
                                       &\implies \norm{a} \geq \norm{\pi(a)},
    \end{align*}
    that is, for all \(a \in \algebra{A}\) it is true that \(\norm{a} \geq \norm{\pi(a)}\). This is enough to guarantee \(\pi\) is continuous with respect to the uniform topology.
\end{proof}
\begin{corollary}
    If, in addition, \(\pi\) is bijective, then \(\pi\) is an isometry.
\end{corollary}
\begin{proof}
    As \(\pi\) and \(\pi^{-1}\) are *-isomorphisms, then
    \begin{equation*}
        \norm{\pi(a)} \leq \norm{a} = \norm{\pi^{-1}\circ \pi(a)} \leq \norm{\pi(a)},
    \end{equation*}
    that is, \(\norm{\pi(a)} = \norm{a}\) for all \(a \in \algebra{A}\).
\end{proof}

It is an elementary result that the kernel of a group morphism \(\pi : G \to H\) is a normal subgroup of \(G\), and that the coset \(G / \ker\pi\) is isomorphic to the subgroup \(\ran\pi \subset H\). We will now show a similar result with the additional structure of C*-algebras.
\begin{proposition}{Kernel of a *-morphism is a closed two-sided ideal}{kernel_ideal}
    Let \(\pi : \algebra{A} \to \algebra{B}\) be a *-morphism between the C*-algebras \(\algebra{A}\) and \(\algebra{B}\). Then \(\ker \pi\) is a closed two-sided ideal of \(\algebra{A}\).
\end{proposition}
\begin{proof}
    As \(\pi\) is continuous, we know from \cref{prop:kernel_closed} that its kernel is a linear subspace that is closed in the uniform topology. Let \(a \in \ker\pi\) and \(b \in \algebra{A}\), then \(\pi(ab) = \pi(a)\pi(b) = 0\pi(b) = 0\) and \(\pi(ba) = \pi(b)\pi(a) = 0\), hence \(\ker\pi\) is a two-sided ideal.
\end{proof}
\begin{lemma}{Projector in a Banach *-algebra defines a unital Banach *-subalgebra}{projector_star_subalgebra}
    Let \(\algebra{A}\) be a Banach *-algebra. If \(p \in \algebra{A}\) is an orthogonal projector, then
    \begin{equation*}
        \algebra{B} = \setc{b \in \algebra{A}}{\exists a \in \algebra{A} : b = pap}
    \end{equation*}
    is a Banach *-algebra. If \(p \neq 0\), then \(p\) is the identity element for \(\algebra{B}\).
\end{lemma}
\begin{proof}
    We may assume without loss of generality that \(p \neq 0\), for if it were, then \(\algebra{B} = \set{0}\) is obviously a Banach *-algebra. The set \(\algebra{B}\) is not empty as it clearly contains at least \(0\) and \(p\). Let \(b_1,b_2 \in \algebra{B}\) and \(\alpha \in \mathbb{C}\), then there exist \(a_1,a_2 \in \algebra{A}\) such that \(b_i = pa_ip\), which yields
    \begin{equation*}
        b_1 + \alpha b_2 = p(a_1 + \alpha a_2)p \in \algebra{B},
        \quad
        b_1 b_2 = pa_1 p^2 a_2 p \in \algebra{B},
        \quad\text{and}\quad
        b_1^* = (pa_1p)^* = pa_1^*p \in \algebra{B},
    \end{equation*}
    hence \(\algebra{B}\) is a *-subalgebra of \(\algebra{A}\). Since \(p\) is idempotent, then \(p = p^3 \in \algebra{B}\) and it is the identity element of \(\algebra{B}\) as
    \begin{equation*}
        pb = p^2ap = pap = b
        \quad\text{and}\quad
        bp = pap^2 = pap = b
    \end{equation*}
    for all \(b = pap \in \algebra{B}\).

    Let \(x : \mathbb{N} \to \algebra{B}\) be a sequence of elements in \(\algebra{B}\) that converges to some \(\tilde{x} \in \algebra{A}\). \cref{prop:norm_projector} guarantees \(\norm{p} = 1\), then
    \begin{equation*}
        \norm{x_n - p\tilde{x}p} = \norm{px_np - p\tilde{x}p} =
        \norm{p(x_n - \tilde{x})p} \leq \norm{p}^2 \norm{x_n - \tilde{x}} = \norm{x_n - \tilde{x}},
    \end{equation*}
    for all \(n \in \mathbb{N}\), hence \(\tilde{x} = p \tilde{x} p \in \algebra{B}\). As a closed *-subalgebra of \(\algebra{A}\), \(\algebra{B}\) is a unital Banach *-algebra.
\end{proof}

\begin{lemma}{*-morphism maps identity to an orthogonal projector}{morphism_projector}
    Let \(\algebra{A}\) be a unital *-algebra and \(\algebra{B}\) a *-algebra. If \(\pi : \algebra{A} \to \algebra{B}\) is a *-morphism, then the following statements hold
    \begin{enumerate}[label=(\alph*)]
        \item \(p = \pi(\unity_{\algebra{A}})\) is an orthogonal projector;
        \item \(p = 0\) if and only if \(\ker{\pi} = \algebra{A}\);
        \item \(\ran{\pi}\) is a *-subalgebra contained in \(\algebra{C} = \setc{c \in \algebra{B}}{\exists b \in \algebra{B} : c = pbp}\).
    \end{enumerate}
\end{lemma}
\begin{proof}
    Notice
    \begin{equation*}
        p = \pi(\unity_\algebra{A}) = \pi(\unity_\algebra{A}^2) = \pi(\unity_\algebra{A})^2 = p^2
    \end{equation*}
    \begin{equation*}
        p^* = \pi(\unity_\algebra{A})^* = \pi(\unity_\algebra{A}^*) = \pi(\unity_\algebra{A}) = p,
    \end{equation*}
    and
    \begin{equation*}
        p = 0 \iff \forall a \in \algebra{A}: p \pi(a) = 0 \iff a \in \algebra{A} : \pi(a) = 0 \iff \ker{\pi} = \algebra{A}
    \end{equation*}
    hence \(p\) is an orthogonal projector and it is the zero operator if and only if \(\pi\) is identically zero. Clearly \(\ran{\pi}\) is a *-subalgebra of \(\algebra{B}\), so it remains to show it is contained in \(\algebra{C}\). Let \(b \in \ran\pi\), then there exists \(a \in \algebra{A}\) such that \(b = \pi(a)\). As such, we have \(b = \pi(\unity_{\algebra{A}} a \unity_{\algebra{A}}) = p\pi(a)p = pbp \in \algebra{C}\), as desired.
\end{proof}

\begin{proposition}{*-morphism is non-expansive}{morphism_non_expansive}
    Let \(\algebra{A}\) be a *-subalgebra with identity \(\unity\) of a unital Banach *-algebra, and let \(\algebra{B}\) be a C*-algebra. If \(\pi : \algebra{A} \to \algebra{B}\) is a *-morphism, then \(\pi\) is non-expansive, that is, for all \(a \in \algebra{A}\) it is true that \(\norm{\pi(a)} \leq \norm{a}\).
\end{proposition}
\begin{proof}
    It is clear that \(\pi\) is non-expansive with respect to its kernel, so we may assume without loss of generality that \(\ker\pi \neq \algebra{A}\), then by \cref{lem:projector_star_subalgebra,lem:morphism_projector} we know \(\algebra{C} = \setc{c \in \algebra{B}}{\exists b \in \algebra{B}: c = \pi(\unity)b\pi(\unity)}\) is a C*-algebra with identity \(p = \pi(\unity)\) that contains the *-subalgebra \(\ran\pi\). By \cref{prop:norm_cstar_spectral_radius}, we know the restriction of the norm for \(\algebra{B}\) to the C*-subalgebra \(\algebra{C}\) is equal to the spectral radius norm, that is, \(\norm{c} = \sqrt{r_{\algebra{C}}(c^*c)}\) for all \(c \in \algebra{C}\), where the spectral radius is defined with the identity \(p\).

    Let \(a \in \algebra{A}\) be a self-adjoint operator, then
    \begin{equation*}
        \lambda \notin \sigma_{\algebra{A}}(a) \implies a - \lambda \unity \in \invertible{\algebra{A}} \implies \pi(a) - \lambda p \in \invertible{\algebra{C}} \implies \lambda \notin \sigma_{\algebra{C}}(\pi(a)),
    \end{equation*}
    that is, \(\sigma_{\algebra{C}}(\pi(a))\subset \sigma_{\algebra{A}}(a)\). \cref{thm:spectral_radius_cstar} then yields
    \begin{equation*}
        \norm{\pi(a)} = r_{\algebra{C}}(\pi(a)) \leq r_{\algebra{A}}(a) \leq \norm{a}.
    \end{equation*}
    If we now consider \(a \in \algebra{A}\) not self-adjoint, then
    \begin{equation*}
        \norm{\pi(a)}^2 = \norm{\pi(a)^*\pi(a)} = \norm{\pi(a^*a)} \leq \norm{a^*a} \leq \norm{a^*}\norm{a} = \norm{a}^2,
    \end{equation*}
    hence \(\norm{\pi(a)} \leq \norm{a}\).
\end{proof}
\begin{remark}
    This result shows that a *-morphism between C*-algebras is continuous with respect to the uniform topology.
\end{remark}



\begin{proposition}{Range of *-morphism is a C*-subalgebra of its codomain}{range_cstar}
    Let \(\pi : \algebra{A} \to \algebra{B}\) be a *-morphism between the unital C*-algebra \(\algebra{A}\) and the C*-algebra \(\algebra{B}\). Then \(\ran\pi\) is a C*-subalgebra of \(\algebra{B}\) and the map
    \begin{align*}
        \hat{\pi} : \algebra{A}/\ker\pi &\to \ran\pi\\
                                    [a] &\mapsto \pi(a)
    \end{align*}
    is a *-isomorphism.
\end{proposition}
\begin{proof}
    We claim the map \(\tilde{\pi} : \algebra{A}/\ker\pi \to \algebra{B}\) defined by \([a] \mapsto \pi(a)\) is well-defined, since if \(a \in \algebra{A}\) and \(\tilde{a} \in [a]\), then \(\tilde{a} - a \in \ker\pi\), hence \(\tilde{\pi}([\tilde{a}]) = \pi(\tilde{a}) = \pi(a) = \tilde{\pi}([a])\). Let \(a_1, a_2 \in \algebra{A}\) and \(\alpha \in \mathbb{C}\), then \(\tilde{\pi}([0]) = \pi(0) = 0\),
    \begin{equation*}
        \tilde{\pi}([a_1] + \alpha[a_2]) = \tilde{\pi}([a_1 + \alpha a_2]) = \pi(a_1 + \alpha a_2) = \pi(a_1) + \alpha \pi(a_2) = \tilde{\pi}([a_1]) + \alpha \tilde{\pi}([a_2]),
    \end{equation*}
    \begin{equation*}
        \tilde{\pi}([a_1][a_2]) = \tilde{\pi}([a_1a_2]) = \pi(a_1a_2) = \pi(a_1)\pi(a_2) = \tilde{\pi}([a_1])\tilde{\pi}([a_2]),
    \end{equation*}
    and
    \begin{equation*}
        \tilde{\pi}([a_1])^* = \pi(a_1)^* = \pi(a_1)^* = \pi(a_1^*) = \tilde{\pi}([a_1^*]) = \tilde{\pi}([a_1]^*),
    \end{equation*}
    thus showing \(\tilde{\pi}\) is a *-morphism.

    By construction \(\tilde{\pi}\) is injective, hence the map with the codomain restricted to its range, \(\hat{\pi} : \algebra{A}/\ker{\pi} \to \ran\pi\), is bijective, thus a *-isomorphism. In particular, the inverse map \(\hat{\pi}^{-1} : \ran\pi \to \algebra{A}/\ker{\pi}\) is a *-morphism from the *-subalgebra with identity \(\ran\pi\) of the unital Banach *-algebra \(\algebra{C} = \setc{\pi(\unity_{\algebra{A}})b\pi(\unity_{\algebra{A}})}{b \in \algebra{B}}\) to the C*-algebra \(\algebra{A}/\ker\pi\), hence \(\hat{\pi}^{-1}\) is non-expansive by \cref{prop:morphism_non_expansive}. As a result and since \(\tilde{\pi}\) is also non-expansive as a *-morphism between C*-algebras, we have
    \begin{equation*}
        \norm{[a]} = \norm{\hat{\pi}^{-1}\circ \hat{\pi}([a])} \leq \norm{\hat{\pi}([a])} = \norm{\tilde{\pi}([a])} \leq \norm{[a]},
    \end{equation*}
    for all \(a \in \algebra{A}\), that is, the map \(\hat{\pi}\) is an isometry. From \cref{prop:isometry_Banach}, \(\ran{\pi}\) is a C*-algebra.
\end{proof}

\begin{definition}{C*-algebra representation}{representation}
    Let \(\algebra{A}\) be a C*-algebra. A \emph{representation \((\hilbert, \pi)\) of \(\algebra{A}\) on the Hilbert space \(\hilbert\)} is a *-morphism \(\pi : \algebra{A} \to \bounded(\hilbert)\). If \(\pi\) is injective, the representation is \emph{faithful}.
\end{definition}
\begin{remark}
    In an abuse of language, we refer to the representation simply to the *-morphism.
\end{remark}


\begin{proposition}{Faithful representation is isometric}{faithful_isometric}
    Let \(\pi : \algebra{A} \to \algebra{B}\) be a *-morphism between C*-algebras \(\algebra{A}\) and \(\algebra{B}\). The following statements are equivalent
    \begin{enumerate}[label=(\alph*)]
        \item \(\pi\) is injective;
        \item \(\pi\) is an isometry;
        \item if \(a \in \algebra{A}_+ \setminus\set{0}\), then \(\pi(a) \in \algebra{B}_+\setminus\set{0}\).
    \end{enumerate}
\end{proposition}
\begin{proof}
    Suppose \(\pi\) is injective, then the map \(\pi^{-1} : \ran\pi \to \algebra{A}\) is a *-isomorphism, and so must be \(\pi : \algebra{A} \to \ran\pi\), hence \(\pi\) is isometric.

    Suppose \(\pi\) is an isometry, then if \(a > 0\), we have \(\pi(a) \in \algebra{B}_+\) with \(a \neq 0\), and as a result \(\norm{\pi(a)} = \norm{a} > 0\), hence \(\pi(a) \neq 0\).

    Suppose \(\pi\) is not injective, then there exists \(a \in \ker{\algebra{A}}\setminus\set{0}\) and as a result \(\pi(a^*a) = 0\). As \(\norm{a^*a} = \norm{a}^2 > 0\), we have \(a^*a > 0\) with \(\pi(a^*a) = 0\).
\end{proof}
\begin{definition}{Stable subspace under a representation}{stable_under_representation}
    Let \(\algebra{A}\) be a C*-algebra and \((\hilbert, \pi)\) a representation. A linear subspace \(\mathscr{V} \subset \hilbert\) is \emph{stable}, or \emph{invariant}, \emph{under the representation \(\pi\)} if \(\pi(\algebra{A})\mathscr{V} \subset \mathscr{V}\).
\end{definition}

\begin{proposition}{Orthogonal decomposition stable under a representation}{orthogonal_stable_representation}
    Let \(\algebra{A}\) be a C*-algebra and \(\hilbert, \pi\) a representation. A closed linear subspace \(\mathscr{V}\subset \hilbert\) is stable under \(\pi\) if and only if \(\mathscr{V}^{\perp}\) is stable under \(\pi\).
\end{proposition}
\begin{proof}
    We have
    \begin{align*}
        \pi(\algebra{A})\mathscr{V} \subset \mathscr{V}
        &\iff
        \forall a \in \algebra{A}, \forall \psi \in \mathscr{V} : \pi(a)\psi \in \mathscr{V}\\
        &\iff
        \forall a \in \algebra{A}, \forall \psi \in \mathscr{V}, \forall \phi \in \mathscr{V}^{\perp} : \inner{\phi}{\pi(a)\psi} = 0\\
        &\iff
        \forall a \in \algebra{A}, \forall \psi \in \mathscr{V}, \forall \phi \in \mathscr{V}^{\perp}: \inner{\psi}{\pi(a^*)\phi} = 0\\
        &\iff
        \forall a \in \algebra{A}, \forall \psi \in \mathscr{V}, \forall \phi \in \mathscr{V}^{\perp}: \inner{\psi}{\pi(a)\phi} = 0\\
        &\iff
        \forall a \in \algebra{A},\forall \phi \in \mathscr{V}^{\perp} : \pi(a)\phi \in \mathscr{V}^{\perp}\\
        &\iff
        \pi(\algebra{A})\mathscr{V}^{\perp} \subset \mathscr{V}^{\perp}
    \end{align*}
    as desired.
\end{proof}

\begin{proposition}{Necessary and sufficient condition for a closed stable subspace}{subspace_stable}
    Let \(\algebra{A}\) be a C*-algebra and \((\hilbert, \pi)\) be a representation. The closed linear subspace \(\mathscr{V} \subset \hilbert\) is stable under \(\pi\) if and only if the orthogonal projector \(p_{\mathscr{V}}\) onto \(\mathscr{V}\) lies in the commutant \(\pi(\algebra{A})'\).
\end{proposition}
\begin{proof}
    Suppose \(p_{\mathscr{V}}\) commutes with every representant of \(\algebra{A}\), then for all \(\psi \in \mathscr{V}\) we have
    \begin{equation*}
        \pi(a) \psi = \pi(a)p_\mathscr{V} \psi = p_{\mathscr{V}} \pi(a) \psi \subset \mathscr{V},
    \end{equation*}
    hence \(\mathscr{V}\) is stable under \(\pi\).

    If \(\mathscr{V}\) is stable under \(\pi\), then \(p_\mathscr{V} \pi(a) p_{\mathscr{V}} = \pi(a) p_{\mathscr{V}}\) for all \(a \in \algebra{A}\), since \(\pi(a) p_{\mathscr{V}}\psi \in \mathscr{V}\) for all \(\psi \in \hilbert\). This yields
    \begin{equation*}
        \pi(a)p_{\mathscr{V}} = (p_{\mathscr{V}}\pi(a^*)p_{\mathscr{V}})^* = (\pi(a^*)p_{\mathscr{V}})^* = p_{\mathscr{V}} \pi(a),
    \end{equation*}
    hence \(p_{\mathscr{V}} \in \pi(\algebra{A})'\).
\end{proof}

\begin{proposition}{Subrepresentation}{subrepresentation}
    Let \(\algebra{A}\) be a C*-algebra and \((\hilbert, \pi)\) be a representation. If \(\mathscr{V}\) is a closed linear subspace stable under \(\pi\), then the map
    \begin{align*}
        \pi_{\mathscr{V}} : \algebra{A} &\to \bounded(\mathscr{V})\\
                                      a &\mapsto p_\mathscr{V} \pi(a) p_{\mathscr{V}}
    \end{align*}
    is a *-morphism, and the representation \((\mathscr{V}, \pi_{\mathscr{V}})\) is referred to as a \emph{subrepresentation} of \((\hilbert, \pi)\).
\end{proposition}
\begin{proof}
    The map is clearly linear, as we have similarly shown in \cref{lem:projector_star_subalgebra}. Let \(a, b \in \algebra{A}\), then
    \begin{equation*}
        \pi_{\mathscr{V}}(a)\pi_{\mathscr{V}}(b) = p_{\mathscr{V}} \pi(a) p_{\mathscr{V}}^2 \pi(b) p_{\mathscr{V}} = p_{\mathscr{V}}^2 \pi(a)\pi(b) p_{\mathscr{V}}^2 = p_{\mathscr{V}} \pi(ab) p_\mathscr{V} = \pi_{\mathscr{V}}(ab)
    \end{equation*}
    and
    \begin{equation*}
        \pi_{\mathscr{V}}(a)^* = p_{\mathscr{V}} \pi(a)^* p_{\mathscr{V}} = p_{\mathscr{V}}\pi(a^*) p_{\mathscr{V}} = \pi_{\mathscr{V}}(a^*),
    \end{equation*}
    hence \(\pi_\mathscr{V}\) is a *-morphism.
\end{proof}

\begin{definition}{Nondegenerate representation}{nondegenerate_representation}
    Let \(\algebra{A}\) be a C*-algebra. A representation \((\hilbert, \pi)\) of \(\algebra{A}\) is \emph{nondegenerate} if for any nonzero \(\psi \in \hilbert\) there exists \(a \in \algebra{A}\) such that \(\pi(a)\psi \neq 0\).
\end{definition}

A useful and equivalent characterization of nondegeneracy is shown in the
\begin{lemma}{Necessary and sufficient condition for a nondegenerate representation}{nondegenerate_representation}
    Let \(\algebra{A}\) be a C*-algebra and \((\hilbert, \pi)\) a representation. The set \(\pi(\algebra{A})\hilbert\) is dense in \(\hilbert\) if and only if \(\pi\) is nondegenerate.
\end{lemma}
\begin{proof}
    Suppose \(\pi\) is nondegenerate and let \(\psi \in \pi(\algebra{A})\hilbert^\perp\). Then,
    \begin{equation*}
        \inner{\psi}{\pi(a^*a)\psi} = \norm{\pi(a)\psi} = 0
    \end{equation*}
    for all \(a \in \algebra{A}\), hence \(\psi = 0.\) That is, \(\pi(\algebra{A})\hilbert^{\perp}= \set{0}\) and we conclude \(\pi(\algebra{A})\hilbert\) is dense in \(\hilbert\).
    
    Suppose \(\pi(\algebra{A})\hilbert\) is dense in \(\hilbert\), and let \(\psi \in \hilbert\) such that \(\pi(a)\psi = 0\) for all \(a \in \algebra{A}\). Then
    \begin{equation*}
        \forall a \in \algebra{A}, \varphi \in \hilbert : \inner{\pi(a^*)\psi}{\varphi} = 0 \implies \forall a \in \algebra{A}, \varphi \in \hilbert: \inner{\psi}{\pi(a)\varphi} = 0 \implies \psi \in \pi(\algebra{A})\hilbert^{\perp} = \set{0},
    \end{equation*}
    hence we conclude \(\pi\) is nondegenerate.
\end{proof}

\begin{proposition}{Non-degenerate representation preserve approximate identities}{representation_approximate_identity}
    Let \(\algebra{A}\) be a C*-algebra and let \(e : \Lambda \to \algebra{A}_+\) be an approximate identity on \(\algebra{A}\). If \((\hilbert, \pi)\) is a non-degenerate representation of \(\algebra{A}\), then \(\pi \circ e : \Lambda \to \bounded(\hilbert)\) is an approximate identity on \(\bounded(\hilbert)\) with respect to the strong operator topology, that is, for all \(b \in \bounded(\hilbert)\) and \(\psi \in \hilbert\) we have \(\lim_{\lambda \in \Lambda}{\norm{b\psi - \tilde{e}_{\lambda}b\psi}} = 0\).
\end{proposition}
\begin{proof}
    Let us denote \(\tilde{e} = \pi \circ e\). It is clear the range of \(e\) lies in \(\setc{a \in \bounded(\hilbert)_+}{\norm{a} \leq 1}\) as \(\pi\) preserves positivity and is non-expansive. By the same argument, we have 
    \begin{equation*}
        \lambda, \lambda' \in \Lambda : \lambda \succeq \lambda' \implies e_{\lambda} - e_{\lambda'} \in \algebra{A}_+ \implies \tilde{e}_{\lambda} \geq \tilde{e}_{\lambda'}.
    \end{equation*}
    Let \(b \in \bounded(\hilbert)\) and let \(\varepsilon > 0\). By \cref{lem:nondegenerate_representation}, for all \(\psi \in \hilbert\), there exist \(a_{\varepsilon} \in \algebra{A}\) and \(\psi_\varepsilon \in \hilbert\) such that \(\norm{\pi(a_{\varepsilon})\psi_{\varepsilon} - b\psi} < \frac\varepsilon3\). Notice we have
    \begin{align*}
        \norm{b \psi - \tilde{e}_{\lambda}b \psi} &\leq \norm{b\psi - \pi(a_\varepsilon)\psi_\varepsilon} + \norm{\pi(a_\varepsilon)\psi_\varepsilon - \tilde{e}_{\lambda}\pi(a_\varepsilon)\psi_\varepsilon} + \norm{\tilde{e}_\lambda\pi(a_\varepsilon)\psi_\varepsilon - \tilde{e}_\lambda b\psi}\\
                                                  &< \frac\varepsilon3 + \norm{\pi(a_\varepsilon - e_\lambda a_\varepsilon)\psi_\varepsilon} + \norm{\tilde{e}_\lambda} \norm{\pi(a_\varepsilon) \psi_\varepsilon - b \psi}\\
                                                  &\leq \frac{2\varepsilon}{3} + \norm{a_\varepsilon - e_\lambda a_\varepsilon} \norm{\psi_\varepsilon}
    \end{align*}
    for all \(\lambda \in \Lambda\). As \(e\) is an approximate identity with respect to the uniform topology, there exists \(\tilde{\lambda} \in \Lambda\) such that for all \(\lambda \succeq \tilde{\lambda}\) we have \(\norm{a_\varepsilon - e_\lambda a_\varepsilon} < \frac{\varepsilon}{3\norm{\psi_\varepsilon}}\), hence
    \begin{equation*}
        \norm{b\psi - \tilde{e}_{\lambda}b\psi} < \varepsilon,
    \end{equation*}
    that is, \(\tilde{e}_\lambda b\psi\) converges strongly against \(b\psi\).
\end{proof}

An important class of nondegenerate representations is the one of \emph{cyclic representations}.
\begin{definition}{Cyclic vector and cyclic representation}{cyclic_representation}
    Let \(\algebra{A}\) be a C*-algebra and \(\hilbert\) be a Hilbert space. A vector \(\psi \in \hilbert\) is \emph{cyclic} for a set of bounded operators \(\algebra{M} \subset \bounded(\hilbert)\) if \(\algebra{M}\psi\) is dense in \(\hilbert\). A \emph{cyclic representation \((\hilbert, \pi, \Omega)\) of \(\algebra{A}\)} is a representation \((\hilbert, \pi)\) for which \(\Omega \in \hilbert\) is a cyclic vector.
\end{definition}

% TODO: study direct sum of hilbert spaces, including uncountable ones
A subrepresentation allows for a decomposition of a representation. If \(\hilbert_1 \subset \hilbert\) is a closed linear subspace of a Hilbert space \(\hilbert\), then we set \(\hilbert_2 = \hilbert_1^{\perp}\), and then \(\hilbert = \hilbert_1 \oplus \hilbert_2\). By the previous propositions, \(\pi_1 = p_{\hilbert_1} \pi p_{\hilbert_1}\) and \(\pi_2 = p_{\hilbert_2} \pi p_{\hilbert_2}\) are representations and we may consider \(\pi = \pi_1 \oplus \pi_2\) and write \((\hilbert, \pi) = (\hilbert_1, \pi_1) \oplus (\hilbert_2, \pi_2)\). We extend this definition to general direct sums of representations as follows: if \(\family{(\hilbert_{\lambda}, \pi_{\lambda})}{\lambda \in \Lambda}\) is a family of representations of a C*-algebra \(\algebra{A}\), then \(\hilbert = \bigoplus_{\lambda \in \Lambda} \hilbert_{\lambda}\) is \todo[defined in the usual manner] and \(\pi = \bigoplus_{\lambda \in \Lambda} \pi_{\lambda}\) is defined by setting \(\pi(a)\) equal to \(\pi_{\lambda}(a)\) in each component \(\hilbert_{\lambda}\), for all \(a \in \algebra{A}\). It follows that \(\pi : \algebra{A} \to \hilbert\) is a representation since each \(\pi_{\lambda}\) is non-expansive by \cref{prop:morphism_non_expansive}.
\begin{proposition}{Non-degenerate representation is a direct sum of cyclic representations}{direct_sum_representation}
    Let \((\hilbert, \pi)\) be a representation of the C*-algebra \(\algebra{A}\). If \(\pi\) is non-degenerate, then there exists a family of cyclic subrepresentations \family{(\hilbert_{\lambda}, \pi_{\lambda}, \Omega_\lambda)}{\lambda \in \Lambda} such that \((\hilbert, \pi) = \bigoplus_{\lambda \in \Lambda}(\hilbert_{\lambda}, \pi)\).
\end{proposition}
\begin{proof}
    We consider the set
    \begin{equation*}
        \mathfrak{O} = \setc*{\mathscr{V} \in \mathbb{P}(\hilbert\setminus\set{0})}{\forall \psi,\phi \in \mathscr{V} : \psi \neq \phi \implies \pi(\algebra{A})\psi \perp \pi(\algebra{A})\phi},
    \end{equation*}
    partially ordered by inclusion. Notice that if \(\phi \in \lspan\set{\psi}\setminus \set{0,\psi}\), then \(\set{\phi,\psi} \notin \mathfrak{O}\). If \(\hilbert\) is unidimensional, then there exists \(\xi \in \hilbert\setminus\set{0}\) such that \(\hilbert = \lspan\set{\xi}\) and as \(\pi\) is nondegenerate, we have \(\pi(\algebra{A})\xi \neq \set{0}\), and as a result \(\set{\xi}\) is a maximal element of \(\mathfrak{O}\). Suppose \(\hilbert\) has dimension greater than one. Let \(\mathcal{F} \subset \mathfrak{O}\) be a non-empty linearly ordered subset of \(\mathfrak{O}\), then its union 
    \begin{equation*}
        \bigcup \mathcal{F} = \bigcup_{\mathscr{V} \in \mathcal{F}} \mathscr{V} = \setc{\psi \in \hilbert}{\exists \mathscr{X} \in \mathcal{F} : \psi \in \mathscr{X}}
    \end{equation*}
    is an upper bound for \(\mathcal{F}\). Let \(\psi, \phi \in \bigcup \mathcal{F}\) with \(\psi \neq \phi\), then there exist \(\mathscr{U},\mathscr{V} \in \mathcal{F}\) such that \(\psi \in \mathscr{U}\) and \(\phi \in \mathscr{V}\). By linear order, we may assume without loss of generality that \(\mathscr{U} \subset \mathscr{V}\), hence \(\psi \in \mathscr{V}\), and we conclude \(\pi(\algebra{A})\psi \perp \pi(\algebra{A}\phi)\) as \(\mathscr{V} \in \mathfrak{O}\), therefore \(\bigcup \mathcal{F} \in \mathfrak{O}\). 

    We have shown every linearly ordered subset of \(\mathfrak{O}\) has an upper bound in \(\mathfrak{O}\), hence by \nameref{thm:zorn} there exists a maximal family of nonzero vectors \(\family{\Omega_{\lambda}}{\lambda \in \Lambda} \subset \hilbert \setminus \set{0}\) such that
    \begin{equation*}
        \alpha, \beta \in \Lambda : \alpha \neq \beta \implies \forall a, b \in \algebra{A} : \inner{\pi(a)\Omega_{\alpha}}{\pi(b)\Omega_{\beta}} = 0.
    \end{equation*}
    Then \(\hilbert_{\lambda} = \cl_{\hilbert}\left(\pi(\algebra{A})\Omega_{\lambda}\right)\) defines a closed linear subspace that is stable under \(\pi\). Indeed, if \(\tilde{\psi} \in \hilbert_{\lambda}\) then there exists a convergent sequence \(\psi : \mathbb{N} \to \hilbert_{\lambda}\) such that \(\psi \to \tilde{\psi}\), and as a result, for all \(a \in \algebra{A}\) we have \(\pi(a)\tilde{\psi} = \lim_{n\to\infty}{\pi(a)\psi_n} \in \hilbert_{\lambda}\) as \(\pi(a)\) is continuous. Following \cref{prop:subrepresentation}, we define the subrepresentation \((\hilbert_{\lambda}, \pi_{\lambda})\) with
    \begin{align*}
        \pi_{\lambda} : \algebra{A} &\to \bounded(\hilbert)\\
                                  a &\mapsto p_{\hilbert_{\lambda}}\pi(a)p_{\hilbert_{\lambda}}
    \end{align*}
    where \(p_{\hilbert_{\lambda}} \in \bounded(\hilbert)\) is the orthogonal projector onto \(\hilbert_{\lambda}\), and for which \(\Omega_\lambda\) is the cyclic vector by construction. Defining the Hilbert space \(\tilde{\hilbert} = \bigoplus_{\lambda \in \Lambda} \hilbert_\lambda\) and its representation \(\tilde{\pi} : \algebra{A} \to \bounded(\hilbert)\) defined by
    \begin{align*}
        \tilde{\pi}(a) : \hilbert &\to \hilbert\\
                             \psi &\mapsto \bigoplus_{\lambda \in \Lambda} \pi_{\lambda}(a)\psi
    \end{align*}
    for all \(a \in \algebra{A}\), we aim to show that \((\hilbert, \pi) = (\tilde{\hilbert}, \tilde{\pi})\).

    Notice that if \(\lambda, \mu \in \Lambda\) with \(\lambda \neq \mu\), then \(\hilbert_{\lambda} \perp \hilbert_{\mu}\) by construction, as the inner product is continuous. Suppose, by contradiction, there exists \(\tilde{\Phi} \in \hilbert\) that does not lie in the direct sum \(\tilde{\hilbert} = \bigoplus_{\lambda \in \Lambda} \hilbert_\lambda\), hence \(\tilde{\Phi} \in \hilbert_\lambda^\perp\) for all \(\lambda \in \Lambda\), and in particular, \(\tilde{\Phi} \in \algebra{A}\Omega_{\lambda}\) for all \(\lambda \in \Lambda\). That is, we have
    \begin{align*}
        \tilde{\Phi} \notin \tilde{\hilbert} &\implies \forall \lambda \in \Lambda, \forall a \in \algebra{A} : \inner{\tilde{\Phi}}{\pi(a^{\ast})\pi(b)\Omega_\lambda} = 0\\
                                             &\implies \forall \lambda \in \Lambda, \forall a,b\in \algebra{A} : \inner{\tilde{\Phi}}{\pi(a)^{\ast}\pi(b)\Omega_{\lambda}} = 0\\
                                             &\implies \forall \lambda \in \Lambda, \forall a,b \in \algebra{A}: \inner{\pi(a)\tilde{\Phi}}{\pi(b)\Omega_\lambda} = 0,
    \end{align*}
    therefore \(\set{\tilde{\Phi}}\cup \family{\Omega_\lambda}{\lambda \in \Lambda}\) is an upper bound for the maximal family. By definition, this means \(\tilde{\Phi} \in \family{\Omega_\lambda}{\lambda \in \Lambda}\), which is absurd as \(\tilde{\Phi} \in \hilbert_{\lambda}^\perp\) for all \(\lambda \in \Lambda\). This contradiction shows us \(\hilbert = \tilde{\hilbert}\).

    Notice for all \(a \in \algebra{A}\) and \(\lambda \in \Lambda\) the operators \(\pi(a)\) and \(p_{\hilbert_\lambda}\) commute. Indeed, since \(\pi(a) \hilbert_\lambda \subset \hilbert_\lambda\) we know \(\pi(a)p_{\hilbert_\lambda} = p_{\hilbert_\lambda}\pi(a)p_{\hilbert_\lambda}\), then taking the adjoint and using the fact that \(\algebra{A}\) is self-adjoint yields \(\pi(a)p_{\hilbert_\lambda} = p_{\hilbert_\lambda}\pi(a)\) for all \(a \in \algebra{A}\) and \(\lambda \in \Lambda\). Let \(\psi \in \hilbert\), then we may decompose it as \(\psi = \oplus_{\lambda \in \Lambda}p_{\hilbert_\lambda}\psi\), yielding
    \begin{align*}
        \tilde{\pi}(a)\psi &= \tilde{\pi}(a)\bigoplus_{\lambda \in \Lambda} p_{\hilbert_\lambda}\psi\\
                           &= \bigoplus_{\lambda \in \Lambda} \pi_{\lambda}(a) p_{\hilbert_\lambda}\psi\\
                           &= \bigoplus_{\lambda \in \Lambda} p_{\hilbert_\lambda} \pi(a) p_{\hilbert_\lambda}^2 \psi\\
                           &= \bigoplus_{\lambda \in \Lambda} p_{\hilbert_\lambda} \pi(a) \psi\\
                           &= \pi(a) \psi
    \end{align*}
    for all \(a \in \algebra{A}\), hence \(\tilde{\pi} = \pi\).
\end{proof}

\begin{definition}{Topologically Irreducible set of bounded operators}{irreducible}
    Let \(\hilbert\) be a Hilbert space. A set \(\algebra{M}\) of bounded operators on \(\hilbert\) is \emph{topologically irreducible} if the only closed subspaces of \(\hilbert\) which are invariant under the action of \(\algebra{M}\) are the trivial subspaces \(\set{0}\) and \(\hilbert\). A representation \((\hilbert, \pi)\) of a C*-algebra \(\algebra{A}\) is \emph{topologically irreducible} if \(\pi(\algebra{A})\) is irreducible.
\end{definition}
\begin{remark}
    We will refer to topological irreducibility simply by irreducibility. The notion of \emph{algebraic} irreducibility requires that the only invariant subspaces are the trivial ones. 
    %TODO: find proof of this
    In fact, the both notions coincide for representations of C*-algebras.
\end{remark}

\begin{proposition}{Irreducible self-adjoint set of bounded operators}{irreducible_self_adjoint}
    Let \(\hilbert\) be a Hilbert space and let \(\algebra{M} \subset \bounded(\hilbert)\) be a self-adjoint set of bounded operators on \(\hilbert\). The following statements are equivalent:
    \begin{enumerate}[label=(\alph*)]
        \item \(\algebra{M}\) is irreducible;
        \item \(\psi \in \hilbert \setminus \set{0}\) is cyclic for \(\algebra{M}\), or \(\algebra{M} = \set{0}\) and \(\hilbert = \mathbb{C}\); and
        \item \(\algebra{M}' = \mathbb{C} \unity\).
    \end{enumerate}
\end{proposition}
\begin{proof}
    Assume \(\algebra{M}\) is irreducible. If \(\algebra{M} = \set{0}\), then the only closed subspaces of \(\hilbert\) are \(\set{0}\) and itself, hence \(\hilbert\) is unidimensional and we conclude \(\hilbert = \mathbb{C}\), up to isomorphism. If \(\algebra{M} \neq \set{0}\), then we suppose, by contradiction, there exists \(\psi \in \hilbert \setminus \set{0}\) such that \(\algebra{M}\psi\) is not dense in \(\hilbert\). Then \((\algebra{M}\psi)^\perp\) contains a non-zero vector and it is invariant by the action of \(\algebra{M}\) as we have
    \begin{align*}
        \phi \in (\algebra{M}\psi)^{\perp}\setminus\set{0} &\iff \forall a,b \in \algebra{M} : \inner{\phi}{a^{\ast}b\psi} = 0\\
                                                           &\iff \forall a,b \in \algebra{M} : \inner{a\phi}{b\psi} = 0\\
                                                           &\iff \forall a \in \algebra{M} : a\phi \in (\algebra{M}\psi)^\perp,
    \end{align*}
    since \(\algebra{M}\) is self-adjoint. Notice we must have \((\algebra{M}\psi)^\perp \neq \hilbert\), otherwise this would imply \(\algebra{M}\psi = \set{0}\), and as a result \(\lspan\set{\psi}\) would be an invariant subspace under \(\algebra{M}\), contradicting its irreducibility. The only other possibility is that \((\algebra{M}\psi)^\perp\) is a non-trivial closed subspace invariant under \(\algebra{M}\), which also contradicts the hypothesis. The contradiction thus shows every non-zero vector of \(\hilbert\) must be cyclic for \(\algebra{M}\).

    %TODO: try and come up with a more elementary proof
    Clearly if \(\algebra{M} = \set{0}\) and \(\hilbert = \mathbb{C}\), we must have \(\algebra{M}' = \mathbb{C} \unity\), so we may assume \(\algebra{M} \neq \set{0}\) and \(\dim\hilbert > 1\). Suppose every non-zero vector of \(\hilbert\) is cyclic for \(\algebra{M}\). \todo[spectral projectors?]

    Suppose \(\algebra{M}\) is not irreducible, then there exists a non-trivial closed subspace \(\mathscr{V}\subset \hilbert\) such that \(\algebra{M}\mathscr{V} \subset \mathscr{V}\). Let \(p_{\mathscr{V}}\in \bounded(\hilbert)\) be the orthogonal projector onto \(\mathscr{V}\), then it follows that \(ap_{\mathscr{V}} = p_{\mathscr{V}}a p_{\mathscr{V}}\) for all \(a \in \algebra{A}\) and taking the adjoint yields \(ap_{\mathscr{V}} = p_{\mathscr{V}} a\), that is, \(p_{\mathscr{V}} \in \algebra{M}'\). As \(\mathscr{V}\) is non-trivial, it follows that \(p_{\mathscr{V}} \notin \lspan\set\unity\), hence \(\algebra{M}' \neq \mathbb{C} \unity\).
\end{proof}

Unitary operators in \(\bounded(\hilbert)\) may be used together with a representation to define yet another representation.
\begin{proposition}{Representations and unitary operators}{representation_unitary}
    Let \(\algebra{A}\) be a C*-algebra and let \(u : \hilbert_1 \to \hilbert_2\) be a unitary operator between the Hilbert spaces \(\hilbert_1\) and \(\hilbert_2\). If \((\hilbert_1, \pi)\) is a representation of \(\algebra{A}\), then the *-morphism \(\pi_u : \algebra{A} \to \bounded(\hilbert_2)\) defined by \(\pi_u(a) = u\pi(a)u^*\) for all \(a \in \algebra{A}\) establishes a representation of \(\algebra{A}\) in \(\hilbert_2\).
\end{proposition}
\begin{proof}
    Clearly \(\pi_u\) is linear and for all \(a, b \in \algebra{A}\) we have
    \begin{equation*}
        \pi_u(ab) = u \pi(ab) u^* = u \pi(a) \pi(b) u^* = u \pi(a) u^* u \pi(b) u^* = \pi_u(a)\pi_u(b)
    \end{equation*}
    and 
    \begin{equation*}
        \pi_u(a)^* = (u \pi(a) u^*)^* = u \pi(a)^* u^* = u \pi(a^*)u^* = \pi_u(a^*),
    \end{equation*}
    hence \(\pi_u\) is a *-morphism.
\end{proof}
In fact, this construction establishes an equivalence relation on the set of representations of a C*-algebra.
\begin{proposition}{Unitarily equivalent representations of a C*-algebra}{unitary_equivalence_representation}
    Let \(\algebra{A}\) be a C*-algebra and let \(\mathfrak{R}_{\algebra{A}}\) denote the set of representations of \(\algebra{A}\). If \((\hilbert_1, \pi_1), (\hilbert_2, \pi_2) \in \mathfrak{R}_{\algebra{A}}\) are such that there exists a unitary operator \(u : \hilbert_1 \to \hilbert_2\) that satisfies \(\pi_2(a) = u\pi_1(a) u^*\) for all \(a \in \algebra{A}\), then the representations are \emph{unitarily equivalent} and we write \(\pi_1 \simeq \pi_2\). Unitary equivalence is an equivalent relation on \(\mathfrak{R}_{\algebra{A}}\).
\end{proposition}
\begin{proof}
    Let \((\hilbert_1, \pi_1), (\hilbert_2, \pi_2), (\hilbert_3, \pi_3) \in \mathfrak{R}_{\algebra{A}}\) be representations. Clearly, \(\pi(a) = \id{\hilbert_1} \pi(a) \id{\hilbert_1}^*\) for all \(a \in \algebra{A},\) hence unitary equivalence is reflexive. If \((\hilbert_1, \pi_1) \simeq (\hilbert_2, \pi_2\), then there exists a unitary operator \(u_{12} : \hilbert_1 \to \hilbert_2\) such that \(\pi_2(a) = u_{12}\pi_1(a)u_{12}^*\) for all \(a \in \algebra{A}\), hence the unitary map \(u_{21} = u_{12}^*\) satisfies \(\pi_1(a) = u_{21}\pi_2(a) u_{21}^*\) for all \(a \in \algebra{A}\), hence unitary equivalence is symmetric. If \((\hilbert_1, \pi_1) \simeq (\hilbert_2, \pi_2)\) and \((\hilbert_2, \pi_2) \simeq (\hilbert_3, \pi_3)\), then with the notation used so far we have 
    \begin{equation*}
        \pi_3(a) = u_{23}\pi_2(a) u_{23}^* = u_{23} u_{12} \pi_1(a) u_{12}^* u_{23}^* = u_{13} \pi_1(a) u_{13}^*,
    \end{equation*}
    for all \(a \in \algebra{A}\), where \(u_{13} = u_{23} u_{12}\) is a composition of unitary maps, thus unitary, that is, unitary equivalence is transitive.
\end{proof}

% vim: spl=en_us
\section{States and positive linear functionals on C*-algebras}
As usual we denote the topological dual of a C*-algebra \(\algebra{A}\) by \(\algebra{A}^\dag\), which is a Banach space with respect to the operator norm, and contained in the algebraic dual \(\algebra{A}'\), which consists of all linear functionals \(\omega : \algebra{A} \to \mathbb{C}\).
\begin{definition}{Positive linear functional and states}{state}
    Let \(\algebra{A}\) be a C*-algebra. The linear functional \(\omega \in \algebra{A}'\) is \emph{positive} if \(\omega(a^*a) \geq 0\) for all \(a \in \algebra{A}\). A \emph{state} is a positive linear functional \(\omega\) with \(\norm{\omega} = 1\).
\end{definition}

A positive linear functional defines a positive sesquilinear form on \(\algebra{A}\), then, in this sense, this linear functional satisfies the Cauchy-Schwarz inequality.
\begin{proposition}{Properties of a positive sesquilinear form}{sesquilinear_positive_form}
    Let \(\inner{\noarg}{\noarg} : V \to \mathbb{C}\) be a sesquilinear form on a complex linear space \(V\), that is, it is anti-linear in the first argument, linear in the second argument. If it is positive, that is, it satisfies \(\inner{v}{v} \geq 0\) for all \(v \in V\), then
    \begin{enumerate}[label=(\alph*)]
        \item this sesquilinear form is Hermitian, \(\inner{u}{v} = \conj{\inner{v}{u}}\); and
        \item the Cauchy-Schwarz inequality holds, \(\abs{\inner{u}{v}}^2 \leq \inner{u}{u} \inner{v}{v}\),
    \end{enumerate}
    for all \(u,v \in V.\)
\end{proposition}
\begin{proof}
    As the sesquilinear form is positive, we have
    \begin{equation*}
        \inner{u+\lambda v}{u + \lambda v} = \inner{u}{u} + \lambda \inner{u}{v} + \conj{\lambda} \inner{v}{u} + \abs{\lambda}^2\inner{v}{v} \geq 0,
    \end{equation*}
    for all \(u,v \in V\) and \(\lambda \in \mathbb{C}\). The imaginary part of the left-hand side must be zero, hence
    \begin{equation*}
        0 = \Im\left(\lambda \inner{u}{v} + \conj{\lambda} \inner{v}{u}\right) = \Re(\lambda) \Im\left(\inner{u}{v} + \inner{v}{u}\right) + \Im(\lambda)\Re\left(\inner{u}{v} - \inner{v}{u}\right),
    \end{equation*}
    and by setting \(\lambda = 1\) and \(\lambda = i\) we conclude
    \begin{equation*}
        \Im\left(\inner{u}{v}\right) = - \Im\left(\inner{v}{u}\right)
        \quad\text\quad
        \Re\left(\inner{u}{v}\right) = \Re\left(\inner{v}{u}\right),
    \end{equation*}
    that is, \(\inner{u}{v} = \conj{\inner{v}{u}}\). As a consequence we have
    \begin{align*}
        0 \leq \inner{u+\lambda v}{u + \lambda v} &= \inner{u}{u} + 2 \Re\left(\lambda \inner{u}{v}\right) + \abs{\lambda}^2\inner{v}{v}\\
                                                  &= \inner{u}{u} + 2 \Re(\lambda) \Re\left(\inner{u}{v}\right) - 2 \Im(\lambda) \Im\left(\inner{u}{v}\right) + \abs{\lambda}^2 \inner{v}{v}.
    \end{align*}
    If \(\inner{v}{v} = 0\), then the inequality can only hold for all \(\lambda\in \mathbb{C}\) if \(\inner{u}{v} = 0\), case for which the Cauchy-Schwarz inequality holds trivially, so we must assume \(\inner{v}{v} \neq 0\) without loss of generality. Notice
    \begin{equation*}
        \inner{v}{v}\left[\Re(\lambda) + \frac{\Re(\inner{u}{v})}{\inner{v}{v}}\right]^2 = \inner{v}{v} \Re(\lambda)^2 + \frac{\Re(\inner{u}{v})^2}{\inner{v}{v}} + 2\Re(\lambda) \Re(\inner{u}{v}
    \end{equation*}
    and
    \begin{equation*}
        \inner{v}{v}\left[\Im(\lambda) - \frac{\Im(\inner{u}{v})}{\inner{v}{v}}\right]^2 = \inner{v}{v} \Im(\lambda)^2 + \frac{\Im(\inner{u}{v})^2}{\inner{v}{v}} - 2\Im(\lambda) \Im(\inner{u}{v},
    \end{equation*}
    then
    \begin{equation*}
        \inner{u+\lambda v}{u + \lambda v} = \inner{u}{u} + \inner{v}{v}\left[\left(\Re(\lambda) + \frac{\Re(\inner{u}{v})}{\inner{v}{v}}\right)^2 + \left(\Im(\lambda) - \frac{\Im(\inner{u}{v})}{\inner{v}{v}}\right)^2\right] - \frac{\abs{\inner{u}{v}}^2}{\inner{v}{v}},
    \end{equation*}
    and thus positivity yields
    \begin{equation*}
        \inner{u}{u} \geq \frac{\abs{\inner{u}{v}}^2}{\inner{v}{v}}
    \end{equation*}
    and the Cauchy-Schwarz inequality follows.
\end{proof}
\begin{corollary}
    Let \(\algebra{A}\) be a C*-algebra and let \(\omega \in \algebra{A}'\) be a positive linear functional. Then
    \begin{enumerate}[label=(\alph*)]
        \item \(\omega(a^*b) = \conj{\omega(b^*a)}\), and
        \item \(\abs{\omega(a^*b)}^2 \leq \omega(a^*a) \omega(b^*b)\),
    \end{enumerate}
    for all \(a, b \in \algebra{A}\).
\end{corollary}
\begin{proof}
    The map
    \begin{align*}
        \inner{\noarg}{\noarg} : \algebra{A} \times \algebra{A} &\to \mathbb{C}\\
                                                          (a,b) &\mapsto \omega(a^*b),
    \end{align*}
    is a positive sesquilinear form. Indeed, let \(a, b,c,d \in \algebra{A}\) and \(\lambda,\mu \in \mathbb{C}\), then
    \begin{align*}
        \inner{a + \lambda b}{c + \mu d} &= \omega\left(a^*c + \mu a^*d + \conj{\lambda} b^* c + \conj{\lambda}\mu b^*d\right)\\
                                         &= \omega(a^*c) + \mu \omega(a^*d) + \conj{\lambda} \omega(b^*c) + \conj{\lambda}\mu \omega(b^*d)\\
                                         &= \inner{a}{c} + \mu \inner{a}{d} + \conj{\lambda} \inner{b}{c} + \conj{\lambda}\mu \inner{b}{d},
    \end{align*}
    hence it is a sesquilinear form and it is positive since \(\inner{a}{a} = \omega(a^*a) \geq 0\) by hypothesis. The corollary then follows from the previous proposition.
\end{proof}

\begin{proposition}{Positive linear functionals are continuous}{state_continuous}
    Let \(\omega : \algebra{A} \to \mathbb{C}\) be a linear functional over a C*-algebra \(\algebra{A}\). The following statements are equivalent:
    \begin{enumerate}[label=(\alph*)]
        \item \(\omega\) is positive;
        \item \(\omega\) is continuous and \(\norm{\omega} = \lim_{\lambda\in\Lambda} \omega(e_{\lambda}^2)\) for some approximate identity \(\family{e_{\lambda}}{\lambda \in \Lambda} \subset \algebra{A}_+\).
    \end{enumerate}
    If \(\omega\) is positive, then \(\norm{\omega} = \lim_{\lambda \in \Lambda}{\omega(e_\lambda)}\).
\end{proposition}
\begin{proof}[Proof that (a) \(\implies\) (b)]
    Consider the set \(B_+ = \setc{a \in \algebra{A}_+}{\norm{a} \leq 1}\), which is closed with respect to the uniform topology as the intersection of two closed sets. If \(\lambda : \mathbb{N} \to \mathbb{R}\) is a summable sequence of non-negative numbers, then for any sequence \(a : \mathbb{N} \to B_+\) we have \(s : \mathbb{N} \to \algebra{A}_+\) defined by \(s_n = \sum_{k = 1}^{n}\lambda_k a_k\) uniformly and monotonically convergent to some positive element \(\tilde{s} \in \algebra{A}_+\). Indeed, for all \(n, m \in \mathbb{N}\) with \(n \leq m\) we have
    \begin{equation*}
        s_m - s_n = \sum_{k=n+1}^m \lambda_k a_k \in \algebra{A}_+
    \end{equation*}
    and
    \begin{equation*}
        \norm{s_m - s_n} \leq \sum_{k = n+1}^m \norm{\lambda_k a_k} \leq \sum_{k = n+1}^M \abs{\lambda_k},
    \end{equation*}
    hence \(s\) is Cauchy, and thus converges uniformly and is monotonic. As a result, by positivity and linearity we have
    \begin{equation*}
        \sum_{k = 1}^{n} \lambda_k \omega(a_k) \leq \omega(\tilde{s}) < \infty
    \end{equation*}
    for all \(n \in \mathbb{N}\).

    Suppose, by contradiction, the set \(\omega(B_+)\) is unbounded, then there must exist a sequence \(a : \mathbb{N} \to B_+\) such that \(\omega(a_n) \geq n\) for all \(n \in \mathbb{N}\). Consider the summable sequence \(\lambda : \mathbb{N} \to \mathbb{R}\) defined by \(\lambda_n = \frac1{n^2}\), then by the previous discussion, the sequence \(s : \mathbb{N} \to \algebra{A}_+\) defined by \(s_n = \sum_{k = 1}^{n} \lambda_k a_k\) converges uniformly and monotonically in \(\algebra{A}_+\) to some \(\tilde{s} \in \algebra{A}_+\) and we have
    \begin{equation*}
        \sum_{k = 1}^{n} \frac{1}{k^2} \omega(a_k) \leq \omega(\tilde{s}) < \infty
    \end{equation*}
    for all \(n \in \mathbb{N}\). However, the sum on the left is bounded below by \(\sum_{k} \frac1k\), which diverges. This contradiction informs us there exists \(M = \sup{\setc{\omega(a)}{a \in B_+}} \geq 0\) with \(M < \infty\).

    %TODO: add non-unital case for lemma 4.17 and add information about norm of the decomposition
    Recall that any operator \(a \in \algebra{A}\) may be written as a sum of two self-adjoint operators with \todo[norms no greater than \(\norm{a}\)] and in turn these self-adjoint operators may be written as a difference of positive operators with \todo[norm no greater than \(\norm{a}\)]. As a result we have \(\omega(a) = \norm{a} \omega(\frac1{\norm{a}}a) \leq 4M \norm{a}\) for all \(a \in \algebra{A} \setminus \set{0}\). We have thus shown \(\omega\) is bounded, thus continuous with \(M \leq \norm{\omega} \leq 4M\).

    Let \(e : \Lambda \to \algebra{A}_+\) be an approximate identity on \(\algebra{A}\), where \(\Lambda\) is some directed set. For all \(\lambda \in \Lambda\) we have by the Cauchy-Schwarz inequality that
    \begin{equation*}
        \abs{\omega(a e_\lambda)}^2 \leq \omega(a^*a) \omega(e_\lambda^2) \leq M \norm{a^*a} \omega(e_{\lambda}^2) = M \norm{a}^2\omega(e_\lambda^2),
    \end{equation*}
    which yields for all \(a \in \algebra{A}\) that
    \begin{equation*}
        \abs{\omega(a)}^2 \leq M L \abs{a}^2,
    \end{equation*}
    where \(L = \sup\setc{\omega(e_\lambda^2)}{\lambda \in \Lambda}\). We may infer, then, that \(L \leq M\) as \(e_{\lambda}^2 \in B_+\), hence \(M^2 \leq \norm{\omega}^2 \leq ML \leq M^2\). That is, \(M = L = \norm{\omega}\). As \(e_\lambda^2 \leq e_\lambda\), we get \(\norm{\omega} = L \leq \sup\setc{\omega(e_\lambda)}{\lambda \in \Lambda} \leq M = \norm{\omega}\), and we conclude the proof.
\end{proof}
\begin{proof}[Proof that (b) \(\implies\) (a)]
    We may assume \(\norm{\omega} > 0,\) otherwise \(\omega\) is trivially positive. In this case, \(\frac{1}{\norm{\omega}}\omega\) is a state if and only if \(\omega\) is positive, so we may assume without loss of generality that \(\norm{\omega} = 1\). If \(\algebra{A}\) is unital, then it is clear that \(\omega(\unity) = \norm{\omega} = 1\). If \(\algebra{A}\) is not unital, we consider the continuous linear functional
    \begin{align*}
        \hat{\omega} : \mathbb{C} \ltimes \algebra{A} &\to \mathbb{C}\\
                                           (\alpha,a) &\mapsto \alpha + \omega(a)
    \end{align*}
    which extends \(\omega\) to \(\mathbb{C} \ltimes \algebra{A}\) and clearly satisfies \(\hat{\omega}(\unity) = 1\). For all \(a \in \algebra{A}\) and \(\lambda \in \Lambda\) we have
    \begin{equation*}
        a - ae_{\lambda}^2 = a - ae_{\lambda} + (a - ae_{\lambda})e_{\lambda}
    \end{equation*}
    hence \(ae_{\lambda}^2 \to a\), thus yielding
    \begin{align*}
        \abs{\hat{\omega}((\alpha, a))} = \abs{\alpha + \omega(a)} 
        &= \lim_{\lambda \in \Lambda}{\abs*{\alpha \omega(e_{\lambda}^2) + \omega(a e_{\lambda}^2)}}\\
        &\leq \limsup_{\lambda \in \Lambda}{\norm*{\alpha e_{\lambda}^2 + a e_{\lambda}^2}}\\
        &= \limsup_{\lambda \in \Lambda}{\norm*{(\alpha, a) (0, e_{\lambda}^2)}}\\
        &\leq \norm{(\alpha, a)}
    \end{align*}
    for all \((\alpha, a) \in \mathbb{C} \ltimes \algebra{A}\), and we conclude \(\norm{\hat{\omega}} = 1\). In either case, we may assume \(\algebra{A}\) is unital with \(\omega(\unity) = \norm{\omega} = 1.\)

    Consider the self-adjoint operator \(a \in \algebra{A} \setminus \set{0}\), then the operator \(a + i\alpha \unity\) is normal for all \(\alpha \in \mathbb{R}\) with spectrum \(\sigma(a + i \alpha \unity) = \setc{\lambda + i \alpha}{\lambda \in [-\norm{a}, \norm{a}]}\). By \cref{thm:spectral_radius_cstar} we have
    \begin{equation*}
        \norm{a + i \alpha \unity} = r(a + i \alpha \unity) = \sqrt{\norm{a}^2 + \alpha^2}
    \end{equation*}
    and thus
    \begin{equation*}
        \abs{\Im\left(\omega(a)\right) + \alpha} = \abs{\Im\left(\omega(a + i \alpha \unity)\right)} \leq \abs{\omega(a + i \alpha \unity)} \leq \norm{a + i \alpha \unity} = \sqrt{\norm{a}^2 + \alpha^2}
    \end{equation*}
    for all \(\alpha \in \mathbb{R}\). This yields
    \begin{equation*}
        \Im\left(\omega(a)\right)^2 \leq \norm{a}^2 - 2 \alpha \Im\left(\omega(a)\right)
    \end{equation*}
    for all \(\alpha \in \mathbb{R}\), hence we must have \(\omega(a) \in \mathbb{R}\). In particular, this means \(\omega(\algebra{A}_+)\) is real. Let \(a \in \algebra{A}_+\setminus{0}\), then by \cref{thm:square_root_cstar} we have
    \begin{equation*}
        \norm*{\unity - \norm{a}^{-1}a} \leq 1,
    \end{equation*}
    hence
    \begin{equation*}
        \abs{1 - \norm{a}^{-1} \omega(a)} = \abs*{\omega\left(\unity - \norm{a}^{-1}a\right)} \leq \norm{\unity - \norm{a}^{-1}a} \leq 1
    \end{equation*}
    which yields \(\omega(a) \in [0, \norm{a}]\). In particular, we have shown \(\omega(\algebra{A}_+) \subset [0, \infty)\), and thus \(\omega\) is a state, concluding the proof.
\end{proof}
\begin{proposition}{Properties of positive linear functionals over C*-algebras}{state_properties}
    Let \(\algebra{A}\) be a C*-algebra. If \(\omega : \algebra{A} \to \mathbb{C}\) is a positive linear functional, then
    \begin{enumerate}[label=(\alph*)]
        \item \(\omega(a^*) = \conj{\omega(a)}\);
        \item \(\abs{\omega(a)}^2 \leq \omega(a^*a)\norm{\omega}\);
        \item \(\abs{\omega(a^*ba)} \leq \omega(a^*a) \norm{b}\); and
        \item \(\norm{\omega} = \sup\setc{\omega(a^*a)}{a \in \algebra{A} : \norm{a} = 1}\);
    \end{enumerate}
    for all \(a, b \in \algebra{A}\).
\end{proposition}
\begin{proof}
Let \(a, b \in \algebra{A}\)  and \(e : \Lambda \to \algebra{A}_+\) be an approximate identity on \(\algebra{A}\). \cref{prop:sesquilinear_positive_form} yields
    \begin{equation*}
        \omega(a^* e_\lambda) = \conj{\omega(e_\lambda a)}
        \quad\text{and}\quad
        \abs{\omega(e_\lambda a)}^2 \leq \omega(e_\lambda^2) \omega(a^*a)
    \end{equation*}
    hence we conclude (a) and (b) by taking the limit and using continuity. The Cauchy-Schwarz inequality also informs us that
    \begin{equation*}
        \abs{\omega(a^*ba)}^2 \leq \omega(a^*a) \omega(a^*b^*ba),
    \end{equation*}
    and the \cref{prop:properties_order,prop:congruence_order} yield
    \begin{equation*}
        \todo[b^*b \leq \norm{b}^2 \unity \implies] a^* b^*b a \leq \norm{b}^2 a^*a \implies \omega(a^*b^*ba) \leq \norm{b}^2 \omega(a^*a),
    \end{equation*}
    hence we conclude (c). 

    Let \(m = \sup\setc{\omega(a^*a)}{a \in \algebra{A} : \norm{a} = 1},\) then clearly \(m \leq \norm{\omega}\). From (b) we have
    \begin{equation*}
        m = \sup_{\substack{a \in \algebra{A}\\\norm{a} = 1}}{\omega(a^*a)} \geq \norm{\omega}^{-1} \sup_{\substack{a \in \algebra{A}\\\norm{a} = 1}} \abs{\omega(a)}^2 = \norm{\omega},
    \end{equation*}
    hence \(m = \norm{\omega}\), and we conclude (d).
\end{proof}

\begin{proposition}{States over a C*-algebra form a convex set}{states_convex}
    Let \(\algebra{A}\) be a C*-algebra. If \(\omega_1, \omega_2 \in \algebra{A}^\dag\) are positive linear functionals over \(\algebra{A}\), then so is \(\omega_1 + \omega_2\), which satisfies \(\norm{\omega_1 + \omega_2} = \norm{\omega_1} + \norm{\omega_2}\). The set \(E_{\algebra{A}}\) of states over \(\algebra{A}\) for a convex subset of \(\algebra{A}^\dag\).
\end{proposition}
\begin{proof}
    Let \(e : \Lambda \to \algebra{A}_+\) be an approximate identity on \(\algebra{A}\), then 
    \begin{equation*}
        \norm{\omega_1 + \omega_2} = \lim_{\lambda \in \Lambda}{\left[\omega_1(e_\lambda^2) + \omega_2(e_\lambda^2)\right]} = \lim_{\lambda \in \Lambda}{\omega_1(e_\lambda^2)} + \lim_{\lambda \in \Lambda}{\omega_2(e_\lambda^2)} = \norm{\omega_1} + \norm{\omega_2}
    \end{equation*}
    and the positivity of \(\omega_1 + \omega_2\) follows trivially. Let \(\alpha \in [0,1],\) and suppose \(\omega_1\) and \(\omega_2\) are states, then \(\alpha \omega_1 + (1 - \alpha) \omega_2\) is positive with
    \begin{equation*}
        \norm{\alpha \omega_1 + (1 - \alpha)\omega_2} = \alpha \norm{\omega_1} + (1 - \alpha) \norm{\omega_2} = 1,
    \end{equation*}
    hence \(\alpha \omega_1 + (1 - \alpha) \omega_2\) is a state.
\end{proof}

Similar to positive elements in a C*-algebra \(\algebra{A}\), the set of positive linear functionals has a natural order. Indeed, if \(\omega_1, \omega_2 \in \algebra{A}^\dag\) are positive linear functionals, then we write \(\omega_1 \geq \omega_2\) whenever \(\omega_1 - \omega_2\) is a positive linear functional and say \(\omega_1\) \emph{majorizes} \(\omega_2\). If \(\omega_1, \omega_2\) are states and \(\alpha \in [0,1]\), then \(\omega = \alpha \omega_1 + (1 - \alpha) \omega_2\) is a state with \(\omega \geq \alpha \omega_1\) and \(\omega \geq (1 - \alpha) \omega_2\).
\begin{definition}{Pure states}{pure_state}
    Let \(\algebra{A}\) be a C*-algebra. A state \(\omega \in \algebra{A}^\dag\) is \emph{pure} if the only positive linear functionals majorized by \(\omega\) are of the form \(\alpha \omega\) with \(\alpha \in [0,1]\). The set of all pure states is denoted by \(P_{\algebra{A}}\).
\end{definition}

%TODO: topology of states and pure states.

\begin{proposition}{States defined by a representation of a C*-algebra}{vector_state}
    Let \(\algebra{A}\) be a C*-algebra. If \((\hilbert, \pi)\) is a representation and \(\Omega \in \hilbert\) is a non-zero vector, then
    \begin{align*}
        \omega_{\Omega} : \algebra{A} &\to \mathbb{C}\\
                                    a &\mapsto \inner{\Omega}{\pi(a)\Omega}
    \end{align*}
    is a positive linear functional. This linear functional is a state if \(\norm{\Omega} = 1\) and \(\pi\) is non-degenerate.
\end{proposition}
\begin{proof}
    It is clear that \(\omega_{\Omega}\) is a linear functional, but it is positive since
    \begin{equation*}
        \omega_{\Omega}(a^*a) = \inner{\Omega}{\pi(a^*a)\Omega} = \inner{\pi(a)\Omega}{\pi(a)\Omega} = \norm{\pi(a)\Omega}^2 \geq 0
    \end{equation*}
    for all \(a \in \algebra{A}\).
\end{proof}

% vim: spl=en_us
\section{Construction of representations}
As an application of the Hahn-Banach theorem, we can show the existence of states and, thus, of positive linear functionals on a C*-algebra.
\begin{proposition}{Existence of states}{states_existence}
    Let \(\algebra{A}\) be a C*-algebra and let \(a \in \algebra{A}\) be an operator. There exists a state \(\omega_a \in P_{\algebra{A}}\) such that \(\omega_a(a^*a) = \norm{a}^2\).
\end{proposition}
\begin{remark}
    In fact, such a state is a pure state.
\end{remark}
\begin{proof}
    If \(\algebra{A}\) does not have an identity, adjoin one. Consider the linear subspace \(B = \lspan\set{\unity, a^*a}\) of normal operators and its linear functional
    \begin{align*}
        f : B &\to \mathbb{C}\\
        \alpha \unity + \beta a^*a &\mapsto \alpha + \beta \norm{a}^2,
    \end{align*}
    which satisfies \(f(a^*a) = \norm{a}^2\) and \(f(\unity) = 1.\) Let \(\alpha \unity + \beta a^*a \in B\), then
    \begin{equation*}
        \abs{f(\alpha \unity + \beta a^*a)} = \abs{\alpha + \beta \norm{a}^2} \leq \sup_{\lambda \in \sigma(a)}{\abs{\alpha + \beta \lambda}} = r(\alpha \unity + \beta a^*a) = \norm{\alpha \unity + \beta a^*a},
    \end{equation*}
    that is, there exists \(m \in [0,1]\) such that \(\abs{f(b)}\leq m \norm{b}\) for all \(b \in B\). The \nameref{thm:Hahn_Banach_normed} guarantees the existence of a bounded linear functional \(\omega : \algebra{A} \to \mathbb{C}\) that extends \(f\) with \(\norm{\omega} = m\). As \(\omega(\unity) = f(\unity) = 1\) and \(\omega\) is continuous, then \(\norm{\omega} = 1\) and \(\omega\) is a state by \cref{prop:state_continuous}.
\end{proof}

Moreover, given a representation of a C*-algebra, there exists a \emph{vector state} which uses the structure of the representation. Its definition is shown in the
\begin{proposition}{States defined by a representation of a C*-algebra}{vector_state}
    Let \(\algebra{A}\) be a C*-algebra. If \((\hilbert, \pi)\) is a representation and \(\Omega \in \hilbert\) is a non-zero vector, then
    \begin{align*}
        \omega_{\Omega} : \algebra{A} &\to \mathbb{C}\\
                                    a &\mapsto \inner{\Omega}{\pi(a)\Omega}
    \end{align*}
    is a positive linear functional. This linear functional is a state if \(\norm{\Omega} = 1\) and \(\pi\) is non-degenerate, called a \emph{vector state}. 
\end{proposition}
\begin{proof}
    It is clear that \(\omega_{\Omega}\) is a linear functional, but it is positive since
    \begin{equation*}
        \omega_{\Omega}(a^*a) = \inner{\Omega}{\pi(a^*a)\Omega} = \inner{\pi(a)\Omega}{\pi(a)\Omega} = \norm{\pi(a)\Omega}^2 \geq 0
    \end{equation*}
    for all \(a \in \algebra{A}\). If \(e : \Lambda \to \algebra{A}_+\) is an approximate identity on \(\algebra{A}\) we have by \cref{prop:state_continuous} that
    \begin{equation*}
        \norm{\omega_\Omega} = \lim_{\lambda \in \Lambda}{\omega_\Omega(e_\lambda^2)} = \lim_{\lambda \in \Lambda}{\norm{\pi(e_{\lambda})\Omega}^2}.
    \end{equation*}
    If \(\pi\) is non-degenerate we know by \cref{prop:representation_approximate_identity} that \(\pi \circ e\) is an approximate identity in \(\bounded(\hilbert)\) with respect to the strong operator topology, hence \(\pi(e_\lambda)\Omega\) converges against \(\Omega\). Hence \(\omega_\Omega\) is a state if \(\pi\) is non degenerate and \(\norm{\Omega} = 1\).
\end{proof}

We have not shown the existence of representations, and we will prove by concluding every state of a C*-algebra is a vector state of some cyclic representation. 
\begin{theorem}{Canonical cyclic representation}{gns_construction}
    Let \(\algebra{A}\) be a C*-algebra. If \(\omega \in E_\algebra{A}\) is a state, then there exists a cyclic representation \((\hilbert_\omega, \pi_\omega, \Omega_\omega)\) such that 
    \begin{equation*}
        \omega(a) = \inner{\Omega_\omega}{\pi_{\omega}(a)\Omega_\omega}_{\hilbert_{\omega}}
    \end{equation*}
    for all \(a \in \algebra{A}\). Up to unitary equivalence, this cyclic representation is unique.
\end{theorem}
\begin{remark}
    This representation is referred to as \emph{GNS construction}, due to Gelfand, Naimark, and Segal. The triple \((\hilbert_{\omega}, \pi_{\omega}, \Omega_{\omega})\) is referred to as the \emph{GNS triple} associated with the state \(\omega\) over the C*-algebra \(\algebra{A}\).
\end{remark}
\begin{proof}
    We consider the set \(\algebra{J}_\omega = \setc{a \in \algebra{A}}{\omega(a^*a) = 0} \subset \setc{a \in \algebra{A}}{\forall b \in \algebra{A} : \omega(b^*a) = 0}\). Let \(a \in \algebra{J}_\omega\) and \(b \in \algebra{A}\), then \(\abs{\omega(b^*a)}^2 \leq \omega(a^*a) \omega(b^*b) = 0,\) hence \(\omega(b^*a) = 0\) and we conclude
    \begin{equation*}
        \algebra{J}_\omega = \setc{a \in \algebra{A}}{\forall b \in \algebra{A} : \omega(b^*a) = 0}.
    \end{equation*}
    Let \(a, b \in \algebra{J}_\omega\) and \(\lambda \in \mathbb{C}\), then
    \begin{equation*}
        \forall c \in \algebra{A} : \omega(c^*a) + \lambda \omega(c^*b) = 0 \implies \forall c \in \algebra{A} : \omega(c^*(a + \lambda b)) = 0 \implies a + \lambda b \in \algebra{J}_\omega,
    \end{equation*}
    hence \(\algebra{J}_\omega\) is a linear subspace of \(\algebra{A}\). Let \(a : \mathbb{N} \to \algebra{J}_\omega\) be a convergent sequence to some \(\tilde{a} \in \algebra{A},\) then
    \begin{equation*}
        \omega(b^*\tilde{a}) = \lim_{n \to \mathbb{N}}{\omega(b^*a_n)} = 0
    \end{equation*}
    for all \(b \in \algebra{A},\) hence \(\tilde{a} \in \algebra{J}_\omega\) and we conclude \(\algebra{J}_\omega\) is closed. Furthermore, \(\algebra{J}_\omega\) is a left ideal of \(\algebra{A}\) since 
    \begin{equation*}
        \omega((ba)^*(ba)) = \omega((a^*b^*b)a) = 0
    \end{equation*}
    for all \(b \in \algebra{A}\) and \(a \in \algebra{J}_\omega\). 

    We now consider the linear space on the coset \(\algebra{A}/\algebra{J}_\omega\). Let \([a], [b] \in \algebra{A}/\algebra{J}_\omega\) and let \(\tilde{a} \in [a]\) and \(\tilde{b} \in [b]\), then there exist \(j_a, j_b \in \algebra{J}_\omega\) such that \(\tilde{a} = a + j_a\) and \(\tilde{b} = b + j_b\), hence
    \begin{equation*}
        \omega(\tilde{a}^*\tilde{b}) = \omega(\tilde{a}^*(b + j_b)) = \omega(\tilde{a}^*b) = \omega((a + j_a)^*b) = \omega(a^*b).
    \end{equation*}
    We have thus shown the map
    \begin{align*}
        \inner{\noarg}{\noarg} : \algebra{A}/\algebra{J}_\omega \times \algebra{A}/\algebra{J}_\omega &\to \mathbb{C}\\
        ([a],[b]) &\mapsto \omega(a^*b)
    \end{align*}
    is well-defined. We have already established this is a positive sesquilinear form, so \((\algebra{A}/\algebra{J}_\omega, \inner{\noarg}{\noarg})\) is an inner product space as we have
    \begin{equation*}
        \inner{[a]}{[a]} = 0 \implies \omega(a^*a) = 0 \implies a \in \algebra{J}_\omega \implies [a] = [0].
    \end{equation*}
    \todo[We define \(\hilbert_\omega\) as the canonical completion of \(\algebra{A}/\algebra{J}_\omega\),] and we'll denote the embedding of \(\algebra{A}/\algebra{J}_\omega\) in \(\hilbert_\omega\) by the linear isometry
    \begin{align*}
        \Psi : \algebra{A} &\to \hilbert_\omega\\
                         a &\mapsto \Psi_{[a]}
    \end{align*}
    with \(\preim{\Psi}{\set{\Psi_{[a]}}} = [a]\) for all \(a \in \algebra{A}\).

    Let \(a,b \in \algebra{A}\), and \(\tilde{b} \in [b]\), then there exists \(j \in \algebra{I}_\omega\) and we have
    \begin{equation*}
        \Psi(a \tilde{b}) = \Psi(ab + aj) = \Psi(ab)
    \end{equation*}
    since \(\algebra{I}_\omega\) is a left ideal. This means for all \(a \in \algebra{A}\) the linear map
    \begin{align*}
        p_a : \Psi(\algebra{A}) &\to \Psi(\algebra{A})\\
                         \Psi_b &\mapsto \Psi_{ab}
    \end{align*}
    is well-defined. Let \(b \in \algebra{A} \setminus \algebra{I}_\omega\), then \(\norm{b} \neq 0\) and we have
    \begin{equation*}
        \norm{p_a \Psi_b}^2 = \norm{\Psi_{ab}}^2 = \omega(b^*a^*a b) \leq \omega(b^*b) \norm{a^*a} = \norm{a}^2\norm{\Psi_b}^2
    \end{equation*}
    by \cref{prop:state_properties}, hence \(p_a\) is bounded and densely defined. We thus define the map
    \begin{align*}
        \pi_{\omega} : \algebra{A} &\to \bounded(\hilbert_\omega)\\
                                 a &\mapsto \pi_\omega(a),
    \end{align*}
    where, for all \(a \in \algebra{A}\), \(\pi_\omega(a)\) is the bounded linear extension of \(p_a\) to \(\hilbert_\omega\) defined in \cref{thm:blt}.
\end{proof}

% \appendix
% \include{appendix_01}
\end{document}

% vim: spl=en_us
\section{Closed graph theorem}
We recall the definition of a map and its graph.
\begin{definition}{Map and graph}{graph}
    Let \(S\) be a set and let \(T\) be a non-empty set. A \emph{map} \(f\) is a triple \((S, T, \graph{f})\), where the \emph{graph \(\graph{f} \subset S \times T\) of \(f\)} satisfies
    \begin{enumerate}[label=(\alph*)]
        \item for all \(s \in S\) and for all \(t_1, t_2 \in T\), if \((s, t_1) \in \graph{f}\) and \((s, t_2) \in \graph{f}\), then \(t_1 = t_2\); and
        \item for all \(s \in S\), there exists \(t \in T\) such that \((s, t) \in \graph{f}\).
    \end{enumerate}
    By abuse of notation we write \((s, t) \in f\) and, in view of (a), denote this by \(f(s) = t\). For this reason, by abuse of notation we denote the map by \(f : S \to T\).
\end{definition}
Note that given non-empty sets \(S, T\) and a non-empty collection \(G \subset S \times T\) satisfying the properties of a graph, then there exists a map \(f\) with graph \(G\) such that the diagram
\begin{equation*}
    \begin{tikzcd}[column sep = large, row sep = large]
        G \arrow{rd}{\pi_2} \arrow[swap]{d}{\pi_1} &\\
        S \arrow[swap]{r}{f} & T
    \end{tikzcd}
\end{equation*}
commutes, where \(\pi_1 : S \times T \to S\) and \(\pi_2 : S\times T \to T\) are the coordinate projections, that is, \(f \circ \pi_1 = \pi_2\).

An important consequence of the open mapping theorem is the closed graph theorem, which relates the continuity of a map \(T : X \to Y\) with its graph \(\graph{T}\subset X \times Y\) and a topology on \(X \times Y\).

\begin{proposition}{Topological direct sum}{direct_sum}
    Let \((X, \norm{\noarg}_X), (Y, \norm{\noarg}_Y)\) be normed linear spaces. The topological direct sum \(X \oplus Y\) is the normed linear space \((X \times Y, \norm{\noarg}_{X \times Y})\), where addition, scalar multiplication, and norm are defined by
    \begin{align*}
        +_{X \times Y} : (X \times Y) \times (X \times Y) &\to X \times Y\\
        \left((x_1,y_1),(x_2,y_2)\right)&\mapsto (x_1 +_X x_2, y_1 +_Y y_2),
    \end{align*}
    \begin{align*}
        \cdot_{X \times Y} : \mathbb{C} \times (X \times Y) &\to X \times Y\\
        \left(\alpha,(x,y)\right)&\mapsto (\alpha \cdot_X x, \alpha \cdot_Y y),
    \end{align*}
    and
    \begin{align*}
        \norm{\noarg}_{X \times Y} : (X \times Y) &\to \mathbb{R}\\
        (x,y)&\mapsto \norm{x}_X + \norm{y}_Y.
    \end{align*}
    Moreover, if \((X, \norm{\noarg}_X)\) and \((Y, \norm{\noarg}_Y)\) are Banach spaces, then so is \(X \oplus Y\).
\end{proposition}
\begin{proof}
    We first show \(X \oplus Y\) is a normed linear space. It is trivial to verify the linear space axioms are satisfied since \(X\) and \(Y\) are linear spaces, with the main insight being \((0,0) \in X\times Y\) is unsurprisingly the neutral element of addition. Let us verify \(\norm{\noarg}_{X\ \times Y}\) is indeed a norm. Since the norms on \(X\) and \(Y\) are non-negative, it is clear that \(\norm{\noarg}_{X\times Y}\) is non-negative. It is also positive-definite since
    \begin{align*}
        \norm{(x, y)}_{X \times Y} = 0 &\iff \norm{x}_X + \norm{y}_{Y} = 0\\
                                       &\iff \norm{x}_X = 0 \land \norm{y}_Y = 0\\
                                       &\iff x = 0 \land y = 0\\
                                       &\iff (x,y) = (0,0).
    \end{align*}
    Absolute homogeneity follows from the absolute homogeneity of the norms on \(X\) and \(Y\) as
    \begin{align*}
        \norm{\lambda (x, y)}_{X \times Y} = \norm{\lambda x}_X + \norm{\lambda y}_Y = \abs{\lambda} \left(\norm{x}_X + \norm{y}_Y\right) = \abs{\lambda} \norm{(x,y)}_{X \times Y}
    \end{align*}
    holds for all \(\lambda \in \mathbb{C}\) and all \((x,y) \in X \times Y\). Finally, let \((x_1, y_1), (x_2, y_2) \in X\times Y\), then
    \begin{align*}
        \norm{(x_1, y_1) + (x_2, y_2)}_{X \times Y} &= \norm{x_1 + x_2}_X + \norm{y_1 + y_2}_Y\\ &\leq \left(\norm{x_1}_X + \norm{y_1}_Y\right) + \left(\norm{x_2}_X + \norm{y_2}_Y\right) \\&= \norm{(x_1, y_1)}_{X \times Y} + \norm{(x_2, y_2)}_{X\times Y},
    \end{align*}
    hence \(\norm{\noarg}_{X\times Y}\) is subadditive, and we conclude \(X \oplus Y\) is indeed a normed linear space.

    Suppose \(X\) and \(Y\) are Banach spaces and let \(\family{v_n}{n \in \mathbb{N}} \subset X \oplus Y\) be a Cauchy sequence in \(X \oplus Y\). For each \(n \in \mathbb{N}\), there exists \(x_n \in X\) and \(y_n \in Y\) such that \(v_n = (x_n, y_n)\). Let \(\varepsilon > 0\), then there exists \(N \in \mathbb{N}\) such that for all \(n, m \geq N\) we have \(\norm{v_n - v_m}_{X \times Y} < \varepsilon\), hence \(\norm{x_n - x_m}_X + \norm{y_n - y_m}_Y < \varepsilon\). Since the norms are non-negative, we have \(\norm{x_n - x_m}_X < \varepsilon\) and \(\norm{y_n - y_m} < \varepsilon\), thus showing \(\family{x_n}{n \in \mathbb{N}} \subset X\) and \(\family{y_n}{n \in \mathbb{N}}\) are Cauchy sequences. By completeness, there exists \(\tilde{x} \in X\) and \(\tilde{y} \in Y\) such that \(x_n \to \tilde{x}\) and \(y_n \to \tilde{y}\).

    We consider \(\tilde{v} = (\tilde{x}, \tilde{y}) \in X \oplus Y\) and let \(\eta > 0\). By convergence, there exists \(M_X, M_Y \in \mathbb{N}\) such that \(\norm{\tilde{x} - x_n}_X < \frac12 \eta\) for all \(n > M_X\) and such that \(\norm{\tilde{y} - y_n}_Y < \frac12 \eta\) for all \(n > M_Y\). Then, we set \(M = \max\set{M_X, M_Y}\) such that for all \(n > M\) we have
    \begin{equation*}
        \norm{\tilde{v} - v_n}_{X\times Y} = \norm{\tilde{x} - x_n}_X + \norm{\tilde{y} - y_n}_Y < \eta,
    \end{equation*}
    that is, \(v_n \to \tilde{v}\), hence \(X \oplus Y\) is complete.
\end{proof}

\begin{proposition}{Graph of a linear operator is a linear subspace}{graph_linear_subspace}
    Let \(X, Y\) be normed linear spaces. If \(T : \domain{T} \subset X \to Y\) is a linear operator, then \(\graph{T}\) is a linear subspace of \(X \oplus Y\).
\end{proposition}
\begin{proof}
    Since \(\domain{T}\) is a linear subspace of \(X\), then the graph \(\graph{T}\) is a linear subspace of \(X \oplus Y\). Indeed, let \(v_1, v_2 \in \graph{T}\), then there exists \(x_1, x_2 \in \domain{T}\) such that \(v_1 = (x_1, Tx_1)\) and \(v_2 = (x_2, Tx_2)\), hence for all \(\alpha \in \mathbb{C}\) we have \(v_1 + \alpha v_2 = (x_1 + \alpha x_2, T(x_1 + \alpha x_2)) \in \graph{T}\).
\end{proof}
\begin{proposition}{Coordinate projections are bounded operators}{coordinate_projection_bounded}
    Let \(X_1, X_2\) be normed linear spaces. The coordinate projections \(\pi_i : X_1 \oplus X_2 \to X_i\) are bounded linear operators, with \(i \in \set{1,2}.\)
\end{proposition}
\begin{proof}
    Notice that for all \(w \in X_1 \oplus X_2\), we have \(w = (\pi_1(w), \pi_2(w))\) by the definitions of the topological direct sum and the coordinate projections. Let \(u, v \in X_1 \oplus X_2\) and \(\alpha, \beta \in \mathbb{C}\), then
    \begin{equation*}
        \pi_i(\alpha u + \beta v) = \pi_i\left(\alpha \pi_1(u) + \beta \pi_1(v), \alpha \pi_2(u) + \beta \pi_2(v)\right) = \alpha \pi_i(u) + \beta \pi_i(v),
    \end{equation*}
    hence \(\pi_i\) is linear, with \(i \in \set{1,2}\). Moreover, for all \(w \in X_1 \oplus X_2\) we have
    \begin{equation*}
        \norm{\pi_iw}_{X_i} \leq \norm{\pi_1 w}_{X_1} + \norm{\pi_2 w}_{X_2} = \norm{w}_{X_1\oplus X_2},
    \end{equation*}
    hence \(\pi_i \in \bounded(X_1 \oplus X_2, X_i)\).
\end{proof}
\begin{proposition}{Coordinate projections restricted to a graph}{coordinate_projection_graph}
    Let \(X_1, X_2\) be normed linear spaces and let \(T : X_1 \to X_2\) be a linear operator. Then the restricted coordinate projection \(\restrict{\pi_i}{\graph{T}} : X_1 \oplus X_2 \to X_i\) is a bounded linear operator for \(i \in \set{1,2}\). In addition, \(\restrict{\pi_1}{\graph{T}}\) is bijective.
\end{proposition}
\begin{proof}
    By the previous result, we conclude \(\restrict{\pi_i}{\graph{T}}\) is linear. Moreover, \(\pi_i\) clearly extends \(\restrict{\pi_i}{\graph{T}}\), then we have
    \begin{equation*}
        \sup_{w \in \graph{T}} \frac{\norm{\restrict{\pi_i}{\graph{T}}w}_{X_i}}{\norm{w}_{\graph{T}}} \leq \sup_{w \in X_1 \oplus X_2} \frac{\norm{\pi_i w}_{X_i}}{\norm{x}_{X_1 \oplus X_2}} < \infty,
    \end{equation*}
    hence \(\restrict{\pi_i}{\graph{T}} \in \bounded(\graph{T}, X_i)\). By properties (a) and (b) of \cref{def:graph}, it follows that \(\restrict{\pi_1}{\graph{T}}\) is bijective. Indeed, (b) states that \(\restrict{\pi_1}{\graph{T}}(\domain{T}) = \domain{T}\) and (a) states that \(\restrict{\pi_1}{\graph{T}}(w) = \restrict{\pi_1}{\graph{T}}(v)\) implies \(w = v\).
\end{proof}

We may describe a linear operator \(T : \domain{T} \subset X \to Y\) in terms of the standing of its graph \(\graph{T}\) in the metric topology on \(X \oplus Y\).
\begin{definition}{Closed linear operator}{closed_linear_operator}
    Let \(X, Y\) be normed linear spaces. A linear operator \(T : \domain{T} \subset X \to Y\) is \emph{closed} if its graph \(\graph{T} \subset \domain{T} \times Y\)  is a closed linear subspace of \(X \oplus Y\).
\end{definition}


Recall \cref{exam:derivative_unbounded}, where we presented the derivative operator as an unbounded linear operator on the Banach space of continuous complex-valued functions. We now show this operator is also closed.
\begin{example}{The derivative operator of continuous complex-valued functions is closed}{derivative_closed}
    Let \(Y = \mathcal{C}([0,1]; \mathbb{C})\) be the Banach space of continuous complex-valued functions equipped with the sup norm and let the set \(X = \mathcal{C}^1([0,1];\mathbb{C})\subset Y\) of continuously differentiable complex-valued functions in the interval \([0,1]\). The derivative operator
    \begin{align*}
        P : X \subset Y &\to Y\\
            f &\mapsto f'
    \end{align*}
    is closed.
\end{example}
\begin{proof}
    Let \(\family{v_n}{n \in \mathbb{N}} \subset \graph{T}\) be a convergent sequence with \(v_n \to (\tilde{x}, \tilde{y}) \in Y \oplus Y\). Then, there exists convergent sequences \(\family{x_n}{n \in \mathbb{N}} \subset X\) and \(\family{Tx_n}{n \in \mathbb{N}} \subset Y\) such that we have \(v_n = (x_n, Tx_n)\) for all \(n \in \mathbb{N}\) and such that \(x_n \to \tilde{x}\) and \(Tx_n = x'_n \to \tilde{y}\). Notice \(x'_n\) converges uniformly to \(\tilde{y}\), then \(\tilde{x}\) must be differentiable with \(\tilde{x}' = \tilde{y}\). That is, \((\tilde{x}, \tilde{y}) \in \graph{T}\), thus showing \(\graph{T}\) is closed in \(Y \oplus Y\).
\end{proof}

\begin{theorem}{Closed graph theorem}{closed_graph}
    Let \(X, Y\) be Banach spaces. A linear map \(T : X \to Y\) is continuous if and only if it is closed.
\end{theorem}
\begin{proof}
    Suppose \(T\) is continuous and let \(\family{v_n}{n \in \mathbb{N}}\subset \graph{T}\) be a sequence that converges to \(\tilde{v} \in X \oplus Y\). Then there exist convergent sequences \(\family{x_n}{n \in \mathbb{N}} \subset X\) and \(\family{Tx_n}{n \in \mathbb{N}} \subset Y\), such that \(v_n = (x_n, Tx_n)\) for all \(n \in \mathbb{N}\) and such that \(x_n \to \tilde{x}\) and \(Tx_n \to \tilde{y}\), where \(\tilde{v} = (\tilde{x}, \tilde{y})\). Since \(T\) is continuous, we have
    \begin{equation*}
        \tilde{y} = \lim_{n\to\infty} Tx_n = T\left(\lim_{n\to\infty} x_n\right)= T\tilde{x},
    \end{equation*}
    hence \(\tilde{v} \in \graph{T}\). That is, \(\graph{T}\) is closed in \(X \oplus Y\).

    Suppose \(T\) is closed. Then \(\graph{T}\) is a closed subset of the Banach space \(X \oplus Y\), hence \(\graph{T}\) is a Banach space. Let \(S_1 = \restrict{\pi_1}{\graph{T}}\) and \(S_2 = \restrict{\pi_2}{\graph{T}}\) be the restricted coordinate projections considered in \cref{prop:coordinate_projection_graph}, then \(S_1 \in \bounded(\graph{T}, X)\) is a homeomorphism, and \(S_2 \in \bounded(\graph{T}, Y)\) is a continuous map. Then, \(T = S_2 \circ S_1^{-1}\) is a composition of continuous linear operators, hence \(T \in \bounded(X, Y).\)
\end{proof}

\begin{proposition}{Closed subspace of \(\mathcal{C}([0,1])\) contained in \(\mathcal{C}^1([0,1])\) has finite dimension}{derivative_finite_dimension}
    Every closed subspace of \((\mathcal{C}([0,1], \mathbb{C}), \norm{\noarg}_\infty)\) contained in \(\mathcal{C}^1([0,1], \mathbb{C})\) has finite dimension. Furthermore, every infinite dimensional closed subspace of \(\mathcal{C}([0,1], \mathbb{C}, \norm{\noarg}_{\infty})\) has at least one function that is not differentiable or that has discontinuous derivative.
\end{proposition}
\begin{proof}
    We keep the notation from \cref{exam:derivative_closed}. Let \(W \subset X\) be a closed subspace of \(Y\), then by the previous result \(T = \restrict{P}{W} : W \to Y\) is closed. Since \((W, \restrict{\norm{\noarg}_\infty}{W})\) is a Banach space, \(T\) is bounded by the closed graph theorem.

    Let \(N = \inf\setc{n}{n \in \mathbb{N} \land n \geq \norm{T}}\) and consider the map
    \begin{align*}
        S : W &\to \mathbb{R}^{N + 1}\\
            f &\mapsto \sum_{n = 1}^{N+1} f\left(\frac{n-1}{N}\right) e_n,
    \end{align*}
    where \(\set{e_1, \dots e_{N+1}} \subset \mathbb{R}^{N+1}\) is the canonical basis of \(\mathbb{R}^{N+1}\). Notice \(S\) is linear, since
    \begin{equation*}
        S(f + \alpha g) = \sum_{n=1}^{N+1}(f + \alpha g)\left(\frac{n-1}{N}\right) e_n = \sum_{n=1}^{N+1} \left[f\left(\frac{n-1}{N}\right) + \alpha g\left(\frac{n-1}{N}\right)\right] e_n = Sf + \alpha Sg,
    \end{equation*}
    for all \(f, g \in W\) and \(\alpha \in \mathbb{C}\). Suppose, by contradiction, \(S\) is not injective, then there exists a nontrivial continuously differentiable function \(h \in W\setminus \set{0}\) such that \(h \in \ker{S}\). Since \(h\) is defined on a compact set, there exists \(x_* \in [0,1]\) such that \(\abs{h(x_*)} = \norm{h}_\infty > 0\). As \(Sh = 0\), we know \(h\left(\frac{n - 1}{N}\right) = 0\) for every \(n \in \set{1,2, \dots, N+1}\), then there exists \(m \in \set{1,2,\dots, N}\) such that
    \begin{equation*}
        \frac{m-1}{N} < x_* < \frac{m}{N} \implies 0 < x_* - \frac{m - 1}{N} < \frac{1}{N}.
    \end{equation*}
    The fundamental theorem of calculus yields
    \begin{equation*}
        \norm{h}_{\infty} = \abs{h(x_*)} = \abs*{\int_{\frac{m - 1}{N}}^{x_*} \dli{x} h'(x)} \leq \int_{\frac{m-1}{N}}^{x_*} \abs{h'(x)} \leq \norm{Th}_{\infty} \left(x_* - \frac{m-1}{N}\right) \leq \frac{\norm{T}}{N}\norm{h}_{\infty} \leq \norm{h}_\infty,
    \end{equation*}
    that is, \(\norm{h}_\infty < \norm{h}_\infty\). This contradiction shows the kernel of \(S\) is trivial, i.e., that \(S\) is injective. Since there is a linear injection from \(W\) to a finite dimensional linear space, we may conclude \(W\) is finite dimensional.

    Suppose \(V \subset Y\) is a infinite dimensional closed subspace of \(Y\). Then \(V\) is not contained in \(X\), otherwise it would be finite dimensional. That is, \(V \cap (Y \setminus X) = V \setminus X\) is non-empty, and we infer there exists \(f \in V\) such that \(f \notin X\), and the result follows.
\end{proof}

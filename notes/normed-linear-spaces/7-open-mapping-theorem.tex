% vim: spl=en_us
\section{Open mapping theorem}
Baire's category argument can also be used to show the open mapping theorem, which states surjective bounded maps defined on Banach spaces map open sets into open sets. To prove the theorem we need the following lemma.
\begin{lemma}{Image of a neighborhood of the zero vector}{}
    Let \(X,Y\) be linear normed spaces and let \(T \in \bounded(X,Y)\). If the range \(\range{T}\) is a set of the second category in \(Y\), then to each neighborhood \(U \in \tau_X\) of \(0\) there corresponds some neighborhood \(V \in \tau_Y\) of \(0\) such that \(V \subset \cl_Y(T(U))\).
\end{lemma}
\begin{proof}
    Let \(U \in \tau_X\) be a neighborhood of \(0\). Then, there exists an open ball \(W = B_r(0) \in \tau_X\) with radius \(r > 0\) centered at the zero vector and is contained in \(U\) such that for all \(u, v \in W\) we have \(u + v \in U\). For every \(x \in X\), there exists \(n_x \in \mathbb{N}\) such that \(x \in n_x W\). That is, we may write \(X = \bigcup_{n\in \mathbb{N}} nW\), hence \(\range{T} = \bigcup_{n\in \mathbb{N}} T(nW)\). \nameref{thm:Baire_Hausdorff}, therefore there exists \(m \in \mathbb{N}\) such that \(\cl_Y(T(mW))\) has non-empty interior.

    \cref{prop:scaling_subset} shows \(\cl_Y(T(mW)) = m \cl_Y(T(W))\). Since the map \(v \mapsto \frac1m v\) is a homeomorphism in any topological linear space, \todo[then \(S = \inte_Y\cl_Y(T(W))\) has non-empty interior.] Notice an element of \(S\) is also a point of closure of \(T(W)\), so \(S \cap T(W) \neq \emptyset\), as \(S\) is an open neighborhood for its points. Let \(y \in S \cap T(W)\), then there exists \(x \in W\) such that \(y = Tx\).

    % The map \(v \mapsto v - y\) is a homeomorphism, hence the image of \(S\) under this map is a neighborhood of \(0\)
\end{proof}

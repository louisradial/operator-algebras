% vim spl=en_us
\section{Topological linear spaces}
Throughout these notes we will consider linear spaces equipped with some topology.
\begin{definition}{Topological linear space}{tvs}
    Let \((X, +, \cdot)\) be a linear space over a field \(\mathbb{K}\) (either \(\mathbb{R}\) or \(\mathbb{C}\)). A \emph{vector topology} \(\tau\) is a topology on \(X\) such that
    \begin{enumerate}[label=(\alph*)]
        \item for all \(x \in X\), \(X \setminus \set{x} \in \tau\); and
        \item the linear space operations are continuous with respect to \(\tau\);
    \end{enumerate}
    and we say \((X, \tau)\) is a \emph{topological linear space}.
\end{definition}

Some basic results follow from requiring the linear space operations to be continuous. First, let \(S,T \subset X\) be a subset of a topological linear space \(X\), let \(x \in X\), and let \(\alpha \in \mathbb{K}\), then we denote
\begin{align*}
    S + T &= \setc{v \in X}{\exists s \in S, \exists t \in T : v = s + t} = \setc{s + t}{s \in S, t \in T},\\
    x + S &= \setc{v \in X}{\exists s \in S : v = x + s} = \setc{x + s}{s \in S},\quad\text{and}\\
    \alpha S &= \setc{v \in X}{\exists s \in S : v = \alpha s} = \setc{\alpha s}{s \in S}
\end{align*}
as the images of these subsets under the basic linear space operations.

\begin{proposition}{Characterization of open sets in topological linear spaces}{tvs}
    Let \(X\) be a linear space and let \(\tau\) be a topology on \(X\) such that the linear space operations are continuous. Let us denote the translation and multiplication maps by
    \begin{align*}
        T_x : X &\to X&
        M_\alpha : X &\to X\\
        y &\mapsto x + y&
              y &\mapsto \alpha y,
    \end{align*}
    where \(x \in X\) and \(\alpha \in \mathbb{K}\). Then
    \begin{enumerate}[label=(\alph*)]
        \item for all \(x \in X,\) the translation \(T_x\) is a homeomorphism;
        \item a subset \(S \subset X\) is open if and only if \(x + S\) is open for all \(x \in X\);
        \item for all \(\alpha \in \mathbb{K}\setminus\set{0}\), the multiplication \(M_\alpha\) is a homeomorphism; and
        \item a subset \(S \subset X\) is open if and only if \(\alpha S\) is open for all \(\alpha \in \mathbb{K} \setminus\set{0}\).
    \end{enumerate}
\end{proposition}
\begin{proof}
    Let \(x \in X\), then \(T_x\) is a bijection as it has the inverse \(T_{-x}\). Let \(\jmath_{x} : X \to \set{x} \times X\) be the continuous map defined by \(y \mapsto (x,y)\), then \(T_x = \restrict{+}{\set{x}\times X} \circ \jmath_{x}\) is continuous. As \(x\) is arbitrary, then the inverse \(T_{-x}\) is also continuous, therefore translations are homeomorphisms. Let \(S \subset X\), then
    \begin{equation*}
        S = \preim{T_{x}}{S + x} = T_{-x}(S + x)
    \end{equation*}
    hence \(S\) is open if and only if \(S + x\) is open for all \(x \in X\). An analogous argument shows (c) and (d).
\end{proof}


Since a set is open if and only if its translates are open, we may characterize a vector topology with a system of neighborhoods of a vector.
\begin{definition}{Local base}{local_basis}
    Let \((X,\tau)\) be a topological space. A \emph{local basis at a point \(x \in X\)} is a collection \(\mathcal{B} \subset \tau\) such that if \(U \in \tau\) is a neighborhood of \(x\), then there exists \(B \in \mathcal{B}\) such that \(B \subset U\).
\end{definition}
In the context of linear topological spaces, we will use only local basis at the zero vector.
\begin{proposition}{Open sets and local basis}{local_basis}
    Let \((X, \tau)\) be a topological linear space and let \(\mathcal{B} \subset \tau\) be a local basis. Then the translates of the elements of \(\mathcal{B}\) form a basis for \(\tau,\) that is, if \(U \in \tau,\) then there exists \(\family{x_{\lambda}}{\lambda \in \Lambda} \subset X\) and \(\family{B_\lambda}{\lambda \in \Lambda} \subset \mathcal{B}\) such that \(U = \bigcup_{\lambda \in \Lambda}{x_\lambda + B_\lambda}\).
\end{proposition}
\begin{proof}
    Let \(U\in \tau,\) then for every \(u \in U\) there exists \(V_u \in \tau\) with \(u \in V_u\) and \(V_u \subset U,\) hence \(U = \bigcup_{u \in U} V_u\). Moreover, each \(V_u - u\) is a neighborhood of zero, hence there exists \(B_u \in \mathcal{B}\) such that \(B_u \subset V_u - u,\) thus \(u \in B_u + u \subset V_u\) and we conclude \(U = \bigcup_{u \in U} u + B_u,\) as desired.
\end{proof}

In a linear space \(X,\) a subset \(S \subset X\) is \emph{balanced} if \(\alpha S \subset S\) for all \(\alpha \in \mathbb{K}\) with \(\abs{\alpha} \leq 1\); and it is \emph{convex} if \(\alpha S + (1 - \alpha) S \subset S\) for all \(\alpha \in [0,1].\) 
\begin{proposition}{Properties}{}
    Let \(X\) be a linear space and let \(\tau\) be a topology such that the linear space operations are continuous. Then
    \begin{enumerate}[label=(\alph*)]
        \item if \(S \subset X,\) then \(\cl_XS = \bigcap_{V \in \mathcal{V}} (S + V),\) where \(\mathcal{V} \subset \tau\) is the set of neighborhoods of zero;
        \item if \(S, T \subset X,\) then \(\cl_XS + \cl_XT \subset \cl_X(S + T)\);
        \item if \(S \subset X\) is a linear subspace, then so is \(\cl_XS\);
        \item if \(S \subset X\) is convex, then so are \(\cl_XS\) and \(\inte_XS\);
        \item if \(S \subset X\) is balanced, then so is \(\cl_XS\); and
        \item if \(S \subset X\) is balanced and \(0 \in S,\) then \(\inte_XS\) is balanced.
    \end{enumerate}
\end{proposition}
\begin{proof}
    Notice
    \begin{equation*}
        x \in \cl_XS \iff \forall V \in \mathcal{V} : (x + V) \cap S \neq \emptyset \iff \forall V \in \mathcal{V} : x \in S - V \iff x \in \bigcap_{V \in \mathcal{V}}(S + V),
    \end{equation*}
    and we conclude (a). From \cref{thm:closure_continuity} and the continuity of addition, (b) follows. 

    Suppose \(S\) is a linear subspace. As multiplication is a homeomorphism, we have \(\cl_X(\lambda S) = \lambda\cl_XS\) for all \(\lambda \in \mathbb{K}\), hence the continuity of addition yields
    \begin{equation*}
        \alpha \cl_XS + \beta \cl_XS = \cl_X(\alpha S) + \cl_X(\beta S) \subset \cl_X(\alpha S + \beta S) \subset \cl_X(S),
    \end{equation*}
    for all \(\alpha, \beta \in \mathbb{K}\). If we suppose instead that \(S\) is convex, we have similarly
    \begin{equation*}
        \alpha \cl_XS + (1 - \alpha) \cl_XS \subset \cl_X\left(\alpha S + (1 - \alpha) S\right) \subset \cl_X S
    \end{equation*}
    for all \(\alpha \in [0,1]\). If we suppose that \(S\) is balanced
\end{proof}

\subsection{Separation properties of a topological linear space}
Let us now show requiring the singletons to be closed is equivalent to requiring the topology to be Hausdorff.
\begin{lemma}{Every neighborhood of zero contains a symmetric neighborhood of zero}{symmetric_neighborhood}
    Let \(X\) be a linear space and let \(\tau\) be a topology on \(X\) such that the linear space operations are continuous. If \(U \in \tau\) is a neighborhood of zero, then there exists an open set \(W \in \tau\) such that
    \begin{enumerate}[label=(\alph*)]
        \item \(W\) is a neighborhood of zero;
        \item \(W + W \subset U\); and
        \item \(W = -W\).
    \end{enumerate}
\end{lemma}
\begin{proof}
    Notice \((0,0) \in \preim{+}{U}\), hence there exists neighborhoods of zero \(V_1, V_2 \in \tau\) such that \(V_1 + V_2 \subset U.\) As \(0 \in V_1 \cap V_2\) and \(0 \in (-V_1) \cap (-V_2)\), the non-empty set \(W = V_1 \cap V_2 \cap (-V_1) \cap (-V_2)\) is an open neighborhood of \(0\). It is clear this set satisfies the desired properties.
\end{proof}

\begin{lemma}{Existence of disjoint open sets that contain disjoint compact and closed sets}{tvs_hausdorff}
    Let \(X\) be a linear space and let \(\tau\) be a topology on \(X\) such that the linear space operations are continuous. Suppose \(K \subset X\) is compact, \(C \subset X\) is closed, and that \(K \cap C = \emptyset\). Then there exists a neighborhood of zero \(V \in \tau\) such that \((K + V) \cap (C + V) = \emptyset\) and \(\cl_X(K+V) \cap C = \emptyset\).
\end{lemma}
\begin{proof}
    We may assume without loss of generality that \(K\) is non-empty, for \(\empty + V = \emptyset\) and the lemma follows trivially. Let \(x \in K\), then there exists a neighborhood of \(x\) that does not intersect \(C,\) hence there exists a symmetric neighborhood of zero \(V_x \in \tau\) such that \(C \cap (x + V_x + V_x + V_x) = \emptyset\), by \cref{lem:symmetric_neighborhood}.

    Clearly the sets \(x + V_x\) form an open cover of \(K\), hence there exists \(N \in \mathbb{N}\) and a finite family \(\family{x_n}{n = 1}{N} \subset K\) such that \(\family{x_n + V_{x_n}}{n = 1}{N}\) is an open cover of \(K\). Let \(V = \bigcap_{n = 1}^{N} V_n\), then
    \begin{equation*}
        K + V \subset V + \bigcup_{n = 1}^N (x_n + V_{x_n}) = \bigcup_{n = 1}^N (x_n + V_{x_n} + V \subset \bigcup_{n = 1}^N (x_n + V_{x_n} + V_{x_n}),
    \end{equation*}
    hence
    \begin{equation*}
        (K + V)\cap(C + V) \subset (C + V) \cap \bigcup_{n = 1}^N (x_n + V_{x_n} + V_{x_n}) \subset \bigcup_{n = 1}^{N} C \cap (x_n + V_{x_n} + V_{x_n} + V_{x_n}) = \emptyset.
    \end{equation*} 
    Note \(C + V\) is open as we have \(C + V = \bigcup_{x \in C} (x + V)\), then we conclude \((C + V) \cap \cl_{X}(K + V) = \emptyset\), otherwise \(C + V\) would be a neighborhood for a point of closure of \(K + V\) that does not intersect \(K + V\). Furthermore, \(C \subset C + V\) does not intersect \(\cl_{X}(K+V)\).
\end{proof}

\begin{proposition}{Necessary and sufficient conditions for a vector topology}{vector_topology}
    Let \(X\) be a linear space with a topology \(\tau\) such that the linear space operations are continuous. The following statements are equivalent:
    \begin{enumerate}[label=(\alph*)]
        \item \((X, \tau)\) is Hausdorff;
        \item \((X, \tau)\) is a topological linear space; and
        \item for every \(x \in X\setminus\set{0}\), there exists an neighborhood of zero \(V \in \tau\) such that \(x \notin \cl_XV.\)
    \end{enumerate}
\end{proposition}
\begin{proof}
    Suppose \(\tau\) is Hausdorff. Let \(v \in X,\) and consider the set \(X \setminus \set{v}\). Let \(u \in X\setminus \set{v}\), then there exists \(S, T \in \tau\) such that \(u \in S\), \(v \in T\), and \(S \cap T = \emptyset\), hence \(u \in S \subset X \setminus \set{v}\), and we conclude \(X \setminus \set{v}\) is open, as desired.

    Suppose \(\tau\) is a vector topology and let \(x \in X\setminus \set{0}\). As \(\set{x}\) is closed and \(\set{0}\) is compact, then by \cref{lem:tvs_hausdorff} there exists a neighborhood \(V \in \tau\) of zero such that \(\set{x} \cap (\set{0} + \cl_XV) = \emptyset,\) hence \(x \notin \cl_XV.\)

    Suppose for every nonzero vector there exists an open neighborhood of zero that does not contain it and let \(u, v \in X\) with \(u \neq v\). Then there exists \(U \in \tau\) such that \(v - u \notin \cl_XU\) and \(0 \in U\), hence \(v \notin u + \cl_XU \subset \cl_X(u + U)\) by \cref{prop:tvs,thm:closure_continuity}. In particular, there exists a neighborhood \(V \in\ \tau\) of \(v\) that does not intersect \(u + U\), which is a neighborhood of \(U\), therefore \(\tau\) is Hausdorff.
\end{proof}

The \cref{lem:tvs_hausdorff} also lets us infer a result about local basis.
\begin{proposition}{Every element of a local basis contains the closure of some basic element}{tvs_local_basis}
    Let \((X, \tau)\) be a topological linear space and let \(\mathcal{B} \subset \tau\) be a local basis. If \(U \in \mathcal{B}\), then there exists \(V \in \mathcal{B}\) such that \(\cl_X V \subset U\).
\end{proposition}
\begin{proof}
    Let \(U \in \mathcal{B}\), then by \cref{lem:tvs_hausdorff} there exists \(W \in \tau\) such that \(0 \in W\) and \(\cl_XW \cap (X\setminus U) = \emptyset\). That is, \(\cl_X W \subset U\). As \(W\) is a neighborhood of \(0,\) there exists \(V \in \mathcal{B}\) such that \(V \subset W,\) therefore \(\cl_X V \subset \cl_X W \subset U\).
\end{proof}

\subsection{Topology generated by a family of semi-norms on a linear space}
We begin by studying semi-norms defined on a linear space.
\begin{definition}{Seminorm on a linear space}{seminorm}
    Let \(V\) be a linear space over \(\mathbb{K}\). A map \(p : V \to \mathbb{R}\) satisfying
    \begin{enumerate}[label=(\alph*)]
        \item subadditivity: \(p(u + v) \leq p(u) + p(v)\);
        \item absolute homogeneity: \(p(\alpha v) = \abs{\alpha} p(v)\);
    \end{enumerate}
    for all \(u,v \in V\) and \(\alpha \in \mathbb{K}\), is a \emph{semi-norm on \(V\)}.
\end{definition}

\begin{proposition}{Semi-norms are positive}{seminorm_positive}
    Let \(V\) be a linear space. A semi-norm \(p : V \to \mathbb{R}\) satisfies 
    \begin{enumerate}[label=(\alph*)]
        \item \(p(0) = 0\);
        \item \(p(u - v) \geq \abs{p(u) - p(v)}\) for all \(u, v \in V\);
        \item \(v \in V \implies p(v) \geq 0\); and
        \item \(N = \setc{v \in V}{p(v) = 0}\) is a linear subspace of \(V\).
    \end{enumerate}
\end{proposition}
\begin{proof}
    It is clear (a) follows from absolute homogeneity and that (c) follows from (b). Let \(u, v \in V\), then by subadditivity we have
    \begin{equation*}
        p(u) \leq p(u - v) + p(v) \implies p(u - v) \geq p(u) - p(v)
    \end{equation*}
    and thus
    \begin{equation*}
        p(u - v) = \abs{-1} p(v - u) = p(v - u) \geq p(v) - p(u),
    \end{equation*}
    yielding (b). From (a), we know \(0 \in N\), then for all \(u, v \in N\) and \(\lambda \in \mathbb{K}\) we have
    \begin{equation*}
        0 \leq p(u + \alpha v) \leq p(u) + \abs{\alpha} p(v) = 0,
    \end{equation*}
    hence \(N\) is a linear subspace.
\end{proof}

\begin{proposition}{Open ball with respect to a semi-norm}{ball_seminorm}
    Let \(V\) be a linear space. An \emph{open ball \(B_p(x, r)\) with respect to the semi-norm \(p : V \to \mathbb{R}\) of radius \(r > 0\) centered at \(x \in V\)} is the set
    \begin{equation*}
        B_p(x, r) = \setc{v \in V}{p(v - x) < r}.
    \end{equation*}
    Such a set enjoys the following properties:
    \begin{enumerate}[label=(\alph*)]
        \item \(x \in B_p(x, r)\);
        \item \(B_p(0, r)\) is convex: \(u, v \in B_p(0,r), \alpha \in (0,1) \implies \alpha u + (1 - \alpha)v \in B_p(0,r)\);
        \item \(B_p(0, r)\) is balanced: \(v \in B_p(0,r), \alpha \in \mathbb{K} : \abs{\alpha} \leq 1\implies \alpha v \in B_p(0,r)\);
        \item \(B_p(0, r)\) is absorbing: \(\forall v \in V : \exists \alpha > 0 : \alpha^{-1}v\in B_p(0,r)\); and
        \item \(p(v) = \inf\setc{\alpha r}{\alpha > 0 : \alpha^{-1} v \in B_p(0, r)}\) for all \(v \in V\).
    \end{enumerate}
\end{proposition}
\begin{proof}
    Property (a) follows from \cref{prop:seminorm_positive}. Let \(u, v \in B_p(0,r)\) and \(\alpha \in (0,1)\), then 
    \begin{equation*}
        p(\alpha u + (1 - \alpha)v) \leq p(\alpha u) + p((1 - \alpha)v) = \alpha p(u) + (1 - \alpha) p(v) < \alpha r + (1 - \alpha) r = r,
    \end{equation*}
    hence \(B_p(0,r)\) is convex. It is clear (c) follows from absolute homogeneity. Let \(x \in V,\) then if \(p(x) < r\), we have \(1^{-1}x \in B_p(0,r)\), and if \(p(x) \geq r\), we may set \(\alpha = \frac{2p(x)}{r} > 2\) such that \(p(\alpha^{-1} x) = \frac12r,\) that is, \(B_p(0,r)\) is absorbing. Let \(v \in V\), then for all \(\alpha > 0\), we have
    \begin{equation*}
        \alpha^{-1}v \in B_p(0,r) \iff p(\alpha^{-1}v) < r \iff p(v) < \alpha r,
    \end{equation*}
    hence \(p(v)\) is the greatest lower bound of \(\setc{\alpha r}{\alpha > 0 : \alpha^{-1} v \in B_p(0,r)}\), and (e) follows.
\end{proof}

\begin{proposition}{Topology generated by a family of semi-norms}{topology_seminorms}
    Let \(V\) be a linear space and let \(\mathcal{P}\) be a family of semi-norms on \(V\). The set
    \begin{equation*}
        \mathcal{S} = \bigcup_{x \in V}{\bigcup_{p \in \mathcal{P}}{\bigcup_{r > 0}{\setc{v \in V}{p(v - x) < r}}}}
    \end{equation*}
    is a subbasis for a topology on \(V\) relative to which the vector operations are continuous and every semi-norm \(p \in \mathcal{P}\) is continuous.
\end{proposition}
\begin{proof}
    Notice we have \(\bigcup_{x \in V}\bigcup_{p \in \mathcal{P}} B_p(x, 1) = V,\) then \(\mathcal{S}\) is a subbasis for a topology, \(\tau\) say. Notice
    \begin{equation*}
        \mathcal{S}_\mathcal{P} = \bigcup_{p \in \mathcal{P}}\bigcup_{r > 0} \setc{v \in V}{p(v) < r} = \bigcup_{p \in \mathcal{P}}\bigcup_{r > 0} B_p(0, r) = \bigcup_{p \in \mathcal{P}}\bigcup_{r > 0} \preim{p}{(-\infty,r)} \subset \mathcal{S}
    \end{equation*}
    is also a subbasis for a topology, namely the initial topology \(\tau_{\mathcal{P}}\) relative to \(\mathcal{P}\). By \cref{prop:subbasis_topology} we have \(\tau_{\mathcal{P}} \subset \tau\), hence every \(p \in \mathcal{P}\) is continuous with respect to \(\tau\).

    Let \(S \in \mathcal{S},\) then there exists \(x \in V,\) \(p \in \mathcal{P}\), and \(\varepsilon > 0\) such that \(S = B_p(x, \varepsilon)\). We consider the preimage of \(S\) under scalar multiplication, \(U = \setc{(\alpha, v) \in \mathbb{K} \times V}{p(\alpha v - x) < \varepsilon}\). Let \((\alpha, v) \in U\) and set \(r = p(\alpha v - x) < \varepsilon\), then notice for all \((\tilde{\alpha}, \tilde{v}) \in \mathbb{K} \times V\) we have
    \begin{equation*}
        p(\tilde{\alpha}\tilde{v} - x) \leq p(\tilde{\alpha}\tilde{v} - \tilde{\alpha} v) + p(\tilde{\alpha} v -  \alpha v) + p(\alpha v - x) = \abs{\tilde{\alpha}} p(\tilde{v} - v) + \abs{\tilde{\alpha} - \alpha} p(v) + r.
    \end{equation*}
    If \(p(v) = 0,\) we set \(A = \setc{\tilde{\alpha} \in \mathbb{K}}{\abs{\tilde{\alpha}} < 1}\) and \(B = B_p(v, \varepsilon - r)\), then \(A \times B \subset U\), hence \(U\) is open. If \(p(v) \neq 0\), we set \(A = \setc{\tilde{\alpha} \in \mathbb{K}}{\abs{\tilde{\alpha} - \alpha} < \frac{\varepsilon - r}{2p(v)}}\) and \(B = B_p(v, \frac{\varepsilon - r}{2\abs{a} + 2\frac{\varepsilon - r}{p(v)}}),\) then for all \(\tilde{\alpha} \in A\) and \(\tilde{v} \in B\) we have
    \begin{equation*}
        \abs{\tilde{\alpha}} p(\tilde{v} - v) \leq \left(\abs{\tilde{\alpha} - \alpha} + \abs{\alpha}\right) p(\tilde{v} - v) < \frac{\varepsilon - r}{2}
        \quad\text{and}\quad
        \abs{\tilde{\alpha} - \alpha} p(v) < \frac{\varepsilon - r}{2},
    \end{equation*}
    hence \(p(\tilde{\alpha} \tilde{v} - x) < \varepsilon\) and we conclude \(A \times B \subset U,\) hence \(U\) is open. Let us consider now the preimage of \(S\) under addition, \(\tilde{U} = \setc{(u,v) \in V\times V}{p(u + v - x) < \varepsilon}\). Let \((u,v) \in \tilde{U}\) and set \(\tilde{r} = p(u + v - x) < \varepsilon,\) then for all \(\tilde{u} \in C = B_p(u, \frac{\varepsilon - \tilde{r}}{2}\) and \(\tilde{v} \in D = B_p(v, \frac{\varepsilon - \tilde{r}}{2})\) we have
    \begin{equation*}
        p(\tilde{u} + \tilde{v} - x) \leq p(\tilde{u} - u) + p(\tilde{v} - v) + p(u + v - x) < \varepsilon,
    \end{equation*}
    hence \(C \times D \in \tilde{U}\), and we conclude \(\tilde{U}\) is open. By \cref{prop:continuity_subbasis}, it follows that addition and scalar multiplication is continuous.
\end{proof}

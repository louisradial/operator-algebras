% vim spl=en_us
\section{Topological linear spaces}
Throughout these notes we will consider linear spaces equipped with some topology. We begin by studying semi-norms defined on a linear space.

\begin{definition}{Seminorm on a linear space}{seminorm}
    Let \(V\) be a linear space over \(\mathbb{K}\). A map \(p : V \to \mathbb{R}\) satisfying
    \begin{enumerate}[label=(\alph*)]
        \item subadditivity: \(p(u + v) \leq p(u) + p(v)\);
        \item absolute homogeneity: \(p(\alpha v) = \abs{\alpha} p(v)\);
    \end{enumerate}
    for all \(u,v \in V\) and \(\alpha \in \mathbb{K}\), is a \emph{semi-norm on \(V\)}.
\end{definition}

\begin{proposition}{Semi-norms are positive}{seminorm_positive}
    Let \(V\) be a linear space. A semi-norm \(p : V \to \mathbb{R}\) satisfies 
    \begin{enumerate}[label=(\alph*)]
        \item \(p(0) = 0\);
        \item \(p(u - v) \geq \abs{p(u) - p(v)}\) for all \(u, v \in V\);
        \item \(v \in V \implies p(v) \geq 0\); and
        \item \(N = \setc{v \in V}{p(v) = 0}\) is a linear subspace of \(V\).
    \end{enumerate}
\end{proposition}
\begin{proof}
    It is clear (a) follows from absolute homogeneity and that (c) follows from (b). Let \(u, v \in V\), then by subadditivity we have
    \begin{equation*}
        p(u) \leq p(u - v) + p(v) \implies p(u - v) \geq p(u) - p(v)
    \end{equation*}
    and thus
    \begin{equation*}
        p(u - v) = \abs{-1} p(v - u) = p(v - u) \geq p(v) - p(u),
    \end{equation*}
    yielding (b). From (a), we know \(0 \in N\), then for all \(u, v \in N\) and \(\lambda \in \mathbb{K}\) we have
    \begin{equation*}
        0 \leq p(u + \alpha v) \leq p(u) + \abs{\alpha} p(v) = 0,
    \end{equation*}
    hence \(N\) is a linear subspace.
\end{proof}

\begin{proposition}{Open ball with respect to a semi-norm}{ball_seminorm}
    Let \(V\) be a linear space. An \emph{open ball \(B_p(x, r)\) with respect to the semi-norm \(p : V \to \mathbb{R}\) of radius \(r > 0\) centered at \(x \in V\)} is the set
    \begin{equation*}
        B_p(x, r) = \setc{v \in V}{p(v - x) < r}.
    \end{equation*}
    Such a set enjoys the following properties:
    \begin{enumerate}[label=(\alph*)]
        \item \(x \in B_p(x, r)\);
        \item \(B_p(0, r)\) is convex: \(u, v \in B_p(0,r), \alpha \in (0,1) \implies \alpha u + (1 - \alpha)v \in B_p(0,r)\);
        \item \(B_p(0, r)\) is balanced: \(v \in B_p(0,r), \alpha \in \mathbb{K} : \abs{\alpha} \leq 1\implies \alpha v \in B_p(0,r)\);
        \item \(B_p(0, r)\) is absorbing: \(\forall v \in V : \exists \alpha > 0 : \alpha^{-1}v\in B_p(0,r)\); and
        \item \(p(v) = \inf\setc{\alpha r}{\alpha > 0 : \alpha^{-1} v \in B_p(0, r)}\) for all \(v \in V\).
    \end{enumerate}
\end{proposition}
\begin{proof}
    Property (a) follows from \cref{prop:seminorm_positive}. Let \(u, v \in B_p(0,r)\) and \(\alpha \in (0,1)\), then 
    \begin{equation*}
        p(\alpha u + (1 - \alpha)v) \leq p(\alpha u) + p((1 - \alpha)v) = \alpha p(u) + (1 - \alpha) p(v) < \alpha r + (1 - \alpha) r = r,
    \end{equation*}
    hence \(B_p(0,r)\) is convex. It is clear (c) follows from absolute homogeneity. Let \(x \in V,\) then if \(p(x) < r\), we have \(1^{-1}x \in B_p(0,r)\), and if \(p(x) \geq r\), we may set \(\alpha = \frac{2p(x)}{r} > 2\) such that \(p(\alpha^{-1} x) = \frac12r,\) that is, \(B_p(0,r)\) is absorbing. Let \(v \in V\), then for all \(\alpha > 0\), we have
    \begin{equation*}
        \alpha^{-1}v \in B_p(0,r) \iff p(\alpha^{-1}v) < r \iff p(v) < \alpha r,
    \end{equation*}
    hence \(p(v)\) is the greatest lower bound of \(\setc{\alpha r}{\alpha > 0 : \alpha^{-1} v \in B_p(0,r)}\), and (e) follows.
\end{proof}

\begin{proposition}{Topology generated by a family of semi-norms}{topology_seminorms}
    Let \(V\) be a linear space and let \(\mathcal{P}\) be a family of semi-norms on \(V\). The set
    \begin{equation*}
        \mathcal{S} = \bigcup_{x \in V}{\bigcup_{p \in \mathcal{P}}{\bigcup_{r > 0}{\setc{v \in V}{p(v - x) < r}}}}
    \end{equation*}
    is a subbasis for a topology on \(V\) relative to which the vector operations are continuous and every semi-norm \(p \in \mathcal{P}\) is continuous.
\end{proposition}
\begin{proof}
    Notice we have \(\bigcup_{x \in V}\bigcup_{p \in \mathcal{P}} B_p(x, 1) = V,\) then \(\mathcal{S}\) is a subbasis for a topology, \(\tau\) say. Notice
    \begin{equation*}
        \mathcal{S}_\mathcal{P} = \bigcup_{p \in \mathcal{P}}\bigcup_{r > 0} \setc{v \in V}{p(v) < r} = \bigcup_{p \in \mathcal{P}}\bigcup_{r > 0} B_p(0, r) = \bigcup_{p \in \mathcal{P}}\bigcup_{r > 0} \preim{p}{(-\infty,r)} \subset \mathcal{S}
    \end{equation*}
    is also a subbasis for a topology, namely the initial topology \(\tau_{\mathcal{P}}\) relative to \(\mathcal{P}\). By \cref{prop:subbasis_topology} we have \(\tau_{\mathcal{P}} \subset \tau\), hence every \(p \in \mathcal{P}\) is continuous with respect to \(\tau\).

    Let \(S \in \mathcal{S},\) then there exists \(x \in V,\) \(p \in \mathcal{P}\), and \(\varepsilon > 0\) such that \(S = B_p(x, \varepsilon)\). We consider the preimage of \(S\) under scalar multiplication, \(U = \setc{(\alpha, v) \in \mathbb{K} \times V}{p(\alpha v - x) < \varepsilon}\). Let \((\alpha, v) \in U\) and set \(r = p(\alpha v - x) < \varepsilon\), then notice for all \((\tilde{\alpha}, \tilde{v}) \in \mathbb{K} \times V\) we have
    \begin{equation*}
        p(\tilde{\alpha}\tilde{v} - x) \leq p(\tilde{\alpha}\tilde{v} - \tilde{\alpha} v) + p(\tilde{\alpha} v -  \alpha v) + p(\alpha v - x) = \abs{\tilde{\alpha}} p(\tilde{v} - v) + \abs{\tilde{\alpha} - \alpha} p(v) + r.
    \end{equation*}
    If \(p(v) = 0,\) we set \(A = \setc{\tilde{\alpha} \in \mathbb{K}}{\abs{\tilde{\alpha}} < 1}\) and \(B = B_p(v, \varepsilon - r)\), then \(A \times B \subset U\), hence \(U\) is open. If \(p(v) \neq 0\), we set \(A = \setc{\tilde{\alpha} \in \mathbb{K}}{\abs{\tilde{\alpha} - \alpha} < \frac{\varepsilon - r}{2p(v)}}\) and \(B = B_p(v, \frac{\varepsilon - r}{2\abs{a} + 2\frac{\varepsilon - r}{p(v)}}),\) then for all \(\tilde{\alpha} \in A\) and \(\tilde{v} \in B\) we have
    \begin{equation*}
        \abs{\tilde{\alpha}} p(\tilde{v} - v) \leq \left(\abs{\tilde{\alpha} - \alpha} + \abs{\alpha}\right) p(\tilde{v} - v) < \frac{\varepsilon - r}{2}
        \quad\text{and}\quad
        \abs{\tilde{\alpha} - \alpha} p(v) < \frac{\varepsilon - r}{2},
    \end{equation*}
    hence \(p(\tilde{\alpha} \tilde{v} - x) < \varepsilon\) and we conclude \(A \times B \subset U,\) hence \(U\) is open. Let us consider now the preimage of \(S\) under addition, \(\tilde{U} = \setc{(u,v) \in V\times V}{p(u + v - x) < \varepsilon}\). Let \((u,v) \in \tilde{U}\) and set \(\tilde{r} = p(u + v - x) < \varepsilon,\) then for all \(\tilde{u} \in C = B_p(u, \frac{\varepsilon - \tilde{r}}{2}\) and \(\tilde{v} \in D = B_p(v, \frac{\varepsilon - \tilde{r}}{2})\) we have
    \begin{equation*}
        p(\tilde{u} + \tilde{v} - x) \leq p(\tilde{u} - u) + p(\tilde{v} - v) + p(u + v - x) < \varepsilon,
    \end{equation*}
    hence \(C \times D \in \tilde{U}\), and we conclude \(\tilde{U}\) is open. By \cref{prop:continuity_subbasis}, it follows that addition and scalar multiplication is continuous.
\end{proof}

We now define a topological linear space.
\begin{definition}{Topological linear space}{tvs}
    Let \((V, +, \cdot)\) be a linear space over a field \(\mathbb{K}\) (either \(\mathbb{R}\) or \(\mathbb{C}\)). A \emph{vector topology} \(\tau\) is a topology on \(V\) such that
    \begin{enumerate}[label=(\alph*)]
        \item for all \(v \in V\), \(V \setminus \set{v} \in \tau\); and
        \item the linear space operations are continuous with respect to \(\tau\);
    \end{enumerate}
    and we say \((V, \tau)\) is a \emph{topological linear space}.
\end{definition}
The first condition is equivalent to requiring a Hausdorff topology, as we now show.
\begin{proposition}{Necessary and sufficient conditions for a vector topology}{vector_topology}
    Let \(V\) be a linear space with a topology \(\tau\) such that the linear space operations are continuous. The following statements are equivalent:
    \begin{enumerate}[label=(\alph*)]
        \item \((V, \tau)\) is Hausdorff;
        \item \((V, \tau)\) is a topological linear space;
        \item for each \(v \in V \setminus\set{0}\), there exists \(U \in \tau\) such that \(0 \in U\) and \(x \notin U\).
    \end{enumerate}
\end{proposition}
\begin{proof}
    Suppose \(\tau\) is Hausdorff. Let \(v \in V,\) and consider the set \(V \setminus \set{v}\). Let \(u \in V\setminus \set{v}\), then there exists \(S, T \in \tau\) such that \(u \in S\), \(v \in T\), and \(S \cap T = \emptyset\), hence \(u \in S \subset V \setminus \set{v}\), and we conclude \(V \setminus \set{v}\) is open, as desired.

    Suppose \(\tau\) is a vector topology. 
\end{proof}
